%===================================================================================
\chapter{Description of the SUNLinearSolver module}\label{s:sunlinsol}
%===================================================================================
\index{SUNLinearSolver@\texttt{SUNLinearSolver} module}
% This is a shared SUNDIALS TEX file with description of
% the generic sunlinsol abstraction
%
For problems that involve the solution of linear systems of equations,
the {\sundials} solvers operate using generic linear solver modules
(of type \Id{SUNLinearSolver}), through a set of operations defined by
the particular {\sunlinsol} implementation.  These work in
coordination with the {\sundials} generic {\nvector} and {\sunmatrix}
modules to provide a set of compatible data structures and solvers for
the solution of linear systems using direct or iterative methods.
Moreover, users can provide their own specific {\sunlinsol}
implementation to each {\sundials} solver, particularly in cases where
they provide their own {\nvector} and/or {\sunmatrix} modules, and the
customized linear solver leverages these additional data structures
to create highly efficient and/or scalable solvers for their
particular problem.  Additionally, {\sundials} provides native
implementations {\sunlinsol} modules, as well as {\sunlinsol} modules
that interface between {\sundials} and external
linear solver libraries.

The various {\sundials} solvers have been designed to specifically
leverage the use of either \emph{direct linear solvers}
or \emph{scaled, preconditioned, iterative linear solvers}, through
their ``Dls'' and ``Spils'' interfaces, respectively.  Additionally,
{\sundials} solvers can make use of user-supplied custom linear
solvers, whether these are problem-specific or come from external
solver libraries.

For iterative (and possibly custom) linear solvers, the {\sundials}
solvers leverage scaling and preconditioning, as applicable, to
balance error between solution components and to accelerate
convergence of the linear solver.  To this end, instead of solving the 
linear system $Ax = b$ directly, we apply the underlying iterative
algorithm to the transformed system  
\begin{equation}
  \label{eq:transformed_linear_system}
  \tilde{A} \tilde{x} = \tilde{b}
\end{equation}
where
\begin{align}
  \notag
  \tilde{A} &= S_1 P_1^{-1} A P_2^{-1} S_2^{-1},\\
  \label{eq:transformed_linear_system_components}
  \tilde{b} &= S_1 P_1^{-1} b,\\
  \notag
  \tilde{x} &= S_2 P_2 x,
\end{align} 
and where
\begin{itemize}
\item $P_1$ is the left preconditioner,
\item $P_2$ is the right preconditioner,
\item $S_1$ is a diagonal matrix of scale factors for $P_1^{-1} b$,
\item $S_2$ is a diagonal matrix of scale factors for $P_2 x$.
\end{itemize}
The {\sundials} solvers request that iterative linear solvers stop
based on the 2-norm of the scaled preconditioned residual meeting a
prescribed tolerance
\[
  \left\| \tilde{b} - \tilde{A} \tilde{x} \right\|_2  <  \text{tol}.
\]
We note that not all of the iterative linear solvers implemented in
{\sundials} support the full range of the above options.  Similarly,
some of the {\sundials} integrators only utilize a subset of these
options.  Exceptions to the operators shown above are described in
the documentation for each {\sunlinsol} implementation, or for each
{\sundials} solver ``Spils'' interface.

The generic \ID{SUNLinearSolver} type has been modeled after the
object-oriented style of the generic \id{N\_Vector} type.
Specifically, a generic \ID{SUNLinearSolver} is a pointer to a structure
that has an implementation-dependent {\em content} field containing
the description and actual data of the linear solver, and an {\em ops}
field pointing to a structure with generic linear solver operations.
The type \id{SUNLinearSolver} is defined as
%%
%%
\begin{verbatim}
typedef struct _generic_SUNLinearSolver *SUNLinearSolver;

struct _generic_SUNLinearSolver {
  void *content;
  struct _generic_SUNLinearSolver_Ops *ops;
};
\end{verbatim}
%%
%%
The \id{\_generic\_SUNLinearSolver\_Ops} structure is essentially a
list of pointers to the various actual linear solver operations, and
is defined as 
%%
\begin{verbatim}
struct _generic_SUNLinearSolver_Ops {
  SUNLinearSolver_Type (*gettype)(SUNLinearSolver);
  int                  (*setatimes)(SUNLinearSolver, void*, ATimesFn);
  int                  (*setpreconditioner)(SUNLinearSolver, void*, 
                                            PSetupFn, PSolveFn);
  int                  (*setscalingvectors)(SUNLinearSolver,
                                            N_Vector, N_Vector);
  int                  (*initialize)(SUNLinearSolver);
  int                  (*setup)(SUNLinearSolver, SUNMatrix);
  int                  (*solve)(SUNLinearSolver, SUNMatrix, N_Vector, 
                                N_Vector, realtype);
  int                  (*numiters)(SUNLinearSolver);
  realtype             (*resnorm)(SUNLinearSolver);
  long int             (*lastflag)(SUNLinearSolver);
  int                  (*space)(SUNLinearSolver, long int*, long int*);
  N_Vector             (*resid)(SUNLinearSolver);
  int                  (*free)(SUNLinearSolver);
};
\end{verbatim}

The generic {\sunlinsol} module defines and implements the linear
solver operations acting on \id{SUNLinearSolver} objects.  These
routines are in fact only wrappers for the linear solver operations
defined by a particular {\sunlinsol} implementation, which are
accessed through the {\em ops} field of the \id{SUNLinearSolver}
structure. To illustrate this point we show below the implementation
of a typical linear solver operation from the generic {\sunlinsol}
module, namely \id{SUNLinSolInitialize}, which initializes a
{\sunlinsol} object for use after it has been created and configured,
and returns a flag denoting a successful/failed operation:
%%
%%
\begin{verbatim}
int SUNLinSolInitialize(SUNLinearSolver S)
{
  return ((int) S->ops->initialize(S));
}
\end{verbatim}
%%
%%
Table \ref{t:sunlinsolops} contains a complete list of all linear
solver operations defined by the generic {\sunlinsol} module.  In
order to support both direct and iterative linear solver types, the
generic {\sunlinsol} module defines linear solver routines (or 
arguments) that may be specific to individual use cases.  As such,
for each routine we specify its intended use.  If a custom
{\sunlinsol} module is provided, the function pointers for
non-required routines may be set to \id{NULL} to indicate that they
are not provided.

A particular implementation of the {\sunlinsol} module must:
\begin{itemize}
\item Specify the {\em content} field of the \id{SUNLinearSolver} object.
\item Define and implement a minimal subset of the linear solver
  operations. See the documentation for each {\sundials} linear solver
  interface to determine which {\sunlinsol} operations they require.

  Note that the names of these routines should be unique to that
  implementation in order to permit using more than one {\sunlinsol}
  module (each with different \id{SUNLinearSolver} internal data
  representations) in the same code. 
\item Define and implement user-callable constructor and destructor
  routines to create and free a \id{SUNLinearSolver} with
  the new {\em content} field and with {\em ops} pointing to the
  new linear solver operations.
\item Optionally, define and implement additional user-callable routines
  acting on the newly defined \id{SUNLinearSolver} (e.g., routines to
  set various configuration options for tuning the linear solver to a
  particular problem).
\item Optionally, provide functions as needed for that particular
  implementation to access different parts in the {\em
  content} field of the newly defined \id{SUNLinearSolver} object
  (e.g., routines to return various statistics from the solver).
\end{itemize}

Each {\sunlinsol} implementation included in {\sundials} has a ``type''
identifier specified in enumeration and shown in Table \ref{t:linsolIDs}.
It is recommended that a user-supplied {\sunlinsol} implementation set
this identifier based on the {\sundials} solver interface they intend
to use: ``Dls'' interfaces require \ID{SUNLINEARSOLVER\_DIRECT}
{\sunlinsol} objects, ``Spils'' interfaces require 
\ID{SUNLINEARSOLVER\_ITERATIVE} objects, and ``Cls'' interfaces 
require \ID{SUNLINEARSOLVER\_CUSTOM} objects. 

\begin{table}
\centering
\caption{Identifiers associated with linear solver kernels supplied with {\sundials}.}
\label{t:linsolIDs}
\medskip
\begin{tabular}{|l|l|c|}
\hline
{\bf Linear Solver ID} & {\bf Solver type} & {\bf ID Value} \\
\hline
SUNLINEARSOLVER\_DIRECT      & Direct solvers     & 0 \\ 
SUNLINEARSOLVER\_ITERATIVE   & Iterative solvers  & 1 \\
SUNLINEARSOLVER\_CUSTOM      & Custom solvers     & 2 \\
\hline
\end{tabular}
\end{table}




%---------------------------------------------------------------------------
% Table of matrix kernels
%---------------------------------------------------------------------------

\newlength{\colOne}
\settowidth{\colOne}{\id{SUNLinSolSetScalingVectors}}
\newlength{\colTwo}
\setlength{\colTwo}{\textwidth}
\addtolength{\colTwo}{-0.5in}
\addtolength{\colTwo}{-\colOne}

\tablecaption{Description of the \id{SUNLinearSolver} operations}\label{t:sunlinsolops}
\tablehead{\hline {\rule{0mm}{5mm}}{\bf Name} & {\bf Usage and  Description} \\[3mm] \hline\hline}
\tabletail{\hline \multicolumn{2}{|r|}{\small\slshape continued on next page} \\ \hline}
\begin{xtabular}{|p{\colOne}|p{\colTwo}|}
%%
\id{SUNLinSolGetType} & \id{type = SUNLinSolGetType(LS);} \\ 
& Returns the type identifier for the linear solver \id{LS}. It is
  used to determine the solver type (direct, iterative, or custom) from
  the abstract \id{SUNLinearSolver} interface.  This is used to assess
  compatibility with {\sundials}-provided linear solver interfaces.
  Returned values are given in the Table \ref{t:linsolIDs}. 
\\[2mm]
%%
\id{SUNLinSolInitialize} & \id{ier = SUNLinSolInitialize(LS);} \\ 
& Performs linear solver initialization (assumes that all
  solver-specific options have been set).  This should return zero for a
  successful call, and a negative value for a failure, ideally
  returning one of the generic error codes listed in Table 
  \ref{t:sunlinsolerr}. 
\\[2mm]
%%
\id{SUNLinSolSetup} & \id{ier = SUNLinSolSetup(LS, A);} \\
& Performs any linear solver setup needed, based on an updated system
  {\sunmatrix} \id{A}.  This may be called frequently (e.g.~with a full
  Newton method) or infrequently (for a modified Newton method), based
  on the type of integrator and/or nonlinear solver requesting the
  solves.  This should return zero for a successful call, a positive
  value for a recoverable failure and a negative value for an
  unrecoverable failure, ideally returning one of the generic error
  codes listed in Table \ref{t:sunlinsolerr}. 
\\[2mm]
%%
\id{SUNLinSolSolve} & \id{ier = SUNLinSolSolve(LS, A, x, b, tol);} \\
& Solves a linear system $Ax = b$.  This should return zero for a
  successful call, a positive value for a recoverable failure and a
  negative value for an unrecoverable failure, ideally returning one
  of the generic error codes listed in Table \ref{t:sunlinsolerr}.\\ 
& {\bf Direct solvers:} can ignore the \id{realtype}
  argument \id{tol}. \\
& {\bf Iterative solvers:} can ignore the {\sunmatrix} input \id{A}
  since a \id{NULL} argument will be passed (these should instead rely
  on the matrix-vector product function supplied through the
  routine \id{SUNLinSolSetATimes}).  These should attempt to solve to
  the specified \id{realtype} tolerance \id{tol} in a weighted 2-norm.
  If the solver does not support scaling then it should just use a
  2-norm. \\
& {\bf Custom solvers:} all arguments will be supplied, and if the
  solver is approximate then it should attempt to solve to
  the specified \id{realtype} tolerance \id{tol} in a weighted 2-norm.
  If the solver does not support scaling then it should just use a
  2-norm. \\
\\[2mm]
%%
\id{SUNLinSolFree} & \id{ier = SUNLinSolFree(LS);} \\
& Frees memory allocated by the linear solver.  This should return
  zero for a successful call, and a negative value for a failure.
\\[2mm]
%%
\id{SUNLinSolSetATimes} & \id{ier = SUNLinSolSetATimes(LS, A\_data, ATimes);} \\
& (Iterative/Custom linear solvers only)
  Provides \id{ATimesFn} function pointer, as well as a \id{void *} pointer 
  to a data structure used by this routine, to a linear solver object.
  {\sundials} solvers will call this function to set the matrix-vector
  product function to either a solver-provided difference-quotient via
  vector operations or a user-supplied solver-specific routine. This
  routine should return zero for a successful call, and a negative
  value for a failure, ideally returning one of the generic error
  codes listed in Table \ref{t:sunlinsolerr}. 
\\[2mm]
%%
\id{SUNLinSolSetPreconditioner} & \id{ier = SUNLinSolSetPreconditioner(LS, Pdata, Pset, Psol);} \\
& (Optional; Iterative/Custom linear solvers only)
  Provides \id{PSetupFn} and \id{PSolveFn} function pointers that
  implement the preconditioner solves $P_1^{-1}$ and $P_2^{-1}$ from
  equations \eqref{eq:transformed_linear_system}-\eqref{eq:transformed_linear_system_components}.
  This routine will be called by a {\sundials} 
  solver, which will provide translation between the generic \id{Pset}
  and \id{Psol} calls and the integrator-specific and integrator- or
  user-supplied routines.  This routine should return zero for a
  successful call, and a negative value for a failure, ideally
  returning one of the generic error codes listed in
  Table \ref{t:sunlinsolerr}. 
\\[2mm]
%%
\id{SUNLinSolSetScalingVectors} & \id{ier = SUNLinSolSetScalingVectors(LS, s1, s2);} \\
& (Optional; Iterative/Custom linear solvers only)
  Sets pointers to left/right scaling vectors for the linear system
  solve.  Here, \id{s1} is an {\nvector} of positive scale factors
  containing the diagonal of the matrix $S_1$ from
  equations \eqref{eq:transformed_linear_system}-\eqref{eq:transformed_linear_system_components}.
  Similarly, \id{s2} is an {\nvector} containing the diagonal of $S_2$
  from equations \eqref{eq:transformed_linear_system}-\eqref{eq:transformed_linear_system_components}.
  Neither of these vectors are tested for positivity, and a \id{NULL}
  argument for either indicates that the corresponding scaling matrix
  is the identity. This routine should return zero for a successful call,
  and a negative value for a failure, ideally returning one of the
  generic error codes listed in Table \ref{t:sunlinsolerr}. 
\\[2mm]
%%
\id{SUNLinSolNumIters} & \id{its = SUNLinSolNumIters(LS);} \\
& (Optional; Iterative/Custom linear solvers only)
  Should return the \id{int} number of linear iterations performed in
  the last `solve' call. 
\\[2mm]
%%
\id{SUNLinSolResNorm} & \id{rnorm = SUNLinSolResNorm(LS);} \\
& (Optional; Iterative/Custom linear solvers only)
  Should return the \id{realtype} final residual norm from the last
  `solve' call.
\\[2mm]
%%
\id{SUNLinSolResid} & \id{rvec = SUNLinSolResid(LS);} \\
& (Optional; Iterative/Custom linear solvers only)
  If an iterative method computes the preconditioned initial residual
  and returns with a successful solve without performing any
  iterations (i.e.~either the initial guess or the preconditioner is
  sufficiently accurate), then this function may be called by the
  {\sundials} solver.  This routine should return the {\nvector}
  containing the preconditioned initial residual vector.
\\[2mm]
%%
\id{SUNLinLastFlag} & \id{lflag = SUNLinLastFlag(LS);} \\
& (Optional) Should return the last error flag encountered within the
  linear solver. This is not called by the {\sundials} solvers 
  directly; it allows the user to investigate linear solver issues
  after a failed solve.
\\[2mm]
%%
\id{SUNLinSolSpace} & \id{ier = SUNLinSolSpace(LS, \&lrw, \&liw);} \\
& (Optional) Returns the storage requirements for the linear
  solver \id{LS}.  \id{lrw} is a \id{long int} containing the number
  of realtype words and \id{liw} is a \id{long int} containing the
  number of integer words.  The return value is an integer flag
  denoting success/failure of the operation. 

  This function is advisory only, for use in determining a user's
  total space requirements.
\\[2mm]
%%
\end{xtabular}
\bigskip



%---------------------------------------------------------------------------
% Tables listing compatibility with matrix/vector types
%---------------------------------------------------------------------------

We note that not all {\sunlinsol} types are compatible with all
{\sunmatrix} and {\nvector} types provided with {\sundials}.  In Table 
\ref{t:linsol-matrix} we show the direct linear solver interfaces
available as {\sunlinsol} modules, and the compatible matrix
implementations.  In Table \ref{t:linsol-vector} we show the
compatibility between all {\sunlinsol} modules and vector
implementations.

\tablecaption{{\sundials} direct linear solvers and matrix
              implementations that can be used for each.}\label{t:linsol-matrix} 
\tablehead{\hline \multicolumn{1}{|p{2cm}|}{{Linear Solver Interface}} &
                  \multicolumn{1}{p{1.3cm}|}{{Dense Matrix}} & 
                  \multicolumn{1}{p{1.3cm}|}{{Banded Matrix}} & 
                  \multicolumn{1}{p{1.4cm}|}{{Sparse Matrix}} & 
                  \multicolumn{1}{p{1.4cm}|}{{User Supplied}} \\ \hline}
\tabletail{\hline \multicolumn{5}{|r|}{\small\slshape continued on next page} \\ \hline}
{\renewcommand{\arraystretch}{1.2}
\begin{xtabular}{|l|c|c|c|c|}
%   Linear Solver &  Dense   & Banded & Sparse & User     \\
%   Interface     &          &        &        & Supplied \\ 
    Dense         &  \cm     &        &        & \cm      \\ 
    Band          &          & \cm    &        & \cm      \\
    Diagonal      &          &        &        & \cm      \\ 
    LapackDense   &  \cm     &        &        & \cm      \\ 
    LapackBand    &          & \cm    &        & \cm      \\
    \klu          &          &        &  \cm   & \cm      \\
    \superlumt    &          &        &  \cm   & \cm      \\
    User supplied &  \cm     & \cm    &  \cm   & \cm      \\ 
    \hline
\end{xtabular}}
\bigskip


\tablecaption{{\sundials} linear solver interfaces and vector
             implementations that can be used for each.}\label{t:linsol-vector}
\tablehead{\hline \multicolumn{1}{|p{2cm}|}{{Linear Solver Interface}} &
                  \multicolumn{1}{p{1.3cm}|}{{Serial}} & 
                  \multicolumn{1}{p{1.3cm}|}{{Parallel (MPI)}} & 
                  \multicolumn{1}{p{1.4cm}|}{{OpenMP}} & 
                  \multicolumn{1}{p{1.4cm}|}{{pThreads}} & 
                  \multicolumn{1}{p{1.3cm}|}{{{\hypre} Vector}} & 
                  \multicolumn{1}{p{1.3cm}|}{{{\petsc} Vector}} & 
                  \multicolumn{1}{p{1.4cm}|}{{User Supplied}} \\ \hline }
\tabletail{\hline \multicolumn{8}{|r|}{\small\slshape continued on next page} \\ \hline}
{\renewcommand{\arraystretch}{1.2}
\begin{xtabular}{|l|c|c|c|c|c|c|c|}
%   Linear Solver &  Serial  & Parallel  &  OpenMP  &  pThreads  & {\hypre}    & {\petsc} & User     \\
%   Interface     &          & (MPI)     &          &            &  Vector     & Vector   & Supplied \\ 
    Dense         &  \cm     &           & \cm      &  \cm       &             &          & \cm      \\ 
    Band          &  \cm     &           & \cm      &  \cm       &             &          & \cm      \\
    LapackDense   &  \cm     &           & \cm      &  \cm       &             &          & \cm      \\
    LapackBand    &  \cm     &           & \cm      &  \cm       &             &          & \cm      \\
    \klu          &  \cm     &           & \cm      &  \cm       &             &          & \cm      \\
    \superlumt    &  \cm     &           & \cm      &  \cm       &             &          & \cm      \\
    \spgmr        &  \cm     &  \cm      &  \cm     &  \cm       & \cm         &  \cm     & \cm      \\
    \spfgmr       &  \cm     &  \cm      &  \cm     &  \cm       & \cm         &  \cm     & \cm      \\
    \spbcg        &  \cm     &  \cm      &  \cm     &  \cm       & \cm         &  \cm     & \cm      \\
    \sptfqmr      &  \cm     &  \cm      &  \cm     &  \cm       & \cm         &  \cm     & \cm      \\ 
    \pcg          &  \cm     &  \cm      &  \cm     &  \cm       & \cm         &  \cm     & \cm      \\
    User supplied &  \cm     &  \cm      &  \cm     &  \cm       & \cm         &  \cm     & \cm      \\ 
    \hline
\end{xtabular}}
\bigskip


%---------------------------------------------------------------------------
% Table of linear solver error codes
%---------------------------------------------------------------------------

\newlength{\ColumnOne}
\settowidth{\ColumnOne}{\id{SUNLS\_PACKAGE\_FAIL\_UNREC}}
\newlength{\ColumnTwo}
\settowidth{\ColumnTwo}{\id{Value}}
\newlength{\ColumnThree}
\setlength{\ColumnThree}{\textwidth}
\addtolength{\ColumnThree}{-0.5in}
\addtolength{\ColumnThree}{-\ColumnOne}
\addtolength{\ColumnThree}{-\ColumnTwo}

\tablecaption{Description of the \id{SUNLinearSolver} error codes}\label{t:sunlinsolerr}
\tablehead{\hline {\rule{0mm}{5mm}}{\bf Name} & {\bf Value} & {\bf Description} \\[3mm] \hline\hline}
\tabletail{\hline \multicolumn{3}{|r|}{\small\slshape continued on next page} \\ \hline}
\begin{xtabular}{|p{\ColumnOne}|p{\ColumnTwo}|p{\ColumnThree}|}
%%
\id{SUNLS\_SUCCESS} & \id{0} & successful call or converged solve
\\[1mm]
%%
\id{SUNLS\_MEM\_NULL} & \id{-1} & the memory argument to the function is \id{NULL}
\\[2mm]
%%
\id{SUNLS\_ILL\_INPUT} & \id{-2} & an illegal input has been provided to the function 
\\[2mm]
%%
\id{SUNLS\_MEM\_FAIL} & \id{-3} & failed memory access or allocation
\\[2mm]
%%
\id{SUNLS\_ATIMES\_FAIL\_UNREC} & \id{-4} & an unrecoverable failure
  occurred in the \id{ATimes} routine
\\[2mm]
%%
\id{SUNLS\_PSET\_FAIL\_UNREC} & \id{-5} & an unrecoverable failure
  occurred in the \id{Pset} routine
\\[2mm]
%%
\id{SUNLS\_PSOLVE\_FAIL\_UNREC} & \id{-6} & an unrecoverable failure
  occurred in the \id{Psolve} routine
\\[2mm]
%%
\id{SUNLS\_PACKAGE\_FAIL\_UNREC} & \id{-7} & an unrecoverable failure
  occurred in an external linear solver package
\\[2mm]
%%
\id{SUNLS\_GS\_FAIL} & \id{-8} & a failure occurred during
  Gram-Schmidt orthogonalization ({\sunlinsolspgmr}/{\sunlinsolspfgmr})
\\[2mm]
%%
\id{SUNLS\_QRSOL\_FAIL} & \id{-9} & a singular $R$ matrix was
  encountered in a QR factorization ({\sunlinsolspgmr}/{\sunlinsolspfgmr})
\\[2mm]
%%
\id{SUNLS\_RES\_REDUCED} & \id{1} &  an iterative solver reduced the
  residual, but did not converge to the desired tolerance
\\[2mm]
%%
\id{SUNLS\_CONV\_FAIL} & \id{2} &  an iterative solver did not
converge (and the residual was not reduced)
\\[2mm]
%%
\id{SUNLS\_ATIMES\_FAIL\_REC} & \id{3} & a recoverable failure occurred
  in the \id{ATimes} routine
\\[2mm]
%%
\id{SUNLS\_PSET\_FAIL\_REC} & \id{4} & a recoverable failure occurred
  in the \id{Pset} routine
\\[2mm]
%%
\id{SUNLS\_PSOLVE\_FAIL\_REC} & \id{5} & a recoverable failure occurred
  in the \id{Psolve} routine
\\[2mm]
%%
\id{SUNLS\_PACKAGE\_FAIL\_REC} & \id{6} &  a recoverable failure
  occurred in an external linear solver package
\\[2mm]
%%
\id{SUNLS\_QRFACT\_FAIL} & \id{7} & a singular matrix was encountered
  during a QR factorization ({\sunlinsolspgmr}/{\sunlinsolspfgmr})
\\[2mm]
%%
\id{SUNLS\_LUFACT\_FAIL} & \id{8} & a singular matrix was encountered
  during a LU factorization ({\sunlinsoldense}/{\sunlinsolband})
\\[2mm]
%%
%%
\end{xtabular}
\bigskip



%---------------------------------------------------------------------------
\section{The SUNLinearSolver\_Dense implementation}\label{ss:sunlinsol_dense}
%% This is a shared SUNDIALS TEX file with a description of the
%% dense sunlinsol implementation
%%

The dense implementation of the {\sunlinsol} module provided with
{\sundials}, {\sunlinsoldense}, is designed to be used with the
corresponding {\sunmatdense} matrix type, and one of the serial or
shared-memory {\nvector} implementations ({\nvecs}, {\nvecopenmp} or
{\nvecpthreads}).  The {\sunlinsoldense} module defines the {\em
content} field of a \id{SUNLinearSolver} to be the following structure:
%%
\begin{verbatim} 
struct _SUNLinearSolverContent_Dense {
  sunindextype N;
  sunindextype *pivots;
  long int last_flag;
};
\end{verbatim}
%%
These entries of the \emph{content} field contain the following
information:
\begin{description}
  \item[N] - size of the linear system,
  \item[pivots] - index array for partial pivoting in LU factorization,
  \item[last\_flag] - last error return flag from internal function evaluations.
\end{description}

This solver is constructed to perform the following operations:
\begin{itemize}
\item In the ``setup'' call, this performs a $LU$ factorization with
  partial (row) pivoting ($\mathcal O(N^3)$ cost), $PA=LU$, where $P$
  is a permutation matrix, $L$ is a lower triangular matrix with 1's
  on the diagonal, and $U$ is an upper triangular matrix.  This
  factorization is stored in-place on the input {\sunmatdense} object
  $A$, with pivoting information encoding $P$ stored in
  the \id{pivots} array.
\item In the ``solve'' call, this performs pivoting, forward and
  backward substitution using the stored \id{pivots} array and the
  $LU$ factors held in the {\sunmatdense} object ($\mathcal O(N^2)$
  cost).
\end{itemize}

\noindent The header file to be included when using this module 
is \id{sunlinsol/sunlinsol\_dense.h}. \\
%%
%%----------------------------------------------
%%
The {\sunlinsoldense} module defines dense implementations of all
``direct'' linear solver operations listed in
Table \ref{t:sunlinsolops}:
\begin{itemize}
\item \id{SUNLinSolGetType\_Dense}
\item \id{SUNLinSolInitialize\_Dense} -- this does nothing, since all
  consistency checks were performed at solver creation.
\item \id{SUNLinSolSetup\_Dense} -- this performs the $LU$ factorization.
\item \id{SUNLinSolSolve\_Dense} -- this uses the $LU$ factors
  and \id{pivots} array to perform the solve.
\item \id{SUNLinSolLastFlag\_Dense}
\item \id{SUNLinSolSpace\_Dense} -- this only returns information for
  the storage \emph{within} the solver object, i.e.~storage
  for \id{N}, \id{last\_flag} and \id{pivots}.
\item \id{SUNLinSolFree\_Dense}
\end{itemize}
The module {\sunlinsoldense} provides the following additional
user-callable routine: 
%%
\begin{itemize}

%%--------------------------------------

\item \ID{SUNDenseLinearSolver}

  This function creates and allocates memory for a dense \id{SUNLinearSolver}.
  Its arguments are an {\nvector} and {\sunmatrix}, that it uses to
  determine the linear system size and to assess compatibility with
  the linear solver implementation.

  This routine will perform consistency checks to ensure that it is
  called with consistent {\nvector} and {\sunmatrix} implementations.
  These are currently limited to the {\sunmatdense} matrix type, and
  the {\nvecs}, {\nvecopenmp} and {\nvecpthreads} vector types.  As
  additional compatible matrix and vector implementations are added to
  {\sundials}, these will be included within this compatibility check.

  If either \id{A} or \id{y} are incompatible then this routine will
  return \id{NULL}.

  \verb|SUNLinearSolver SUNDenseLinearSolver(N_Vector y, SUNMatrix A);|

\end{itemize}
%%
%%------------------------------------
%%
For solvers that include a Fortran interface module, the {\sunlinsoldense}
module also includes the Fortran-callable
function \id{FSUNDenseLinSolInit(code, ier)} to initialize
this {\sunlinsoldense} module for a given {\sundials} solver.
Here \id{code} is an input solver id (1 for {\cvode}, 2 for {\ida}, 3
for {\kinsol}, 4 for {\arkode}); \id{ier} is an error return flag 
equal 0 for success and -1 for failure (declared so as to match C type
\id{int}).  This routine must be called \emph{after} both the
{\nvector} and {\sunmatrix} objects have been initialized.
Additionally, when using {\arkode} with non-identity mass matrix, the
Fortran-callable function \id{FSUNMassDenseLinSolInit(ier)}  
initializes this {\sunlinsoldense} module for solving mass matrix
linear systems.


%---------------------------------------------------------------------------
\section{The SUNLinearSolver\_Band implementation}\label{ss:sunlinsol_band}
%% This is a shared SUNDIALS TEX file with a description of the
%% band sunlinsol implementation
%%

The band implementation of the {\sunlinsol} module provided with
{\sundials}, {\sunlinsolband}, is designed to be used with the
corresponding {\sunmatband} matrix type, and one of the serial or
shared-memory {\nvector} implementations ({\nvecs}, {\nvecopenmp} or
{\nvecpthreads}).


%---------------------------------------------------------------------------
\subsection{{\sunlinsolband} usage}\label{ss:sunlinsol_band_usage}

The header file to include when using this module is
\id{sunlinsol/sunlinsol\_band.h}. The {\sunlinsolband} module 
is accessible from all {\sundials} solvers \textit{without}
linking to the \\ \noindent
\id{libsundials\_sunlinsolband} module library.

The module {\sunlinsolband} provides the following user-callable constructor routine: 
%%
% --------------------------------------------------------------------
\ucfunction{SUNLinSol\_Band}
{
  LS = SUNLinSol\_Band(y, A);
}
{
  The function \ID{SUNLinSol\_Band} creates and allocates memory for
  a band \id{SUNLinearSolver} object.
}
{
  \begin{args}[y]
  \item[y] (\id{N\_Vector})
    a template for cloning vectors needed within the solver
  \item[A] (\id{SUNMatrix})
    a {\sunmatband} matrix template for cloning matrices needed
    within the solver 
  \end{args}
}
{
  This returns a \id{SUNLinearSolver} object.  If either \id{A} or
  \id{y} are incompatible then this routine will return \id{NULL}.
}
{
  This routine will perform consistency checks to ensure that it is
  called with consistent {\nvector} and {\sunmatrix} implementations.
  These are currently limited to the {\sunmatdense} matrix type and
  the {\nvecs}, {\nvecopenmp}, and {\nvecpthreads} vector types.  As
  additional compatible matrix and vector implementations are added to
  {\sundials}, these will be included within this compatibility check.

  Additionally, this routine will verify that the input matrix \id{A}
  is allocated with appropriate upper bandwidth storage for the $LU$
  factorization.
}
% --------------------------------------------------------------------
%%
For backwards compatibility, we also provide the wrapper functions:
\begin{itemize}

\item \ID{SUNBandLinearSolver}

  Wrapper function for \ID{SUNLinSol\_Band}, with identical input and
  output arguments.

\end{itemize}
%%
%%------------------------------------
%%
For solvers that include a Fortran interface module, the {\sunlinsolband}
module also includes a Fortran-callable function for creating a
\id{SUNLinearSolver} object.
\ucfunction{FSUNBANDLINSOLINIT}
{
  FSUNBANDLINSOLINIT(code, ier)
}
{
  The function \ID{FSUNBANDLINSOLINIT} can be called for Fortran programs
  to create a band \id{SUNLinearSolver} object.
}
{
  \begin{args}[code]
  \item[code] (\id{int*})
    is an integer input specifying the solver id (1 for {\cvode}, 2
    for {\ida}, 3 for {\kinsol}, and 4 for {\arkode}).
  \end{args}
}
{
  \id{ier} is a return completion flag equal to \id{0} for a success
  return and \id{-1} otherwise. See printed message for details in case
  of failure.
}
{
  This routine must be
  called \emph{after} both the {\nvector} and {\sunmatrix} objects have
  been initialized.
}
Additionally, when using {\arkode} with a non-identity
mass matrix, the {\sunlinsolband} module includes a Fortran-callable
function for creating a \id{SUNLinearSolver} mass matrix solver
object.
\ucfunction{FSUNMASSBANDLINSOLINIT}
{
  FSUNMASSBANDLINSOLINIT(ier)
}
{
  The function \ID{FSUNMASSBANDLINSOLINIT} can be called for Fortran programs
  to create a band \id{SUNLinearSolver} object for mass matrix linear
  systems.
}
{
}
{
  \id{ier} is a \id{int} return completion flag equal to \id{0} for a success
  return and \id{-1} otherwise. See printed message for details in case
  of failure.
}
{
  This routine must be
  called \emph{after} both the {\nvector} and {\sunmatrix} mass-matrix
  objects have been initialized.
}

%---------------------------------------------------------------------------
\subsection{{\sunlinsolband} description}\label{ss:sunlinsol_band_description}



The {\sunlinsolband} module defines the {\em
content} field of a \id{SUNLinearSolver} to be the following structure:
%%
\begin{verbatim} 
struct _SUNLinearSolverContent_Band {
  sunindextype N;
  sunindextype *pivots;
  long int last_flag;
};
\end{verbatim}
%%
These entries of the \emph{content} field contain the following
information:
\begin{description}
  \item[N] - size of the linear system,
  \item[pivots] - index array for partial pivoting in LU factorization,
  \item[last\_flag] - last error return flag from internal function evaluations.
\end{description}

This solver is constructed to perform the following operations:
\begin{itemize}
\item The ``setup'' call performs a $LU$ factorization with
  partial (row) pivoting, $PA=LU$, where $P$ is a permutation matrix,
  $L$ is a lower triangular matrix with 1's on the diagonal, and $U$
  is an upper triangular matrix.  This factorization is stored
  in-place on the input {\sunmatband} object $A$, with pivoting
  information encoding $P$ stored in the \id{pivots} array.
\item The ``solve'' call performs pivoting and forward and
  backward substitution using the stored \id{pivots} array and the
  $LU$ factors held in the {\sunmatband} object.
\item
  {\warn} $A$ must be allocated to accommodate the increase in upper
  bandwidth that occurs during factorization.  More precisely, if $A$
  is a band matrix with upper bandwidth \id{mu} and lower bandwidth
  \id{ml}, then the upper triangular factor $U$ can have upper
  bandwidth as big as \id{smu = MIN(N-1,mu+ml)}. The lower triangular
  factor $L$ has lower bandwidth \id{ml}.
\end{itemize}


%%
%%----------------------------------------------
%%

\noindent The {\sunlinsolband} module defines band implementations of all
``direct'' linear solver operations listed in Sections
\ref{ss:sunlinsol_CoreFn}-\ref{ss:sunlinsol_GetFn}:
\begin{itemize}
\item \id{SUNLinSolGetType\_Band}
\item \id{SUNLinSolInitialize\_Band} -- this does nothing, since all
  consistency checks are performed at solver creation.
\item \id{SUNLinSolSetup\_Band} -- this performs the $LU$ factorization.
\item \id{SUNLinSolSolve\_Band} -- this uses the $LU$ factors
  and \id{pivots} array to perform the solve.
\item \id{SUNLinSolLastFlag\_Band}
\item \id{SUNLinSolSpace\_Band} -- this only returns information for
  the storage \emph{within} the solver object, i.e.~storage
  for \id{N}, \id{last\_flag}, and \id{pivots}.
\item \id{SUNLinSolFree\_Band}
\end{itemize}


%---------------------------------------------------------------------------
\section{The SUNLinearSolver\_LapackDense implementation}\label{ss:sunlinsol_lapdense}
%% This is a shared SUNDIALS TEX file with a description of the
%% lapackdense sunlinsol implementation
%%
\section{The SUNLinearSolver\_LapackDense implementation}\label{ss:sunlinsol_lapdense}

The LAPACK dense implementation of the {\sunlinsol} module provided
with {\sundials}, {\sunlinsollapdense}, is designed to be used with the 
corresponding {\sunmatdense} matrix type, and one of the serial or
shared-memory {\nvector} implementations ({\nvecs}, {\nvecopenmp}, or
{\nvecpthreads}).

%---------------------------------------------------------------------------
\subsection{{\sunlinsollapdense} usage}\label{ss:sunlinsol_lapdense_usage}

The header file to include when using this module 
is \id{sunlinsol/sunlinsol\_lapackdense.h}. The installed module
library to link to is
\id{libsundials\_sunlinsollapackdense.\textit{lib}}
where \id{\em.lib} is typically \id{.so} for shared libraries and
\id{.a} for static libraries.

The module {\sunlinsollapdense} provides the following user-callable constructor routine: 
%%
% --------------------------------------------------------------------
\ucfunction{SUNLinSol\_LapackDense}
{
  LS = SUNLinSol\_LapackDense(y, A);
}
{
  The function \ID{SUNLinSol\_LapackDense} creates and allocates
  memory for a LAPACK-based, dense \id{SUNLinearSolver} object.
}
{
  \begin{args}[y]
  \item[y] (\id{N\_Vector})
    a template for cloning vectors needed within the solver
  \item[A] (\id{SUNMatrix})
    a {\sunmatdense} matrix template for cloning matrices needed
    within the solver 
  \end{args}
}
{
  This returns a \id{SUNLinearSolver} object.  If either \id{A} or
  \id{y} are incompatible then this routine will return \id{NULL}.
}
{
  This routine will perform consistency checks to ensure that it is
  called with consistent {\nvector} and {\sunmatrix} implementations.
  These are currently limited to the {\sunmatdense} matrix type and
  the {\nvecs}, {\nvecopenmp}, and {\nvecpthreads} vector types.  As
  additional compatible matrix and vector implementations are added to
  {\sundials}, these will be included within this compatibility check.
}
% --------------------------------------------------------------------
%%
For backwards compatibility, we also provide the wrapper function,
\begin{itemize}

\item \ID{SUNLapackDense}

  Wrapper function for \ID{SUNLinSol\_LapackDense}, with identical input and
  output arguments.
  
\end{itemize}
%%
%%------------------------------------
%%
For solvers that include a Fortran interface module, the
{\sunlinsollapdense} module also includes a Fortran-callable function
for creating a \id{SUNLinearSolver} object.
\ucfunction{FSUNLAPACKDENSEINIT}
{
  FSUNLAPACKDENSEINIT(code, ier)
}
{
  The function \ID{FSUNLAPACKDENSEINIT} can be called for Fortran programs
  to create a LAPACK-based dense \id{SUNLinearSolver} object.
}
{
  \begin{args}[code]
  \item[code] (\id{int*})
    is an integer input specifying the solver id (1 for {\cvode}, 2
    for {\ida}, 3 for {\kinsol}, and 4 for {\arkode}).
  \end{args}
}
{
  \id{ier} is a return completion flag equal to \id{0} for a success
  return and \id{-1} otherwise. See printed message for details in case
  of failure.
}
{
  This routine must be
  called \emph{after} both the {\nvector} and {\sunmatrix} objects have
  been initialized.
}
Additionally, when using {\arkode} with a non-identity
mass matrix, the {\sunlinsollapdense} module includes a Fortran-callable
function for creating a \id{SUNLinearSolver} mass matrix solver
object.
\ucfunction{FSUNMASSLAPACKDENSEINIT}
{
  FSUNMASSLAPACKDENSEINIT(ier)
}
{
  The function \ID{FSUNMASSLAPACKDENSEINIT} can be called for Fortran programs
  to create a LAPACK-based, dense \id{SUNLinearSolver} object for mass
  matrix linear systems.
}
{
}
{
  \id{ier} is a \id{int} return completion flag equal to \id{0} for a success
  return and \id{-1} otherwise. See printed message for details in case
  of failure.
}
{
  This routine must be
  called \emph{after} both the {\nvector} and {\sunmatrix} mass-matrix
  objects have been initialized.
}


%---------------------------------------------------------------------------
\subsection{{\sunlinsollapdense} description}\label{ss:sunlinsol_lapdense_description}

The {\sunlinsollapdense} module defines the {\em
content} field of a \\
\noindent\id{SUNLinearSolver} to be the following structure:
%%
\begin{verbatim} 
struct _SUNLinearSolverContent_Dense {
  sunindextype N;
  sunindextype *pivots;
  long int last_flag;
};
\end{verbatim}
%%
These entries of the \emph{content} field contain the following
information:
\begin{description}
  \item[N] - size of the linear system,
  \item[pivots] - index array for partial pivoting in LU factorization,
  \item[last\_flag] - last error return flag from internal function evaluations.
\end{description}

{\warn} The {\sunlinsollapdense} module is a {\sunlinsol} wrapper for
the LAPACK dense matrix factorization and solve routines, \id{*GETRF}
and \id{*GETRS}, where \id{*} is either \id{D} or \id{S}, depending on
whether {\sundials} was configured to have \id{realtype} set to
\id{double} or \id{single}, respectively (see Section \ref{s:types}).
In order to use the {\sunlinsollapdense} module it is assumed
that LAPACK has been installed on the system prior to installation of
{\sundials}, and that {\sundials} has been configured appropriately to
link with LAPACK (see Appendix \ref{c:install} for details).  
We note that since there do not exist 128-bit floating-point
factorization and solve routines in LAPACK, this interface cannot be
compiled when using \id{extended} precision for \id{realtype}.
Similarly, since there do not exist 64-bit integer LAPACK routines,
the {\sunlinsollapdense} module also cannot be compiled when using
\id{int64\_t} for the \id{sunindextype}.

This solver is constructed to perform the following operations:
\begin{itemize}
\item The ``setup'' call performs a $LU$ factorization with
  partial (row) pivoting ($\mathcal O(N^3)$ cost), $PA=LU$, where $P$
  is a permutation matrix, $L$ is a lower triangular matrix with 1's
  on the diagonal, and $U$ is an upper triangular matrix.  This
  factorization is stored in-place on the input {\sunmatdense} object
  $A$, with pivoting information encoding $P$ stored in
  the \id{pivots} array.
\item The ``solve'' call performs pivoting and forward and
  backward substitution using the stored \id{pivots} array and the
  $LU$ factors held in the {\sunmatdense} object ($\mathcal O(N^2)$
  cost).
\end{itemize}

%%
%%----------------------------------------------
%%

\noindent The {\sunlinsollapdense} module defines dense implementations of all
``direct'' linear solver operations listed in Sections
\ref{ss:sunlinsol_CoreFn}-\ref{ss:sunlinsol_GetFn}:
\begin{itemize}
\item \id{SUNLinSolGetType\_LapackDense}
\item \id{SUNLinSolInitialize\_LapackDense} -- this does nothing, since all
  consistency checks are performed at solver creation.
\item \id{SUNLinSolSetup\_LapackDense} -- this calls either
  \id{DGETRF} or \id{SGETRF} to perform the $LU$ factorization.
\item \id{SUNLinSolSolve\_LapackDense} -- this calls either
  \id{DGETRS} or \id{SGETRS} to use the $LU$ factors and \id{pivots}
  array to perform the solve.
\item \id{SUNLinSolLastFlag\_LapackDense}
\item \id{SUNLinSolSpace\_LapackDense} -- this only returns information for
  the storage \emph{within} the solver object, i.e.~storage
  for \id{N}, \id{last\_flag}, and \id{pivots}.
\item \id{SUNLinSolFree\_LapackDense}
\end{itemize}


%---------------------------------------------------------------------------
\section{The SUNLinearSolver\_LapackBand implementation}\label{ss:sunlinsol_lapband}
%% This is a shared SUNDIALS TEX file with a description of the
%% lapackband sunlinsol implementation
%%

The LAPACK band implementation of the {\sunlinsol} module provided
with {\sundials}, {\sunlinsollapband}, is designed to be used with the
corresponding {\sunmatband} matrix type, and one of the serial or
shared-memory {\nvector} implementations ({\nvecs}, {\nvecopenmp}, or
{\nvecpthreads}).  The {\sunlinsollapband} module defines the {\em
content} field of a\\
\noindent\id{SUNLinearSolver} to be the following structure:
%%
\begin{verbatim} 
struct _SUNLinearSolverContent_Band {
  sunindextype N;
  sunindextype *pivots;
  long int last_flag;
};
\end{verbatim}
%%
These entries of the \emph{content} field contain the following
information:
\begin{description}
  \item[N] - size of the linear system,
  \item[pivots] - index array for partial pivoting in LU factorization,
  \item[last\_flag] - last error return flag from internal function evaluations.
\end{description}

{\warn} The {\sunlinsollapband} module is a {\sunlinsol} wrapper for
the LAPACK band matrix factorization and solve routines, \id{*GBTRF}
and \id{*GBTRS}, where \id{*} is either \id{D} or \id{S}, depending on
whether {\sundials} was configured to have \id{realtype} set to
\id{double} or \id{single}, respectively (see Section \ref{s:types}).
In order to use the {\sunlinsollapband} module it is assumed
that LAPACK has been installed on the system prior to installation of
{\sundials}, and that {\sundials} has been configured appropriately to
link with LAPACK (see Appendix \ref{c:install} for details).  We note
that since there do not exist 128-bit floating-point factorization and
solve routines in LAPACK, this interface cannot be compiled when
using \id{extended} precision for \id{realtype}.  Similarly, since
there do not exist 64-bit integer LAPACK routines, the
{\sunlinsollapband} module also cannot be compiled when using 
\id{int64\_t} for the \id{sunindextype}.

This solver is constructed to perform the following operations:
\begin{itemize}
\item The ``setup'' call performs a $LU$ factorization with
  partial (row) pivoting, $PA=LU$, where $P$ is a permutation matrix,
  $L$ is a lower triangular matrix with 1's on the diagonal, and $U$
  is an upper triangular matrix.  This factorization is stored
  in-place on the input {\sunmatband} object $A$, with pivoting
  information encoding $P$ stored in the \id{pivots} array.
\item The ``solve'' call performs pivoting and forward and
  backward substitution using the stored \id{pivots} array and the
  $LU$ factors held in the {\sunmatband} object.
\item
  {\warn} $A$ must be allocated to accommodate the increase in upper
  bandwidth that occurs during factorization.  More precisely, if $A$
  is a band matrix with upper bandwidth \id{mu} and lower bandwidth
  \id{ml}, then the upper triangular factor $U$ can have upper
  bandwidth as big as \id{smu = MIN(N-1,mu+ml)}. The lower triangular
  factor $L$ has lower bandwidth \id{ml}.
\end{itemize}


\noindent The header file to be included when using this module 
is \id{sunlinsol/sunlinsol\_lapackband.h}. \\
%%
%%----------------------------------------------
%%
The {\sunlinsollapband} module defines band implementations of all
``direct'' linear solver operations listed in
Table \ref{t:sunlinsolops}:
\begin{itemize}
\item \id{SUNLinSolGetType\_LapackBand}
\item \id{SUNLinSolInitialize\_LapackBand} -- this does nothing, since all
  consistency checks are performed at solver creation.
\item \id{SUNLinSolSetup\_LapackBand} -- this calls either
  \id{DGBTRF} or \id{SGBTRF} to perform the $LU$ factorization.
\item \id{SUNLinSolSolve\_LapackBand} -- this calls either
  \id{DGBTRS} or \id{SGBTRS} to use the $LU$ factors and \id{pivots}
  array to perform the solve.
\item \id{SUNLinSolLastFlag\_LapackBand}
\item \id{SUNLinSolSpace\_LapackBand} -- this only returns information for
  the storage \emph{within} the solver object, i.e.~storage
  for \id{N}, \id{last\_flag}, and \id{pivots}.
\item \id{SUNLinSolFree\_LapackBand}
\end{itemize}
The module {\sunlinsollapband} provides the following additional
user-callable routine: 
%%
\begin{itemize}

%%--------------------------------------

\item \ID{SUNLapackBand}

  This function creates and allocates memory for a LAPACK band
  \id{SUNLinearSolver}.  Its arguments are an {\nvector} and
  {\sunmatrix}, that it uses to determine the linear system size and
  to assess compatibility with the linear solver implementation.

  This routine will perform consistency checks to ensure that it is
  called with consistent {\nvector} and {\sunmatrix} implementations.
  These are currently limited to the {\sunmatband} matrix type and
  the {\nvecs}, {\nvecopenmp}, and {\nvecpthreads} vector types.  As
  additional compatible matrix and vector implementations are added to
  {\sundials}, these will be included within this compatibility check.

  Additionally, this routine will verify that the input matrix \id{A}
  is allocated with appropriate upper bandwidth storage for the $LU$
  factorization.

  If either \id{A} or \id{y} are incompatible then this routine will
  return \id{NULL}.

  \verb|SUNLinearSolver SUNLapackBand(N_Vector y, SUNMatrix A);|

\end{itemize}
%%
%%------------------------------------
%%
For solvers that include a Fortran interface module, the
{\sunlinsollapband} module also includes the Fortran-callable
function \id{FSUNLapackBandInit(code, ier)} to initialize
this\\
\noindent {\sunlinsollapband} module for a given {\sundials} solver.
Here \id{code} is an integer input solver id (1 for {\cvode}, 2 for {\ida}, 3
for {\kinsol}, 4 for {\arkode}); \id{ier} is an error return flag 
equal to 0 for success and -1 for failure. Both \id{code} and \id{ier}
are declared to match C type \id{int}.
This routine must be called \emph{after} both the
{\nvector} and {\sunmatrix} objects have been initialized.
Additionally, when using {\arkode} with a non-identity mass matrix, the
Fortran-callable function \id{FSUNMassLapackBandInit(ier)}  
initializes this {\sunlinsollapband} module for solving mass matrix
linear systems.


%---------------------------------------------------------------------------
\section{The SUNLinearSolver\_KLU implementation}\label{ss:sunlinsol_klu}
%% This is a shared SUNDIALS TEX file with a description of the
%% klu sunlinsol implementation
%%

The {\klu} implementation of the {\sunlinsol} module provided with
{\sundials}, {\sunlinsolklu}, is designed to be used with the
corresponding {\sunmatsparse} matrix type, and one of the serial or
shared-memory {\nvector} implementations ({\nvecs}, {\nvecopenmp}, or 
{\nvecpthreads}).


%---------------------------------------------------------------------------
\subsection{{\sunlinsolklu} usage}\label{ss:sunlinsol_klu_usage}

The header file to include when using this module 
is \id{sunlinsol/sunlinsol\_klu.h}. The installed module
library to link to is
\id{libsundials\_sunlinsolklu.\textit{lib}}
where \id{\em.lib} is typically \id{.so} for shared libraries and
\id{.a} for static libraries. 

The module {\sunlinsolklu} provides the following user-callable routines: 
%%
% --------------------------------------------------------------------
\ucfunction{SUNLinSol\_KLU}
{
  LS = SUNLinSol\_KLU(y, A);
}
{
  The function \ID{SUNLinSol\_KLU} creates and allocates memory for a
  {\sunlinsolklu} object.
}
{
  \begin{args}[y]
  \item[y] (\id{N\_Vector})
    a template for cloning vectors needed within the solver
  \item[A] (\id{SUNMatrix})
    a {\sunmatsparse} matrix template for cloning matrices needed
    within the solver 
  \end{args}
}
{
  This returns a \id{SUNLinearSolver} object.  If either \id{A} or
  \id{y} are incompatible then this routine will return \id{NULL}.
}
{
  This routine will perform consistency checks to ensure that it is
  called with consistent {\nvector} and {\sunmatrix} implementations.
  These are currently limited to the {\sunmatsparse} matrix type
  (using either CSR or CSC storage formats) and the {\nvecs},
  {\nvecopenmp}, and {\nvecpthreads} vector types.  As additional
  compatible matrix and vector implementations are added to
  {\sundials}, these will be included within this compatibility
  check. 
}
% --------------------------------------------------------------------
\ucfunction{SUNLinSol\_KLUReInit}
{
  retval = SUNLinSol\_KLUReInit(LS, A, nnz, reinit\_type);
}
{
  The function \ID{SUNLinSol\_KLUReInit} reinitializes memory and
  flags for a new factorization (symbolic and numeric) to be conducted
  at the next solver setup call.  This routine is useful in the cases
  where the number of nonzeroes has changed or if the structure of the
  linear system has changed which would require a new symbolic (and
  numeric factorization). 
}
{
  \begin{args}[reinit\_type]
  \item[LS] (\id{SUNLinearSolver})
    a template for cloning vectors needed within the solver
  \item[A] (\id{SUNMatrix})
    a {\sunmatsparse} matrix template for cloning matrices needed
    within the solver 
  \item[nnz] (\id{sunindextype})
    the new number of nonzeros in the matrix
  \item[reinit\_type] (\id{int})
    flag governing the level of reinitialization.  The allowed values
    are:
    \begin{itemize}
    \item \texttt{SUNKLU\_REINIT\_FULL} -- The Jacobian matrix will be
      destroyed and a new one will be allocated based on the \id{nnz}
      value passed to this call.  New symbolic and numeric
      factorizations will be completed at the next solver setup. 
    \item \texttt{SUNKLU\_REINIT\_PARTIAL} -- Only symbolic and numeric 
      factorizations will be completed.  It is assumed that the
      Jacobian size has not exceeded the size of \id{nnz} given in the
      sparse matrix provided to the original constructor routine (or
      the previous \id{SUNLinSol\_KLUReInit} call). 
    \end{itemize}
  \end{args}
}
{
  The return values from this function are \id{SUNLS\_MEM\_NULL}
  (either \id{S} or \id{A} are \id{NULL}), \id{SUNLS\_ILL\_INPUT}
  (\id{A} does not have type \id{SUNMATRIX\_SPARSE} or
  \id{reinit\_type} is invalid), \id{SUNLS\_MEM\_FAIL} (reallocation
  of the sparse matrix failed) or \id{SUNLS\_SUCCESS}.
}
{
  This routine will perform consistency checks to ensure that it is
  called with consistent {\nvector} and {\sunmatrix} implementations.
  These are currently limited to the {\sunmatsparse} matrix type
  (using either CSR or CSC storage formats) and the {\nvecs},
  {\nvecopenmp}, and {\nvecpthreads} vector types.  As additional
  compatible matrix and vector implementations are added to
  {\sundials}, these will be included within this compatibility
  check.

  This routine assumes no other changes to solver use are necessary.
}
% --------------------------------------------------------------------
\ucfunction{SUNLinSol\_KLUSetOrdering}
{
  retval = SUNLinSol\_KLUSetOrdering(LS, ordering);
}
{
  This function sets the ordering used by {\klu} for reducing fill in
  the linear solve.
}
{
  \begin{args}[ordering]
  \item[LS] (\id{SUNLinearSolver})
    the {\sunlinsolklu} object
  \item[ordering] (\id{int})
    flag indication the reordering algorithm to use.  Options include:
    \begin{itemize}
    \item[0] AMD,
    \item[1] COLAMD, and
    \item[2] the natural ordering.
    \end{itemize}
    The default is 1 for COLAMD.
  \end{args}
}
{
  The return values from this function are \id{SUNLS\_MEM\_NULL}
  (\id{S} is \id{NULL}), \id{SUNLS\_ILL\_INPUT}
  (invalid \id{ordering}), or \id{SUNLS\_SUCCESS}.
}
{
}
% --------------------------------------------------------------------
%%
For backwards compatibility, we also provide the wrapper functions,
each with identical input and output arguments to the routines that
they wrap:
\begin{itemize}

\item \ID{SUNKLU}

  Wrapper function for \id{SUNLinSol\_KLU}

\item \ID{SUNKLUReInit}

  Wrapper function for \id{SUNLinSol\_KLUReInit}

\item \ID{SUNKLUSetOrdering}

  Wrapper function for \id{SUNLinSol\_KLUSetOrdering}

\end{itemize}
%%
%%------------------------------------
%%
For solvers that include a Fortran interface module, the
{\sunlinsolklu} module also includes a Fortran-callable function
for creating a \id{SUNLinearSolver} object.
% --------------------------------------------------------------------
\ucfunction{FSUNKLUINIT}
{
  FSUNKLUINIT(code, ier)
}
{
  The function \ID{FSUNKLUINIT} can be called for Fortran programs
  to create a {\sunlinsolklu} object.
}
{
  \begin{args}[code]
  \item[code] (\id{int*})
    is an integer input specifying the solver id (1 for {\cvode}, 2
    for {\ida}, 3 for {\kinsol}, and 4 for {\arkode}).
  \end{args}
}
{
  \id{ier} is a return completion flag equal to \id{0} for a success
  return and \id{-1} otherwise. See printed message for details in case
  of failure.
}
{
  This routine must be
  called \emph{after} both the {\nvector} and {\sunmatrix} objects have
  been initialized.
}
% --------------------------------------------------------------------
Additionally, when using
{\arkode} with a non-identity mass matrix, the {\sunlinsolklu} module
includes a Fortran-callable function for creating a
\id{SUNLinearSolver} mass matrix solver object.
% --------------------------------------------------------------------
\ucfunction{FSUNMASSKLUINIT}
{
  FSUNMASSKLUINIT(ier)
}
{
  The function \ID{FSUNMASSKLUINIT} can be called for Fortran programs
  to create a {\sunlinsolklu} object for mass matrix linear systems.
}
{
}
{
  \id{ier} is a \id{int} return completion flag equal to \id{0} for a success
  return and \id{-1} otherwise. See printed message for details in case
  of failure.
}
{
  This routine must be
  called \emph{after} both the {\nvector} and {\sunmatrix} mass-matrix
  objects have been initialized.
}
% --------------------------------------------------------------------
The \id{SUNLinSol\_KLUReInit} and \ID{SUNLinSol\_KLUSetOrdering}
routines also support Fortran interfaces for the system and mass
matrix solvers: 
% --------------------------------------------------------------------
\ucfunction{FSUNKLUREINIT}
{
  FSUNKLUREINIT(code, nnz, reinit\_type, ier)
}
{
  The function \ID{FSUNKLUREINIT} can be called for Fortran programs
  to re-initialize a {\sunlinsolklu} object.
}
{
  \begin{args}[reinit\_type]
  \item[code] (\id{int*})
    is an integer input specifying the solver id (1 for {\cvode}, 2
    for {\ida}, 3 for {\kinsol}, and 4 for {\arkode}).
  \item[nnz] (\id{sunindextype*})
    the new number of nonzeros in the matrix
  \item[reinit\_type] (\id{int*})
    flag governing the level of reinitialization.  The allowed values
    are:
    \begin{itemize}
    \item[1] -- The Jacobian matrix will be
      destroyed and a new one will be allocated based on the \id{nnz}
      value passed to this call.  New symbolic and numeric
      factorizations will be completed at the next solver setup. 
    \item[2] -- Only symbolic and numeric 
      factorizations will be completed.  It is assumed that the
      Jacobian size has not exceeded the size of \id{nnz} given in the
      sparse matrix provided to the original constructor routine (or
      the previous \id{SUNLinSol\_KLUReInit} call). 
    \end{itemize}
  \end{args}
}
{
  \id{ier} is a \id{int} return completion flag equal to \id{0} for a success
  return and \id{-1} otherwise. See printed message for details in case
  of failure.
}
{
  See \id{SUNLinSol\_KLUReInit} for complete further documentation of
  this routine. 
}
% --------------------------------------------------------------------
\ucfunction{FSUNMASSKLUREINIT}
{
  FSUNMASSKLUREINIT(nnz, reinit\_type, ier)
}
{
  The function \ID{FSUNMASSKLUREINIT} can be called for Fortran programs
  to re-initialize a {\sunlinsolklu} object for mass matrix linear systems.
}
{
  The arguments are identical to \id{FSUNKLUREINIT} above, except that
  \id{code} is not needed since mass matrix linear systems only arise
  in {\arkode}.
}
{
  \id{ier} is a \id{int} return completion flag equal to \id{0} for a success
  return and \id{-1} otherwise. See printed message for details in case
  of failure.
}
{
  See \id{SUNLinSol\_KLUReInit} for complete further documentation of
  this routine. 
}
% --------------------------------------------------------------------
\ucfunction{FSUNKLUSETORDERING}
{
  FSUNKLUSETORDERING(code, ordering, ier)
}
{
  The function \ID{FSUNKLUSETORDERING} can be called for Fortran programs
  to change the reordering algorithm used by {\klu}.
}
{
  \begin{args}[ordering]
  \item[code] (\id{int*})
    is an integer input specifying the solver id (1 for {\cvode}, 2
    for {\ida}, 3 for {\kinsol}, and 4 for {\arkode}).
  \item[ordering] (\id{int*})
    flag indication the reordering algorithm to use.  Options include:
    \begin{itemize}
    \item[0] AMD,
    \item[1] COLAMD, and
    \item[2] the natural ordering.
    \end{itemize}
    The default is 1 for COLAMD.
  \end{args}
}
{
  \id{ier} is a \id{int} return completion flag equal to \id{0} for a success
  return and \id{-1} otherwise. See printed message for details in case
  of failure.
}
{
  See \id{SUNLinSol\_KLUSetOrdering} for complete further documentation of
  this routine. 
}
% --------------------------------------------------------------------
\ucfunction{FSUNMASSKLUSETORDERING}
{
  FSUNMASSKLUSETORDERING(ier)
}
{
  The function \ID{FSUNMASSKLUSETORDERING} can be called for Fortran programs
  to change the reordering algorithm used by {\klu} for mass matrix linear systems.
}
{
  The arguments are identical to \id{FSUNKLUSETORDERING} above, except that
  \id{code} is not needed since mass matrix linear systems only arise
  in {\arkode}.
}
{
  \id{ier} is a \id{int} return completion flag equal to \id{0} for a success
  return and \id{-1} otherwise. See printed message for details in case
  of failure.
}
{
  See \id{SUNLinSol\_KLUSetOrdering} for complete further documentation of
  this routine. 
}


%---------------------------------------------------------------------------
\subsection{{\sunlinsolklu} description}\label{ss:sunlinsol_klu_description}


The {\sunlinsolklu} module defines the {\em
content} field of a \id{SUNLinearSolver} to be the following structure:
%%
\begin{verbatim} 
struct _SUNLinearSolverContent_KLU {
  long int         last_flag;
  int              first_factorize;
  sun_klu_symbolic *symbolic;
  sun_klu_numeric  *numeric;
  sun_klu_common   common;
  sunindextype     (*klu_solver)(sun_klu_symbolic*, sun_klu_numeric*,
                                 sunindextype, sunindextype,
                                 double*, sun_klu_common*);
};
\end{verbatim}
%%
These entries of the \emph{content} field contain the following
information:
\begin{description}
  \item[last\_flag] - last error return flag from internal function evaluations,
  \item[first\_factorize] - flag indicating whether the factorization
    has ever been performed, 
  \item[symbolic] - {\klu} storage structure for symbolic factorization components,
  \item[numeric] - {\klu} storage structure for numeric factorization components,
  \item[common] - storage structure for common {\klu} solver components,
  \item[klu\_solver] -- pointer to the appropriate {\klu} solver function
    (depending on whether it is using a CSR or CSC sparse matrix).
\end{description}

{\warn} The {\sunlinsolklu} module is a {\sunlinsol} wrapper for
the {\klu} sparse matrix factorization and solver library written by Tim
Davis \cite{KLU_site,DaPa:10}.  In order to use the
{\sunlinsolklu} interface to {\klu}, it is assumed that {\klu} has
been installed on the system prior to installation of {\sundials}, and
that {\sundials} has been configured appropriately to link with {\klu}
(see Appendix \ref{c:install} for details).  Additionally, this
wrapper only supports double-precision calculations, and therefore
cannot be compiled if {\sundials} is configured to have \id{realtype}
set to either \id{extended} or \id{single} (see Section \ref{s:types}).
Since the {\klu} library supports both 32-bit and 64-bit integers, this
interface will be compiled for either of the available \id{sunindextype} options.

The {\klu} library has a symbolic factorization routine that computes
the permutation of the linear system matrix to block triangular form
and the permutations that will pre-order the diagonal blocks (the only
ones that need to be factored) to reduce fill-in (using AMD, COLAMD,
CHOLAMD, natural, or an ordering given by the user).  Of these
ordering choices, the default value in the {\sunlinsolklu} 
module is the COLAMD ordering.

{\klu} breaks the factorization into two separate parts.  The first is
a symbolic factorization and the second is a numeric factorization
that returns the factored matrix along with final pivot information.   
{\klu} also has a refactor routine that can be called instead of the numeric 
factorization.  This routine will reuse the pivot information.  This routine 
also returns diagnostic information that a user can examine to determine if 
numerical stability is being lost and a full numerical factorization should 
be done instead of the refactor.

Since the linear systems that arise within the context of {\sundials}
calculations will typically have identical sparsity patterns, the
{\sunlinsolklu} module is constructed to perform the
following operations:
\begin{itemize}
\item The first time that the ``setup'' routine is called, it
  performs the symbolic factorization, followed by an initial
  numerical factorization.  
\item On subsequent calls to the ``setup'' routine, it calls the
  appropriate {\klu} ``refactor'' routine, followed by estimates of
  the numerical conditioning using the relevant ``rcond'', and if
  necessary ``condest'', routine(s).  If these estimates of the
  condition number are larger than $\varepsilon^{-2/3}$ (where
  $\varepsilon$ is the double-precision unit roundoff), then a new
  factorization is performed.
\item The module includes the routine \id{SUNKLUReInit}, that 
  can be called by the user to force a full or partial refactorization
  at the next ``setup'' call. 
\item The ``solve'' call performs pivoting and forward and
  backward substitution using the stored {\klu} data structures.  We
  note that in this solve {\klu} operates on the native data arrays
  for the right-hand side and solution vectors, without requiring
  costly data copies.
\end{itemize}


%%
%%----------------------------------------------
%%

\noindent The {\sunlinsolklu} module defines implementations of all
``direct'' linear solver operations listed in Sections
\ref{ss:sunlinsol_CoreFn}-\ref{ss:sunlinsol_GetFn}:
\begin{itemize}
\item \id{SUNLinSolGetType\_KLU}
\item \id{SUNLinSolInitialize\_KLU} -- this sets the
  \id{first\_factorize} flag to 1, forcing both symbolic and numerical
  factorizations on the subsequent ``setup'' call.
\item \id{SUNLinSolSetup\_KLU} -- this performs either a $LU$
  factorization or refactorization of the input matrix.
\item \id{SUNLinSolSolve\_KLU} -- this calls the appropriate {\klu}
  solve routine to utilize the $LU$ factors to solve the linear
  system. 
\item \id{SUNLinSolLastFlag\_KLU}
\item \id{SUNLinSolSpace\_KLU} -- this only returns information for
  the storage within the solver \emph{interface}, i.e.~storage for the
  integers \id{last\_flag} and \id{first\_factorize}.  For additional
  space requirements, see the {\klu} documentation.
\item \id{SUNLinSolFree\_KLU}
\end{itemize}


%---------------------------------------------------------------------------
\section{The SUNLinearSolver\_SuperLUMT implementation}\label{ss:sunlinsol_superlumt}
% ====================================================================
\section{The SUNLinearSolver\_SuperLUMT implementation}
\label{ss:sunlinsol_superlumt}
% ====================================================================

This section describes the {\sunlinsol} implementation for solving sparse linear
systems with SuperLU\_MT. The {\superlumt} module is designed to be used with the
corresponding {\sunmatsparse} matrix type, and one of the serial or
shared-memory {\nvector} implementations ({\nvecs}, {\nvecopenmp}, or
{\nvecpthreads}). While these are compatible, it is not recommended
to use a threaded vector module with {\sunlinsolslumt} unless it is
the {\nvecopenmp} module and the {\superlumt} library has also been
compiled with OpenMP.

The header file to include when using this module
is \id{sunlinsol/sunlinsol\_superlumt.h}. The installed module
library to link to is
\id{libsundials\_sunlinsolsuperlumt.\textit{lib}}
where \id{\em.lib} is typically \id{.so} for shared libraries and
\id{.a} for static libraries.

The {\sunlinsolslumt} module is a {\sunlinsol} wrapper for
the {\superlumt} sparse matrix factorization and solver library
written by X. Sherry Li \cite{SuperLUMT_site,Li:05,DGL:99}.  The
package performs matrix factorization using threads to enhance
efficiency in shared memory parallel environments.  It should be noted
that threads are only used in the factorization step.  In
order to use the {\sunlinsolslumt} interface to {\superlumt}, it is
assumed that {\superlumt} has been installed on the system prior to
installation of {\sundials}, and that {\sundials} has been configured
appropriately to link with {\superlumt} (see Appendix \ref{c:install}
for details).  Additionally, this wrapper only supports single- and
double-precision calculations, and therefore cannot be compiled if
{\sundials} is configured to have \id{realtype} set to \id{extended}
(see Section \ref{s:types}).  Moreover, since the {\superlumt} library
may be installed to support either 32-bit or 64-bit integers, it is
assumed that the {\superlumt} library is installed using the same
integer precision as the {\sundials} \id{sunindextype} option. {\warn}

% ====================================================================
\subsection{SUNLinearSolver\_SuperLUMT description}
\label{ss:sunlinsol_slumt_usage}
% ====================================================================

The {\superlumt} library has a symbolic factorization routine that
computes the permutation of the linear system matrix to reduce fill-in
on subsequent $LU$ factorizations (using COLAMD, minimal degree
ordering on $A^T*A$, minimal degree ordering on $A^T+A$, or natural
ordering).  Of these ordering choices, the default value in the
{\sunlinsolslumt} module is the COLAMD ordering.

Since the linear systems that arise within the context of {\sundials}
calculations will typically have identical sparsity patterns, the
{\sunlinsolslumt} module is constructed to perform the
following operations:
\begin{itemize}
\item The first time that the ``setup'' routine is called, it
  performs the symbolic factorization, followed by an initial
  numerical factorization.
\item On subsequent calls to the ``setup'' routine, it skips the
  symbolic factorization, and only refactors the input matrix.
\item The ``solve'' call performs pivoting and forward and
  backward substitution using the stored {\superlumt} data
  structures.  We note that in this solve {\superlumt} operates on the
  native data arrays for the right-hand side and solution vectors,
  without requiring costly data copies.
\end{itemize}


% ====================================================================
\subsection{SUNLinearSolver\_SuperLUMT functions}
\label{ss:sunlinsol_slumt_functions}
% ====================================================================

The module {\sunlinsolslumt} provides the following user-callable constructor
for creating a \newline \id{SUNLinearSolver} object.
%
% --------------------------------------------------------------------
%
\ucfunctiond{SUNLinSol\_SuperLUMT}
{
  LS = SUNLinSol\_SuperLUMT(y, A, num\_threads);
}
{
  The function \ID{SUNLinSol\_SuperLUMT} creates and allocates memory for
  a \newline SuperLU\_MT-based \id{SUNLinearSolver} object.
}
{
  \begin{args}[num\_threads]
  \item[y] (\id{N\_Vector})
    a template for cloning vectors needed within the solver
  \item[A] (\id{SUNMatrix})
    a {\sunmatsparse} matrix template for cloning matrices needed
    within the solver
  \item[num\_threads] (\id{int})
    desired number of threads (OpenMP or Pthreads, depending on how
    {\superlumt} was installed) to use during the factorization steps
  \end{args}
}
{
  This returns a \id{SUNLinearSolver} object.  If either \id{A} or
  \id{y} are incompatible then this routine will return \id{NULL}.
}
{
  This routine analyzes the input matrix and vector to determine the
  linear system size and to assess compatibility with the {\superlumt}
  library.

  This routine will perform consistency checks to ensure that it is
  called with consistent {\nvector} and {\sunmatrix} implementations.
  These are currently limited to the {\sunmatsparse} matrix type
  (using either CSR or CSC storage formats) and the {\nvecs},
  {\nvecopenmp}, and {\nvecpthreads} vector types.  As additional
  compatible matrix and vector implementations are added to
  {\sundials}, these will be included within this compatibility
  check.

  The \id{num\_threads} argument is not checked and is passed directly
  to {\superlumt} routines.
}
{SUNSuperLUMT}
%
% --------------------------------------------------------------------
%
\noindent The {\sunlinsolslumt} module defines implementations of all
``direct'' linear solver operations listed in Sections
\ref{ss:sunlinsol_CoreFn} -- \ref{ss:sunlinsol_GetFn}:
\begin{itemize}
\item \id{SUNLinSolGetType\_SuperLUMT}
\item \id{SUNLinSolInitialize\_SuperLUMT} -- this sets the
  \id{first\_factorize} flag to 1 and resets the internal {\superlumt}
  statistics variables.
\item \id{SUNLinSolSetup\_SuperLUMT} -- this performs either a $LU$
  factorization or refactorization of the input matrix.
\item \id{SUNLinSolSolve\_SuperLUMT} -- this calls the appropriate
  {\superlumt} solve routine to utilize the $LU$ factors to solve the
  linear system.
\item \id{SUNLinSolLastFlag\_SuperLUMT}
\item \id{SUNLinSolSpace\_SuperLUMT} -- this only returns information for
  the storage within the solver \emph{interface}, i.e.~storage for the
  integers \id{last\_flag} and \id{first\_factorize}.  For additional
  space requirements, see the {\superlumt} documentation.
\item \id{SUNLinSolFree\_SuperLUMT}
\end{itemize}

The {\sunlinsolslumt} module also defines the following additional
user-callable function.
%
% --------------------------------------------------------------------
%
\ucfunctiond{SUNLinSol\_SuperLUMTSetOrdering}
{
  retval = SUNLinSol\_SuperLUMTSetOrdering(LS, ordering);
}
{
  This function sets the ordering used by {\superlumt} for reducing fill in
  the linear solve.
}
{
  \begin{args}[ordering]
  \item[LS] (\id{SUNLinearSolver})
    the {\sunlinsolslumt} object
  \item[ordering] (\id{int})
    a flag indicating the ordering algorithm to use, the options are:
    \begin{itemize}
    \item[0] natural ordering
    \item[1] minimal degree ordering on $A^TA$
    \item[2] minimal degree ordering on $A^T+A$
    \item[3] COLAMD ordering for unsymmetric matrices
    \end{itemize}
    The default is 3 for COLAMD.
  \end{args}
}
{
  The return values from this function are \id{SUNLS\_MEM\_NULL}
  (\id{S} is \id{NULL}), \newline \id{SUNLS\_ILL\_INPUT}
  (invalid ordering choice), or \id{SUNLS\_SUCCESS}.
}
{}
{SUNSuperLUMTSetOrdering}

% ====================================================================
\subsection{SUNLinearSolver\_SuperLUMT Fortran interfaces}
\label{ss:sunlinsol_slumt_fortran}
% ====================================================================

For solvers that include a Fortran interface module, the
{\sunlinsolslumt} module also includes a Fortran-callable function
for creating a \id{SUNLinearSolver} object.
%
% --------------------------------------------------------------------
%
\ucfunction{FSUNSUPERLUMTINIT}
{
  FSUNSUPERLUMTINIT(code, num\_threads, ier)
}
{
  The function \ID{FSUNSUPERLUMTINIT} can be called for Fortran programs
  to create a {\sunlinsolklu} object.
}
{
  \begin{args}[num\_threads]
  \item[code] (\id{int*})
    is an integer input specifying the solver id (1 for {\cvode}, 2
    for {\ida}, 3 for {\kinsol}, and 4 for {\arkode}).
  \item[num\_threads] (\id{int*})
    desired number of threads (OpenMP or Pthreads, depending on how
    {\superlumt} was installed) to use during the factorization steps
  \end{args}
}
{
  \id{ier} is a return completion flag equal to \id{0} for a success
  return and \id{-1} otherwise. See printed message for details in case
  of failure.
}
{
  This routine must be
  called \emph{after} both the {\nvector} and {\sunmatrix} objects have
  been initialized.
}
Additionally, when using {\arkode} with a non-identity
mass matrix, the {\sunlinsolslumt} module includes a Fortran-callable
function for creating a \id{SUNLinearSolver} mass matrix solver
object.
%
% --------------------------------------------------------------------
%
\ucfunction{FSUNMASSSUPERLUMTINIT}
{
  FSUNMASSSUPERLUMTINIT(num\_threads, ier)
}
{
  The function \ID{FSUNMASSSUPERLUMTINIT} can be called for Fortran programs
  to create a SuperLU\_MT-based \id{SUNLinearSolver} object for mass matrix linear
  systems.
}
{
  \begin{args}[num\_threads]
  \item[num\_threads] (\id{int*})
    desired number of threads (OpenMP or Pthreads, depending on how
    {\superlumt} was installed) to use during the factorization steps.
  \end{args}
}
{
  \id{ier} is a \id{int} return completion flag equal to \id{0} for a success
  return and \id{-1} otherwise. See printed message for details in case
  of failure.
}
{
  This routine must be
  called \emph{after} both the {\nvector} and {\sunmatrix} mass-matrix
  objects have been initialized.
}
The \ID{SUNLinSol\_SuperLUMTSetOrdering} routine also supports Fortran
interfaces for the system and mass matrix solvers:
%
% --------------------------------------------------------------------
%
\ucfunction{FSUNSUPERLUMTSETORDERING}
{
  FSUNSUPERLUMTSETORDERING(code, ordering, ier)
}
{
  The function \ID{FSUNSUPERLUMTSETORDERING} can be called for Fortran programs
  to update the ordering algorithm in a {\sunlinsolslumt} object.
}
{
  \begin{args}[ordering]
  \item[code] (\id{int*})
    is an integer input specifying the solver id (1 for {\cvode}, 2
    for {\ida}, 3 for {\kinsol}, and 4 for {\arkode}).
  \item[ordering] (\id{int*})
    a flag indicating the ordering algorithm, options are:
    \begin{itemize}
    \item[0] natural ordering
    \item[1] minimal degree ordering on $A^TA$
    \item[2] minimal degree ordering on $A^T+A$
    \item[3] COLAMD ordering for unsymmetric matrices
    \end{itemize}
    The default is 3 for COLAMD.
  \end{args}
}
{
  \id{ier} is a \id{int} return completion flag equal to \id{0} for a success
  return and \id{-1} otherwise. See printed message for details in case
  of failure.
}
{
  See \id{SUNLinSol\_SuperLUMTSetOrdering} for complete further
  documentation of this routine.
}
%
% --------------------------------------------------------------------
%
\ucfunction{FSUNMASSUPERLUMTSETORDERING}
{
  FSUNMASSUPERLUMTSETORDERING(ordering, ier)
}
{
  The function \ID{FSUNMASSUPERLUMTSETORDERING} can be called for Fortran
  programs to update the ordering algorithm in a {\sunlinsolslumt}
  object for mass matrix linear systems.
}
{
  \begin{args}[ordering]
  \item[ordering] (\id{int*})
    a flag indicating the ordering algorithm, options are:
    \begin{itemize}
    \item[0] natural ordering
    \item[1] minimal degree ordering on $A^TA$
    \item[2] minimal degree ordering on $A^T+A$
    \item[3] COLAMD ordering for unsymmetric matrices
    \end{itemize}
    The default is 3 for COLAMD.
  \end{args}
}
{
  \id{ier} is a \id{int} return completion flag equal to \id{0} for a success
  return and \id{-1} otherwise. See printed message for details in case
  of failure.
}
{
  See \id{SUNLinSol\_SuperLUMTSetOrdering} for complete further
  documentation of this routine.
}


% ====================================================================
\subsection{SUNLinearSolver\_SuperLUMT content}
\label{ss:sunlinsol_slumt_content}
% ====================================================================

The {\sunlinsolslumt} module defines the \textit{content} field of a
\id{SUNLinearSolver} as the following structure:
%%
\begin{verbatim}
struct _SUNLinearSolverContent_SuperLUMT {
  int          last_flag;
  int          first_factorize;
  SuperMatrix  *A, *AC, *L, *U, *B;
  Gstat_t      *Gstat;
  sunindextype *perm_r, *perm_c;
  sunindextype N;
  int          num_threads;
  realtype     diag_pivot_thresh;
  int          ordering;
  superlumt_options_t *options;
};
\end{verbatim}
%%
These entries of the \emph{content} field contain the following
information:
\begin{args}[diag\_pivot\_thresh]
  \item[last\_flag] - last error return flag from internal function evaluations,
  \item[first\_factorize] - flag indicating whether the factorization
    has ever been performed,
  \item[A, AC, L, U, B] - \id{SuperMatrix} pointers used in solve,
  \item[Gstat] - \id{GStat\_t} object used in solve,
  \item[perm\_r, perm\_c] - permutation arrays used in solve,
  \item[N] - size of the linear system,
  \item[num\_threads] - number of OpenMP/Pthreads threads to use,
  \item[diag\_pivot\_thresh] - threshold on diagonal pivoting,
  \item[ordering] - flag for which reordering algorithm to use,
  \item[options] - pointer to {\superlumt} options structure.
\end{args}


%---------------------------------------------------------------------------
\section{The SUNLinearSolver\_SPGMR implementation}\label{ss:sunlinsol_spgmr}
%% This is a shared SUNDIALS TEX file with a description of the
%% spgmr sunlinsol implementation
%%

The {\spgmr} (Scaled, Preconditioned, Generalized Minimum
Residual \cite{SaSc:86}) implementation of the {\sunlinsol} module
provided with {\sundials}, {\sunlinsolspgmr}, is an iterative linear
solver that is designed to be compatible with any {\nvector}
implementation (serial, threaded, parallel, and user-supplied) that
supports a minimal subset of operations (\id{N\_VClone}, 
\id{N\_VDotProd}, \id{N\_VScale}, \id{N\_VLinearSum}, \id{N\_VProd},
\id{N\_VConst}, \id{N\_VDiv}, and \id{N\_VDestroy}).  

The {\sunlinsolspgmr} module defines the {\em content} field of a
\id{SUNLinearSolver} to be the following structure:
%%
\begin{verbatim} 
struct _SUNLinearSolverContent_SPGMR {
  int maxl;
  int pretype;
  int gstype;
  int max_restarts;
  int numiters;
  realtype resnorm;
  long int last_flag;
  ATimesFn ATimes;
  void* ATData;
  PSetupFn Psetup;
  PSolveFn Psolve;
  void* PData;
  N_Vector s1;
  N_Vector s2;
  N_Vector *V;
  realtype **Hes;
  realtype *givens;
  N_Vector xcor;
  realtype *yg;
  N_Vector vtemp;
};
\end{verbatim}
%%
These entries of the \emph{content} field contain the following
information:
\begin{description}
  \item[maxl] - number of GMRES basis vectors to use (default is 5),
  \item[pretype] - flag for type of preconditioning to employ
    (default is none),
  \item[gstype] - flag for type of Gram-Schmidt orthogonalization
    (default is modified Gram-Schmidt),
  \item[max\_restarts] - number of GMRES restarts to allow
    (default is 0),
  \item[numiters] - number of iterations from the most-recent solve,
  \item[resnorm] - final linear residual norm from the most-recent solve,
  \item[last\_flag] - last error return flag from an internal function,
  \item[ATimes] - function pointer to perform $Av$ product,
  \item[ATData] - pointer to structure for \id{ATimes},
  \item[Psetup] - function pointer to preconditioner setup routine,
  \item[Psolve] - function pointer to preconditioner solve routine,
  \item[PData] - pointer to structure for \id{Psetup} and \id{Psolve},
  \item[s1, s2] - vector pointers for supplied scaling matrices
    (default is \id{NULL}),
  \item[V] - the array of Krylov basis vectors
    $v_1, \ldots, v_{\text{\id{maxl}}+1}$, stored in \id{V[0]},
    \ldots, \id{V[maxl]}. Each $v_i$ is a vector of type {\nvector}.,
  \item[Hes] - the $(\text{\id{maxl}}+1)\times\text{\id{maxl}}$
    Hessenberg matrix. It is stored row-wise so that the (i,j)th
    element is given by \id{Hes[i][j]}.,
  \item[givens] - a length \id{2*maxl} array which represents the
    Givens rotation matrices that arise in the GMRES algorithm. These
    matrices are $F_0, F_1, \ldots, F_j$, where
    $F_i = \begin{bmatrix}
      1 &        &   &     &      &   &        &   \\
        & \ddots &   &     &      &   &        &   \\
        &        & 1 &     &      &   &        &   \\
        &        &   & c_i & -s_i &   &        &   \\
        &        &   & s_i &  c_i &   &        &   \\
        &        &   &     &      & 1 &        &   \\
        &        &   &     &      &   & \ddots &   \\
        &        &   &     &      &   &        & 1\end{bmatrix}$,
    are represented in the \id{givens} vector as \id{givens[0] =}
    $c_0$, \id{givens[1] = } $s_0$, \id{givens[2] = } $c_1$,
    \id{givens[3] = } $s_1$, \ldots \id{givens[2j] = } $c_j$,
    \id{givens[2j+1] = } $s_j$.,
  \item[xcor] - a vector which holds the scaled, preconditioned
    correction to the initial guess,
  \item[yg] - a length \id{(maxl+1)} array of \id{realtype} values
    used to hold ``short'' vectors (e.g. $y$ and $g$),
  \item[vtemp] - temporary vector storage.
\end{description}

This solver is constructed to perform the following operations:
\begin{itemize}
\item During construction, the \id{xcor} and \id{vtemp} arrays are
  cloned from a template {\nvector} that is input, and default solver
  parameters are set.
\item User-facing ``set'' routines may be called to modify default
  solver parameters.
\item Additional ``set'' routines are called by the {\sundials} solver
  that interfaces with {\sunlinsolspgmr} to supply the 
  \id{ATimes}, \id{PSetup}, and \id{Psolve} function pointers and
  \id{s1} and \id{s2} scaling vectors.
\item In the ``initialize'' call, the remaining solver data is
  allocated (\id{V}, \id{Hes}, \id{givens}, and \id{yg} )
\item In the ``setup'' call, any non-\id{NULL} 
  \id{PSetup} function is called.  Typically, this is provided by
  the {\sundials} solver itself, that translates between the
  generic \id{PSetup} function and the
  solver-specific routine (solver-supplied or user-supplied).
\item In the ``solve'' call, the GMRES iteration is performed.  This
  will include scaling, preconditioning, and restarts if those options
  have been supplied.
\end{itemize}

\noindent The header file to include when using this module 
is \id{sunlinsol/sunlinsol\_spgmr.h}. The {\sunlinsolspgmr} module
is accessible from all {\sundials} solvers \textit{without}
linking to the \\
\id{libsundials\_sunlinsolspgmr} module library. \\

%%
%%----------------------------------------------
%%

\noindent The {\sunlinsolspgmr} module defines implementations of all
``iterative'' linear solver operations listed in Table
\ref{t:sunlinsolops}:
\begin{itemize}
\item \id{SUNLinSolGetType\_SPGMR}
\item \id{SUNLinSolInitialize\_SPGMR}
\item \id{SUNLinSolSetATimes\_SPGMR}
\item \id{SUNLinSolSetPreconditioner\_SPGMR}
\item \id{SUNLinSolSetScalingVectors\_SPGMR}
\item \id{SUNLinSolSetup\_SPGMR}
\item \id{SUNLinSolSolve\_SPGMR}
\item \id{SUNLinSolNumIters\_SPGMR}
\item \id{SUNLinSolResNorm\_SPGMR}
\item \id{SUNLinSolResid\_SPGMR}
\item \id{SUNLinSolLastFlag\_SPGMR}
\item \id{SUNLinSolSpace\_SPGMR}
\item \id{SUNLinSolFree\_SPGMR}
\end{itemize}
The module {\sunlinsolspgmr} provides the following additional
user-callable routines: 
%%
\begin{itemize}

%%--------------------------------------

\item \ID{SUNSPGMR}

  This constructor function creates and allocates memory for a {\spgmr}
  \id{SUNLinearSolver}.  Its arguments are an {\nvector}, the desired
  type of preconditioning, and the number of Krylov basis vectors to use.

  This routine will perform consistency checks to ensure that it is
  called with a consistent {\nvector} implementation (i.e.~that it
  supplies the requisite vector operations).  If \id{y} is
  incompatible, then this routine will return \id{NULL}.

  A \id{maxl} argument that is $\le0$ will result in the default
  value (5).

  Allowable inputs for \id{pretype} are \id{PREC\_NONE} (0),
  \id{PREC\_LEFT} (1), \id{PREC\_RIGHT} (2) and \id{PREC\_BOTH} (3);
  any other integer input will result in the default (no
  preconditioning).
  We note that some {\sundials} solvers are designed to only work
  with left preconditioning ({\ida} and {\idas}) and others with only
  right preconditioning ({\kinsol}). While it is possible to configure
  a {\sunlinsolspgmr} object to use any of the preconditioning options
  with these solvers, this use mode is not supported and may result in
  inferior performance.

  \verb|SUNLinearSolver SUNSPGMR(N_Vector y, int pretype, int maxl);|

%%--------------------------------------

\item \ID{SUNSPGMRSetPrecType}

  This function updates the type of preconditioning to use.  Supported
  values are \id{PREC\_NONE} (0), \id{PREC\_LEFT} (1),
  \id{PREC\_RIGHT} (2) and \id{PREC\_BOTH} (3).  

  This routine will return with one of the error codes
  \id{SUNLS\_ILL\_INPUT} (illegal \id{pretype}), \id{SUNLS\_MEM\_NULL}
  (\id{S} is \id{NULL}) or \id{SUNLS\_SUCCESS}.
  
  \verb|int SUNSPGMRSetPrecType(SUNLinearSolver S, int pretype);|

%%--------------------------------------

\item \ID{SUNSPGMRSetGSType}

  This function sets the type of Gram-Schmidt orthogonalization to
  use.  Supported values are \id{MODIFIED\_GS} (1) and
  \id{CLASSICAL\_GS} (2).  Any other integer input will result in a
  failure, returning error code \id{SUNLS\_ILL\_INPUT}.

  This routine will return with one of the error codes
  \id{SUNLS\_ILL\_INPUT} (illegal \id{gstype}), \id{SUNLS\_MEM\_NULL}
  (\id{S} is \id{NULL}) or \id{SUNLS\_SUCCESS}.
  
  \verb|int SUNSPGMRSetGSType(SUNLinearSolver S, int gstype);|


%%--------------------------------------

\item \ID{SUNSPGMRSetMaxRestarts}

  This function sets the number of GMRES restarts to 
  allow.  A negative input will result in the default of 0.

  This routine will return with one of the error codes
  \id{SUNLS\_MEM\_NULL} (\id{S} is \id{NULL}) or \id{SUNLS\_SUCCESS}.
  
  \verb|int SUNSPGMRSetMaxRestarts(SUNLinearSolver S, int maxrs);|

\end{itemize}
%%
%%------------------------------------
%%
For solvers that include a Fortran interface module, the
{\sunlinsolspgmr} module also includes the Fortran-callable
function \id{FSUNSPGMRInit(code, pretype, maxl, ier)} to initialize
this {\sunlinsolspgmr} module for a given {\sundials} solver.
Here \id{code} is an integer input solver id (1 for {\cvode}, 2 for {\ida}, 3
for {\kinsol}, 4 for {\arkode}); \id{pretype} and \id{maxl} are the
same as for the C function \ID{SUNSPGMR}; \id{ier} is an error return
flag equal to 0 for success and -1 for failure.  All of these input
arguments should be declared so as to match C type \id{int}.  This
routine must be called \emph{after} the {\nvector} object has been
initialized.  Additionally, when using {\arkode} with a non-identity
mass matrix, the Fortran-callable
function \id{FSUNMassSPGMRInit(pretype, maxl, ier)} initializes this 
{\sunlinsolspgmr} module for solving mass matrix linear systems.

The \id{SUNSPGMRSetPrecType}, \id{SUNSPGMRSetGSType} and
\id{SUNSPGMRSetMaxRestarts} routines also support Fortran interfaces
for the system and mass matrix solvers (all arguments should be
commensurate with a C \id{int}):
\begin{itemize}
\item \id{FSUNSPGMRSetGSType(code, gstype, ier)}
\item \id{FSUNMassSPGMRSetGSType(gstype, ier)}
\item \id{FSUNSPGMRSetPrecType(code, pretype, ier)}
\item \id{FSUNMassSPGMRSetPrecType(pretype, ier)}
\item \id{FSUNSPGMRSetMaxRS(code, maxrs, ier)}
\item \id{FSUNMassSPGMRSetMaxRS(maxrs, ier)}
\end{itemize}


%---------------------------------------------------------------------------
\section{The SUNLinearSolver\_SPFGMR implementation}\label{ss:sunlinsol_spfgmr}
%% This is a shared SUNDIALS TEX file with a description of the
%% spfgmr sunlinsol implementation
%%

The {\spfgmr} (Scaled, Preconditioned, Flexible, Generalized Minimum
Residual \cite{Saa:93}) implementation of the {\sunlinsol} module
provided with {\sundials}, {\sunlinsolspfgmr}, is an iterative linear
solver that is designed to be compatible with any {\nvector}
implementation (serial, threaded, parallel, and user-supplied) that
supports a minimal subset of operations (\id{N\_VClone},
\id{N\_VDotProd}, \id{N\_VScale}, \id{N\_VLinearSum}, \id{N\_VProd},
\id{N\_VConst}, \id{N\_VDiv}, and \id{N\_VDestroy}).  When using
Classical Gram-Schmidt, the optional function \id{N\_VDotProdMulti}
may be supplied for increased efficiency.  Unlike the other
Krylov iterative linear solvers supplied with {\sundials}, FGMRES is
specifically designed to work with a changing preconditioner
(e.g.~from an iterative method).

%---------------------------------------------------------------------------
\subsection{{\sunlinsolspfgmr} usage}\label{ss:sunlinsol_spfgmr_usage}

The header file to include when using this module
is \id{sunlinsol/sunlinsol\_spfgmr.h}. The {\sunlinsolspfgmr} module
is accessible from all {\sundials} solvers \textit{without}
linking to the \\ \noindent
\id{libsundials\_sunlinsolspfgmr} module library.

The module {\sunlinsolspfgmr} provides the following user-callable routines:
% --------------------------------------------------------------------
\ucfunction{SUNLinSol\_SPFGMR}
{
  LS = SUNLinSol\_SPFGMR(y, pretype, maxl);
}
{
  The function \ID{SUNLinSol\_SPFGMR} creates and allocates memory for
  a {\spfgmr} \id{SUNLinearSolver}.
}
{
  \begin{args}[pretype]
  \item[y] (\id{N\_Vector})
    a template for cloning vectors needed within the solver
  \item[pretype] (\id{int})
    flag indicating the desired type of preconditioning, allowed
    values are:
    \begin{itemize}
    \item \id{PREC\_NONE} (0)
    \item \id{PREC\_LEFT} (1)
    \item \id{PREC\_RIGHT} (2)
    \item \id{PREC\_BOTH} (3)
    \end{itemize}
    Any other integer input will result in the default (no
    preconditioning).
  \item[maxl] (\id{int})
    the number of Krylov basis vectors to use.  values $\le0$ will
    result in the default value (5).
  \end{args}
}
{
  This returns a \id{SUNLinearSolver} object.  If either \id{y} is
  incompatible then this routine will return \id{NULL}.
}
{
  This routine will perform consistency checks to ensure that it is
  called with a consistent {\nvector} implementation (i.e.~that it
  supplies the requisite vector operations).  If \id{y} is
  incompatible, then this routine will return \id{NULL}.

  We note that some {\sundials} solvers are designed to only work with
  left preconditioning ({\ida} and {\idas}). While it is possible to
  use a right-preconditioned {\sunlinsolspfgmr} object for these
  packages, this use mode is not supported and may result in inferior
  performance.
}
% --------------------------------------------------------------------
\ucfunction{SUNLinSol\_SPFGMRSetPrecType}
{
  retval = SUNLinSol\_SPFGMRSetPrecType(LS, pretype);
}
{
  The function \ID{SUNLinSol\_SPFGMRSetPrecType} updates the type of
  preconditioning to use in the {\sunlinsolspfgmr} object.
}
{
  \begin{args}[pretype]
  \item[LS] (\id{SUNLinearSolver})
    the {\sunlinsolspfgmr} object to update
  \item[pretype] (\id{int})
    flag indicating the desired type of preconditioning, allowed
    values match those discussed in \id{SUNLinSol\_SPFGMR}.
  \end{args}
}
{
  This routine will return with one of the error codes
  \id{SUNLS\_ILL\_INPUT} (illegal \id{pretype}), \id{SUNLS\_MEM\_NULL}
  (\id{S} is \id{NULL}) or \id{SUNLS\_SUCCESS}.
}
{
}
% --------------------------------------------------------------------
\ucfunction{SUNLinSol\_SPFGMRSetGSType}
{
  retval = SUNLinSol\_SPFGMRSetGSType(LS, gstype);
}
{
  The function \ID{SUNLinSol\_SPFGMRSetPrecType} sets the type of
  Gram-Schmidt orthogonalization to use in the {\sunlinsolspfgmr}
  object.
}
{
  \begin{args}[gstype]
  \item[LS] (\id{SUNLinearSolver})
    the {\sunlinsolspfgmr} object to update
  \item[gstype] (\id{int})
    flag indicating the desired orthogonalization algorithm; allowed
    values are:
    \begin{itemize}
    \item \id{MODIFIED\_GS} (1)
    \item \id{CLASSICAL\_GS} (2)
    \end{itemize}
    Any other integer input will result in a
    failure, returning error code \id{SUNLS\_ILL\_INPUT}.
  \end{args}
}
{
  This routine will return with one of the error codes
  \id{SUNLS\_ILL\_INPUT} (illegal \id{pretype}), \id{SUNLS\_MEM\_NULL}
  (\id{S} is \id{NULL}) or \id{SUNLS\_SUCCESS}.
}
{
}
% --------------------------------------------------------------------
\ucfunction{SUNLinSol\_SPFGMRSetMaxRestarts}
{
  retval = SUNLinSol\_SPFGMRSetMaxRestarts(LS, maxrs);
}
{
  The function \ID{SUNLinSol\_SPFGMRSetMaxRestarts} sets the number of
  GMRES restarts to allow in the {\sunlinsolspfgmr} object.
}
{
  \begin{args}[maxrs]
  \item[LS] (\id{SUNLinearSolver})
    the {\sunlinsolspfgmr} object to update
  \item[maxrs] (\id{int})
    integer indicating number of restarts to allow.  A negative input
    will result in the default of 0.
  \end{args}
}
{
  This routine will return with one of the error codes
  \id{SUNLS\_MEM\_NULL} (\id{S} is \id{NULL}) or \id{SUNLS\_SUCCESS}.
}
{
}
% --------------------------------------------------------------------
%%
For backwards compatibility, we also provide the wrapper functions,
each with identical input and output arguments to the routines that
they wrap:
\begin{itemize}

\item \ID{SUNSPFGMR}

  Wrapper function for \ID{SUNLinSol\_SPFGMR}

\item \ID{SUNSPFGMRSetPrecType}

  Wrapper function for \ID{SUNLinSol\_SPFGMRSetPrecType}

\item \ID{SUNSPFGMRSetGSType}

  Wrapper function for \ID{SUNLinSol\_SPFGMRSetGSType}

\item \ID{SUNSPFGMRSetMaxRestarts}

  Wrapper function for \ID{SUNLinSol\_SPFGMRSetMaxRestarts}

\end{itemize}
%%
%%------------------------------------
%%
For solvers that include a Fortran interface module, the
{\sunlinsolspfgmr} module also includes a Fortran-callable function
for creating a \id{SUNLinearSolver} object.
% --------------------------------------------------------------------
\ucfunction{FSUNSPFGMRINIT}
{
  FSUNSPFGMRINIT(code, pretype, maxl, ier)
}
{
  The function \ID{FSUNSPFGMRINIT} can be called for Fortran programs
  to create a {\sunlinsolspfgmr} object.
}
{
  \begin{args}[pretype]
  \item[code] (\id{int*})
    is an integer input specifying the solver id (1 for {\cvode}, 2
    for {\ida}, 3 for {\kinsol}, and 4 for {\arkode}).
  \item[pretype] (\id{int*})
    flag indicating desired preconditioning type
  \item[maxl] (\id{int*})
    flag indicating Krylov subspace size
  \end{args}
}
{
  \id{ier} is a return completion flag equal to \id{0} for a success
  return and \id{-1} otherwise. See printed message for details in case
  of failure.
}
{
  This routine must be called \emph{after} the {\nvector} object has
  been initialized.

  Allowable values for \id{pretype} and \id{maxl} are the same as for
  the C function \ID{SUNLinSol\_SPFGMR}.
}
% --------------------------------------------------------------------
Additionally, when using
{\arkode} with a non-identity mass matrix, the {\sunlinsolspfgmr} module
includes a Fortran-callable function for creating a
\id{SUNLinearSolver} mass matrix solver object.
% --------------------------------------------------------------------
\ucfunction{FSUNMASSSPFGMRINIT}
{
  FSUNMASSSPFGMRINIT(pretype, maxl, ier)
}
{
  The function \ID{FSUNMASSSPFGMRINIT} can be called for Fortran programs
  to create a {\sunlinsolspfgmr} object for mass matrix linear systems.
}
{
  \begin{args}[pretype]
  \item[pretype] (\id{int*})
    flag indicating desired preconditioning type
  \item[maxl] (\id{int*})
    flag indicating Krylov subspace size
  \end{args}
}
{
  \id{ier} is a \id{int} return completion flag equal to \id{0} for a success
  return and \id{-1} otherwise. See printed message for details in case
  of failure.
}
{
  This routine must be called \emph{after} the {\nvector} object has
  been initialized.

  Allowable values for \id{pretype} and \id{maxl} are the same as for
  the C function \ID{SUNLinSol\_SPFGMR}.
}
% --------------------------------------------------------------------
The \id{SUNLinSol\_SPFGMRSetPrecType}, \id{SUNLinSol\_SPFGMRSetGSType}
and \id{SUNLinSol\_SPFGMRSetMaxRestarts} routines also support Fortran
interfaces for the system and mass matrix solvers

% --------------------------------------------------------------------
\ucfunction{FSUNSPFGMRSETGSTYPE}
{
  FSUNSPFGMRSETGSTYPE(code, gstype, ier)
}
{
  The function \ID{FSUNSPFGMRSETGSTYPE} can be called for Fortran
  programs to change the Gram-Schmidt orthogonaliation algorithm.
}
{
  \begin{args}[gstype]
  \item[code] (\id{int*})
    is an integer input specifying the solver id (1 for {\cvode}, 2
    for {\ida}, 3 for {\kinsol}, and 4 for {\arkode}).
  \item[gstype] (\id{int*})
    flag indicating the desired orthogonalization algorithm.
  \end{args}
}
{
  \id{ier} is a \id{int} return completion flag equal to \id{0} for a success
  return and \id{-1} otherwise. See printed message for details in case
  of failure.
}
{
  See \id{SUNLinSol\_SPFGMRSetGSType} for complete further documentation of
  this routine.
}
% --------------------------------------------------------------------
\ucfunction{FSUNMASSSPFGMRSETGSTYPE}
{
  FSUNMASSSPFGMRSETGSTYPE(gstype, ier)
}
{
  The function \ID{FSUNMASSSPFGMRSETGSTYPE} can be called for Fortran
  programs to change the Gram-Schmidt orthogonaliation algorithm for
  mass matrix linear systems.
}
{
  The arguments are identical to \id{FSUNSPFGMRSETGSTYPE} above, except that
  \id{code} is not needed since mass matrix linear systems only arise
  in {\arkode}.
}
{
  \id{ier} is a \id{int} return completion flag equal to \id{0} for a success
  return and \id{-1} otherwise. See printed message for details in case
  of failure.
}
{
  See \id{SUNLinSol\_SPFGMRSetGSType} for complete further documentation of
  this routine.
}
% --------------------------------------------------------------------
\ucfunction{FSUNSPFGMRSETPRECTYPE}
{
  FSUNSPFGMRSETPRECTYPE(code, pretype, ier)
}
{
  The function \ID{FSUNSPFGMRSETPRECTYPE} can be called for Fortran
  programs to change the type of preconditioning to use.
}
{
  \begin{args}[pretype]
  \item[code] (\id{int*})
    is an integer input specifying the solver id (1 for {\cvode}, 2
    for {\ida}, 3 for {\kinsol}, and 4 for {\arkode}).
  \item[pretype] (\id{int*})
    flag indication the type of preconditioning to use.
  \end{args}
}
{
  \id{ier} is a \id{int} return completion flag equal to \id{0} for a success
  return and \id{-1} otherwise. See printed message for details in case
  of failure.
}
{
  See \id{SUNLinSol\_SPFGMRSetPrecType} for complete further documentation of
  this routine.
}
% --------------------------------------------------------------------
\ucfunction{FSUNMASSSPFGMRSETPRECTYPE}
{
  FSUNMASSSPFGMRSETPRECTYPE(pretype, ier)
}
{
  The function \ID{FSUNMASSSPFGMRSETPRECTYPE} can be called for Fortran
  programs to change the type of preconditioning for mass matrix
  linear systems.
}
{
  The arguments are identical to \id{FSUNSPFGMRSETPRECTYPE} above, except that
  \id{code} is not needed since mass matrix linear systems only arise
  in {\arkode}.
}
{
  \id{ier} is a \id{int} return completion flag equal to \id{0} for a success
  return and \id{-1} otherwise. See printed message for details in case
  of failure.
}
{
  See \id{SUNLinSol\_SPFGMRSetPrecType} for complete further documentation of
  this routine.
}
% --------------------------------------------------------------------
\ucfunction{FSUNSPFGMRSETMAXRS}
{
  FSUNSPFGMRSETMAXRS(code, maxrs, ier)
}
{
  The function \ID{FSUNSPFGMRSETMAXRS} can be called for Fortran programs
  to change the maximum number of restarts allowed for {\spfgmr}.
}
{
  \begin{args}[maxrs]
  \item[code] (\id{int*})
    is an integer input specifying the solver id (1 for {\cvode}, 2
    for {\ida}, 3 for {\kinsol}, and 4 for {\arkode}).
  \item[maxrs] (\id{int*})
    maximum allowed number of restarts.
  \end{args}
}
{
  \id{ier} is a \id{int} return completion flag equal to \id{0} for a success
  return and \id{-1} otherwise. See printed message for details in case
  of failure.
}
{
  See \id{SUNLinSol\_SPFGMRSetMaxRestarts} for complete further
  documentation of this routine.
}
% --------------------------------------------------------------------
\ucfunction{FSUNMASSSPFGMRSETMAXRS}
{
  FSUNMASSSPFGMRSETMAXRS(maxrs, ier)
}
{
  The function \ID{FSUNMASSSPFGMRSETMAXRS} can be called for Fortran
  programs to change the maximum number of restarts allowed for
  {\spfgmr} for mass matrix linear systems.
}
{
  The arguments are identical to \id{FSUNSPFGMRSETMAXRS} above, except that
  \id{code} is not needed since mass matrix linear systems only arise
  in {\arkode}.
}
{
  \id{ier} is a \id{int} return completion flag equal to \id{0} for a success
  return and \id{-1} otherwise. See printed message for details in case
  of failure.
}
{
  See \id{SUNLinSol\_SPFGMRSetMaxRestarts} for complete further
  documentation of this routine.
}
% --------------------------------------------------------------------


%---------------------------------------------------------------------------
\subsection{{\sunlinsolspfgmr} description}\label{ss:sunlinsol_spfgmr_description}


The {\sunlinsolspfgmr} module defines the {\em content} field of a
\id{SUNLinearSolver} to be the following structure:
%%
\begin{verbatim}
struct _SUNLinearSolverContent_SPFGMR {
  int maxl;
  int pretype;
  int gstype;
  int max_restarts;
  int numiters;
  realtype resnorm;
  long int last_flag;
  ATimesFn ATimes;
  void* ATData;
  PSetupFn Psetup;
  PSolveFn Psolve;
  void* PData;
  N_Vector s1;
  N_Vector s2;
  N_Vector *V;
  N_Vector *Z;
  realtype **Hes;
  realtype *givens;
  N_Vector xcor;
  realtype *yg;
  N_Vector vtemp;
};
\end{verbatim}
%%
These entries of the \emph{content} field contain the following
information:
\begin{description}
  \item[maxl] - number of FGMRES basis vectors to use (default is 5),
  \item[pretype] - flag for use of preconditioning (default is none),
  \item[gstype] - flag for type of Gram-Schmidt orthogonalization
    (default is modified Gram-Schmidt),
  \item[max\_restarts] - number of FGMRES restarts to allow
    (default is 0),
  \item[numiters] - number of iterations from the most-recent solve,
  \item[resnorm] - final linear residual norm from the most-recent solve,
  \item[last\_flag] - last error return flag from an internal function,
  \item[ATimes] - function pointer to perform $Av$ product,
  \item[ATData] - pointer to structure for \id{ATimes},
  \item[Psetup] - function pointer to preconditioner setup routine,
  \item[Psolve] - function pointer to preconditioner solve routine,
  \item[PData] - pointer to structure for \id{Psetup} and \id{Psolve},
  \item[s1, s2] - vector pointers for supplied scaling matrices
    (default is \id{NULL}),
  \item[V] - the array of Krylov basis vectors
    $v_1, \ldots, v_{\text{\id{maxl}}+1}$, stored in \id{V[0]},
    \ldots, \id{V[maxl]}. Each $v_i$ is a vector of type {\nvector}.,
  \item[Z] - the array of preconditioned Krylov basis vectors
    $z_1, \ldots, z_{\text{\id{maxl}}+1}$, stored in \id{Z[0]},
    \ldots, \id{Z[maxl]}. Each $z_i$ is a vector of type {\nvector}.,
  \item[Hes] - the $(\text{\id{maxl}}+1)\times\text{\id{maxl}}$
    Hessenberg matrix. It is stored row-wise so that the (i,j)th
    element is given by \id{Hes[i][j]}.,
  \item[givens] - a length \id{2*maxl} array which represents the
    Givens rotation matrices that arise in the FGMRES algorithm. These
    matrices are $F_0, F_1, \ldots, F_j$, where
    $F_i = \begin{bmatrix}
      1 &        &   &     &      &   &        &   \\
        & \ddots &   &     &      &   &        &   \\
        &        & 1 &     &      &   &        &   \\
        &        &   & c_i & -s_i &   &        &   \\
        &        &   & s_i &  c_i &   &        &   \\
        &        &   &     &      & 1 &        &   \\
        &        &   &     &      &   & \ddots &   \\
        &        &   &     &      &   &        & 1\end{bmatrix}$,
    are represented in the \id{givens} vector as \id{givens[0] =}
    $c_0$, \id{givens[1] = } $s_0$, \id{givens[2] = } $c_1$,
    \id{givens[3] = } $s_1$, \ldots \id{givens[2j] = } $c_j$,
    \id{givens[2j+1] = } $s_j$.,
  \item[xcor] - a vector which holds the scaled, preconditioned
    correction to the initial guess,
  \item[yg] - a length \id{(maxl+1)} array of \id{realtype} values
    used to hold ``short'' vectors (e.g. $y$ and $g$),
  \item[vtemp] - temporary vector storage.
\end{description}

This solver is constructed to perform the following operations:
\begin{itemize}
\item During construction, the \id{xcor} and \id{vtemp} arrays are
  cloned from a template {\nvector} that is input, and default solver
  parameters are set.
\item User-facing ``set'' routines may be called to modify default
  solver parameters.
\item Additional ``set'' routines are called by the {\sundials} solver
  that interfaces with {\sunlinsolspfgmr} to supply the
  \id{ATimes}, \id{PSetup}, and \id{Psolve} function pointers and
  \id{s1} and \id{s2} scaling vectors.
\item In the ``initialize'' call, the remaining solver data is
  allocated (\id{V}, \id{Hes}, \id{givens}, and \id{yg} )
\item In the ``setup'' call, any non-\id{NULL}
  \id{PSetup} function is called.  Typically, this is provided by
  the {\sundials} solver itself, that translates between the
  generic \id{PSetup} function and the
  solver-specific routine (solver-supplied or user-supplied).
\item In the ``solve'' call, the FGMRES iteration is performed.  This
  will include scaling, preconditioning, and restarts if those options
  have been supplied.
\end{itemize}

%%
%%----------------------------------------------
%%

\noindent The {\sunlinsolspfgmr} module defines implementations of all
``iterative'' linear solver operations listed in Sections
\ref{ss:sunlinsol_CoreFn}-\ref{ss:sunlinsol_GetFn}:
\begin{itemize}
\item \id{SUNLinSolGetType\_SPFGMR}
\item \id{SUNLinSolInitialize\_SPFGMR}
\item \id{SUNLinSolSetATimes\_SPFGMR}
\item \id{SUNLinSolSetPreconditioner\_SPFGMR}
\item \id{SUNLinSolSetScalingVectors\_SPFGMR}
\item \id{SUNLinSolSetup\_SPFGMR}
\item \id{SUNLinSolSolve\_SPFGMR}
\item \id{SUNLinSolNumIters\_SPFGMR}
\item \id{SUNLinSolResNorm\_SPFGMR}
\item \id{SUNLinSolResid\_SPFGMR}
\item \id{SUNLinSolLastFlag\_SPFGMR}
\item \id{SUNLinSolSpace\_SPFGMR}
\item \id{SUNLinSolFree\_SPFGMR}
\end{itemize}


%---------------------------------------------------------------------------
\section{The SUNLinearSolver\_SPBCGS implementation}\label{ss:sunlinsol_spbcgs}
%% This is a shared SUNDIALS TEX file with a description of the
%% spbcgs sunlinsol implementation
%%

The {\spbcg} (Scaled, Preconditioned, Bi-Conjugate Gradient,
Stabilized \cite{Van:92}) implementation of the {\sunlinsol} module 
provided with {\sundials}, {\sunlinsolspbcgs}, is an iterative linear
solver that is designed to be compatible with any {\nvector}
implementation (serial, threaded, parallel, and user-supplied) that
supports a minimal subset of operations (\id{N\_VClone}, 
\id{N\_VDotProd}, \id{N\_VScale}, \id{N\_VLinearSum}, \id{N\_VProd},
\id{N\_VDiv}, and \id{N\_VDestroy}).  Unlike the {\spgmr} and {\spfgmr}
algorithms, {\spbcg} requires a fixed amount of memory that does not
increase with the number of allowed iterations.

The {\sunlinsolspbcgs} module defines the {\em content} field of a
\id{SUNLinearSolver} to be the following structure:
%%
\begin{verbatim} 
struct _SUNLinearSolverContent_SPBCGS {
  int maxl;
  int pretype;
  int numiters;
  realtype resnorm;
  long int last_flag;
  ATimesFn ATimes;
  void* ATData;
  PSetupFn Psetup;
  PSolveFn Psolve;
  void* PData;
  N_Vector s1;
  N_Vector s2;
  N_Vector r;
  N_Vector r_star;
  N_Vector p;
  N_Vector q;
  N_Vector u;
  N_Vector Ap;
  N_Vector vtemp;
};
\end{verbatim}
%%
These entries of the \emph{content} field contain the following
information:
\begin{description}
  \item[maxl] - number of {\spbcg} iterations to allow (default is 5),
  \item[pretype] - flag for type of preconditioning to employ
    (default is none),
  \item[numiters] - number of iterations from the most-recent solve,
  \item[resnorm] - final linear residual norm from the most-recent solve,
  \item[last\_flag] - last error return flag from an internal function,
  \item[ATimes] - function pointer to perform $Av$ product,
  \item[ATData] - pointer to structure for \id{ATimes},
  \item[Psetup] - function pointer to preconditioner setup routine,
  \item[Psolve] - function pointer to preconditioner solve routine,
  \item[PData] - pointer to structure for \id{Psetup} and \id{Psolve},
  \item[s1, s2] - vector pointers for supplied scaling matrices
    (default is \id{NULL}),
  \item[r] - a {\nvector} which holds the current scaled,
    preconditioned linear system residual,
  \item[r\_star] - a {\nvector} which holds the initial scaled,
    preconditioned linear system residual,
  \item[p, q, u, Ap, vtemp] - {\nvector}s used for workspace by the
    {\spbcg} algorithm.
\end{description}

This solver is constructed to perform the following operations:
\begin{itemize}
\item During construction all {\nvector} solver data is allocated,
  with vectors cloned from a template {\nvector} that is input, and
  default solver parameters are set.
\item User-facing ``set'' routines may be called to modify default
  solver parameters.
\item Additional ``set'' routines are called by the {\sundials} solver
  that interfaces with {\sunlinsolspbcgs} to supply the 
  \id{ATimes}, \id{PSetup}, and \id{Psolve} function pointers and
  \id{s1} and \id{s2} scaling vectors.
\item In the ``initialize'' call, the solver parameters are checked
  for validity.
\item In the ``setup'' call, any non-\id{NULL} 
  \id{PSetup} function is called.  Typically, this is provided by
  the {\sundials} solver itself, that translates between the
  generic \id{PSetup} function and the
  solver-specific routine (solver-supplied or user-supplied).
\item In the ``solve'' call the {\spbcg} iteration is performed.  This
  will include scaling and preconditioning if those options have been
  supplied.
\end{itemize}

\noindent The header file to be included when using this module 
is \id{sunlinsol/sunlinsol\_spbcgs.h}. \\
%%
%%----------------------------------------------
%%
The {\sunlinsolspbcgs} module defines implementations of all
``iterative'' linear solver operations listed in Table
\ref{t:sunlinsolops}:
\begin{itemize}
\item \id{SUNLinSolGetType\_SPBCGS}
\item \id{SUNLinSolInitialize\_SPBCGS}
\item \id{SUNLinSolSetATimes\_SPBCGS}
\item \id{SUNLinSolSetPreconditioner\_SPBCGS}
\item \id{SUNLinSolSetScalingVectors\_SPBCGS}
\item \id{SUNLinSolSetup\_SPBCGS}
\item \id{SUNLinSolSolve\_SPBCGS}
\item \id{SUNLinSolNumIters\_SPBCGS}
\item \id{SUNLinSolResNorm\_SPBCGS}
\item \id{SUNLinSolResid\_SPBCGS}
\item \id{SUNLinSolLastFlag\_SPBCGS}
\item \id{SUNLinSolSpace\_SPBCGS}
\item \id{SUNLinSolFree\_SPBCGS}
\end{itemize}
The module {\sunlinsolspbcgs} provides the following additional
user-callable routines: 
%%
\begin{itemize}

%%--------------------------------------

\item \ID{SUNSPBCGS}

  This constructor function creates and allocates memory for a {\spbcg}
  \id{SUNLinearSolver}.  Its arguments are an {\nvector}, the desired
  type of preconditioning, and the number of linear iterations to allow.

  This routine will perform consistency checks to ensure that it is
  called with a consistent {\nvector} implementation (i.e.~that it
  supplies the requisite vector operations).  If \id{y} is
  incompatible, then this routine will return \id{NULL}.

  A \id{maxl} argument that is $\le0$ will result in the default
  value (5).

  Allowable inputs for \id{pretype} are \id{PREC\_NONE} (0),
  \id{PREC\_LEFT} (1), \id{PREC\_RIGHT} (2) and \id{PREC\_BOTH} (3);
  any other integer input will result in the default (no
  preconditioning).
  We note that some {\sundials} solvers are designed to only work
  with left preconditioning ({\ida} and {\idas}) and others with only
  right preconditioning ({\kinsol}). While it is possible to configure
  a {\sunlinsolspbcgs} object to use any of the preconditioning options
  with these solvers, this use mode is not supported and may result in
  inferior performance.

  \verb|SUNLinearSolver SUNSPBCGS(N_Vector y, int pretype, int maxl);|

%%--------------------------------------

\item \ID{SUNSPBCGSSetPrecType}

  This function updates the type of preconditioning to use.  Supported
  values are \id{PREC\_NONE} (0), \id{PREC\_LEFT} (1),
  \id{PREC\_RIGHT} (2), and \id{PREC\_BOTH} (3).  

  This routine will return with one of the error codes
  \id{SUNLS\_ILL\_INPUT} (illegal \id{pretype}), \id{SUNLS\_MEM\_NULL}
  (\id{S} is \id{NULL}), or \id{SUNLS\_SUCCESS}.
  
  \verb|int SUNSPBCGSSetPrecType(SUNLinearSolver S, int pretype);|

%%--------------------------------------

\item \ID{SUNSPBCGSSetMaxl}

  This function updates the number of linear solver iterations to allow.  

  A \id{maxl} argument that is $\le0$ will result in the default
  value (5).

  This routine will return with one of the error codes
  \id{SUNLS\_MEM\_NULL} (\id{S} is \id{NULL}) or \id{SUNLS\_SUCCESS}.
  
  \verb|int SUNSPBCGSSetMaxl(SUNLinearSolver S, int maxl);|

\end{itemize}
%%
%%------------------------------------
%%
For solvers that include a Fortran interface module, the
{\sunlinsolspbcgs} module also includes the Fortran-callable
function \id{FSUNSPBCGSInit(code, pretype, maxl, ier)} to initialize
this {\sunlinsolspbcgs} module for a given {\sundials} solver.
Here \id{code} is an integer input solver id (1 for {\cvode}, 2 for {\ida}, 3
for {\kinsol}, 4 for {\arkode}); \id{pretype} and \id{maxl} are the
same as for the C function \ID{SUNSPBCGS}; \id{ier} is an error return
flag equal to 0 for success and -1 for failure.  All of these input
arguments should be declared so as to match C type \id{int}.  This
routine must be called \emph{after} the {\nvector} object has been
initialized.  Additionally, when using {\arkode} with a non-identity
mass matrix, the Fortran-callable function 
\id{FSUNMassSPBCGSInit(pretype, maxl, ier)} initializes this
{\sunlinsolspbcgs} module for solving mass matrix linear systems.

The \id{SUNSPBCGSSetPrecType} and \id{SUNSPBCGSSetMaxl} routines also
support Fortran interfaces for the system and mass matrix solvers (all
arguments should be commensurate with a C \id{int}):
\begin{itemize}
\item \id{FSUNSPBCGSSetPrecType(code, pretype, ier)}
\item \id{FSUNMassSPBCGSSetPrecType(pretype, ier)}
\item \id{FSUNSPBCGSSetMaxl(code, maxl, ier)}
\item \id{FSUNMassSPBCGSSetMaxl(maxl, ier)}
\end{itemize}


%---------------------------------------------------------------------------
\section{The SUNLinearSolver\_SPTFQMR implementation}\label{ss:sunlinsol_sptfqmr}
%% This is a shared SUNDIALS TEX file with a description of the
%% sptfqmr sunlinsol implementation
%%
\section{The SUNLinearSolver\_SPTFQMR implementation}\label{ss:sunlinsol_sptfqmr}

The {\sptfqmr} (Scaled, Preconditioned, Transpose-Free Quasi-Minimum
Residual \cite{Fre:93}) implementation of the {\sunlinsol} module
provided with {\sundials}, {\sunlinsolsptfqmr}, is an iterative linear
solver that is designed to be compatible with any {\nvector}
implementation (serial, threaded, parallel, and user-supplied) that
supports a minimal subset of operations (\id{N\_VClone},
\id{N\_VDotProd}, \id{N\_VScale}, \id{N\_VLinearSum}, \id{N\_VProd},
\id{N\_VConst}, \id{N\_VDiv}, and \id{N\_VDestroy}).  Unlike the
{\spgmr} and {\spfgmr} algorithms, {\sptfqmr} requires a fixed amount of
memory that does not increase with the number of allowed iterations.

%---------------------------------------------------------------------------
\subsection{{\sunlinsolsptfqmr} usage}\label{ss:sunlinsol_sptfqmr_usage}

The header file to include when using this module
is \id{sunlinsol/sunlinsol\_sptfqmr.h}. The {\sunlinsolsptfqmr} module
is accessible from all {\sundials} solvers \textit{without}
linking to the \\ \noindent
\id{libsundials\_sunlinsolsptfqmr} module library.


The module {\sunlinsolsptfqmr} provides the following
user-callable routines:
% --------------------------------------------------------------------
\ucfunction{SUNLinSol\_SPTFQMR}
{
  LS = SUNLinSol\_SPTFQMR(y, pretype, maxl);
}
{
  The function \ID{SUNLinSol\_SPTFQMR} creates and allocates memory for
  a {\sptfqmr} \\ \noindent \id{SUNLinearSolver}.
}
{
  \begin{args}[pretype]
  \item[y] (\id{N\_Vector})
    a template for cloning vectors needed within the solver
  \item[pretype] (\id{int})
    flag indicating the desired type of preconditioning, allowed
    values are:
    \begin{itemize}
    \item \id{PREC\_NONE} (0)
    \item \id{PREC\_LEFT} (1)
    \item \id{PREC\_RIGHT} (2)
    \item \id{PREC\_BOTH} (3)
    \end{itemize}
    Any other integer input will result in the default (no
    preconditioning).
  \item[maxl] (\id{int})
    the number of linear iterations to allow; values $\le0$ will
    result in the default value (5).
  \end{args}
}
{
  This returns a \id{SUNLinearSolver} object.  If either \id{y} is
  incompatible then this routine will return \id{NULL}.
}
{
  This routine will perform consistency checks to ensure that it is
  called with a consistent {\nvector} implementation (i.e.~that it
  supplies the requisite vector operations).  If \id{y} is
  incompatible, then this routine will return \id{NULL}.

  We note that some {\sundials} solvers are designed to only work
  with left preconditioning ({\ida} and {\idas}) and others with only
  right preconditioning ({\kinsol}). While it is possible to configure
  a {\sunlinsolsptfqmr} object to use any of the preconditioning options
  with these solvers, this use mode is not supported and may result in
  inferior performance.
}
% --------------------------------------------------------------------
\ucfunction{SUNLinSol\_SPTFQMRSetPrecType}
{
  retval = SUNLinSol\_SPTFQMRSetPrecType(LS, pretype);
}
{
  The function \ID{SUNLinSol\_SPTFQMRSetPrecType} updates the type of
  preconditioning to use in the {\sunlinsolsptfqmr} object.
}
{
  \begin{args}[pretype]
  \item[LS] (\id{SUNLinearSolver})
    the {\sunlinsolsptfqmr} object to update
  \item[pretype] (\id{int})
    flag indicating the desired type of preconditioning, allowed
    values match those discussed in \id{SUNLinSol\_SPTFQMR}.
  \end{args}
}
{
  This routine will return with one of the error codes
  \id{SUNLS\_ILL\_INPUT} (illegal \id{pretype}), \id{SUNLS\_MEM\_NULL}
  (\id{S} is \id{NULL}) or \id{SUNLS\_SUCCESS}.
}
{
}
% --------------------------------------------------------------------
\ucfunction{SUNLinSol\_SPTFQMRSetMaxl}
{
  retval = SUNLinSol\_SPTFQMRSetMaxl(LS, maxl);
}
{
  The function \ID{SUNLinSol\_SPTFQMRSetMaxl} updates the number of
  linear solver iterations to allow.
}
{
  \begin{args}[maxl]
  \item[LS] (\id{SUNLinearSolver})
    the {\sunlinsolsptfqmr} object to update
  \item[maxl] (\id{int})
    flag indicating the number of iterations to allow; values $\le0$
    will result in the default value (5)
  \end{args}
}
{
  This routine will return with one of the error codes
  \id{SUNLS\_MEM\_NULL} (\id{S} is \id{NULL}) or \id{SUNLS\_SUCCESS}.
}
{
}
% --------------------------------------------------------------------
%%
For backwards compatibility, we also provide the wrapper functions,
each with identical input and output arguments to the routines that
they wrap:
\begin{itemize}

\item \ID{SUNSPTFQMR}

  Wrapper function for \ID{SUNLinSol\_SPTFQMR}

\item \ID{SUNSPTFQMRSetPrecType}

  Wrapper function for \ID{SUNLinSol\_SPTFQMRSetPrecType}

\item \ID{SUNSPTFQMRSetMaxl}

  Wrapper function for \ID{SUNLinSol\_SPTFQMRSetMaxl}

\end{itemize}
%%
%%------------------------------------
%%
For solvers that include a Fortran interface module, the
{\sunlinsolsptfqmr} module also includes a Fortran-callable function
for creating a \id{SUNLinearSolver} object.
% --------------------------------------------------------------------
\ucfunction{FSUNSPTFQMRINIT}
{
  FSUNSPTFQMRINIT(code, pretype, maxl, ier)
}
{
  The function \ID{FSUNSPTFQMRINIT} can be called for Fortran programs
  to create a {\sunlinsolsptfqmr} object.
}
{
  \begin{args}[pretype]
  \item[code] (\id{int*})
    is an integer input specifying the solver id (1 for {\cvode}, 2
    for {\ida}, 3 for {\kinsol}, and 4 for {\arkode}).
  \item[pretype] (\id{int*})
    flag indicating desired preconditioning type
  \item[maxl] (\id{int*})
    flag indicating number of iterations to allow
  \end{args}
}
{
  \id{ier} is a return completion flag equal to \id{0} for a success
  return and \id{-1} otherwise. See printed message for details in case
  of failure.
}
{
  This routine must be called \emph{after} the {\nvector} object has
  been initialized.

  Allowable values for \id{pretype} and \id{maxl} are the same as for
  the {\CC} function \\ \noindent \ID{SUNLinSol\_SPTFQMR}.
}
% --------------------------------------------------------------------
Additionally, when using
{\arkode} with a non-identity mass matrix, the {\sunlinsolsptfqmr} module
includes a Fortran-callable function for creating a
\id{SUNLinearSolver} mass matrix solver object.
% --------------------------------------------------------------------
\ucfunction{FSUNMASSSPTFQMRINIT}
{
  FSUNMASSSPTFQMRINIT(pretype, maxl, ier)
}
{
  The function \ID{FSUNMASSSPTFQMRINIT} can be called for Fortran programs
  to create a {\sunlinsolsptfqmr} object for mass matrix linear systems.
}
{
  \begin{args}[pretype]
  \item[pretype] (\id{int*})
    flag indicating desired preconditioning type
  \item[maxl] (\id{int*})
    flag indicating number of iterations to allow
  \end{args}
}
{
  \id{ier} is a \id{int} return completion flag equal to \id{0} for a success
  return and \id{-1} otherwise. See printed message for details in case
  of failure.
}
{
  This routine must be called \emph{after} the {\nvector} object has
  been initialized.

  Allowable values for \id{pretype} and \id{maxl} are the same as for
  the {\CC} function \\ \noindent \ID{SUNLinSol\_SPTFQMR}.
}
% --------------------------------------------------------------------
The \id{SUNLinSol\_SPTFQMRSetPrecType} and
\id{SUNLinSol\_SPTFQMRSetMaxl} routines also support Fortran
interfaces for the system and mass matrix solvers.


% --------------------------------------------------------------------
\ucfunction{FSUNSPTFQMRSETPRECTYPE}
{
  FSUNSPTFQMRSETPRECTYPE(code, pretype, ier)
}
{
  The function \ID{FSUNSPTFQMRSETPRECTYPE} can be called for Fortran
  programs to change the type of preconditioning to use.
}
{
  \begin{args}[pretype]
  \item[code] (\id{int*})
    is an integer input specifying the solver id (1 for {\cvode}, 2
    for {\ida}, 3 for {\kinsol}, and 4 for {\arkode}).
  \item[pretype] (\id{int*})
    flag indication the type of preconditioning to use.
  \end{args}
}
{
  \id{ier} is a \id{int} return completion flag equal to \id{0} for a success
  return and \id{-1} otherwise. See printed message for details in case
  of failure.
}
{
  See \id{SUNLinSol\_SPTFQMRSetPrecType} for complete further documentation of
  this routine.
}
% --------------------------------------------------------------------
\ucfunction{FSUNMASSSPTFQMRSETPRECTYPE}
{
  FSUNMASSSPTFQMRSETPRECTYPE(pretype, ier)
}
{
  The function \ID{FSUNMASSSPTFQMRSETPRECTYPE} can be called for Fortran
  programs to change the type of preconditioning for mass matrix
  linear systems.
}
{
  The arguments are identical to \id{FSUNSPTFQMRSETPRECTYPE} above,
  except that \id{code} is not needed since mass matrix linear systems
  only arise in {\arkode}.
}
{
  \id{ier} is a \id{int} return completion flag equal to \id{0} for a success
  return and \id{-1} otherwise. See printed message for details in case
  of failure.
}
{
  See \id{SUNLinSol\_SPTFQMRSetPrecType} for complete further documentation of
  this routine.
}
% --------------------------------------------------------------------
\ucfunction{FSUNSPTFQMRSETMAXL}
{
  FSUNSPTFQMRSETMAXL(code, maxl, ier)
}
{
  The function \ID{FSUNSPTFQMRSETMAXL} can be called for Fortran
  programs to change the maximum number of iterations to allow.
}
{
  \begin{args}[maxl]
  \item[code] (\id{int*})
    is an integer input specifying the solver id (1 for {\cvode}, 2
    for {\ida}, 3 for {\kinsol}, and 4 for {\arkode}).
  \item[maxl] (\id{int*})
    the number of iterations to allow
  \end{args}
}
{
  \id{ier} is a \id{int} return completion flag equal to \id{0} for a success
  return and \id{-1} otherwise. See printed message for details in case
  of failure.
}
{
  See \id{SUNLinSol\_SPTFQMRSetMaxl} for complete further documentation of
  this routine.
}
% --------------------------------------------------------------------
\ucfunction{FSUNMASSSPTFQMRSETMAXL}
{
  FSUNMASSSPTFQMRSETMAXL(maxl, ier)
}
{
  The function \ID{FSUNMASSSPTFQMRSETMAXL} can be called for Fortran
  programs to change the type of preconditioning for mass matrix
  linear systems.
}
{
  The arguments are identical to \id{FSUNSPTFQMRSETMAXL} above, except that
  \id{code} is not needed since mass matrix linear systems only arise
  in {\arkode}.
}
{
  \id{ier} is a \id{int} return completion flag equal to \id{0} for a success
  return and \id{-1} otherwise. See printed message for details in case
  of failure.
}
{
  See \id{SUNLinSol\_SPTFQMRSetMaxl} for complete further documentation of
  this routine.
}
% --------------------------------------------------------------------

%---------------------------------------------------------------------------
\subsection{{\sunlinsolsptfqmr} description}\label{ss:sunlinsol_sptfqmr_description}


The {\sunlinsolsptfqmr} module defines the {\em content} field of a
\id{SUNLinearSolver} to be the following structure:
%%
\begin{verbatim}
struct _SUNLinearSolverContent_SPTFQMR {
  int maxl;
  int pretype;
  int numiters;
  realtype resnorm;
  long int last_flag;
  ATimesFn ATimes;
  void* ATData;
  PSetupFn Psetup;
  PSolveFn Psolve;
  void* PData;
  N_Vector s1;
  N_Vector s2;
  N_Vector r_star;
  N_Vector q;
  N_Vector d;
  N_Vector v;
  N_Vector p;
  N_Vector *r;
  N_Vector u;
  N_Vector vtemp1;
  N_Vector vtemp2;
  N_Vector vtemp3;
};
\end{verbatim}
%%
These entries of the \emph{content} field contain the following
information:
\begin{description}
  \item[maxl] - number of TFQMR iterations to allow (default is 5),
  \item[pretype] - flag for type of preconditioning to employ
    (default is none),
  \item[numiters] - number of iterations from the most-recent solve,
  \item[resnorm] - final linear residual norm from the most-recent solve,
  \item[last\_flag] - last error return flag from an internal function,
  \item[ATimes] - function pointer to perform $Av$ product,
  \item[ATData] - pointer to structure for \id{ATimes},
  \item[Psetup] - function pointer to preconditioner setup routine,
  \item[Psolve] - function pointer to preconditioner solve routine,
  \item[PData] - pointer to structure for \id{Psetup} and \id{Psolve},
  \item[s1, s2] - vector pointers for supplied scaling matrices
    (default is \id{NULL}),
  \item[r\_star] - a {\nvector} which holds the initial scaled,
    preconditioned linear system residual,
  \item[q, d, v, p, u] - {\nvector}s used for workspace by the SPTFQMR
    algorithm,
  \item [r] - array of two {\nvector}s used for workspace within the
    SPTFQMR algorithm,
  \item[vtemp1, vtemp2, vtemp3] - temporary vector storage.
\end{description}

This solver is constructed to perform the following operations:
\begin{itemize}
\item During construction all {\nvector} solver data is allocated,
  with vectors cloned from a template {\nvector} that is input, and
  default solver parameters are set.
\item User-facing ``set'' routines may be called to modify default
  solver parameters.
\item Additional ``set'' routines are called by the {\sundials} solver
  that interfaces with \\ \noindent {\sunlinsolsptfqmr} to supply the
  \id{ATimes}, \id{PSetup}, and \id{Psolve} function pointers and
  \id{s1} and \id{s2} scaling vectors.
\item In the ``initialize'' call, the solver parameters are checked
  for validity.
\item In the ``setup'' call, any non-\id{NULL}
  \id{PSetup} function is called.  Typically, this is provided by
  the {\sundials} solver itself, that translates between the
  generic \id{PSetup} function and the
  solver-specific routine (solver-supplied or user-supplied).
\item In the ``solve'' call the TFQMR iteration is performed.  This
  will include scaling and preconditioning if those options have been
  supplied.
\end{itemize}

%%
%%----------------------------------------------
%%

\noindent The {\sunlinsolsptfqmr} module defines implementations of all
``iterative'' linear solver operations listed in Sections
\ref{ss:sunlinsol_CoreFn}-\ref{ss:sunlinsol_GetFn}:
\begin{itemize}
\item \id{SUNLinSolGetType\_SPTFQMR}
\item \id{SUNLinSolInitialize\_SPTFQMR}
\item \id{SUNLinSolSetATimes\_SPTFQMR}
\item \id{SUNLinSolSetPreconditioner\_SPTFQMR}
\item \id{SUNLinSolSetScalingVectors\_SPTFQMR}
\item \id{SUNLinSolSetup\_SPTFQMR}
\item \id{SUNLinSolSolve\_SPTFQMR}
\item \id{SUNLinSolNumIters\_SPTFQMR}
\item \id{SUNLinSolResNorm\_SPTFQMR}
\item \id{SUNLinSolResid\_SPTFQMR}
\item \id{SUNLinSolLastFlag\_SPTFQMR}
\item \id{SUNLinSolSpace\_SPTFQMR}
\item \id{SUNLinSolFree\_SPTFQMR}
\end{itemize}


%---------------------------------------------------------------------------
\section{The SUNLinearSolver\_PCG implementation}\label{ss:sunlinsol_pcg}
%% This is a shared SUNDIALS TEX file with a description of the
%% pcg sunlinsol implementation
%%

The {\pcg} (Preconditioned Conjugate Gradient \cite{HeSt:52})
implementation of the {\sunlinsol} module provided with {\sundials},
{\sunlinsolpcg}, is an iterative linear solver that is designed to be
compatible with any {\nvector} implementation (serial, threaded,
parallel, user-supplied) that supports a minimal subset of operations
(\id{N\_VClone}, \id{N\_VDotProd}, \id{N\_VScale}, \id{N\_VLinearSum},
\id{N\_VProd} and \id{N\_VDestroy}).  Unlike the {\spgmr} and {\spfgmr}
algorithms, {\pcg} requires a fixed amount of memory that does not
scale with the number of allowed iterations.

Unlike all of the other iterative linear solvers supplied with
{\sundials}, {\pcg} should only be used on \emph{symmetric} linear
systems (e.g.~mass matrix linear systems encountered in
{\arkode}). As a result, the explanation of the role of scaling and
preconditioning matrices given in general must be modified in this
scenario.  The {\pcg} algorithm solves a linear system $Ax = b$ where  
$A$ is a symmetric ($A^T=A$), real-valued matrix.  Preconditioning is
allowed, and is applied in a symmetric fashion on both the right and
left.  Scaling is also allowed and is applied symmetrically.  We
denote the preconditioner and scaling matrices as follows:
\begin{itemize}
\item $P$ is the preconditioner (assumed symmetric),
\item $S$ is a diagonal matrix of scale factors.
\end{itemize}
The matrices $A$ and $P$ are not required explicitly; only routines
that provide $A$ and $P^{-1}$ as operators are required.  The diagonal
of the matrix $S$ is held in a single {\nvector}, supplied by the user
of this module.

In this notation, {\pcg} applies the underlying CG algorithm to the
equivalent transformed system 
\begin{equation}
  \label{eq:transformed_linear_systemPCG}
  \tilde{A} \tilde{x} = \tilde{b}
\end{equation}
where
\begin{align}
  \notag
  \tilde{A} &= S P^{-1} A P^{-1} S,\\
  \label{eq:transformed_linear_system_componentsPCG}
  \tilde{b} &= S P^{-1} b,\\
  \notag
  \tilde{x} &= S^{-1} P x.
\end{align} 
The scaling matrix must be chosen so that the vectors $SP^{-1}b$ and
$S^{-1}Px$ have dimensionless components.

The stopping test for the PCG iterations is on the L2 norm of the
scaled preconditioned residual:
\begin{align*}
  &\| \tilde{b} - \tilde{A} \tilde{x} \|_2  <  \delta\\
  \Leftrightarrow\quad &\\
  &\| S P^{-1} b - S P^{-1} A x \|_2  <  \delta\\
  \Leftrightarrow\quad &\\
  &\| P^{-1} b - P^{-1} A x \|_S  <  \delta
\end{align*}
where $\| v \|_S = \sqrt{v^T S^T S v}$, with an input tolerance $\delta$.

The {\sunlinsolpcg} module defines the {\em content} field of a
\id{SUNLinearSolver} to be the following structure:
%%
\begin{verbatim} 
struct _SUNLinearSolverContent_PCG {
  int maxl;
  int pretype;
  int numiters;
  realtype resnorm;
  long int last_flag;
  ATimesFn ATimes;
  void* ATData;
  PSetupFn Psetup;
  PSolveFn Psolve;
  void* PData;
  N_Vector s;
  N_Vector r;
  N_Vector p;
  N_Vector z;
  N_Vector Ap;
};
\end{verbatim}
%%
These entries of the \emph{content} field contain the following
information:
\begin{description}
  \item[maxl] - number of {\pcg} iterations to allow (default is 5)
  \item[pretype] - flag for use of preconditioning (default is none)
  \item[numiters] - number of iterations from most-recent solve
  \item[resnorm] - final linear residual norm from most-recent solve
  \item[last\_flag] - last error return flag from internal function
  \item[ATimes] - function pointer to perform $Av$ product
  \item[ATData] - pointer to structure for \id{ATimes}
  \item[Psetup] - function pointer to preconditioner setup routine
  \item[Psolve] - function pointer to preconditioner solve routine
  \item[PData] - pointer to structure for \id{Psetup}, \id{Psolve}
  \item[s] - vector pointer for supplied scaling matrix
    (default is \id{NULL})
  \item[r] - a {\nvector} which holds the preconditioned linear system
    residual
  \item[p, z, Ap] - {\nvector}s used for workspace by the
    {\pcg} algorithm. 
\end{description}

This solver is constructed to perform the following operations:
\begin{itemize}
\item During construction all {\nvector} solver data is allocated,
  with vectors cloned from a template {\nvector} that is input, and
  default solver parameters are set.
\item User-facing ``set'' routines may be called to modify default
  solver parameters.
\item Additional ``set'' routines are called by the {\sundials} solver
  that interfaces with {\sunlinsolpcg} to supply the 
  \id{ATimes}, \id{PSetup} and \id{Psolve} function pointers and
  \id{s} scaling vector.
\item In the ``initialize'' call, the solver parameters are checked
  for validity.
\item In the ``setup'' call, any non-\id{NULL} \id{PSetup} function is
  called.  Typically, this is provided by the {\sundials} solver
  itself, that translates between the generic \id{PSetup} function and
  the solver-specific routine (solver-supplied or user-supplied).
\item In the ``solve'' call the {\pcg} iteration is performed.  This
  will include scaling and preconditioning if those options have been
  supplied.
\end{itemize}

\noindent The header file to be included when using this module 
is \id{sunlinsol/sunlinsol\_pcg.h}. \\
%%
%%----------------------------------------------
%%
The {\sunlinsolpcg} module defines implementations of all
``iterative'' linear solver operations listed in Table
\ref{t:sunlinsolops}:
\begin{itemize}
\item \id{SUNLinSolGetType\_PCG}
\item \id{SUNLinSolInitialize\_PCG}
\item \id{SUNLinSolSetATimes\_PCG}
\item \id{SUNLinSolSetPreconditioner\_PCG}
\item \id{SUNLinSolSetScalingVectors\_PCG} -- since {\pcg} only
  supports symmetric scaling, the second {\nvector} argument to this
  function is ignored
\item \id{SUNLinSolSetup\_PCG}
\item \id{SUNLinSolSolve\_PCG}
\item \id{SUNLinSolNumIters\_PCG}
\item \id{SUNLinSolResNorm\_PCG}
\item \id{SUNLinSolResid\_PCG}
\item \id{SUNLinSolLastFlag\_PCG}
\item \id{SUNLinSolSpace\_PCG}
\item \id{SUNLinSolFree\_PCG}
\end{itemize}
The module {\sunlinsolpcg} provides the following additional
user-callable routines: 
%%
\begin{itemize}

%%--------------------------------------

\item \ID{SUNPCG}

  This function creates and allocates memory for a {\pcg}
  \id{SUNLinearSolver}.  Its arguments are an {\nvector}, a flag
  indicating to use preconditioning, and the number of linear
  iterations to allow. 

  This routine will perform consistency checks to ensure that it is
  called with a consistent {\nvector} implementation (i.e.~that it
  supplies the requisite vector operations).  If \id{y} is
  incompatible then this routine will return \id{NULL}.

  A \id{maxl} argument that is $\le0$ will result in the default
  value (5).

  Since the {\pcg} algorithm is designed to only support symmetric
  preconditioning, then any of the \id{pretype} inputs \id{PREC\_LEFT}
  (1), \id{PREC\_RIGHT} (2), or \id{PREC\_BOTH} (3) will result in use
  of the symmetric preconditioner;  any other integer input will
  result in the default (no preconditioning).

  \verb|SUNLinearSolver SUNPCG(N_Vector y, int pretype, int maxl);|

%%--------------------------------------

\item \ID{SUNPCGSetPrecType}

  This function updates the flag indicating use of preconditioning.
  As above, any one of the input values, \id{PREC\_LEFT} (1),
  \id{PREC\_RIGHT} (2) and \id{PREC\_BOTH} (3) will enable
  preconditioning; \id{PREC\_NONE} (0) disables preconditioning.

  This routine will return with one of the error codes
  \id{SUNLS\_ILL\_INPUT} (illegal \id{pretype}), \id{SUNLS\_MEM\_NULL}
  (\id{S} is \id{NULL}) or \id{SUNLS\_SUCCESS}.
  
  \verb|int SUNPCGSetPrecType(SUNLinearSolver S, int pretype);|

%%--------------------------------------

\item \ID{SUNPCGSetMaxl}

  This function updates the number of linear solver iterations to
  allow. 

  A \id{maxl} argument that is $\le0$ will result in the default
  value (5).

  This routine will return with one of the error codes
  \id{SUNLS\_MEM\_NULL} (\id{S} is \id{NULL}) or \id{SUNLS\_SUCCESS}.
  
  \verb|int SUNPCGSetMaxl(SUNLinearSolver S, int maxl);|

\end{itemize}
%%
%%------------------------------------
%%
For solvers that include a Fortran interface module, the
{\sunlinsolpcg} module also includes the Fortran-callable
function \id{FSUNPCGInit(code, pretype, maxl, ier)} to initialize
this {\sunlinsolpcg} module for a given {\sundials} solver.
Here \id{code} is an input solver id (1 for {\cvode}, 2 for {\ida}, 3
for {\kinsol}, 4 for {\arkode}); \id{pretype} and \id{maxl} are the
same as for the C function \ID{SUNPCG}; \id{ier} is an error return
flag equal 0 for success and -1 for failure.  All of these input
arguments should be declared so as to match C type \id{int}).  This
routine must be called \emph{after} the {\nvector} object has been
initialized.  Additionally, when using {\arkode} with non-identity
mass matrix, the Fortran-callable function 
\id{FSUNMassPCGInit(pretype, maxl, ier)} initializes this
{\sunlinsolpcg} module for solving mass matrix linear systems.

The \id{SUNPCGSetPrecType} and \id{SUNPCGSetMaxl} routines also
support Fortran interfaces for the system and mass matrix solvers:
\begin{itemize}
\item \id{FSUNPCGSetPrecType(code, pretype, ier)} -- all arguments
  should be commensurate with a C \id{int}
\item \id{FSUNMassPCGSetPrecType(pretype, ier)}
\item \id{FSUNPCGSetMaxl(code, maxl, ier)} -- all arguments
  should be commensurate with a C \id{int}
\item \id{FSUNMassPCGSetMaxl(maxl, ier)}
\end{itemize}


%---------------------------------------------------------------------------

\section{SUNLinearSolver Examples}\label{ss:sunlinsol_examples}

There are \id{SUNLinearSolver} examples that may be installed for each
implementation; these make use of the functions in \id{test\_sunlinsol.c}. 
These example functions show simple usage of the \id{SUNLinearSolver} family
of functions.  The inputs to the examples depend on the linear solver type,
and are output to \texttt{stdout} if the example is run without the
appropriate number of command-line arguments. 

\noindent The following is a list of the example functions in \id{test\_sunlinsol.c}:
\begin{itemize}
\item \id{Test\_SUNLinSolGetType}: Verifies the returned solver type against
  the value that should be returned.
\item \id{Test\_SUNLinSolInitialize}: Verifies that \id{SUNLinSolInitialize} 
  can be called and returns successfully. 
\item \id{Test\_SUNLinSolSetup}: Verifies that \id{SUNLinSolSetup} can
  be called and returns successfully. 
\item \id{Test\_SUNLinSolSolve}: Given a {\sunmatrix} object $A$,
  {\nvector} objects $x$ and $b$ (where $Ax=b$) and a desired solution
  tolerance \texttt{tol}, this routine clones $x$ into a new vector $y$,
  calls \id{SUNLinSolSolve} to fill $y$ as the solution to $Ay=b$ (to
  the input tolerance), verifies that each entry in $x$ and $y$
  match to within \texttt{10*tol}, and overwrites $x$ with $y$ prior
  to returning (in case the calling routine would like to investigate
  further).
\item \id{Test\_SUNLinSolSetATimes} (iterative solvers only): Verifies that
  \id{SUNLinSolSetATimes} can be called and returns successfully.
\item \id{Test\_SUNLinSolSetPreconditioner} (iterative solvers only):
  Verifies that \id{SUNLinSolSetPreconditioner} can be called and
  returns successfully.
\item \id{Test\_SUNLinSolSetScalingVectors} (iterative solvers only):  
  Verifies that \id{SUNLinSolSetScalingVectors} can be called and
  returns successfully.
\item \id{Test\_SUNLinSolLastFlag}: Verifies that \id{SUNLinSolLastFlag} can
  be called, and outputs the result to \texttt{stdout}.
\item \id{Test\_SUNLinSolNumIters} (iterative solvers only): Verifies that
  \id{SUNLinSolNumIters} can be called, and outputs the result to 
  \texttt{stdout}. 
\item \id{Test\_SUNLinSolResNorm} (iterative solvers only): Verifies that
  \id{SUNLinSolResNorm} can be called, and that the result is
  non-negative. 
\item \id{Test\_SUNLinSolResid} (iterative solvers only): Verifies that
  \id{SUNLinSolResid} can be called.
\item \id{Test\_SUNLinSolSpace} verifies that \id{SUNLinSolSpace} can be
  called, and outputs the results to \texttt{stdout}.
\end{itemize}
We'll note that these tests should be performed in a particular
order.  For either direct or iterative linear
solvers, \id{Test\_SUNLinSolInitialize} must be called
before \id{Test\_SUNLinSolSetup}, which must be called
before \id{Test\_SUNLinSolSolve}.  Additionally, for iterative linear
solvers \id{Test\_SUNLinSolSetATimes}, \id{Test\_SUNLinSolSetPreconditioner}
and \id{Test\_SUNLinSolSetScalingVectors} should be called
before \id{Test\_SUNLinSolInitialize};
similarly \id{Test\_SUNLinSolNumIters}, \id{Test\_SUNLinSolResNorm}
and \id{Test\_SUNLinSolResid} should be called
after \id{Test\_SUNLinSolSolve}.  These are called in the appropriate
order in all of the example problems.




%---------------------------------------------------------------------------
\section{CVODES SUNLinearSolver interface}
\label{s:sunlinsol_interface}
%---------------------------------------------------------------------------

Table \ref{t:sunlinsoluse} below lists the {\sunlinsol} module linear solver
functions used within the {\cvls} interface. As with the {\sunmatrix} module, we
emphasize that the {\cvodes} user does not need to know detailed usage of linear
solver functions by the {\cvodes} code modules in order to use {\cvodes}. The
information is presented as an implementation detail for the interested reader.

The linear solver functions listed below are marked with \cm to
indicate that they are required, or with $\dagger$ to indicate that
they are only called if they are non-\id{NULL} in the {\sunlinsol}
implementation that is being used. Note:
\begin{enumerate}
\item \id{SUNLinSolNumIters} is only used to accumulate overall
  iterative linear solver statistics.  If it is not implemented by
  the {\sunlinsol} module, then {\cvls} will consider all solves as
  requiring zero iterations.
\item Although {\cvls} does not call \id{SUNLinSolLastFlag}
  directly, this routine is available for users to query linear solver
  issues directly.
\item Although {\cvls} does not call \id{SUNLinSolFree}
  directly, this routine should be available for users to call when
  cleaning up from a simulation.
\end{enumerate}

\begin{table}[htb]
\centering
\caption{List of linear solver function usage in the {\cvls} interface}\label{t:sunlinsoluse}
\medskip
\begin{tabular}{|r|c|c|c|} \hline
                                                    & 
\begin{sideways}{DIRECT}             \end{sideways} & 
\begin{sideways}{ITERATIVE}          \end{sideways} & 
\begin{sideways}{MATRIX\_ITERATIVE}  \end{sideways} \\ \hline\hline
%                                  DIRECT       ITER    & MAT-ITER  
\id{SUNLinSolGetType}           &    \cm    &    \cm    &   \cm     \\ \hline
\id{SUNLinSolSetATimes}         & $\dagger$ &    \cm    & $\dagger$ \\ \hline
\id{SUNLinSolSetPreconditioner} & $\dagger$ & $\dagger$ & $\dagger$ \\ \hline
\id{SUNLinSolSetScalingVectors} & $\dagger$ & $\dagger$ & $\dagger$ \\ \hline
\id{SUNLinSolInitialize}        &    \cm    &    \cm    &   \cm     \\ \hline
\id{SUNLinSolSetup}             &    \cm    &    \cm    &   \cm     \\ \hline
\id{SUNLinSolSolve}             &    \cm    &    \cm    &   \cm     \\ \hline
$^1$\id{SUNLinSolNumIters}      &           & $\dagger$ & $\dagger$ \\ \hline
$^2$\id{SUNLinSolLastFlag}      &           &           &           \\ \hline
$^3$\id{SUNLinSolFree}          &           &           &           \\ \hline
\id{SUNLinSolSpace}             & $\dagger$ & $\dagger$ & $\dagger$ \\ \hline
\end{tabular}
\end{table}

Since there are a wide range of potential {\sunlinsol} use cases, the following
subsections describe some details of the {\cvls} interface, in the case that
interested users wish to develop custom {\sunlinsol} modules.

%---------------------------------------------------------------------------
\subsection{Lagged matrix information}
\label{ss:sunlinsol_lagged_matrix}
%---------------------------------------------------------------------------

If the {\sunlinsol} object self-identifies as having type
\id{SUNLINEARSOLVER\_DIRECT} or \\ \noindent
\id{SUNLINEARSOLVER\_MATRIX\_ITERATIVE}, then the {\sunlinsol} object solves a
linear system \emph{defined} by a {\sunmatrix} object. {\cvls} will update the
matrix information infrequently according to the strategies outlined in
\S\ref{ss:ivp_sol}. When solving a linear system
\[
  M\bar{x} = b \quad\Leftrightarrow\quad (I-\bar{\gamma} J)\bar{x} = b
\]
it is likely that the value $\bar{\gamma}$ used to construct $M$
differs from the current value of $\gamma$ in the linear multistep method, since
$M$ is updated infrequently.  Therefore, after calling the
{\sunlinsol}-provided \id{SUNLinSolSolve} routine, we test whether
$\gamma / \bar{\gamma} \ne 1$, and if this is the case we scale
the solution $\bar{x}$ to obtain the desired linear system
solution $x$ via
\begin{equation}
  \label{eq:rescaling}
  x = \frac{2}{1 + \gamma / \bar{\gamma}} \bar{x}.
\end{equation}
For values of $\gamma/\bar{\gamma}$ that are ``close'' to 1, this
rescaling approximately solves the original linear system, as
discussed below.  We first note that the equation \eqref{eq:rescaling}
is equivalent to
\[
  \bar{x} = \frac12 \left(1 + \frac{\gamma}{\bar{\gamma}}\right)x.
\]
Adding the two equations $(I-\gamma J)x=b$ and
$(I-\bar{\gamma}J)\bar{x}=b$, and inserting the above relationship, we
have
\begin{align*}
  2b &= (I-\gamma J)x + (I-\bar{\gamma}J) \\
     &= x - \gamma Jx + \bar{x} - J\left(\bar{\gamma}\bar{x}\right)\\
     &= \frac32\left(I - \gamma J\right)x + \frac12\left(\frac{\gamma}{\bar{\gamma}}I - \bar{\gamma} J\right)x\\
     &= \frac32 b + \frac12\left(\frac{\gamma}{\bar{\gamma}}I - \bar{\gamma} J\right)x.
\end{align*}
When $\gamma/\bar{\gamma}\approx 1$, this latter term is approximately
equal to $\frac12 b$.

%---------------------------------------------------------------------------
\subsection{Iterative linear solver tolerance}
\label{ss:sunlinsol_iterative_tolerance}
%---------------------------------------------------------------------------

If the {\sunlinsol} object self-identifies as having type
\id{SUNLINEARSOLVER\_ITERATIVE} or \newline
\id{SUNLINEARSOLVER\_MATRIX\_ITERATIVE} then {\cvls} will set the input
tolerance \id{delta} as described in \S\ref{ss:ivp_sol}. However, if the
iterative linear solver does not support scaling matrices (i.e., the \newline
\id{SUNLinSolSetScalingVectors} routine is \id{NULL}), then {\cvls} will attempt 
to adjust the linear solver tolerance to account for this lack of functionality.
To this end, the following assumptions are made:
\begin{enumerate}
\item All solution components have similar magnitude; hence the error
  weight vector $W$ used in the WRMS norm (see \S\ref{ss:ivp_sol})
  should satisfy the assumption 
  \[
    W_i \approx W_{mean},\quad \text{for}\quad i=0,\ldots,n-1.
  \]
\item The {\sunlinsol} object uses a standard 2-norm to measure
  convergence.
\end{enumerate}

Since {\cvode} uses identical left and right scaling matrices,
$S_1 = S_2 = S = \operatorname{diag}(W)$, then the linear
solver convergence requirement is converted as follows
(using the notation from equations
\eqref{eq:transformed_linear_system}-\eqref{eq:transformed_linear_system_components}):
\begin{align*}
  &\left\| \tilde{b} - \tilde{A} \tilde{x} \right\|_2  <  \text{tol}\\
  \Leftrightarrow \quad & \left\| S P_1^{-1} b - S P_1^{-1} A x \right\|_2  <  \text{tol}\\
  \Leftrightarrow \quad & \sum_{i=0}^{n-1} \left[W_i \left(P_1^{-1} (b - A x)\right)_i\right]^2  <  \text{tol}^2\\
  \Leftrightarrow \quad & W_{mean}^2 \sum_{i=0}^{n-1} \left[\left(P_1^{-1} (b - A x)\right)_i\right]^2  <  \text{tol}^2\\
  \Leftrightarrow \quad & \sum_{i=0}^{n-1} \left[\left(P_1^{-1} (b - A x)\right)_i\right]^2  <  \left(\frac{\text{tol}}{W_{mean}}\right)^2\\
  \Leftrightarrow \quad & \left\| P_1^{-1} (b - A x)\right\|_2  <  \frac{\text{tol}}{W_{mean}}
\end{align*}
Therefore the tolerance scaling factor
\[
  W_{mean} = \|W\|_2 / \sqrt{n}
\]
is computed and the scaled tolerance \id{delta}$= \text{tol} / W_{mean}$ is
supplied to the {\sunlinsol} object.

%---------------------------------------------------------------------------
% sunlinsol module sections
%---------------------------------------------------------------------------

%% This is a shared SUNDIALS TEX file with a description of the
%% dense sunlinsol implementation
%%

The dense implementation of the {\sunlinsol} module provided with
{\sundials}, {\sunlinsoldense}, is designed to be used with the
corresponding {\sunmatdense} matrix type, and one of the serial or
shared-memory {\nvector} implementations ({\nvecs}, {\nvecopenmp} or
{\nvecpthreads}).  The {\sunlinsoldense} module defines the {\em
content} field of a \id{SUNLinearSolver} to be the following structure:
%%
\begin{verbatim} 
struct _SUNLinearSolverContent_Dense {
  sunindextype N;
  sunindextype *pivots;
  long int last_flag;
};
\end{verbatim}
%%
These entries of the \emph{content} field contain the following
information:
\begin{description}
  \item[N] - size of the linear system,
  \item[pivots] - index array for partial pivoting in LU factorization,
  \item[last\_flag] - last error return flag from internal function evaluations.
\end{description}

This solver is constructed to perform the following operations:
\begin{itemize}
\item In the ``setup'' call, this performs a $LU$ factorization with
  partial (row) pivoting ($\mathcal O(N^3)$ cost), $PA=LU$, where $P$
  is a permutation matrix, $L$ is a lower triangular matrix with 1's
  on the diagonal, and $U$ is an upper triangular matrix.  This
  factorization is stored in-place on the input {\sunmatdense} object
  $A$, with pivoting information encoding $P$ stored in
  the \id{pivots} array.
\item In the ``solve'' call, this performs pivoting, forward and
  backward substitution using the stored \id{pivots} array and the
  $LU$ factors held in the {\sunmatdense} object ($\mathcal O(N^2)$
  cost).
\end{itemize}

\noindent The header file to be included when using this module 
is \id{sunlinsol/sunlinsol\_dense.h}. \\
%%
%%----------------------------------------------
%%
The {\sunlinsoldense} module defines dense implementations of all
``direct'' linear solver operations listed in
Table \ref{t:sunlinsolops}:
\begin{itemize}
\item \id{SUNLinSolGetType\_Dense}
\item \id{SUNLinSolInitialize\_Dense} -- this does nothing, since all
  consistency checks were performed at solver creation.
\item \id{SUNLinSolSetup\_Dense} -- this performs the $LU$ factorization.
\item \id{SUNLinSolSolve\_Dense} -- this uses the $LU$ factors
  and \id{pivots} array to perform the solve.
\item \id{SUNLinSolLastFlag\_Dense}
\item \id{SUNLinSolSpace\_Dense} -- this only returns information for
  the storage \emph{within} the solver object, i.e.~storage
  for \id{N}, \id{last\_flag} and \id{pivots}.
\item \id{SUNLinSolFree\_Dense}
\end{itemize}
The module {\sunlinsoldense} provides the following additional
user-callable routine: 
%%
\begin{itemize}

%%--------------------------------------

\item \ID{SUNDenseLinearSolver}

  This function creates and allocates memory for a dense \id{SUNLinearSolver}.
  Its arguments are an {\nvector} and {\sunmatrix}, that it uses to
  determine the linear system size and to assess compatibility with
  the linear solver implementation.

  This routine will perform consistency checks to ensure that it is
  called with consistent {\nvector} and {\sunmatrix} implementations.
  These are currently limited to the {\sunmatdense} matrix type, and
  the {\nvecs}, {\nvecopenmp} and {\nvecpthreads} vector types.  As
  additional compatible matrix and vector implementations are added to
  {\sundials}, these will be included within this compatibility check.

  If either \id{A} or \id{y} are incompatible then this routine will
  return \id{NULL}.

  \verb|SUNLinearSolver SUNDenseLinearSolver(N_Vector y, SUNMatrix A);|

\end{itemize}
%%
%%------------------------------------
%%
For solvers that include a Fortran interface module, the {\sunlinsoldense}
module also includes the Fortran-callable
function \id{FSUNDenseLinSolInit(code, ier)} to initialize
this {\sunlinsoldense} module for a given {\sundials} solver.
Here \id{code} is an input solver id (1 for {\cvode}, 2 for {\ida}, 3
for {\kinsol}, 4 for {\arkode}); \id{ier} is an error return flag 
equal 0 for success and -1 for failure (declared so as to match C type
\id{int}).  This routine must be called \emph{after} both the
{\nvector} and {\sunmatrix} objects have been initialized.
Additionally, when using {\arkode} with non-identity mass matrix, the
Fortran-callable function \id{FSUNMassDenseLinSolInit(ier)}  
initializes this {\sunlinsoldense} module for solving mass matrix
linear systems.

%% This is a shared SUNDIALS TEX file with a description of the
%% band sunlinsol implementation
%%

The band implementation of the {\sunlinsol} module provided with
{\sundials}, {\sunlinsolband}, is designed to be used with the
corresponding {\sunmatband} matrix type, and one of the serial or
shared-memory {\nvector} implementations ({\nvecs}, {\nvecopenmp} or
{\nvecpthreads}).


%---------------------------------------------------------------------------
\subsection{{\sunlinsolband} usage}\label{ss:sunlinsol_band_usage}

The header file to include when using this module is
\id{sunlinsol/sunlinsol\_band.h}. The {\sunlinsolband} module 
is accessible from all {\sundials} solvers \textit{without}
linking to the \\ \noindent
\id{libsundials\_sunlinsolband} module library.

The module {\sunlinsolband} provides the following user-callable constructor routine: 
%%
% --------------------------------------------------------------------
\ucfunction{SUNLinSol\_Band}
{
  LS = SUNLinSol\_Band(y, A);
}
{
  The function \ID{SUNLinSol\_Band} creates and allocates memory for
  a band \id{SUNLinearSolver} object.
}
{
  \begin{args}[y]
  \item[y] (\id{N\_Vector})
    a template for cloning vectors needed within the solver
  \item[A] (\id{SUNMatrix})
    a {\sunmatband} matrix template for cloning matrices needed
    within the solver 
  \end{args}
}
{
  This returns a \id{SUNLinearSolver} object.  If either \id{A} or
  \id{y} are incompatible then this routine will return \id{NULL}.
}
{
  This routine will perform consistency checks to ensure that it is
  called with consistent {\nvector} and {\sunmatrix} implementations.
  These are currently limited to the {\sunmatdense} matrix type and
  the {\nvecs}, {\nvecopenmp}, and {\nvecpthreads} vector types.  As
  additional compatible matrix and vector implementations are added to
  {\sundials}, these will be included within this compatibility check.

  Additionally, this routine will verify that the input matrix \id{A}
  is allocated with appropriate upper bandwidth storage for the $LU$
  factorization.
}
% --------------------------------------------------------------------
%%
For backwards compatibility, we also provide the wrapper functions:
\begin{itemize}

\item \ID{SUNBandLinearSolver}

  Wrapper function for \ID{SUNLinSol\_Band}, with identical input and
  output arguments.

\end{itemize}
%%
%%------------------------------------
%%
For solvers that include a Fortran interface module, the {\sunlinsolband}
module also includes a Fortran-callable function for creating a
\id{SUNLinearSolver} object.
\ucfunction{FSUNBANDLINSOLINIT}
{
  FSUNBANDLINSOLINIT(code, ier)
}
{
  The function \ID{FSUNBANDLINSOLINIT} can be called for Fortran programs
  to create a band \id{SUNLinearSolver} object.
}
{
  \begin{args}[code]
  \item[code] (\id{int*})
    is an integer input specifying the solver id (1 for {\cvode}, 2
    for {\ida}, 3 for {\kinsol}, and 4 for {\arkode}).
  \end{args}
}
{
  \id{ier} is a return completion flag equal to \id{0} for a success
  return and \id{-1} otherwise. See printed message for details in case
  of failure.
}
{
  This routine must be
  called \emph{after} both the {\nvector} and {\sunmatrix} objects have
  been initialized.
}
Additionally, when using {\arkode} with a non-identity
mass matrix, the {\sunlinsolband} module includes a Fortran-callable
function for creating a \id{SUNLinearSolver} mass matrix solver
object.
\ucfunction{FSUNMASSBANDLINSOLINIT}
{
  FSUNMASSBANDLINSOLINIT(ier)
}
{
  The function \ID{FSUNMASSBANDLINSOLINIT} can be called for Fortran programs
  to create a band \id{SUNLinearSolver} object for mass matrix linear
  systems.
}
{
}
{
  \id{ier} is a \id{int} return completion flag equal to \id{0} for a success
  return and \id{-1} otherwise. See printed message for details in case
  of failure.
}
{
  This routine must be
  called \emph{after} both the {\nvector} and {\sunmatrix} mass-matrix
  objects have been initialized.
}

%---------------------------------------------------------------------------
\subsection{{\sunlinsolband} description}\label{ss:sunlinsol_band_description}



The {\sunlinsolband} module defines the {\em
content} field of a \id{SUNLinearSolver} to be the following structure:
%%
\begin{verbatim} 
struct _SUNLinearSolverContent_Band {
  sunindextype N;
  sunindextype *pivots;
  long int last_flag;
};
\end{verbatim}
%%
These entries of the \emph{content} field contain the following
information:
\begin{description}
  \item[N] - size of the linear system,
  \item[pivots] - index array for partial pivoting in LU factorization,
  \item[last\_flag] - last error return flag from internal function evaluations.
\end{description}

This solver is constructed to perform the following operations:
\begin{itemize}
\item The ``setup'' call performs a $LU$ factorization with
  partial (row) pivoting, $PA=LU$, where $P$ is a permutation matrix,
  $L$ is a lower triangular matrix with 1's on the diagonal, and $U$
  is an upper triangular matrix.  This factorization is stored
  in-place on the input {\sunmatband} object $A$, with pivoting
  information encoding $P$ stored in the \id{pivots} array.
\item The ``solve'' call performs pivoting and forward and
  backward substitution using the stored \id{pivots} array and the
  $LU$ factors held in the {\sunmatband} object.
\item
  {\warn} $A$ must be allocated to accommodate the increase in upper
  bandwidth that occurs during factorization.  More precisely, if $A$
  is a band matrix with upper bandwidth \id{mu} and lower bandwidth
  \id{ml}, then the upper triangular factor $U$ can have upper
  bandwidth as big as \id{smu = MIN(N-1,mu+ml)}. The lower triangular
  factor $L$ has lower bandwidth \id{ml}.
\end{itemize}


%%
%%----------------------------------------------
%%

\noindent The {\sunlinsolband} module defines band implementations of all
``direct'' linear solver operations listed in Sections
\ref{ss:sunlinsol_CoreFn}-\ref{ss:sunlinsol_GetFn}:
\begin{itemize}
\item \id{SUNLinSolGetType\_Band}
\item \id{SUNLinSolInitialize\_Band} -- this does nothing, since all
  consistency checks are performed at solver creation.
\item \id{SUNLinSolSetup\_Band} -- this performs the $LU$ factorization.
\item \id{SUNLinSolSolve\_Band} -- this uses the $LU$ factors
  and \id{pivots} array to perform the solve.
\item \id{SUNLinSolLastFlag\_Band}
\item \id{SUNLinSolSpace\_Band} -- this only returns information for
  the storage \emph{within} the solver object, i.e.~storage
  for \id{N}, \id{last\_flag}, and \id{pivots}.
\item \id{SUNLinSolFree\_Band}
\end{itemize}

%% This is a shared SUNDIALS TEX file with a description of the
%% lapackdense sunlinsol implementation
%%
\section{The SUNLinearSolver\_LapackDense implementation}\label{ss:sunlinsol_lapdense}

The LAPACK dense implementation of the {\sunlinsol} module provided
with {\sundials}, {\sunlinsollapdense}, is designed to be used with the 
corresponding {\sunmatdense} matrix type, and one of the serial or
shared-memory {\nvector} implementations ({\nvecs}, {\nvecopenmp}, or
{\nvecpthreads}).

%---------------------------------------------------------------------------
\subsection{{\sunlinsollapdense} usage}\label{ss:sunlinsol_lapdense_usage}

The header file to include when using this module 
is \id{sunlinsol/sunlinsol\_lapackdense.h}. The installed module
library to link to is
\id{libsundials\_sunlinsollapackdense.\textit{lib}}
where \id{\em.lib} is typically \id{.so} for shared libraries and
\id{.a} for static libraries.

The module {\sunlinsollapdense} provides the following user-callable constructor routine: 
%%
% --------------------------------------------------------------------
\ucfunction{SUNLinSol\_LapackDense}
{
  LS = SUNLinSol\_LapackDense(y, A);
}
{
  The function \ID{SUNLinSol\_LapackDense} creates and allocates
  memory for a LAPACK-based, dense \id{SUNLinearSolver} object.
}
{
  \begin{args}[y]
  \item[y] (\id{N\_Vector})
    a template for cloning vectors needed within the solver
  \item[A] (\id{SUNMatrix})
    a {\sunmatdense} matrix template for cloning matrices needed
    within the solver 
  \end{args}
}
{
  This returns a \id{SUNLinearSolver} object.  If either \id{A} or
  \id{y} are incompatible then this routine will return \id{NULL}.
}
{
  This routine will perform consistency checks to ensure that it is
  called with consistent {\nvector} and {\sunmatrix} implementations.
  These are currently limited to the {\sunmatdense} matrix type and
  the {\nvecs}, {\nvecopenmp}, and {\nvecpthreads} vector types.  As
  additional compatible matrix and vector implementations are added to
  {\sundials}, these will be included within this compatibility check.
}
% --------------------------------------------------------------------
%%
For backwards compatibility, we also provide the wrapper function,
\begin{itemize}

\item \ID{SUNLapackDense}

  Wrapper function for \ID{SUNLinSol\_LapackDense}, with identical input and
  output arguments.
  
\end{itemize}
%%
%%------------------------------------
%%
For solvers that include a Fortran interface module, the
{\sunlinsollapdense} module also includes a Fortran-callable function
for creating a \id{SUNLinearSolver} object.
\ucfunction{FSUNLAPACKDENSEINIT}
{
  FSUNLAPACKDENSEINIT(code, ier)
}
{
  The function \ID{FSUNLAPACKDENSEINIT} can be called for Fortran programs
  to create a LAPACK-based dense \id{SUNLinearSolver} object.
}
{
  \begin{args}[code]
  \item[code] (\id{int*})
    is an integer input specifying the solver id (1 for {\cvode}, 2
    for {\ida}, 3 for {\kinsol}, and 4 for {\arkode}).
  \end{args}
}
{
  \id{ier} is a return completion flag equal to \id{0} for a success
  return and \id{-1} otherwise. See printed message for details in case
  of failure.
}
{
  This routine must be
  called \emph{after} both the {\nvector} and {\sunmatrix} objects have
  been initialized.
}
Additionally, when using {\arkode} with a non-identity
mass matrix, the {\sunlinsollapdense} module includes a Fortran-callable
function for creating a \id{SUNLinearSolver} mass matrix solver
object.
\ucfunction{FSUNMASSLAPACKDENSEINIT}
{
  FSUNMASSLAPACKDENSEINIT(ier)
}
{
  The function \ID{FSUNMASSLAPACKDENSEINIT} can be called for Fortran programs
  to create a LAPACK-based, dense \id{SUNLinearSolver} object for mass
  matrix linear systems.
}
{
}
{
  \id{ier} is a \id{int} return completion flag equal to \id{0} for a success
  return and \id{-1} otherwise. See printed message for details in case
  of failure.
}
{
  This routine must be
  called \emph{after} both the {\nvector} and {\sunmatrix} mass-matrix
  objects have been initialized.
}


%---------------------------------------------------------------------------
\subsection{{\sunlinsollapdense} description}\label{ss:sunlinsol_lapdense_description}

The {\sunlinsollapdense} module defines the {\em
content} field of a \\
\noindent\id{SUNLinearSolver} to be the following structure:
%%
\begin{verbatim} 
struct _SUNLinearSolverContent_Dense {
  sunindextype N;
  sunindextype *pivots;
  long int last_flag;
};
\end{verbatim}
%%
These entries of the \emph{content} field contain the following
information:
\begin{description}
  \item[N] - size of the linear system,
  \item[pivots] - index array for partial pivoting in LU factorization,
  \item[last\_flag] - last error return flag from internal function evaluations.
\end{description}

{\warn} The {\sunlinsollapdense} module is a {\sunlinsol} wrapper for
the LAPACK dense matrix factorization and solve routines, \id{*GETRF}
and \id{*GETRS}, where \id{*} is either \id{D} or \id{S}, depending on
whether {\sundials} was configured to have \id{realtype} set to
\id{double} or \id{single}, respectively (see Section \ref{s:types}).
In order to use the {\sunlinsollapdense} module it is assumed
that LAPACK has been installed on the system prior to installation of
{\sundials}, and that {\sundials} has been configured appropriately to
link with LAPACK (see Appendix \ref{c:install} for details).  
We note that since there do not exist 128-bit floating-point
factorization and solve routines in LAPACK, this interface cannot be
compiled when using \id{extended} precision for \id{realtype}.
Similarly, since there do not exist 64-bit integer LAPACK routines,
the {\sunlinsollapdense} module also cannot be compiled when using
\id{int64\_t} for the \id{sunindextype}.

This solver is constructed to perform the following operations:
\begin{itemize}
\item The ``setup'' call performs a $LU$ factorization with
  partial (row) pivoting ($\mathcal O(N^3)$ cost), $PA=LU$, where $P$
  is a permutation matrix, $L$ is a lower triangular matrix with 1's
  on the diagonal, and $U$ is an upper triangular matrix.  This
  factorization is stored in-place on the input {\sunmatdense} object
  $A$, with pivoting information encoding $P$ stored in
  the \id{pivots} array.
\item The ``solve'' call performs pivoting and forward and
  backward substitution using the stored \id{pivots} array and the
  $LU$ factors held in the {\sunmatdense} object ($\mathcal O(N^2)$
  cost).
\end{itemize}

%%
%%----------------------------------------------
%%

\noindent The {\sunlinsollapdense} module defines dense implementations of all
``direct'' linear solver operations listed in Sections
\ref{ss:sunlinsol_CoreFn}-\ref{ss:sunlinsol_GetFn}:
\begin{itemize}
\item \id{SUNLinSolGetType\_LapackDense}
\item \id{SUNLinSolInitialize\_LapackDense} -- this does nothing, since all
  consistency checks are performed at solver creation.
\item \id{SUNLinSolSetup\_LapackDense} -- this calls either
  \id{DGETRF} or \id{SGETRF} to perform the $LU$ factorization.
\item \id{SUNLinSolSolve\_LapackDense} -- this calls either
  \id{DGETRS} or \id{SGETRS} to use the $LU$ factors and \id{pivots}
  array to perform the solve.
\item \id{SUNLinSolLastFlag\_LapackDense}
\item \id{SUNLinSolSpace\_LapackDense} -- this only returns information for
  the storage \emph{within} the solver object, i.e.~storage
  for \id{N}, \id{last\_flag}, and \id{pivots}.
\item \id{SUNLinSolFree\_LapackDense}
\end{itemize}

%% This is a shared SUNDIALS TEX file with a description of the
%% lapackband sunlinsol implementation
%%

The LAPACK band implementation of the {\sunlinsol} module provided
with {\sundials}, {\sunlinsollapband}, is designed to be used with the
corresponding {\sunmatband} matrix type, and one of the serial or
shared-memory {\nvector} implementations ({\nvecs}, {\nvecopenmp}, or
{\nvecpthreads}).  The {\sunlinsollapband} module defines the {\em
content} field of a\\
\noindent\id{SUNLinearSolver} to be the following structure:
%%
\begin{verbatim} 
struct _SUNLinearSolverContent_Band {
  sunindextype N;
  sunindextype *pivots;
  long int last_flag;
};
\end{verbatim}
%%
These entries of the \emph{content} field contain the following
information:
\begin{description}
  \item[N] - size of the linear system,
  \item[pivots] - index array for partial pivoting in LU factorization,
  \item[last\_flag] - last error return flag from internal function evaluations.
\end{description}

{\warn} The {\sunlinsollapband} module is a {\sunlinsol} wrapper for
the LAPACK band matrix factorization and solve routines, \id{*GBTRF}
and \id{*GBTRS}, where \id{*} is either \id{D} or \id{S}, depending on
whether {\sundials} was configured to have \id{realtype} set to
\id{double} or \id{single}, respectively (see Section \ref{s:types}).
In order to use the {\sunlinsollapband} module it is assumed
that LAPACK has been installed on the system prior to installation of
{\sundials}, and that {\sundials} has been configured appropriately to
link with LAPACK (see Appendix \ref{c:install} for details).  We note
that since there do not exist 128-bit floating-point factorization and
solve routines in LAPACK, this interface cannot be compiled when
using \id{extended} precision for \id{realtype}.  Similarly, since
there do not exist 64-bit integer LAPACK routines, the
{\sunlinsollapband} module also cannot be compiled when using 
\id{int64\_t} for the \id{sunindextype}.

This solver is constructed to perform the following operations:
\begin{itemize}
\item The ``setup'' call performs a $LU$ factorization with
  partial (row) pivoting, $PA=LU$, where $P$ is a permutation matrix,
  $L$ is a lower triangular matrix with 1's on the diagonal, and $U$
  is an upper triangular matrix.  This factorization is stored
  in-place on the input {\sunmatband} object $A$, with pivoting
  information encoding $P$ stored in the \id{pivots} array.
\item The ``solve'' call performs pivoting and forward and
  backward substitution using the stored \id{pivots} array and the
  $LU$ factors held in the {\sunmatband} object.
\item
  {\warn} $A$ must be allocated to accommodate the increase in upper
  bandwidth that occurs during factorization.  More precisely, if $A$
  is a band matrix with upper bandwidth \id{mu} and lower bandwidth
  \id{ml}, then the upper triangular factor $U$ can have upper
  bandwidth as big as \id{smu = MIN(N-1,mu+ml)}. The lower triangular
  factor $L$ has lower bandwidth \id{ml}.
\end{itemize}


\noindent The header file to be included when using this module 
is \id{sunlinsol/sunlinsol\_lapackband.h}. \\
%%
%%----------------------------------------------
%%
The {\sunlinsollapband} module defines band implementations of all
``direct'' linear solver operations listed in
Table \ref{t:sunlinsolops}:
\begin{itemize}
\item \id{SUNLinSolGetType\_LapackBand}
\item \id{SUNLinSolInitialize\_LapackBand} -- this does nothing, since all
  consistency checks are performed at solver creation.
\item \id{SUNLinSolSetup\_LapackBand} -- this calls either
  \id{DGBTRF} or \id{SGBTRF} to perform the $LU$ factorization.
\item \id{SUNLinSolSolve\_LapackBand} -- this calls either
  \id{DGBTRS} or \id{SGBTRS} to use the $LU$ factors and \id{pivots}
  array to perform the solve.
\item \id{SUNLinSolLastFlag\_LapackBand}
\item \id{SUNLinSolSpace\_LapackBand} -- this only returns information for
  the storage \emph{within} the solver object, i.e.~storage
  for \id{N}, \id{last\_flag}, and \id{pivots}.
\item \id{SUNLinSolFree\_LapackBand}
\end{itemize}
The module {\sunlinsollapband} provides the following additional
user-callable routine: 
%%
\begin{itemize}

%%--------------------------------------

\item \ID{SUNLapackBand}

  This function creates and allocates memory for a LAPACK band
  \id{SUNLinearSolver}.  Its arguments are an {\nvector} and
  {\sunmatrix}, that it uses to determine the linear system size and
  to assess compatibility with the linear solver implementation.

  This routine will perform consistency checks to ensure that it is
  called with consistent {\nvector} and {\sunmatrix} implementations.
  These are currently limited to the {\sunmatband} matrix type and
  the {\nvecs}, {\nvecopenmp}, and {\nvecpthreads} vector types.  As
  additional compatible matrix and vector implementations are added to
  {\sundials}, these will be included within this compatibility check.

  Additionally, this routine will verify that the input matrix \id{A}
  is allocated with appropriate upper bandwidth storage for the $LU$
  factorization.

  If either \id{A} or \id{y} are incompatible then this routine will
  return \id{NULL}.

  \verb|SUNLinearSolver SUNLapackBand(N_Vector y, SUNMatrix A);|

\end{itemize}
%%
%%------------------------------------
%%
For solvers that include a Fortran interface module, the
{\sunlinsollapband} module also includes the Fortran-callable
function \id{FSUNLapackBandInit(code, ier)} to initialize
this\\
\noindent {\sunlinsollapband} module for a given {\sundials} solver.
Here \id{code} is an integer input solver id (1 for {\cvode}, 2 for {\ida}, 3
for {\kinsol}, 4 for {\arkode}); \id{ier} is an error return flag 
equal to 0 for success and -1 for failure. Both \id{code} and \id{ier}
are declared to match C type \id{int}.
This routine must be called \emph{after} both the
{\nvector} and {\sunmatrix} objects have been initialized.
Additionally, when using {\arkode} with a non-identity mass matrix, the
Fortran-callable function \id{FSUNMassLapackBandInit(ier)}  
initializes this {\sunlinsollapband} module for solving mass matrix
linear systems.

%% This is a shared SUNDIALS TEX file with a description of the
%% klu sunlinsol implementation
%%

The {\klu} implementation of the {\sunlinsol} module provided with
{\sundials}, {\sunlinsolklu}, is designed to be used with the
corresponding {\sunmatsparse} matrix type, and one of the serial or
shared-memory {\nvector} implementations ({\nvecs}, {\nvecopenmp}, or 
{\nvecpthreads}).


%---------------------------------------------------------------------------
\subsection{{\sunlinsolklu} usage}\label{ss:sunlinsol_klu_usage}

The header file to include when using this module 
is \id{sunlinsol/sunlinsol\_klu.h}. The installed module
library to link to is
\id{libsundials\_sunlinsolklu.\textit{lib}}
where \id{\em.lib} is typically \id{.so} for shared libraries and
\id{.a} for static libraries. 

The module {\sunlinsolklu} provides the following user-callable routines: 
%%
% --------------------------------------------------------------------
\ucfunction{SUNLinSol\_KLU}
{
  LS = SUNLinSol\_KLU(y, A);
}
{
  The function \ID{SUNLinSol\_KLU} creates and allocates memory for a
  {\sunlinsolklu} object.
}
{
  \begin{args}[y]
  \item[y] (\id{N\_Vector})
    a template for cloning vectors needed within the solver
  \item[A] (\id{SUNMatrix})
    a {\sunmatsparse} matrix template for cloning matrices needed
    within the solver 
  \end{args}
}
{
  This returns a \id{SUNLinearSolver} object.  If either \id{A} or
  \id{y} are incompatible then this routine will return \id{NULL}.
}
{
  This routine will perform consistency checks to ensure that it is
  called with consistent {\nvector} and {\sunmatrix} implementations.
  These are currently limited to the {\sunmatsparse} matrix type
  (using either CSR or CSC storage formats) and the {\nvecs},
  {\nvecopenmp}, and {\nvecpthreads} vector types.  As additional
  compatible matrix and vector implementations are added to
  {\sundials}, these will be included within this compatibility
  check. 
}
% --------------------------------------------------------------------
\ucfunction{SUNLinSol\_KLUReInit}
{
  retval = SUNLinSol\_KLUReInit(LS, A, nnz, reinit\_type);
}
{
  The function \ID{SUNLinSol\_KLUReInit} reinitializes memory and
  flags for a new factorization (symbolic and numeric) to be conducted
  at the next solver setup call.  This routine is useful in the cases
  where the number of nonzeroes has changed or if the structure of the
  linear system has changed which would require a new symbolic (and
  numeric factorization). 
}
{
  \begin{args}[reinit\_type]
  \item[LS] (\id{SUNLinearSolver})
    a template for cloning vectors needed within the solver
  \item[A] (\id{SUNMatrix})
    a {\sunmatsparse} matrix template for cloning matrices needed
    within the solver 
  \item[nnz] (\id{sunindextype})
    the new number of nonzeros in the matrix
  \item[reinit\_type] (\id{int})
    flag governing the level of reinitialization.  The allowed values
    are:
    \begin{itemize}
    \item \texttt{SUNKLU\_REINIT\_FULL} -- The Jacobian matrix will be
      destroyed and a new one will be allocated based on the \id{nnz}
      value passed to this call.  New symbolic and numeric
      factorizations will be completed at the next solver setup. 
    \item \texttt{SUNKLU\_REINIT\_PARTIAL} -- Only symbolic and numeric 
      factorizations will be completed.  It is assumed that the
      Jacobian size has not exceeded the size of \id{nnz} given in the
      sparse matrix provided to the original constructor routine (or
      the previous \id{SUNLinSol\_KLUReInit} call). 
    \end{itemize}
  \end{args}
}
{
  The return values from this function are \id{SUNLS\_MEM\_NULL}
  (either \id{S} or \id{A} are \id{NULL}), \id{SUNLS\_ILL\_INPUT}
  (\id{A} does not have type \id{SUNMATRIX\_SPARSE} or
  \id{reinit\_type} is invalid), \id{SUNLS\_MEM\_FAIL} (reallocation
  of the sparse matrix failed) or \id{SUNLS\_SUCCESS}.
}
{
  This routine will perform consistency checks to ensure that it is
  called with consistent {\nvector} and {\sunmatrix} implementations.
  These are currently limited to the {\sunmatsparse} matrix type
  (using either CSR or CSC storage formats) and the {\nvecs},
  {\nvecopenmp}, and {\nvecpthreads} vector types.  As additional
  compatible matrix and vector implementations are added to
  {\sundials}, these will be included within this compatibility
  check.

  This routine assumes no other changes to solver use are necessary.
}
% --------------------------------------------------------------------
\ucfunction{SUNLinSol\_KLUSetOrdering}
{
  retval = SUNLinSol\_KLUSetOrdering(LS, ordering);
}
{
  This function sets the ordering used by {\klu} for reducing fill in
  the linear solve.
}
{
  \begin{args}[ordering]
  \item[LS] (\id{SUNLinearSolver})
    the {\sunlinsolklu} object
  \item[ordering] (\id{int})
    flag indication the reordering algorithm to use.  Options include:
    \begin{itemize}
    \item[0] AMD,
    \item[1] COLAMD, and
    \item[2] the natural ordering.
    \end{itemize}
    The default is 1 for COLAMD.
  \end{args}
}
{
  The return values from this function are \id{SUNLS\_MEM\_NULL}
  (\id{S} is \id{NULL}), \id{SUNLS\_ILL\_INPUT}
  (invalid \id{ordering}), or \id{SUNLS\_SUCCESS}.
}
{
}
% --------------------------------------------------------------------
%%
For backwards compatibility, we also provide the wrapper functions,
each with identical input and output arguments to the routines that
they wrap:
\begin{itemize}

\item \ID{SUNKLU}

  Wrapper function for \id{SUNLinSol\_KLU}

\item \ID{SUNKLUReInit}

  Wrapper function for \id{SUNLinSol\_KLUReInit}

\item \ID{SUNKLUSetOrdering}

  Wrapper function for \id{SUNLinSol\_KLUSetOrdering}

\end{itemize}
%%
%%------------------------------------
%%
For solvers that include a Fortran interface module, the
{\sunlinsolklu} module also includes a Fortran-callable function
for creating a \id{SUNLinearSolver} object.
% --------------------------------------------------------------------
\ucfunction{FSUNKLUINIT}
{
  FSUNKLUINIT(code, ier)
}
{
  The function \ID{FSUNKLUINIT} can be called for Fortran programs
  to create a {\sunlinsolklu} object.
}
{
  \begin{args}[code]
  \item[code] (\id{int*})
    is an integer input specifying the solver id (1 for {\cvode}, 2
    for {\ida}, 3 for {\kinsol}, and 4 for {\arkode}).
  \end{args}
}
{
  \id{ier} is a return completion flag equal to \id{0} for a success
  return and \id{-1} otherwise. See printed message for details in case
  of failure.
}
{
  This routine must be
  called \emph{after} both the {\nvector} and {\sunmatrix} objects have
  been initialized.
}
% --------------------------------------------------------------------
Additionally, when using
{\arkode} with a non-identity mass matrix, the {\sunlinsolklu} module
includes a Fortran-callable function for creating a
\id{SUNLinearSolver} mass matrix solver object.
% --------------------------------------------------------------------
\ucfunction{FSUNMASSKLUINIT}
{
  FSUNMASSKLUINIT(ier)
}
{
  The function \ID{FSUNMASSKLUINIT} can be called for Fortran programs
  to create a {\sunlinsolklu} object for mass matrix linear systems.
}
{
}
{
  \id{ier} is a \id{int} return completion flag equal to \id{0} for a success
  return and \id{-1} otherwise. See printed message for details in case
  of failure.
}
{
  This routine must be
  called \emph{after} both the {\nvector} and {\sunmatrix} mass-matrix
  objects have been initialized.
}
% --------------------------------------------------------------------
The \id{SUNLinSol\_KLUReInit} and \ID{SUNLinSol\_KLUSetOrdering}
routines also support Fortran interfaces for the system and mass
matrix solvers: 
% --------------------------------------------------------------------
\ucfunction{FSUNKLUREINIT}
{
  FSUNKLUREINIT(code, nnz, reinit\_type, ier)
}
{
  The function \ID{FSUNKLUREINIT} can be called for Fortran programs
  to re-initialize a {\sunlinsolklu} object.
}
{
  \begin{args}[reinit\_type]
  \item[code] (\id{int*})
    is an integer input specifying the solver id (1 for {\cvode}, 2
    for {\ida}, 3 for {\kinsol}, and 4 for {\arkode}).
  \item[nnz] (\id{sunindextype*})
    the new number of nonzeros in the matrix
  \item[reinit\_type] (\id{int*})
    flag governing the level of reinitialization.  The allowed values
    are:
    \begin{itemize}
    \item[1] -- The Jacobian matrix will be
      destroyed and a new one will be allocated based on the \id{nnz}
      value passed to this call.  New symbolic and numeric
      factorizations will be completed at the next solver setup. 
    \item[2] -- Only symbolic and numeric 
      factorizations will be completed.  It is assumed that the
      Jacobian size has not exceeded the size of \id{nnz} given in the
      sparse matrix provided to the original constructor routine (or
      the previous \id{SUNLinSol\_KLUReInit} call). 
    \end{itemize}
  \end{args}
}
{
  \id{ier} is a \id{int} return completion flag equal to \id{0} for a success
  return and \id{-1} otherwise. See printed message for details in case
  of failure.
}
{
  See \id{SUNLinSol\_KLUReInit} for complete further documentation of
  this routine. 
}
% --------------------------------------------------------------------
\ucfunction{FSUNMASSKLUREINIT}
{
  FSUNMASSKLUREINIT(nnz, reinit\_type, ier)
}
{
  The function \ID{FSUNMASSKLUREINIT} can be called for Fortran programs
  to re-initialize a {\sunlinsolklu} object for mass matrix linear systems.
}
{
  The arguments are identical to \id{FSUNKLUREINIT} above, except that
  \id{code} is not needed since mass matrix linear systems only arise
  in {\arkode}.
}
{
  \id{ier} is a \id{int} return completion flag equal to \id{0} for a success
  return and \id{-1} otherwise. See printed message for details in case
  of failure.
}
{
  See \id{SUNLinSol\_KLUReInit} for complete further documentation of
  this routine. 
}
% --------------------------------------------------------------------
\ucfunction{FSUNKLUSETORDERING}
{
  FSUNKLUSETORDERING(code, ordering, ier)
}
{
  The function \ID{FSUNKLUSETORDERING} can be called for Fortran programs
  to change the reordering algorithm used by {\klu}.
}
{
  \begin{args}[ordering]
  \item[code] (\id{int*})
    is an integer input specifying the solver id (1 for {\cvode}, 2
    for {\ida}, 3 for {\kinsol}, and 4 for {\arkode}).
  \item[ordering] (\id{int*})
    flag indication the reordering algorithm to use.  Options include:
    \begin{itemize}
    \item[0] AMD,
    \item[1] COLAMD, and
    \item[2] the natural ordering.
    \end{itemize}
    The default is 1 for COLAMD.
  \end{args}
}
{
  \id{ier} is a \id{int} return completion flag equal to \id{0} for a success
  return and \id{-1} otherwise. See printed message for details in case
  of failure.
}
{
  See \id{SUNLinSol\_KLUSetOrdering} for complete further documentation of
  this routine. 
}
% --------------------------------------------------------------------
\ucfunction{FSUNMASSKLUSETORDERING}
{
  FSUNMASSKLUSETORDERING(ier)
}
{
  The function \ID{FSUNMASSKLUSETORDERING} can be called for Fortran programs
  to change the reordering algorithm used by {\klu} for mass matrix linear systems.
}
{
  The arguments are identical to \id{FSUNKLUSETORDERING} above, except that
  \id{code} is not needed since mass matrix linear systems only arise
  in {\arkode}.
}
{
  \id{ier} is a \id{int} return completion flag equal to \id{0} for a success
  return and \id{-1} otherwise. See printed message for details in case
  of failure.
}
{
  See \id{SUNLinSol\_KLUSetOrdering} for complete further documentation of
  this routine. 
}


%---------------------------------------------------------------------------
\subsection{{\sunlinsolklu} description}\label{ss:sunlinsol_klu_description}


The {\sunlinsolklu} module defines the {\em
content} field of a \id{SUNLinearSolver} to be the following structure:
%%
\begin{verbatim} 
struct _SUNLinearSolverContent_KLU {
  long int         last_flag;
  int              first_factorize;
  sun_klu_symbolic *symbolic;
  sun_klu_numeric  *numeric;
  sun_klu_common   common;
  sunindextype     (*klu_solver)(sun_klu_symbolic*, sun_klu_numeric*,
                                 sunindextype, sunindextype,
                                 double*, sun_klu_common*);
};
\end{verbatim}
%%
These entries of the \emph{content} field contain the following
information:
\begin{description}
  \item[last\_flag] - last error return flag from internal function evaluations,
  \item[first\_factorize] - flag indicating whether the factorization
    has ever been performed, 
  \item[symbolic] - {\klu} storage structure for symbolic factorization components,
  \item[numeric] - {\klu} storage structure for numeric factorization components,
  \item[common] - storage structure for common {\klu} solver components,
  \item[klu\_solver] -- pointer to the appropriate {\klu} solver function
    (depending on whether it is using a CSR or CSC sparse matrix).
\end{description}

{\warn} The {\sunlinsolklu} module is a {\sunlinsol} wrapper for
the {\klu} sparse matrix factorization and solver library written by Tim
Davis \cite{KLU_site,DaPa:10}.  In order to use the
{\sunlinsolklu} interface to {\klu}, it is assumed that {\klu} has
been installed on the system prior to installation of {\sundials}, and
that {\sundials} has been configured appropriately to link with {\klu}
(see Appendix \ref{c:install} for details).  Additionally, this
wrapper only supports double-precision calculations, and therefore
cannot be compiled if {\sundials} is configured to have \id{realtype}
set to either \id{extended} or \id{single} (see Section \ref{s:types}).
Since the {\klu} library supports both 32-bit and 64-bit integers, this
interface will be compiled for either of the available \id{sunindextype} options.

The {\klu} library has a symbolic factorization routine that computes
the permutation of the linear system matrix to block triangular form
and the permutations that will pre-order the diagonal blocks (the only
ones that need to be factored) to reduce fill-in (using AMD, COLAMD,
CHOLAMD, natural, or an ordering given by the user).  Of these
ordering choices, the default value in the {\sunlinsolklu} 
module is the COLAMD ordering.

{\klu} breaks the factorization into two separate parts.  The first is
a symbolic factorization and the second is a numeric factorization
that returns the factored matrix along with final pivot information.   
{\klu} also has a refactor routine that can be called instead of the numeric 
factorization.  This routine will reuse the pivot information.  This routine 
also returns diagnostic information that a user can examine to determine if 
numerical stability is being lost and a full numerical factorization should 
be done instead of the refactor.

Since the linear systems that arise within the context of {\sundials}
calculations will typically have identical sparsity patterns, the
{\sunlinsolklu} module is constructed to perform the
following operations:
\begin{itemize}
\item The first time that the ``setup'' routine is called, it
  performs the symbolic factorization, followed by an initial
  numerical factorization.  
\item On subsequent calls to the ``setup'' routine, it calls the
  appropriate {\klu} ``refactor'' routine, followed by estimates of
  the numerical conditioning using the relevant ``rcond'', and if
  necessary ``condest'', routine(s).  If these estimates of the
  condition number are larger than $\varepsilon^{-2/3}$ (where
  $\varepsilon$ is the double-precision unit roundoff), then a new
  factorization is performed.
\item The module includes the routine \id{SUNKLUReInit}, that 
  can be called by the user to force a full or partial refactorization
  at the next ``setup'' call. 
\item The ``solve'' call performs pivoting and forward and
  backward substitution using the stored {\klu} data structures.  We
  note that in this solve {\klu} operates on the native data arrays
  for the right-hand side and solution vectors, without requiring
  costly data copies.
\end{itemize}


%%
%%----------------------------------------------
%%

\noindent The {\sunlinsolklu} module defines implementations of all
``direct'' linear solver operations listed in Sections
\ref{ss:sunlinsol_CoreFn}-\ref{ss:sunlinsol_GetFn}:
\begin{itemize}
\item \id{SUNLinSolGetType\_KLU}
\item \id{SUNLinSolInitialize\_KLU} -- this sets the
  \id{first\_factorize} flag to 1, forcing both symbolic and numerical
  factorizations on the subsequent ``setup'' call.
\item \id{SUNLinSolSetup\_KLU} -- this performs either a $LU$
  factorization or refactorization of the input matrix.
\item \id{SUNLinSolSolve\_KLU} -- this calls the appropriate {\klu}
  solve routine to utilize the $LU$ factors to solve the linear
  system. 
\item \id{SUNLinSolLastFlag\_KLU}
\item \id{SUNLinSolSpace\_KLU} -- this only returns information for
  the storage within the solver \emph{interface}, i.e.~storage for the
  integers \id{last\_flag} and \id{first\_factorize}.  For additional
  space requirements, see the {\klu} documentation.
\item \id{SUNLinSolFree\_KLU}
\end{itemize}

%% This is a shared SUNDIALS TEX file with a description of the
%% superludist sunlinsol implementation
%%
\section{The SUNLinearSolver\_SuperLUDIST implementation}\label{ss:sunlinsol_sludist}

The {\superludist} implementation of the {\sunlinsol} module provided with
{\sundials},\\
\noindent{\sunlinsolsludist}, is designed to be used with the
corresponding {\sunmatslunrloc} matrix type, and one of the serial, threaded
or parallel {\nvector} implementations ({\nvecs}, {\nvecopenmp}, {\nvecpthreads},
{\nvecp}, or {\nvecph}).

The header file to include when using this module
is \id{sunlinsol/sunlinsol\_superludist.h}. The installed module
library to link to is
\id{libsundials\_sunlinsolsuperludist.\textit{lib}}
where \id{\em.lib} is typically \id{.so} for shared libraries and
\id{.a} for static libraries.


% --------------------------------------------------------------------
\subsection{SUNLinearSolver\_SuperLUDIST description}\label{ss:sunlinsol_sludist_description}

The {\sunlinsolsludist} module is a {\sunlinsol} adapter for the
{\superludist} sparse matrix factorization and solver library written by
X. Sherry Li \cite{SuperLUDIST_site,GDL:07,LD:03,SLUUG:99}.
The package uses a SPMD parallel programming model and multithreading
to enhance efficiency in distributed-memory parallel environments with
multicore nodes and possibly GPU accelerators. It uses {\mpi} for communication,
{\openmp} for threading, and {\cuda} for GPU support. In order to use the
{\sunlinsolsludist} interface to {\superludist}, it is assumed that {\superludist}
has been installed on the system prior to installation of {\sundials}, and
that {\sundials} has been configured appropriately to link with {\superludist}
(see Appendix \ref{c:install} for details). Additionally, the adapter only
supports double-precision calculations, and therefore cannot be compiled if {\sundials}
is configured to use single or extended precision. Moreover, since the {\superludist}
library may be installed to support either 32-bit or 64-bit integers,
it is assumed that the {\superludist} library is installed using the same
integer size as {\sundials}.

The {\superludist} library provides many options to control how a linear
system will be solved. These options may be set by a user on an instance
of the \id{superlu\_dist\_options\_t} struct, and then it may be provided
as an argument to the {\sunlinsolsludist} constructor. The {\sunlinsolsludist}
module will respect all options set except for \id{Fact} -- this option is
necessarily modified by the {\sunlinsolsludist} module in the setup and solve routines.

Since the linear systems that arise within the context of {\sundials}
calculations will typically have identical sparsity patterns, the
{\sunlinsolsludist} module is constructed to perform the
following operations:
\begin{itemize}
\item The first time that the ``setup'' routine is called, it
  sets the {\superludist} option \id{Fact} to \id{DOFACT} so that a subsequent
  call to the ``solve'' routine will perform a symbolic factorization,
  followed by an initial numerical factorization before continuing
  to solve the system.
\item On subsequent calls to the ``setup'' routine, it sets the
  {\superludist} option \id{Fact} to \id{SamePattern} so that
  a subsequent call to ``solve'' will perform factorization assuming
  the same sparsity pattern as prior, i.e. it will reuse the column
  permutation vector.
\item If ``setup'' is called prior to the ``solve'' routine, then the ``solve''
  routine will perform a symbolic factorization, followed by an initial
  numerical factorization before continuing to the sparse triangular
  solves, and, potentially, iterative refinement. If ``setup'' is not
  called prior, ``solve'' will skip to the triangular solve step. We
  note that in this solve {\superludist} operates on the native data arrays
  for the right-hand side and solution vectors, without requiring costly data copies.
\end{itemize}


{\warn} Starting with SuperLU\_DIST version 6.3.0, some structures were renamed
to have a prefix for the floating point type. The double precision API functions
have the prefix 'd'. To maintain backwards compatibility with the unprefixed
types, SUNDIALS provides macros to these SuperLU\_DIST types with an 'x' prefix
that expand to the correct prefix. E.g., the SUNDIALS macro \id{xLUstruct\_t}
expands to \id{dLUstruct\_t} or \id{LUstruct\_t} based on the SuperLU\_DIST
version.


\subsection{SUNLinearSolver\_SuperLUDIST functions}\label{ss:sunlinsol_sludist_functions}

The {\sunlinsolsludist} module defines implementations of all
``direct'' linear solver operations listed in Sections
\ref{ss:sunlinsol_CoreFn}-\ref{ss:sunlinsol_GetFn}:
\begin{itemize}
\item \id{SUNLinSolGetType\_SuperLUDIST}
\item \id{SUNLinSolInitialize\_SuperLUDIST} -- this sets the
  \id{first\_factorize} flag to 1 and resets the internal {\superludist}
  statistics variables.
\item \id{SUNLinSolSetup\_SuperLUDIST} -- this sets the appropriate
  {\superludist} options so that a subsequent solve will perform a
  symbolic and numerical factorization before proceeding with the
  triangular solves
\item \id{SUNLinSolSolve\_SuperLUDIST} -- this calls the {\superludist}
  solve routine to perform factorization (if the setup routine
  was called prior) and then use the $LU$ factors to solve the
  linear system.
\item \id{SUNLinSolLastFlag\_SuperLUDIST}
\item \id{SUNLinSolSpace\_SuperLUDIST} -- this only returns information for
  the storage within the solver \emph{interface}, i.e.~storage for the
  integers \id{last\_flag} and \id{first\_factorize}.  For additional
  space requirements, see the {\superludist} documentation.
\item \id{SUNLinSolFree\_SuperLUDIST}
\end{itemize}

In addition, the module {\sunlinsolsludist} provides the following user-callable routines:
%%
% --------------------------------------------------------------------
\ucfunction{SUNLinSol\_SuperLUDIST}
{
  LS = SUNLinSol\_SuperLUDIST(y, A, grid, lu, scaleperm, solve, stat, options);
}
{
  The function \ID{SUNLinSol\_SuperLUDIST} creates and allocates memory for a
  {\sunlinsolsludist} object.
}
{
  \begin{args}[options]
  \item[y] (\id{N\_Vector})
    a template for cloning vectors needed within the solver
  \item[A] (\id{SUNMatrix})
    a {\sunmatslunrloc} matrix template for cloning matrices needed
    within the solver
  \item[grid] (\id{gridinfo\_t*})
  \item[lu] (\id{LUstruct\_t*})
  \item[scaleperm] (\id{ScalePermstruct\_t*})
  \item[solve] (\id{SOLVEstruct\_t*})
  \item[stat] (\id{SuperLUStat\_t*})
  \item[options] (\id{superlu\_dist\_options\_t*})
  \end{args}
}
{
  This returns a \id{SUNLinearSolver} object.  If either \id{A} or
  \id{y} are incompatible then this routine will return \id{NULL}.
}
{
  This routine analyzes the input matrix and vector to determine the
  linear system size and to assess compatibility with the {\superludist}
  library.

  This routine will perform consistency checks to ensure that it is
  called with consistent {\nvector} and {\sunmatrix} implementations.
  These are currently limited to the {\sunmatslunrloc} matrix type
  and the {\nvecs}, {\nvecp}, {\nvecph}, {\nvecopenmp}, and {\nvecpthreads}
  vector types. As additional compatible matrix and vector implementations
  are added to {\sundials}, these will be included within this compatibility
  check.

  The \id{grid}, \id{lu}, \id{scaleperm}, \id{solve}, and \id{options} arguments
  are not checked and are passed directly to {\superludist} routines.

  Some struct members of the \id{options} argument are modified internally
  by the {\sunlinsolsludist} solver. Specifically the member \id{Fact},
  is modified in the setup and solve routines.
}

% --------------------------------------------------------------------
\ucfunction{SUNLinSol\_SuperLUDIST\_GetBerr}
{
  realtype berr = SUNLinSol\_SuperLUDIST\_GetBerr(LS);
}
{
  The function \ID{SUNLinSol\_SuperLUDIST\_GetBerr} returns the componentwise
  relative backward error of the computed solution.
}
{
  \begin{args}[LS]
  \item[LS] (\id{SUNLinearSolver})
    the {\sunlinsolsludist} object
  \end{args}
}
{
  \id{realtype}
}
{
}

% --------------------------------------------------------------------
\ucfunction{SUNLinSol\_SuperLUDIST\_GetGridinfo}
{
  gridinfo\_t *grid = SUNLinSol\_SuperLUDIST\_GetGridinfo(LS);
}
{
  The function \ID{SUNLinSol\_SuperLUDIST\_GetGridinfo} returns the
  {\superludist} structure that contains the 2D process grid.
}
{
  \begin{args}[LS]
  \item[LS] (\id{SUNLinearSolver})
    the {\sunlinsolsludist} object
  \end{args}
}
{
  \id{gridinfo\_t*}
}
{
}

% --------------------------------------------------------------------
\ucfunction{SUNLinSol\_SuperLUDIST\_GetLUstruct}
{
  LUstruct\_t *lu = SUNLinSol\_SuperLUDIST\_GetLUstruct(LS);
}
{
  The function \ID{SUNLinSol\_SuperLUDIST\_GetLUstruct} returns the
  {\superludist} structure that contains the distributed $L$ and $U$ factors.
}
{
  \begin{args}[LS]
  \item[LS] (\id{SUNLinearSolver})
    the {\sunlinsolsludist} object
  \end{args}
}
{
  \id{LUstruct\_t*}
}
{
}

% --------------------------------------------------------------------
\ucfunction{SUNLinSol\_SuperLUDIST\_GetSuperLUOptions}
{
  superlu\_dist\_options\_t *opts = SUNLinSol\_SuperLUDIST\_GetSuperLUOptions(LS);
}
{
  The function \ID{SUNLinSol\_SuperLUDIST\_GetSuperLUOptions} returns the
  {\superludist} structure that contains the options which control how
  the linear system is factorized and solved.
}
{
  \begin{args}[LS]
  \item[LS] (\id{SUNLinearSolver})
    the {\sunlinsolsludist} object
  \end{args}
}
{
  \id{superlu\_dist\_options\_t*}
}
{
}

% --------------------------------------------------------------------
\ucfunction{SUNLinSol\_SuperLUDIST\_GetScalePermstruct}
{
  ScalePermstruct\_t *sp = SUNLinSol\_SuperLUDIST\_GetScalePermstruct(LS);
}
{
  The function \ID{SUNLinSol\_SuperLUDIST\_GetScalePermstruct} returns the
  {\superludist} structure that contains the vectors that describe the
  transformations done to the matrix, $A$.
}
{
  \begin{args}[LS]
  \item[LS] (\id{SUNLinearSolver})
    the {\sunlinsolsludist} object
  \end{args}
}
{
  \id{ScalePermstruct\_t*}
}
{
}

% --------------------------------------------------------------------
\ucfunction{SUNLinSol\_SuperLUDIST\_GetSOLVEstruct}
{
  SOLVEstruct\_t *solve = SUNLinSol\_SuperLUDIST\_GetSOLVEstruct(LS);
}
{
  The function \ID{SUNLinSol\_SuperLUDIST\_GetSOLVEstruct} returns the
  {\superludist} structure that contains information for communication
  during the solution phase.
}
{
  \begin{args}[LS]
  \item[LS] (\id{SUNLinearSolver})
    the {\sunlinsolsludist} object
  \end{args}
}
{
  \id{SOLVEstruct\_t*}
}
{
}

% --------------------------------------------------------------------
\ucfunction{SUNLinSol\_SuperLUDIST\_GetSuperLUStat}
{
  SuperLUStat\_t *stat = SUNLinSol\_SuperLUDIST\_GetSuperLUStat(LS);
}
{
  The function \ID{SUNLinSol\_SuperLUDIST\_GetSuperLUStat} returns the
  {\superludist} structure that stores information about runtime and
  flop count.
}
{
  \begin{args}[LS]
  \item[LS] (\id{SUNLinearSolver})
    the {\sunlinsolsludist} object
  \end{args}
}
{
  \id{SuperLUStat\_t*}
}
{
}


% --------------------------------------------------------------------
\subsection{SUNLinearSolver\_SuperLUDIST content}\label{ss:sunlinsol_sludist_content}

The {\sunlinsolsludist} module defines the {\em
content} field of a \id{SUNLinearSolver} to be the following structure:
%%
\begin{verbatim}
struct _SUNLinearSolverContent_SuperLUDIST {
  booleantype             first_factorize;
  int                     last_flag;
  realtype                berr;
  gridinfo_t              *grid;
  xLUstruct_t             *lu;
  superlu_dist_options_t  *options;
  xScalePermstruct_t      *scaleperm;
  xSOLVEstruct_t          *solve;
  SuperLUStat_t           *stat;
  sunindextype            N;
};
\end{verbatim}
%%
These entries of the \emph{content} field contain the following
information:
\begin{description}
  \item[first\_factorize] - flag indicating whether the factorization
    has ever been performed,
  \item[last\_flag] - last error return flag from calls to internal routines,
  \item[berr] - the componentwise relative backward error of the computed solution,
  \item[grid] - pointer to the {\superludist} structure that stores the 2D process grid,
  \item[lu] - pointer to the {\superludist} structure that stores the distributed $L$
    and $U$ factors,
  \item[options] - pointer to {\superludist} options structure,
  \item[scaleperm] - pointer to the {\superludist} structure that stores vectors describing
    the transformations done to the matrix, $A$,
  \item[solve] - pointer to the {\superludist} solve structure,
  \item[stat] - pointer to the {\superludist} structure that stores information about runtime
    and flop count,
  \item[N] - the number of equations in the system
\end{description}

% ====================================================================
\section{The SUNLinearSolver\_SuperLUMT implementation}
\label{ss:sunlinsol_superlumt}
% ====================================================================

This section describes the {\sunlinsol} implementation for solving sparse linear
systems with SuperLU\_MT. The {\superlumt} module is designed to be used with the
corresponding {\sunmatsparse} matrix type, and one of the serial or
shared-memory {\nvector} implementations ({\nvecs}, {\nvecopenmp}, or
{\nvecpthreads}). While these are compatible, it is not recommended
to use a threaded vector module with {\sunlinsolslumt} unless it is
the {\nvecopenmp} module and the {\superlumt} library has also been
compiled with OpenMP.

The header file to include when using this module
is \id{sunlinsol/sunlinsol\_superlumt.h}. The installed module
library to link to is
\id{libsundials\_sunlinsolsuperlumt.\textit{lib}}
where \id{\em.lib} is typically \id{.so} for shared libraries and
\id{.a} for static libraries.

The {\sunlinsolslumt} module is a {\sunlinsol} wrapper for
the {\superlumt} sparse matrix factorization and solver library
written by X. Sherry Li \cite{SuperLUMT_site,Li:05,DGL:99}.  The
package performs matrix factorization using threads to enhance
efficiency in shared memory parallel environments.  It should be noted
that threads are only used in the factorization step.  In
order to use the {\sunlinsolslumt} interface to {\superlumt}, it is
assumed that {\superlumt} has been installed on the system prior to
installation of {\sundials}, and that {\sundials} has been configured
appropriately to link with {\superlumt} (see Appendix \ref{c:install}
for details).  Additionally, this wrapper only supports single- and
double-precision calculations, and therefore cannot be compiled if
{\sundials} is configured to have \id{realtype} set to \id{extended}
(see Section \ref{s:types}).  Moreover, since the {\superlumt} library
may be installed to support either 32-bit or 64-bit integers, it is
assumed that the {\superlumt} library is installed using the same
integer precision as the {\sundials} \id{sunindextype} option. {\warn}

% ====================================================================
\subsection{SUNLinearSolver\_SuperLUMT description}
\label{ss:sunlinsol_slumt_usage}
% ====================================================================

The {\superlumt} library has a symbolic factorization routine that
computes the permutation of the linear system matrix to reduce fill-in
on subsequent $LU$ factorizations (using COLAMD, minimal degree
ordering on $A^T*A$, minimal degree ordering on $A^T+A$, or natural
ordering).  Of these ordering choices, the default value in the
{\sunlinsolslumt} module is the COLAMD ordering.

Since the linear systems that arise within the context of {\sundials}
calculations will typically have identical sparsity patterns, the
{\sunlinsolslumt} module is constructed to perform the
following operations:
\begin{itemize}
\item The first time that the ``setup'' routine is called, it
  performs the symbolic factorization, followed by an initial
  numerical factorization.
\item On subsequent calls to the ``setup'' routine, it skips the
  symbolic factorization, and only refactors the input matrix.
\item The ``solve'' call performs pivoting and forward and
  backward substitution using the stored {\superlumt} data
  structures.  We note that in this solve {\superlumt} operates on the
  native data arrays for the right-hand side and solution vectors,
  without requiring costly data copies.
\end{itemize}


% ====================================================================
\subsection{SUNLinearSolver\_SuperLUMT functions}
\label{ss:sunlinsol_slumt_functions}
% ====================================================================

The module {\sunlinsolslumt} provides the following user-callable constructor
for creating a \newline \id{SUNLinearSolver} object.
%
% --------------------------------------------------------------------
%
\ucfunctiond{SUNLinSol\_SuperLUMT}
{
  LS = SUNLinSol\_SuperLUMT(y, A, num\_threads);
}
{
  The function \ID{SUNLinSol\_SuperLUMT} creates and allocates memory for
  a \newline SuperLU\_MT-based \id{SUNLinearSolver} object.
}
{
  \begin{args}[num\_threads]
  \item[y] (\id{N\_Vector})
    a template for cloning vectors needed within the solver
  \item[A] (\id{SUNMatrix})
    a {\sunmatsparse} matrix template for cloning matrices needed
    within the solver
  \item[num\_threads] (\id{int})
    desired number of threads (OpenMP or Pthreads, depending on how
    {\superlumt} was installed) to use during the factorization steps
  \end{args}
}
{
  This returns a \id{SUNLinearSolver} object.  If either \id{A} or
  \id{y} are incompatible then this routine will return \id{NULL}.
}
{
  This routine analyzes the input matrix and vector to determine the
  linear system size and to assess compatibility with the {\superlumt}
  library.

  This routine will perform consistency checks to ensure that it is
  called with consistent {\nvector} and {\sunmatrix} implementations.
  These are currently limited to the {\sunmatsparse} matrix type
  (using either CSR or CSC storage formats) and the {\nvecs},
  {\nvecopenmp}, and {\nvecpthreads} vector types.  As additional
  compatible matrix and vector implementations are added to
  {\sundials}, these will be included within this compatibility
  check.

  The \id{num\_threads} argument is not checked and is passed directly
  to {\superlumt} routines.
}
{SUNSuperLUMT}
%
% --------------------------------------------------------------------
%
\noindent The {\sunlinsolslumt} module defines implementations of all
``direct'' linear solver operations listed in Sections
\ref{ss:sunlinsol_CoreFn} -- \ref{ss:sunlinsol_GetFn}:
\begin{itemize}
\item \id{SUNLinSolGetType\_SuperLUMT}
\item \id{SUNLinSolInitialize\_SuperLUMT} -- this sets the
  \id{first\_factorize} flag to 1 and resets the internal {\superlumt}
  statistics variables.
\item \id{SUNLinSolSetup\_SuperLUMT} -- this performs either a $LU$
  factorization or refactorization of the input matrix.
\item \id{SUNLinSolSolve\_SuperLUMT} -- this calls the appropriate
  {\superlumt} solve routine to utilize the $LU$ factors to solve the
  linear system.
\item \id{SUNLinSolLastFlag\_SuperLUMT}
\item \id{SUNLinSolSpace\_SuperLUMT} -- this only returns information for
  the storage within the solver \emph{interface}, i.e.~storage for the
  integers \id{last\_flag} and \id{first\_factorize}.  For additional
  space requirements, see the {\superlumt} documentation.
\item \id{SUNLinSolFree\_SuperLUMT}
\end{itemize}

The {\sunlinsolslumt} module also defines the following additional
user-callable function.
%
% --------------------------------------------------------------------
%
\ucfunctiond{SUNLinSol\_SuperLUMTSetOrdering}
{
  retval = SUNLinSol\_SuperLUMTSetOrdering(LS, ordering);
}
{
  This function sets the ordering used by {\superlumt} for reducing fill in
  the linear solve.
}
{
  \begin{args}[ordering]
  \item[LS] (\id{SUNLinearSolver})
    the {\sunlinsolslumt} object
  \item[ordering] (\id{int})
    a flag indicating the ordering algorithm to use, the options are:
    \begin{itemize}
    \item[0] natural ordering
    \item[1] minimal degree ordering on $A^TA$
    \item[2] minimal degree ordering on $A^T+A$
    \item[3] COLAMD ordering for unsymmetric matrices
    \end{itemize}
    The default is 3 for COLAMD.
  \end{args}
}
{
  The return values from this function are \id{SUNLS\_MEM\_NULL}
  (\id{S} is \id{NULL}), \newline \id{SUNLS\_ILL\_INPUT}
  (invalid ordering choice), or \id{SUNLS\_SUCCESS}.
}
{}
{SUNSuperLUMTSetOrdering}

% ====================================================================
\subsection{SUNLinearSolver\_SuperLUMT Fortran interfaces}
\label{ss:sunlinsol_slumt_fortran}
% ====================================================================

For solvers that include a Fortran interface module, the
{\sunlinsolslumt} module also includes a Fortran-callable function
for creating a \id{SUNLinearSolver} object.
%
% --------------------------------------------------------------------
%
\ucfunction{FSUNSUPERLUMTINIT}
{
  FSUNSUPERLUMTINIT(code, num\_threads, ier)
}
{
  The function \ID{FSUNSUPERLUMTINIT} can be called for Fortran programs
  to create a {\sunlinsolklu} object.
}
{
  \begin{args}[num\_threads]
  \item[code] (\id{int*})
    is an integer input specifying the solver id (1 for {\cvode}, 2
    for {\ida}, 3 for {\kinsol}, and 4 for {\arkode}).
  \item[num\_threads] (\id{int*})
    desired number of threads (OpenMP or Pthreads, depending on how
    {\superlumt} was installed) to use during the factorization steps
  \end{args}
}
{
  \id{ier} is a return completion flag equal to \id{0} for a success
  return and \id{-1} otherwise. See printed message for details in case
  of failure.
}
{
  This routine must be
  called \emph{after} both the {\nvector} and {\sunmatrix} objects have
  been initialized.
}
Additionally, when using {\arkode} with a non-identity
mass matrix, the {\sunlinsolslumt} module includes a Fortran-callable
function for creating a \id{SUNLinearSolver} mass matrix solver
object.
%
% --------------------------------------------------------------------
%
\ucfunction{FSUNMASSSUPERLUMTINIT}
{
  FSUNMASSSUPERLUMTINIT(num\_threads, ier)
}
{
  The function \ID{FSUNMASSSUPERLUMTINIT} can be called for Fortran programs
  to create a SuperLU\_MT-based \id{SUNLinearSolver} object for mass matrix linear
  systems.
}
{
  \begin{args}[num\_threads]
  \item[num\_threads] (\id{int*})
    desired number of threads (OpenMP or Pthreads, depending on how
    {\superlumt} was installed) to use during the factorization steps.
  \end{args}
}
{
  \id{ier} is a \id{int} return completion flag equal to \id{0} for a success
  return and \id{-1} otherwise. See printed message for details in case
  of failure.
}
{
  This routine must be
  called \emph{after} both the {\nvector} and {\sunmatrix} mass-matrix
  objects have been initialized.
}
The \ID{SUNLinSol\_SuperLUMTSetOrdering} routine also supports Fortran
interfaces for the system and mass matrix solvers:
%
% --------------------------------------------------------------------
%
\ucfunction{FSUNSUPERLUMTSETORDERING}
{
  FSUNSUPERLUMTSETORDERING(code, ordering, ier)
}
{
  The function \ID{FSUNSUPERLUMTSETORDERING} can be called for Fortran programs
  to update the ordering algorithm in a {\sunlinsolslumt} object.
}
{
  \begin{args}[ordering]
  \item[code] (\id{int*})
    is an integer input specifying the solver id (1 for {\cvode}, 2
    for {\ida}, 3 for {\kinsol}, and 4 for {\arkode}).
  \item[ordering] (\id{int*})
    a flag indicating the ordering algorithm, options are:
    \begin{itemize}
    \item[0] natural ordering
    \item[1] minimal degree ordering on $A^TA$
    \item[2] minimal degree ordering on $A^T+A$
    \item[3] COLAMD ordering for unsymmetric matrices
    \end{itemize}
    The default is 3 for COLAMD.
  \end{args}
}
{
  \id{ier} is a \id{int} return completion flag equal to \id{0} for a success
  return and \id{-1} otherwise. See printed message for details in case
  of failure.
}
{
  See \id{SUNLinSol\_SuperLUMTSetOrdering} for complete further
  documentation of this routine.
}
%
% --------------------------------------------------------------------
%
\ucfunction{FSUNMASSUPERLUMTSETORDERING}
{
  FSUNMASSUPERLUMTSETORDERING(ordering, ier)
}
{
  The function \ID{FSUNMASSUPERLUMTSETORDERING} can be called for Fortran
  programs to update the ordering algorithm in a {\sunlinsolslumt}
  object for mass matrix linear systems.
}
{
  \begin{args}[ordering]
  \item[ordering] (\id{int*})
    a flag indicating the ordering algorithm, options are:
    \begin{itemize}
    \item[0] natural ordering
    \item[1] minimal degree ordering on $A^TA$
    \item[2] minimal degree ordering on $A^T+A$
    \item[3] COLAMD ordering for unsymmetric matrices
    \end{itemize}
    The default is 3 for COLAMD.
  \end{args}
}
{
  \id{ier} is a \id{int} return completion flag equal to \id{0} for a success
  return and \id{-1} otherwise. See printed message for details in case
  of failure.
}
{
  See \id{SUNLinSol\_SuperLUMTSetOrdering} for complete further
  documentation of this routine.
}


% ====================================================================
\subsection{SUNLinearSolver\_SuperLUMT content}
\label{ss:sunlinsol_slumt_content}
% ====================================================================

The {\sunlinsolslumt} module defines the \textit{content} field of a
\id{SUNLinearSolver} as the following structure:
%%
\begin{verbatim}
struct _SUNLinearSolverContent_SuperLUMT {
  int          last_flag;
  int          first_factorize;
  SuperMatrix  *A, *AC, *L, *U, *B;
  Gstat_t      *Gstat;
  sunindextype *perm_r, *perm_c;
  sunindextype N;
  int          num_threads;
  realtype     diag_pivot_thresh;
  int          ordering;
  superlumt_options_t *options;
};
\end{verbatim}
%%
These entries of the \emph{content} field contain the following
information:
\begin{args}[diag\_pivot\_thresh]
  \item[last\_flag] - last error return flag from internal function evaluations,
  \item[first\_factorize] - flag indicating whether the factorization
    has ever been performed,
  \item[A, AC, L, U, B] - \id{SuperMatrix} pointers used in solve,
  \item[Gstat] - \id{GStat\_t} object used in solve,
  \item[perm\_r, perm\_c] - permutation arrays used in solve,
  \item[N] - size of the linear system,
  \item[num\_threads] - number of OpenMP/Pthreads threads to use,
  \item[diag\_pivot\_thresh] - threshold on diagonal pivoting,
  \item[ordering] - flag for which reordering algorithm to use,
  \item[options] - pointer to {\superlumt} options structure.
\end{args}

%% This is a shared SUNDIALS TEX file with a description of the
%% spgmr sunlinsol implementation
%%

The {\spgmr} (Scaled, Preconditioned, Generalized Minimum
Residual \cite{SaSc:86}) implementation of the {\sunlinsol} module
provided with {\sundials}, {\sunlinsolspgmr}, is an iterative linear
solver that is designed to be compatible with any {\nvector}
implementation (serial, threaded, parallel, and user-supplied) that
supports a minimal subset of operations (\id{N\_VClone}, 
\id{N\_VDotProd}, \id{N\_VScale}, \id{N\_VLinearSum}, \id{N\_VProd},
\id{N\_VConst}, \id{N\_VDiv}, and \id{N\_VDestroy}).  

The {\sunlinsolspgmr} module defines the {\em content} field of a
\id{SUNLinearSolver} to be the following structure:
%%
\begin{verbatim} 
struct _SUNLinearSolverContent_SPGMR {
  int maxl;
  int pretype;
  int gstype;
  int max_restarts;
  int numiters;
  realtype resnorm;
  long int last_flag;
  ATimesFn ATimes;
  void* ATData;
  PSetupFn Psetup;
  PSolveFn Psolve;
  void* PData;
  N_Vector s1;
  N_Vector s2;
  N_Vector *V;
  realtype **Hes;
  realtype *givens;
  N_Vector xcor;
  realtype *yg;
  N_Vector vtemp;
};
\end{verbatim}
%%
These entries of the \emph{content} field contain the following
information:
\begin{description}
  \item[maxl] - number of GMRES basis vectors to use (default is 5),
  \item[pretype] - flag for type of preconditioning to employ
    (default is none),
  \item[gstype] - flag for type of Gram-Schmidt orthogonalization
    (default is modified Gram-Schmidt),
  \item[max\_restarts] - number of GMRES restarts to allow
    (default is 0),
  \item[numiters] - number of iterations from the most-recent solve,
  \item[resnorm] - final linear residual norm from the most-recent solve,
  \item[last\_flag] - last error return flag from an internal function,
  \item[ATimes] - function pointer to perform $Av$ product,
  \item[ATData] - pointer to structure for \id{ATimes},
  \item[Psetup] - function pointer to preconditioner setup routine,
  \item[Psolve] - function pointer to preconditioner solve routine,
  \item[PData] - pointer to structure for \id{Psetup} and \id{Psolve},
  \item[s1, s2] - vector pointers for supplied scaling matrices
    (default is \id{NULL}),
  \item[V] - the array of Krylov basis vectors
    $v_1, \ldots, v_{\text{\id{maxl}}+1}$, stored in \id{V[0]},
    \ldots, \id{V[maxl]}. Each $v_i$ is a vector of type {\nvector}.,
  \item[Hes] - the $(\text{\id{maxl}}+1)\times\text{\id{maxl}}$
    Hessenberg matrix. It is stored row-wise so that the (i,j)th
    element is given by \id{Hes[i][j]}.,
  \item[givens] - a length \id{2*maxl} array which represents the
    Givens rotation matrices that arise in the GMRES algorithm. These
    matrices are $F_0, F_1, \ldots, F_j$, where
    $F_i = \begin{bmatrix}
      1 &        &   &     &      &   &        &   \\
        & \ddots &   &     &      &   &        &   \\
        &        & 1 &     &      &   &        &   \\
        &        &   & c_i & -s_i &   &        &   \\
        &        &   & s_i &  c_i &   &        &   \\
        &        &   &     &      & 1 &        &   \\
        &        &   &     &      &   & \ddots &   \\
        &        &   &     &      &   &        & 1\end{bmatrix}$,
    are represented in the \id{givens} vector as \id{givens[0] =}
    $c_0$, \id{givens[1] = } $s_0$, \id{givens[2] = } $c_1$,
    \id{givens[3] = } $s_1$, \ldots \id{givens[2j] = } $c_j$,
    \id{givens[2j+1] = } $s_j$.,
  \item[xcor] - a vector which holds the scaled, preconditioned
    correction to the initial guess,
  \item[yg] - a length \id{(maxl+1)} array of \id{realtype} values
    used to hold ``short'' vectors (e.g. $y$ and $g$),
  \item[vtemp] - temporary vector storage.
\end{description}

This solver is constructed to perform the following operations:
\begin{itemize}
\item During construction, the \id{xcor} and \id{vtemp} arrays are
  cloned from a template {\nvector} that is input, and default solver
  parameters are set.
\item User-facing ``set'' routines may be called to modify default
  solver parameters.
\item Additional ``set'' routines are called by the {\sundials} solver
  that interfaces with {\sunlinsolspgmr} to supply the 
  \id{ATimes}, \id{PSetup}, and \id{Psolve} function pointers and
  \id{s1} and \id{s2} scaling vectors.
\item In the ``initialize'' call, the remaining solver data is
  allocated (\id{V}, \id{Hes}, \id{givens}, and \id{yg} )
\item In the ``setup'' call, any non-\id{NULL} 
  \id{PSetup} function is called.  Typically, this is provided by
  the {\sundials} solver itself, that translates between the
  generic \id{PSetup} function and the
  solver-specific routine (solver-supplied or user-supplied).
\item In the ``solve'' call, the GMRES iteration is performed.  This
  will include scaling, preconditioning, and restarts if those options
  have been supplied.
\end{itemize}

\noindent The header file to include when using this module 
is \id{sunlinsol/sunlinsol\_spgmr.h}. The {\sunlinsolspgmr} module
is accessible from all {\sundials} solvers \textit{without}
linking to the \\
\id{libsundials\_sunlinsolspgmr} module library. \\

%%
%%----------------------------------------------
%%

\noindent The {\sunlinsolspgmr} module defines implementations of all
``iterative'' linear solver operations listed in Table
\ref{t:sunlinsolops}:
\begin{itemize}
\item \id{SUNLinSolGetType\_SPGMR}
\item \id{SUNLinSolInitialize\_SPGMR}
\item \id{SUNLinSolSetATimes\_SPGMR}
\item \id{SUNLinSolSetPreconditioner\_SPGMR}
\item \id{SUNLinSolSetScalingVectors\_SPGMR}
\item \id{SUNLinSolSetup\_SPGMR}
\item \id{SUNLinSolSolve\_SPGMR}
\item \id{SUNLinSolNumIters\_SPGMR}
\item \id{SUNLinSolResNorm\_SPGMR}
\item \id{SUNLinSolResid\_SPGMR}
\item \id{SUNLinSolLastFlag\_SPGMR}
\item \id{SUNLinSolSpace\_SPGMR}
\item \id{SUNLinSolFree\_SPGMR}
\end{itemize}
The module {\sunlinsolspgmr} provides the following additional
user-callable routines: 
%%
\begin{itemize}

%%--------------------------------------

\item \ID{SUNSPGMR}

  This constructor function creates and allocates memory for a {\spgmr}
  \id{SUNLinearSolver}.  Its arguments are an {\nvector}, the desired
  type of preconditioning, and the number of Krylov basis vectors to use.

  This routine will perform consistency checks to ensure that it is
  called with a consistent {\nvector} implementation (i.e.~that it
  supplies the requisite vector operations).  If \id{y} is
  incompatible, then this routine will return \id{NULL}.

  A \id{maxl} argument that is $\le0$ will result in the default
  value (5).

  Allowable inputs for \id{pretype} are \id{PREC\_NONE} (0),
  \id{PREC\_LEFT} (1), \id{PREC\_RIGHT} (2) and \id{PREC\_BOTH} (3);
  any other integer input will result in the default (no
  preconditioning).
  We note that some {\sundials} solvers are designed to only work
  with left preconditioning ({\ida} and {\idas}) and others with only
  right preconditioning ({\kinsol}). While it is possible to configure
  a {\sunlinsolspgmr} object to use any of the preconditioning options
  with these solvers, this use mode is not supported and may result in
  inferior performance.

  \verb|SUNLinearSolver SUNSPGMR(N_Vector y, int pretype, int maxl);|

%%--------------------------------------

\item \ID{SUNSPGMRSetPrecType}

  This function updates the type of preconditioning to use.  Supported
  values are \id{PREC\_NONE} (0), \id{PREC\_LEFT} (1),
  \id{PREC\_RIGHT} (2) and \id{PREC\_BOTH} (3).  

  This routine will return with one of the error codes
  \id{SUNLS\_ILL\_INPUT} (illegal \id{pretype}), \id{SUNLS\_MEM\_NULL}
  (\id{S} is \id{NULL}) or \id{SUNLS\_SUCCESS}.
  
  \verb|int SUNSPGMRSetPrecType(SUNLinearSolver S, int pretype);|

%%--------------------------------------

\item \ID{SUNSPGMRSetGSType}

  This function sets the type of Gram-Schmidt orthogonalization to
  use.  Supported values are \id{MODIFIED\_GS} (1) and
  \id{CLASSICAL\_GS} (2).  Any other integer input will result in a
  failure, returning error code \id{SUNLS\_ILL\_INPUT}.

  This routine will return with one of the error codes
  \id{SUNLS\_ILL\_INPUT} (illegal \id{gstype}), \id{SUNLS\_MEM\_NULL}
  (\id{S} is \id{NULL}) or \id{SUNLS\_SUCCESS}.
  
  \verb|int SUNSPGMRSetGSType(SUNLinearSolver S, int gstype);|


%%--------------------------------------

\item \ID{SUNSPGMRSetMaxRestarts}

  This function sets the number of GMRES restarts to 
  allow.  A negative input will result in the default of 0.

  This routine will return with one of the error codes
  \id{SUNLS\_MEM\_NULL} (\id{S} is \id{NULL}) or \id{SUNLS\_SUCCESS}.
  
  \verb|int SUNSPGMRSetMaxRestarts(SUNLinearSolver S, int maxrs);|

\end{itemize}
%%
%%------------------------------------
%%
For solvers that include a Fortran interface module, the
{\sunlinsolspgmr} module also includes the Fortran-callable
function \id{FSUNSPGMRInit(code, pretype, maxl, ier)} to initialize
this {\sunlinsolspgmr} module for a given {\sundials} solver.
Here \id{code} is an integer input solver id (1 for {\cvode}, 2 for {\ida}, 3
for {\kinsol}, 4 for {\arkode}); \id{pretype} and \id{maxl} are the
same as for the C function \ID{SUNSPGMR}; \id{ier} is an error return
flag equal to 0 for success and -1 for failure.  All of these input
arguments should be declared so as to match C type \id{int}.  This
routine must be called \emph{after} the {\nvector} object has been
initialized.  Additionally, when using {\arkode} with a non-identity
mass matrix, the Fortran-callable
function \id{FSUNMassSPGMRInit(pretype, maxl, ier)} initializes this 
{\sunlinsolspgmr} module for solving mass matrix linear systems.

The \id{SUNSPGMRSetPrecType}, \id{SUNSPGMRSetGSType} and
\id{SUNSPGMRSetMaxRestarts} routines also support Fortran interfaces
for the system and mass matrix solvers (all arguments should be
commensurate with a C \id{int}):
\begin{itemize}
\item \id{FSUNSPGMRSetGSType(code, gstype, ier)}
\item \id{FSUNMassSPGMRSetGSType(gstype, ier)}
\item \id{FSUNSPGMRSetPrecType(code, pretype, ier)}
\item \id{FSUNMassSPGMRSetPrecType(pretype, ier)}
\item \id{FSUNSPGMRSetMaxRS(code, maxrs, ier)}
\item \id{FSUNMassSPGMRSetMaxRS(maxrs, ier)}
\end{itemize}

%% This is a shared SUNDIALS TEX file with a description of the
%% spfgmr sunlinsol implementation
%%

The {\spfgmr} (Scaled, Preconditioned, Flexible, Generalized Minimum
Residual \cite{Saa:93}) implementation of the {\sunlinsol} module
provided with {\sundials}, {\sunlinsolspfgmr}, is an iterative linear
solver that is designed to be compatible with any {\nvector}
implementation (serial, threaded, parallel, and user-supplied) that
supports a minimal subset of operations (\id{N\_VClone},
\id{N\_VDotProd}, \id{N\_VScale}, \id{N\_VLinearSum}, \id{N\_VProd},
\id{N\_VConst}, \id{N\_VDiv}, and \id{N\_VDestroy}).  When using
Classical Gram-Schmidt, the optional function \id{N\_VDotProdMulti}
may be supplied for increased efficiency.  Unlike the other
Krylov iterative linear solvers supplied with {\sundials}, FGMRES is
specifically designed to work with a changing preconditioner
(e.g.~from an iterative method).

%---------------------------------------------------------------------------
\subsection{{\sunlinsolspfgmr} usage}\label{ss:sunlinsol_spfgmr_usage}

The header file to include when using this module
is \id{sunlinsol/sunlinsol\_spfgmr.h}. The {\sunlinsolspfgmr} module
is accessible from all {\sundials} solvers \textit{without}
linking to the \\ \noindent
\id{libsundials\_sunlinsolspfgmr} module library.

The module {\sunlinsolspfgmr} provides the following user-callable routines:
% --------------------------------------------------------------------
\ucfunction{SUNLinSol\_SPFGMR}
{
  LS = SUNLinSol\_SPFGMR(y, pretype, maxl);
}
{
  The function \ID{SUNLinSol\_SPFGMR} creates and allocates memory for
  a {\spfgmr} \id{SUNLinearSolver}.
}
{
  \begin{args}[pretype]
  \item[y] (\id{N\_Vector})
    a template for cloning vectors needed within the solver
  \item[pretype] (\id{int})
    flag indicating the desired type of preconditioning, allowed
    values are:
    \begin{itemize}
    \item \id{PREC\_NONE} (0)
    \item \id{PREC\_LEFT} (1)
    \item \id{PREC\_RIGHT} (2)
    \item \id{PREC\_BOTH} (3)
    \end{itemize}
    Any other integer input will result in the default (no
    preconditioning).
  \item[maxl] (\id{int})
    the number of Krylov basis vectors to use.  values $\le0$ will
    result in the default value (5).
  \end{args}
}
{
  This returns a \id{SUNLinearSolver} object.  If either \id{y} is
  incompatible then this routine will return \id{NULL}.
}
{
  This routine will perform consistency checks to ensure that it is
  called with a consistent {\nvector} implementation (i.e.~that it
  supplies the requisite vector operations).  If \id{y} is
  incompatible, then this routine will return \id{NULL}.

  We note that some {\sundials} solvers are designed to only work with
  left preconditioning ({\ida} and {\idas}). While it is possible to
  use a right-preconditioned {\sunlinsolspfgmr} object for these
  packages, this use mode is not supported and may result in inferior
  performance.
}
% --------------------------------------------------------------------
\ucfunction{SUNLinSol\_SPFGMRSetPrecType}
{
  retval = SUNLinSol\_SPFGMRSetPrecType(LS, pretype);
}
{
  The function \ID{SUNLinSol\_SPFGMRSetPrecType} updates the type of
  preconditioning to use in the {\sunlinsolspfgmr} object.
}
{
  \begin{args}[pretype]
  \item[LS] (\id{SUNLinearSolver})
    the {\sunlinsolspfgmr} object to update
  \item[pretype] (\id{int})
    flag indicating the desired type of preconditioning, allowed
    values match those discussed in \id{SUNLinSol\_SPFGMR}.
  \end{args}
}
{
  This routine will return with one of the error codes
  \id{SUNLS\_ILL\_INPUT} (illegal \id{pretype}), \id{SUNLS\_MEM\_NULL}
  (\id{S} is \id{NULL}) or \id{SUNLS\_SUCCESS}.
}
{
}
% --------------------------------------------------------------------
\ucfunction{SUNLinSol\_SPFGMRSetGSType}
{
  retval = SUNLinSol\_SPFGMRSetGSType(LS, gstype);
}
{
  The function \ID{SUNLinSol\_SPFGMRSetPrecType} sets the type of
  Gram-Schmidt orthogonalization to use in the {\sunlinsolspfgmr}
  object.
}
{
  \begin{args}[gstype]
  \item[LS] (\id{SUNLinearSolver})
    the {\sunlinsolspfgmr} object to update
  \item[gstype] (\id{int})
    flag indicating the desired orthogonalization algorithm; allowed
    values are:
    \begin{itemize}
    \item \id{MODIFIED\_GS} (1)
    \item \id{CLASSICAL\_GS} (2)
    \end{itemize}
    Any other integer input will result in a
    failure, returning error code \id{SUNLS\_ILL\_INPUT}.
  \end{args}
}
{
  This routine will return with one of the error codes
  \id{SUNLS\_ILL\_INPUT} (illegal \id{pretype}), \id{SUNLS\_MEM\_NULL}
  (\id{S} is \id{NULL}) or \id{SUNLS\_SUCCESS}.
}
{
}
% --------------------------------------------------------------------
\ucfunction{SUNLinSol\_SPFGMRSetMaxRestarts}
{
  retval = SUNLinSol\_SPFGMRSetMaxRestarts(LS, maxrs);
}
{
  The function \ID{SUNLinSol\_SPFGMRSetMaxRestarts} sets the number of
  GMRES restarts to allow in the {\sunlinsolspfgmr} object.
}
{
  \begin{args}[maxrs]
  \item[LS] (\id{SUNLinearSolver})
    the {\sunlinsolspfgmr} object to update
  \item[maxrs] (\id{int})
    integer indicating number of restarts to allow.  A negative input
    will result in the default of 0.
  \end{args}
}
{
  This routine will return with one of the error codes
  \id{SUNLS\_MEM\_NULL} (\id{S} is \id{NULL}) or \id{SUNLS\_SUCCESS}.
}
{
}
% --------------------------------------------------------------------
%%
For backwards compatibility, we also provide the wrapper functions,
each with identical input and output arguments to the routines that
they wrap:
\begin{itemize}

\item \ID{SUNSPFGMR}

  Wrapper function for \ID{SUNLinSol\_SPFGMR}

\item \ID{SUNSPFGMRSetPrecType}

  Wrapper function for \ID{SUNLinSol\_SPFGMRSetPrecType}

\item \ID{SUNSPFGMRSetGSType}

  Wrapper function for \ID{SUNLinSol\_SPFGMRSetGSType}

\item \ID{SUNSPFGMRSetMaxRestarts}

  Wrapper function for \ID{SUNLinSol\_SPFGMRSetMaxRestarts}

\end{itemize}
%%
%%------------------------------------
%%
For solvers that include a Fortran interface module, the
{\sunlinsolspfgmr} module also includes a Fortran-callable function
for creating a \id{SUNLinearSolver} object.
% --------------------------------------------------------------------
\ucfunction{FSUNSPFGMRINIT}
{
  FSUNSPFGMRINIT(code, pretype, maxl, ier)
}
{
  The function \ID{FSUNSPFGMRINIT} can be called for Fortran programs
  to create a {\sunlinsolspfgmr} object.
}
{
  \begin{args}[pretype]
  \item[code] (\id{int*})
    is an integer input specifying the solver id (1 for {\cvode}, 2
    for {\ida}, 3 for {\kinsol}, and 4 for {\arkode}).
  \item[pretype] (\id{int*})
    flag indicating desired preconditioning type
  \item[maxl] (\id{int*})
    flag indicating Krylov subspace size
  \end{args}
}
{
  \id{ier} is a return completion flag equal to \id{0} for a success
  return and \id{-1} otherwise. See printed message for details in case
  of failure.
}
{
  This routine must be called \emph{after} the {\nvector} object has
  been initialized.

  Allowable values for \id{pretype} and \id{maxl} are the same as for
  the C function \ID{SUNLinSol\_SPFGMR}.
}
% --------------------------------------------------------------------
Additionally, when using
{\arkode} with a non-identity mass matrix, the {\sunlinsolspfgmr} module
includes a Fortran-callable function for creating a
\id{SUNLinearSolver} mass matrix solver object.
% --------------------------------------------------------------------
\ucfunction{FSUNMASSSPFGMRINIT}
{
  FSUNMASSSPFGMRINIT(pretype, maxl, ier)
}
{
  The function \ID{FSUNMASSSPFGMRINIT} can be called for Fortran programs
  to create a {\sunlinsolspfgmr} object for mass matrix linear systems.
}
{
  \begin{args}[pretype]
  \item[pretype] (\id{int*})
    flag indicating desired preconditioning type
  \item[maxl] (\id{int*})
    flag indicating Krylov subspace size
  \end{args}
}
{
  \id{ier} is a \id{int} return completion flag equal to \id{0} for a success
  return and \id{-1} otherwise. See printed message for details in case
  of failure.
}
{
  This routine must be called \emph{after} the {\nvector} object has
  been initialized.

  Allowable values for \id{pretype} and \id{maxl} are the same as for
  the C function \ID{SUNLinSol\_SPFGMR}.
}
% --------------------------------------------------------------------
The \id{SUNLinSol\_SPFGMRSetPrecType}, \id{SUNLinSol\_SPFGMRSetGSType}
and \id{SUNLinSol\_SPFGMRSetMaxRestarts} routines also support Fortran
interfaces for the system and mass matrix solvers

% --------------------------------------------------------------------
\ucfunction{FSUNSPFGMRSETGSTYPE}
{
  FSUNSPFGMRSETGSTYPE(code, gstype, ier)
}
{
  The function \ID{FSUNSPFGMRSETGSTYPE} can be called for Fortran
  programs to change the Gram-Schmidt orthogonaliation algorithm.
}
{
  \begin{args}[gstype]
  \item[code] (\id{int*})
    is an integer input specifying the solver id (1 for {\cvode}, 2
    for {\ida}, 3 for {\kinsol}, and 4 for {\arkode}).
  \item[gstype] (\id{int*})
    flag indicating the desired orthogonalization algorithm.
  \end{args}
}
{
  \id{ier} is a \id{int} return completion flag equal to \id{0} for a success
  return and \id{-1} otherwise. See printed message for details in case
  of failure.
}
{
  See \id{SUNLinSol\_SPFGMRSetGSType} for complete further documentation of
  this routine.
}
% --------------------------------------------------------------------
\ucfunction{FSUNMASSSPFGMRSETGSTYPE}
{
  FSUNMASSSPFGMRSETGSTYPE(gstype, ier)
}
{
  The function \ID{FSUNMASSSPFGMRSETGSTYPE} can be called for Fortran
  programs to change the Gram-Schmidt orthogonaliation algorithm for
  mass matrix linear systems.
}
{
  The arguments are identical to \id{FSUNSPFGMRSETGSTYPE} above, except that
  \id{code} is not needed since mass matrix linear systems only arise
  in {\arkode}.
}
{
  \id{ier} is a \id{int} return completion flag equal to \id{0} for a success
  return and \id{-1} otherwise. See printed message for details in case
  of failure.
}
{
  See \id{SUNLinSol\_SPFGMRSetGSType} for complete further documentation of
  this routine.
}
% --------------------------------------------------------------------
\ucfunction{FSUNSPFGMRSETPRECTYPE}
{
  FSUNSPFGMRSETPRECTYPE(code, pretype, ier)
}
{
  The function \ID{FSUNSPFGMRSETPRECTYPE} can be called for Fortran
  programs to change the type of preconditioning to use.
}
{
  \begin{args}[pretype]
  \item[code] (\id{int*})
    is an integer input specifying the solver id (1 for {\cvode}, 2
    for {\ida}, 3 for {\kinsol}, and 4 for {\arkode}).
  \item[pretype] (\id{int*})
    flag indication the type of preconditioning to use.
  \end{args}
}
{
  \id{ier} is a \id{int} return completion flag equal to \id{0} for a success
  return and \id{-1} otherwise. See printed message for details in case
  of failure.
}
{
  See \id{SUNLinSol\_SPFGMRSetPrecType} for complete further documentation of
  this routine.
}
% --------------------------------------------------------------------
\ucfunction{FSUNMASSSPFGMRSETPRECTYPE}
{
  FSUNMASSSPFGMRSETPRECTYPE(pretype, ier)
}
{
  The function \ID{FSUNMASSSPFGMRSETPRECTYPE} can be called for Fortran
  programs to change the type of preconditioning for mass matrix
  linear systems.
}
{
  The arguments are identical to \id{FSUNSPFGMRSETPRECTYPE} above, except that
  \id{code} is not needed since mass matrix linear systems only arise
  in {\arkode}.
}
{
  \id{ier} is a \id{int} return completion flag equal to \id{0} for a success
  return and \id{-1} otherwise. See printed message for details in case
  of failure.
}
{
  See \id{SUNLinSol\_SPFGMRSetPrecType} for complete further documentation of
  this routine.
}
% --------------------------------------------------------------------
\ucfunction{FSUNSPFGMRSETMAXRS}
{
  FSUNSPFGMRSETMAXRS(code, maxrs, ier)
}
{
  The function \ID{FSUNSPFGMRSETMAXRS} can be called for Fortran programs
  to change the maximum number of restarts allowed for {\spfgmr}.
}
{
  \begin{args}[maxrs]
  \item[code] (\id{int*})
    is an integer input specifying the solver id (1 for {\cvode}, 2
    for {\ida}, 3 for {\kinsol}, and 4 for {\arkode}).
  \item[maxrs] (\id{int*})
    maximum allowed number of restarts.
  \end{args}
}
{
  \id{ier} is a \id{int} return completion flag equal to \id{0} for a success
  return and \id{-1} otherwise. See printed message for details in case
  of failure.
}
{
  See \id{SUNLinSol\_SPFGMRSetMaxRestarts} for complete further
  documentation of this routine.
}
% --------------------------------------------------------------------
\ucfunction{FSUNMASSSPFGMRSETMAXRS}
{
  FSUNMASSSPFGMRSETMAXRS(maxrs, ier)
}
{
  The function \ID{FSUNMASSSPFGMRSETMAXRS} can be called for Fortran
  programs to change the maximum number of restarts allowed for
  {\spfgmr} for mass matrix linear systems.
}
{
  The arguments are identical to \id{FSUNSPFGMRSETMAXRS} above, except that
  \id{code} is not needed since mass matrix linear systems only arise
  in {\arkode}.
}
{
  \id{ier} is a \id{int} return completion flag equal to \id{0} for a success
  return and \id{-1} otherwise. See printed message for details in case
  of failure.
}
{
  See \id{SUNLinSol\_SPFGMRSetMaxRestarts} for complete further
  documentation of this routine.
}
% --------------------------------------------------------------------


%---------------------------------------------------------------------------
\subsection{{\sunlinsolspfgmr} description}\label{ss:sunlinsol_spfgmr_description}


The {\sunlinsolspfgmr} module defines the {\em content} field of a
\id{SUNLinearSolver} to be the following structure:
%%
\begin{verbatim}
struct _SUNLinearSolverContent_SPFGMR {
  int maxl;
  int pretype;
  int gstype;
  int max_restarts;
  int numiters;
  realtype resnorm;
  long int last_flag;
  ATimesFn ATimes;
  void* ATData;
  PSetupFn Psetup;
  PSolveFn Psolve;
  void* PData;
  N_Vector s1;
  N_Vector s2;
  N_Vector *V;
  N_Vector *Z;
  realtype **Hes;
  realtype *givens;
  N_Vector xcor;
  realtype *yg;
  N_Vector vtemp;
};
\end{verbatim}
%%
These entries of the \emph{content} field contain the following
information:
\begin{description}
  \item[maxl] - number of FGMRES basis vectors to use (default is 5),
  \item[pretype] - flag for use of preconditioning (default is none),
  \item[gstype] - flag for type of Gram-Schmidt orthogonalization
    (default is modified Gram-Schmidt),
  \item[max\_restarts] - number of FGMRES restarts to allow
    (default is 0),
  \item[numiters] - number of iterations from the most-recent solve,
  \item[resnorm] - final linear residual norm from the most-recent solve,
  \item[last\_flag] - last error return flag from an internal function,
  \item[ATimes] - function pointer to perform $Av$ product,
  \item[ATData] - pointer to structure for \id{ATimes},
  \item[Psetup] - function pointer to preconditioner setup routine,
  \item[Psolve] - function pointer to preconditioner solve routine,
  \item[PData] - pointer to structure for \id{Psetup} and \id{Psolve},
  \item[s1, s2] - vector pointers for supplied scaling matrices
    (default is \id{NULL}),
  \item[V] - the array of Krylov basis vectors
    $v_1, \ldots, v_{\text{\id{maxl}}+1}$, stored in \id{V[0]},
    \ldots, \id{V[maxl]}. Each $v_i$ is a vector of type {\nvector}.,
  \item[Z] - the array of preconditioned Krylov basis vectors
    $z_1, \ldots, z_{\text{\id{maxl}}+1}$, stored in \id{Z[0]},
    \ldots, \id{Z[maxl]}. Each $z_i$ is a vector of type {\nvector}.,
  \item[Hes] - the $(\text{\id{maxl}}+1)\times\text{\id{maxl}}$
    Hessenberg matrix. It is stored row-wise so that the (i,j)th
    element is given by \id{Hes[i][j]}.,
  \item[givens] - a length \id{2*maxl} array which represents the
    Givens rotation matrices that arise in the FGMRES algorithm. These
    matrices are $F_0, F_1, \ldots, F_j$, where
    $F_i = \begin{bmatrix}
      1 &        &   &     &      &   &        &   \\
        & \ddots &   &     &      &   &        &   \\
        &        & 1 &     &      &   &        &   \\
        &        &   & c_i & -s_i &   &        &   \\
        &        &   & s_i &  c_i &   &        &   \\
        &        &   &     &      & 1 &        &   \\
        &        &   &     &      &   & \ddots &   \\
        &        &   &     &      &   &        & 1\end{bmatrix}$,
    are represented in the \id{givens} vector as \id{givens[0] =}
    $c_0$, \id{givens[1] = } $s_0$, \id{givens[2] = } $c_1$,
    \id{givens[3] = } $s_1$, \ldots \id{givens[2j] = } $c_j$,
    \id{givens[2j+1] = } $s_j$.,
  \item[xcor] - a vector which holds the scaled, preconditioned
    correction to the initial guess,
  \item[yg] - a length \id{(maxl+1)} array of \id{realtype} values
    used to hold ``short'' vectors (e.g. $y$ and $g$),
  \item[vtemp] - temporary vector storage.
\end{description}

This solver is constructed to perform the following operations:
\begin{itemize}
\item During construction, the \id{xcor} and \id{vtemp} arrays are
  cloned from a template {\nvector} that is input, and default solver
  parameters are set.
\item User-facing ``set'' routines may be called to modify default
  solver parameters.
\item Additional ``set'' routines are called by the {\sundials} solver
  that interfaces with {\sunlinsolspfgmr} to supply the
  \id{ATimes}, \id{PSetup}, and \id{Psolve} function pointers and
  \id{s1} and \id{s2} scaling vectors.
\item In the ``initialize'' call, the remaining solver data is
  allocated (\id{V}, \id{Hes}, \id{givens}, and \id{yg} )
\item In the ``setup'' call, any non-\id{NULL}
  \id{PSetup} function is called.  Typically, this is provided by
  the {\sundials} solver itself, that translates between the
  generic \id{PSetup} function and the
  solver-specific routine (solver-supplied or user-supplied).
\item In the ``solve'' call, the FGMRES iteration is performed.  This
  will include scaling, preconditioning, and restarts if those options
  have been supplied.
\end{itemize}

%%
%%----------------------------------------------
%%

\noindent The {\sunlinsolspfgmr} module defines implementations of all
``iterative'' linear solver operations listed in Sections
\ref{ss:sunlinsol_CoreFn}-\ref{ss:sunlinsol_GetFn}:
\begin{itemize}
\item \id{SUNLinSolGetType\_SPFGMR}
\item \id{SUNLinSolInitialize\_SPFGMR}
\item \id{SUNLinSolSetATimes\_SPFGMR}
\item \id{SUNLinSolSetPreconditioner\_SPFGMR}
\item \id{SUNLinSolSetScalingVectors\_SPFGMR}
\item \id{SUNLinSolSetup\_SPFGMR}
\item \id{SUNLinSolSolve\_SPFGMR}
\item \id{SUNLinSolNumIters\_SPFGMR}
\item \id{SUNLinSolResNorm\_SPFGMR}
\item \id{SUNLinSolResid\_SPFGMR}
\item \id{SUNLinSolLastFlag\_SPFGMR}
\item \id{SUNLinSolSpace\_SPFGMR}
\item \id{SUNLinSolFree\_SPFGMR}
\end{itemize}

%% This is a shared SUNDIALS TEX file with a description of the
%% spbcgs sunlinsol implementation
%%

The {\spbcg} (Scaled, Preconditioned, Bi-Conjugate Gradient,
Stabilized \cite{Van:92}) implementation of the {\sunlinsol} module 
provided with {\sundials}, {\sunlinsolspbcgs}, is an iterative linear
solver that is designed to be compatible with any {\nvector}
implementation (serial, threaded, parallel, and user-supplied) that
supports a minimal subset of operations (\id{N\_VClone}, 
\id{N\_VDotProd}, \id{N\_VScale}, \id{N\_VLinearSum}, \id{N\_VProd},
\id{N\_VDiv}, and \id{N\_VDestroy}).  Unlike the {\spgmr} and {\spfgmr}
algorithms, {\spbcg} requires a fixed amount of memory that does not
increase with the number of allowed iterations.

The {\sunlinsolspbcgs} module defines the {\em content} field of a
\id{SUNLinearSolver} to be the following structure:
%%
\begin{verbatim} 
struct _SUNLinearSolverContent_SPBCGS {
  int maxl;
  int pretype;
  int numiters;
  realtype resnorm;
  long int last_flag;
  ATimesFn ATimes;
  void* ATData;
  PSetupFn Psetup;
  PSolveFn Psolve;
  void* PData;
  N_Vector s1;
  N_Vector s2;
  N_Vector r;
  N_Vector r_star;
  N_Vector p;
  N_Vector q;
  N_Vector u;
  N_Vector Ap;
  N_Vector vtemp;
};
\end{verbatim}
%%
These entries of the \emph{content} field contain the following
information:
\begin{description}
  \item[maxl] - number of {\spbcg} iterations to allow (default is 5),
  \item[pretype] - flag for type of preconditioning to employ
    (default is none),
  \item[numiters] - number of iterations from the most-recent solve,
  \item[resnorm] - final linear residual norm from the most-recent solve,
  \item[last\_flag] - last error return flag from an internal function,
  \item[ATimes] - function pointer to perform $Av$ product,
  \item[ATData] - pointer to structure for \id{ATimes},
  \item[Psetup] - function pointer to preconditioner setup routine,
  \item[Psolve] - function pointer to preconditioner solve routine,
  \item[PData] - pointer to structure for \id{Psetup} and \id{Psolve},
  \item[s1, s2] - vector pointers for supplied scaling matrices
    (default is \id{NULL}),
  \item[r] - a {\nvector} which holds the current scaled,
    preconditioned linear system residual,
  \item[r\_star] - a {\nvector} which holds the initial scaled,
    preconditioned linear system residual,
  \item[p, q, u, Ap, vtemp] - {\nvector}s used for workspace by the
    {\spbcg} algorithm.
\end{description}

This solver is constructed to perform the following operations:
\begin{itemize}
\item During construction all {\nvector} solver data is allocated,
  with vectors cloned from a template {\nvector} that is input, and
  default solver parameters are set.
\item User-facing ``set'' routines may be called to modify default
  solver parameters.
\item Additional ``set'' routines are called by the {\sundials} solver
  that interfaces with {\sunlinsolspbcgs} to supply the 
  \id{ATimes}, \id{PSetup}, and \id{Psolve} function pointers and
  \id{s1} and \id{s2} scaling vectors.
\item In the ``initialize'' call, the solver parameters are checked
  for validity.
\item In the ``setup'' call, any non-\id{NULL} 
  \id{PSetup} function is called.  Typically, this is provided by
  the {\sundials} solver itself, that translates between the
  generic \id{PSetup} function and the
  solver-specific routine (solver-supplied or user-supplied).
\item In the ``solve'' call the {\spbcg} iteration is performed.  This
  will include scaling and preconditioning if those options have been
  supplied.
\end{itemize}

\noindent The header file to be included when using this module 
is \id{sunlinsol/sunlinsol\_spbcgs.h}. \\
%%
%%----------------------------------------------
%%
The {\sunlinsolspbcgs} module defines implementations of all
``iterative'' linear solver operations listed in Table
\ref{t:sunlinsolops}:
\begin{itemize}
\item \id{SUNLinSolGetType\_SPBCGS}
\item \id{SUNLinSolInitialize\_SPBCGS}
\item \id{SUNLinSolSetATimes\_SPBCGS}
\item \id{SUNLinSolSetPreconditioner\_SPBCGS}
\item \id{SUNLinSolSetScalingVectors\_SPBCGS}
\item \id{SUNLinSolSetup\_SPBCGS}
\item \id{SUNLinSolSolve\_SPBCGS}
\item \id{SUNLinSolNumIters\_SPBCGS}
\item \id{SUNLinSolResNorm\_SPBCGS}
\item \id{SUNLinSolResid\_SPBCGS}
\item \id{SUNLinSolLastFlag\_SPBCGS}
\item \id{SUNLinSolSpace\_SPBCGS}
\item \id{SUNLinSolFree\_SPBCGS}
\end{itemize}
The module {\sunlinsolspbcgs} provides the following additional
user-callable routines: 
%%
\begin{itemize}

%%--------------------------------------

\item \ID{SUNSPBCGS}

  This constructor function creates and allocates memory for a {\spbcg}
  \id{SUNLinearSolver}.  Its arguments are an {\nvector}, the desired
  type of preconditioning, and the number of linear iterations to allow.

  This routine will perform consistency checks to ensure that it is
  called with a consistent {\nvector} implementation (i.e.~that it
  supplies the requisite vector operations).  If \id{y} is
  incompatible, then this routine will return \id{NULL}.

  A \id{maxl} argument that is $\le0$ will result in the default
  value (5).

  Allowable inputs for \id{pretype} are \id{PREC\_NONE} (0),
  \id{PREC\_LEFT} (1), \id{PREC\_RIGHT} (2) and \id{PREC\_BOTH} (3);
  any other integer input will result in the default (no
  preconditioning).
  We note that some {\sundials} solvers are designed to only work
  with left preconditioning ({\ida} and {\idas}) and others with only
  right preconditioning ({\kinsol}). While it is possible to configure
  a {\sunlinsolspbcgs} object to use any of the preconditioning options
  with these solvers, this use mode is not supported and may result in
  inferior performance.

  \verb|SUNLinearSolver SUNSPBCGS(N_Vector y, int pretype, int maxl);|

%%--------------------------------------

\item \ID{SUNSPBCGSSetPrecType}

  This function updates the type of preconditioning to use.  Supported
  values are \id{PREC\_NONE} (0), \id{PREC\_LEFT} (1),
  \id{PREC\_RIGHT} (2), and \id{PREC\_BOTH} (3).  

  This routine will return with one of the error codes
  \id{SUNLS\_ILL\_INPUT} (illegal \id{pretype}), \id{SUNLS\_MEM\_NULL}
  (\id{S} is \id{NULL}), or \id{SUNLS\_SUCCESS}.
  
  \verb|int SUNSPBCGSSetPrecType(SUNLinearSolver S, int pretype);|

%%--------------------------------------

\item \ID{SUNSPBCGSSetMaxl}

  This function updates the number of linear solver iterations to allow.  

  A \id{maxl} argument that is $\le0$ will result in the default
  value (5).

  This routine will return with one of the error codes
  \id{SUNLS\_MEM\_NULL} (\id{S} is \id{NULL}) or \id{SUNLS\_SUCCESS}.
  
  \verb|int SUNSPBCGSSetMaxl(SUNLinearSolver S, int maxl);|

\end{itemize}
%%
%%------------------------------------
%%
For solvers that include a Fortran interface module, the
{\sunlinsolspbcgs} module also includes the Fortran-callable
function \id{FSUNSPBCGSInit(code, pretype, maxl, ier)} to initialize
this {\sunlinsolspbcgs} module for a given {\sundials} solver.
Here \id{code} is an integer input solver id (1 for {\cvode}, 2 for {\ida}, 3
for {\kinsol}, 4 for {\arkode}); \id{pretype} and \id{maxl} are the
same as for the C function \ID{SUNSPBCGS}; \id{ier} is an error return
flag equal to 0 for success and -1 for failure.  All of these input
arguments should be declared so as to match C type \id{int}.  This
routine must be called \emph{after} the {\nvector} object has been
initialized.  Additionally, when using {\arkode} with a non-identity
mass matrix, the Fortran-callable function 
\id{FSUNMassSPBCGSInit(pretype, maxl, ier)} initializes this
{\sunlinsolspbcgs} module for solving mass matrix linear systems.

The \id{SUNSPBCGSSetPrecType} and \id{SUNSPBCGSSetMaxl} routines also
support Fortran interfaces for the system and mass matrix solvers (all
arguments should be commensurate with a C \id{int}):
\begin{itemize}
\item \id{FSUNSPBCGSSetPrecType(code, pretype, ier)}
\item \id{FSUNMassSPBCGSSetPrecType(pretype, ier)}
\item \id{FSUNSPBCGSSetMaxl(code, maxl, ier)}
\item \id{FSUNMassSPBCGSSetMaxl(maxl, ier)}
\end{itemize}

%% This is a shared SUNDIALS TEX file with a description of the
%% sptfqmr sunlinsol implementation
%%
\section{The SUNLinearSolver\_SPTFQMR implementation}\label{ss:sunlinsol_sptfqmr}

The {\sptfqmr} (Scaled, Preconditioned, Transpose-Free Quasi-Minimum
Residual \cite{Fre:93}) implementation of the {\sunlinsol} module
provided with {\sundials}, {\sunlinsolsptfqmr}, is an iterative linear
solver that is designed to be compatible with any {\nvector}
implementation (serial, threaded, parallel, and user-supplied) that
supports a minimal subset of operations (\id{N\_VClone},
\id{N\_VDotProd}, \id{N\_VScale}, \id{N\_VLinearSum}, \id{N\_VProd},
\id{N\_VConst}, \id{N\_VDiv}, and \id{N\_VDestroy}).  Unlike the
{\spgmr} and {\spfgmr} algorithms, {\sptfqmr} requires a fixed amount of
memory that does not increase with the number of allowed iterations.

%---------------------------------------------------------------------------
\subsection{{\sunlinsolsptfqmr} usage}\label{ss:sunlinsol_sptfqmr_usage}

The header file to include when using this module
is \id{sunlinsol/sunlinsol\_sptfqmr.h}. The {\sunlinsolsptfqmr} module
is accessible from all {\sundials} solvers \textit{without}
linking to the \\ \noindent
\id{libsundials\_sunlinsolsptfqmr} module library.


The module {\sunlinsolsptfqmr} provides the following
user-callable routines:
% --------------------------------------------------------------------
\ucfunction{SUNLinSol\_SPTFQMR}
{
  LS = SUNLinSol\_SPTFQMR(y, pretype, maxl);
}
{
  The function \ID{SUNLinSol\_SPTFQMR} creates and allocates memory for
  a {\sptfqmr} \\ \noindent \id{SUNLinearSolver}.
}
{
  \begin{args}[pretype]
  \item[y] (\id{N\_Vector})
    a template for cloning vectors needed within the solver
  \item[pretype] (\id{int})
    flag indicating the desired type of preconditioning, allowed
    values are:
    \begin{itemize}
    \item \id{PREC\_NONE} (0)
    \item \id{PREC\_LEFT} (1)
    \item \id{PREC\_RIGHT} (2)
    \item \id{PREC\_BOTH} (3)
    \end{itemize}
    Any other integer input will result in the default (no
    preconditioning).
  \item[maxl] (\id{int})
    the number of linear iterations to allow; values $\le0$ will
    result in the default value (5).
  \end{args}
}
{
  This returns a \id{SUNLinearSolver} object.  If either \id{y} is
  incompatible then this routine will return \id{NULL}.
}
{
  This routine will perform consistency checks to ensure that it is
  called with a consistent {\nvector} implementation (i.e.~that it
  supplies the requisite vector operations).  If \id{y} is
  incompatible, then this routine will return \id{NULL}.

  We note that some {\sundials} solvers are designed to only work
  with left preconditioning ({\ida} and {\idas}) and others with only
  right preconditioning ({\kinsol}). While it is possible to configure
  a {\sunlinsolsptfqmr} object to use any of the preconditioning options
  with these solvers, this use mode is not supported and may result in
  inferior performance.
}
% --------------------------------------------------------------------
\ucfunction{SUNLinSol\_SPTFQMRSetPrecType}
{
  retval = SUNLinSol\_SPTFQMRSetPrecType(LS, pretype);
}
{
  The function \ID{SUNLinSol\_SPTFQMRSetPrecType} updates the type of
  preconditioning to use in the {\sunlinsolsptfqmr} object.
}
{
  \begin{args}[pretype]
  \item[LS] (\id{SUNLinearSolver})
    the {\sunlinsolsptfqmr} object to update
  \item[pretype] (\id{int})
    flag indicating the desired type of preconditioning, allowed
    values match those discussed in \id{SUNLinSol\_SPTFQMR}.
  \end{args}
}
{
  This routine will return with one of the error codes
  \id{SUNLS\_ILL\_INPUT} (illegal \id{pretype}), \id{SUNLS\_MEM\_NULL}
  (\id{S} is \id{NULL}) or \id{SUNLS\_SUCCESS}.
}
{
}
% --------------------------------------------------------------------
\ucfunction{SUNLinSol\_SPTFQMRSetMaxl}
{
  retval = SUNLinSol\_SPTFQMRSetMaxl(LS, maxl);
}
{
  The function \ID{SUNLinSol\_SPTFQMRSetMaxl} updates the number of
  linear solver iterations to allow.
}
{
  \begin{args}[maxl]
  \item[LS] (\id{SUNLinearSolver})
    the {\sunlinsolsptfqmr} object to update
  \item[maxl] (\id{int})
    flag indicating the number of iterations to allow; values $\le0$
    will result in the default value (5)
  \end{args}
}
{
  This routine will return with one of the error codes
  \id{SUNLS\_MEM\_NULL} (\id{S} is \id{NULL}) or \id{SUNLS\_SUCCESS}.
}
{
}
% --------------------------------------------------------------------
%%
For backwards compatibility, we also provide the wrapper functions,
each with identical input and output arguments to the routines that
they wrap:
\begin{itemize}

\item \ID{SUNSPTFQMR}

  Wrapper function for \ID{SUNLinSol\_SPTFQMR}

\item \ID{SUNSPTFQMRSetPrecType}

  Wrapper function for \ID{SUNLinSol\_SPTFQMRSetPrecType}

\item \ID{SUNSPTFQMRSetMaxl}

  Wrapper function for \ID{SUNLinSol\_SPTFQMRSetMaxl}

\end{itemize}
%%
%%------------------------------------
%%
For solvers that include a Fortran interface module, the
{\sunlinsolsptfqmr} module also includes a Fortran-callable function
for creating a \id{SUNLinearSolver} object.
% --------------------------------------------------------------------
\ucfunction{FSUNSPTFQMRINIT}
{
  FSUNSPTFQMRINIT(code, pretype, maxl, ier)
}
{
  The function \ID{FSUNSPTFQMRINIT} can be called for Fortran programs
  to create a {\sunlinsolsptfqmr} object.
}
{
  \begin{args}[pretype]
  \item[code] (\id{int*})
    is an integer input specifying the solver id (1 for {\cvode}, 2
    for {\ida}, 3 for {\kinsol}, and 4 for {\arkode}).
  \item[pretype] (\id{int*})
    flag indicating desired preconditioning type
  \item[maxl] (\id{int*})
    flag indicating number of iterations to allow
  \end{args}
}
{
  \id{ier} is a return completion flag equal to \id{0} for a success
  return and \id{-1} otherwise. See printed message for details in case
  of failure.
}
{
  This routine must be called \emph{after} the {\nvector} object has
  been initialized.

  Allowable values for \id{pretype} and \id{maxl} are the same as for
  the {\CC} function \\ \noindent \ID{SUNLinSol\_SPTFQMR}.
}
% --------------------------------------------------------------------
Additionally, when using
{\arkode} with a non-identity mass matrix, the {\sunlinsolsptfqmr} module
includes a Fortran-callable function for creating a
\id{SUNLinearSolver} mass matrix solver object.
% --------------------------------------------------------------------
\ucfunction{FSUNMASSSPTFQMRINIT}
{
  FSUNMASSSPTFQMRINIT(pretype, maxl, ier)
}
{
  The function \ID{FSUNMASSSPTFQMRINIT} can be called for Fortran programs
  to create a {\sunlinsolsptfqmr} object for mass matrix linear systems.
}
{
  \begin{args}[pretype]
  \item[pretype] (\id{int*})
    flag indicating desired preconditioning type
  \item[maxl] (\id{int*})
    flag indicating number of iterations to allow
  \end{args}
}
{
  \id{ier} is a \id{int} return completion flag equal to \id{0} for a success
  return and \id{-1} otherwise. See printed message for details in case
  of failure.
}
{
  This routine must be called \emph{after} the {\nvector} object has
  been initialized.

  Allowable values for \id{pretype} and \id{maxl} are the same as for
  the {\CC} function \\ \noindent \ID{SUNLinSol\_SPTFQMR}.
}
% --------------------------------------------------------------------
The \id{SUNLinSol\_SPTFQMRSetPrecType} and
\id{SUNLinSol\_SPTFQMRSetMaxl} routines also support Fortran
interfaces for the system and mass matrix solvers.


% --------------------------------------------------------------------
\ucfunction{FSUNSPTFQMRSETPRECTYPE}
{
  FSUNSPTFQMRSETPRECTYPE(code, pretype, ier)
}
{
  The function \ID{FSUNSPTFQMRSETPRECTYPE} can be called for Fortran
  programs to change the type of preconditioning to use.
}
{
  \begin{args}[pretype]
  \item[code] (\id{int*})
    is an integer input specifying the solver id (1 for {\cvode}, 2
    for {\ida}, 3 for {\kinsol}, and 4 for {\arkode}).
  \item[pretype] (\id{int*})
    flag indication the type of preconditioning to use.
  \end{args}
}
{
  \id{ier} is a \id{int} return completion flag equal to \id{0} for a success
  return and \id{-1} otherwise. See printed message for details in case
  of failure.
}
{
  See \id{SUNLinSol\_SPTFQMRSetPrecType} for complete further documentation of
  this routine.
}
% --------------------------------------------------------------------
\ucfunction{FSUNMASSSPTFQMRSETPRECTYPE}
{
  FSUNMASSSPTFQMRSETPRECTYPE(pretype, ier)
}
{
  The function \ID{FSUNMASSSPTFQMRSETPRECTYPE} can be called for Fortran
  programs to change the type of preconditioning for mass matrix
  linear systems.
}
{
  The arguments are identical to \id{FSUNSPTFQMRSETPRECTYPE} above,
  except that \id{code} is not needed since mass matrix linear systems
  only arise in {\arkode}.
}
{
  \id{ier} is a \id{int} return completion flag equal to \id{0} for a success
  return and \id{-1} otherwise. See printed message for details in case
  of failure.
}
{
  See \id{SUNLinSol\_SPTFQMRSetPrecType} for complete further documentation of
  this routine.
}
% --------------------------------------------------------------------
\ucfunction{FSUNSPTFQMRSETMAXL}
{
  FSUNSPTFQMRSETMAXL(code, maxl, ier)
}
{
  The function \ID{FSUNSPTFQMRSETMAXL} can be called for Fortran
  programs to change the maximum number of iterations to allow.
}
{
  \begin{args}[maxl]
  \item[code] (\id{int*})
    is an integer input specifying the solver id (1 for {\cvode}, 2
    for {\ida}, 3 for {\kinsol}, and 4 for {\arkode}).
  \item[maxl] (\id{int*})
    the number of iterations to allow
  \end{args}
}
{
  \id{ier} is a \id{int} return completion flag equal to \id{0} for a success
  return and \id{-1} otherwise. See printed message for details in case
  of failure.
}
{
  See \id{SUNLinSol\_SPTFQMRSetMaxl} for complete further documentation of
  this routine.
}
% --------------------------------------------------------------------
\ucfunction{FSUNMASSSPTFQMRSETMAXL}
{
  FSUNMASSSPTFQMRSETMAXL(maxl, ier)
}
{
  The function \ID{FSUNMASSSPTFQMRSETMAXL} can be called for Fortran
  programs to change the type of preconditioning for mass matrix
  linear systems.
}
{
  The arguments are identical to \id{FSUNSPTFQMRSETMAXL} above, except that
  \id{code} is not needed since mass matrix linear systems only arise
  in {\arkode}.
}
{
  \id{ier} is a \id{int} return completion flag equal to \id{0} for a success
  return and \id{-1} otherwise. See printed message for details in case
  of failure.
}
{
  See \id{SUNLinSol\_SPTFQMRSetMaxl} for complete further documentation of
  this routine.
}
% --------------------------------------------------------------------

%---------------------------------------------------------------------------
\subsection{{\sunlinsolsptfqmr} description}\label{ss:sunlinsol_sptfqmr_description}


The {\sunlinsolsptfqmr} module defines the {\em content} field of a
\id{SUNLinearSolver} to be the following structure:
%%
\begin{verbatim}
struct _SUNLinearSolverContent_SPTFQMR {
  int maxl;
  int pretype;
  int numiters;
  realtype resnorm;
  long int last_flag;
  ATimesFn ATimes;
  void* ATData;
  PSetupFn Psetup;
  PSolveFn Psolve;
  void* PData;
  N_Vector s1;
  N_Vector s2;
  N_Vector r_star;
  N_Vector q;
  N_Vector d;
  N_Vector v;
  N_Vector p;
  N_Vector *r;
  N_Vector u;
  N_Vector vtemp1;
  N_Vector vtemp2;
  N_Vector vtemp3;
};
\end{verbatim}
%%
These entries of the \emph{content} field contain the following
information:
\begin{description}
  \item[maxl] - number of TFQMR iterations to allow (default is 5),
  \item[pretype] - flag for type of preconditioning to employ
    (default is none),
  \item[numiters] - number of iterations from the most-recent solve,
  \item[resnorm] - final linear residual norm from the most-recent solve,
  \item[last\_flag] - last error return flag from an internal function,
  \item[ATimes] - function pointer to perform $Av$ product,
  \item[ATData] - pointer to structure for \id{ATimes},
  \item[Psetup] - function pointer to preconditioner setup routine,
  \item[Psolve] - function pointer to preconditioner solve routine,
  \item[PData] - pointer to structure for \id{Psetup} and \id{Psolve},
  \item[s1, s2] - vector pointers for supplied scaling matrices
    (default is \id{NULL}),
  \item[r\_star] - a {\nvector} which holds the initial scaled,
    preconditioned linear system residual,
  \item[q, d, v, p, u] - {\nvector}s used for workspace by the SPTFQMR
    algorithm,
  \item [r] - array of two {\nvector}s used for workspace within the
    SPTFQMR algorithm,
  \item[vtemp1, vtemp2, vtemp3] - temporary vector storage.
\end{description}

This solver is constructed to perform the following operations:
\begin{itemize}
\item During construction all {\nvector} solver data is allocated,
  with vectors cloned from a template {\nvector} that is input, and
  default solver parameters are set.
\item User-facing ``set'' routines may be called to modify default
  solver parameters.
\item Additional ``set'' routines are called by the {\sundials} solver
  that interfaces with \\ \noindent {\sunlinsolsptfqmr} to supply the
  \id{ATimes}, \id{PSetup}, and \id{Psolve} function pointers and
  \id{s1} and \id{s2} scaling vectors.
\item In the ``initialize'' call, the solver parameters are checked
  for validity.
\item In the ``setup'' call, any non-\id{NULL}
  \id{PSetup} function is called.  Typically, this is provided by
  the {\sundials} solver itself, that translates between the
  generic \id{PSetup} function and the
  solver-specific routine (solver-supplied or user-supplied).
\item In the ``solve'' call the TFQMR iteration is performed.  This
  will include scaling and preconditioning if those options have been
  supplied.
\end{itemize}

%%
%%----------------------------------------------
%%

\noindent The {\sunlinsolsptfqmr} module defines implementations of all
``iterative'' linear solver operations listed in Sections
\ref{ss:sunlinsol_CoreFn}-\ref{ss:sunlinsol_GetFn}:
\begin{itemize}
\item \id{SUNLinSolGetType\_SPTFQMR}
\item \id{SUNLinSolInitialize\_SPTFQMR}
\item \id{SUNLinSolSetATimes\_SPTFQMR}
\item \id{SUNLinSolSetPreconditioner\_SPTFQMR}
\item \id{SUNLinSolSetScalingVectors\_SPTFQMR}
\item \id{SUNLinSolSetup\_SPTFQMR}
\item \id{SUNLinSolSolve\_SPTFQMR}
\item \id{SUNLinSolNumIters\_SPTFQMR}
\item \id{SUNLinSolResNorm\_SPTFQMR}
\item \id{SUNLinSolResid\_SPTFQMR}
\item \id{SUNLinSolLastFlag\_SPTFQMR}
\item \id{SUNLinSolSpace\_SPTFQMR}
\item \id{SUNLinSolFree\_SPTFQMR}
\end{itemize}

%% This is a shared SUNDIALS TEX file with a description of the
%% pcg sunlinsol implementation
%%

The {\pcg} (Preconditioned Conjugate Gradient \cite{HeSt:52})
implementation of the {\sunlinsol} module provided with {\sundials},
{\sunlinsolpcg}, is an iterative linear solver that is designed to be
compatible with any {\nvector} implementation (serial, threaded,
parallel, user-supplied) that supports a minimal subset of operations
(\id{N\_VClone}, \id{N\_VDotProd}, \id{N\_VScale}, \id{N\_VLinearSum},
\id{N\_VProd} and \id{N\_VDestroy}).  Unlike the {\spgmr} and {\spfgmr}
algorithms, {\pcg} requires a fixed amount of memory that does not
scale with the number of allowed iterations.

Unlike all of the other iterative linear solvers supplied with
{\sundials}, {\pcg} should only be used on \emph{symmetric} linear
systems (e.g.~mass matrix linear systems encountered in
{\arkode}). As a result, the explanation of the role of scaling and
preconditioning matrices given in general must be modified in this
scenario.  The {\pcg} algorithm solves a linear system $Ax = b$ where  
$A$ is a symmetric ($A^T=A$), real-valued matrix.  Preconditioning is
allowed, and is applied in a symmetric fashion on both the right and
left.  Scaling is also allowed and is applied symmetrically.  We
denote the preconditioner and scaling matrices as follows:
\begin{itemize}
\item $P$ is the preconditioner (assumed symmetric),
\item $S$ is a diagonal matrix of scale factors.
\end{itemize}
The matrices $A$ and $P$ are not required explicitly; only routines
that provide $A$ and $P^{-1}$ as operators are required.  The diagonal
of the matrix $S$ is held in a single {\nvector}, supplied by the user
of this module.

In this notation, {\pcg} applies the underlying CG algorithm to the
equivalent transformed system 
\begin{equation}
  \label{eq:transformed_linear_systemPCG}
  \tilde{A} \tilde{x} = \tilde{b}
\end{equation}
where
\begin{align}
  \notag
  \tilde{A} &= S P^{-1} A P^{-1} S,\\
  \label{eq:transformed_linear_system_componentsPCG}
  \tilde{b} &= S P^{-1} b,\\
  \notag
  \tilde{x} &= S^{-1} P x.
\end{align} 
The scaling matrix must be chosen so that the vectors $SP^{-1}b$ and
$S^{-1}Px$ have dimensionless components.

The stopping test for the PCG iterations is on the L2 norm of the
scaled preconditioned residual:
\begin{align*}
  &\| \tilde{b} - \tilde{A} \tilde{x} \|_2  <  \delta\\
  \Leftrightarrow\quad &\\
  &\| S P^{-1} b - S P^{-1} A x \|_2  <  \delta\\
  \Leftrightarrow\quad &\\
  &\| P^{-1} b - P^{-1} A x \|_S  <  \delta
\end{align*}
where $\| v \|_S = \sqrt{v^T S^T S v}$, with an input tolerance $\delta$.

The {\sunlinsolpcg} module defines the {\em content} field of a
\id{SUNLinearSolver} to be the following structure:
%%
\begin{verbatim} 
struct _SUNLinearSolverContent_PCG {
  int maxl;
  int pretype;
  int numiters;
  realtype resnorm;
  long int last_flag;
  ATimesFn ATimes;
  void* ATData;
  PSetupFn Psetup;
  PSolveFn Psolve;
  void* PData;
  N_Vector s;
  N_Vector r;
  N_Vector p;
  N_Vector z;
  N_Vector Ap;
};
\end{verbatim}
%%
These entries of the \emph{content} field contain the following
information:
\begin{description}
  \item[maxl] - number of {\pcg} iterations to allow (default is 5)
  \item[pretype] - flag for use of preconditioning (default is none)
  \item[numiters] - number of iterations from most-recent solve
  \item[resnorm] - final linear residual norm from most-recent solve
  \item[last\_flag] - last error return flag from internal function
  \item[ATimes] - function pointer to perform $Av$ product
  \item[ATData] - pointer to structure for \id{ATimes}
  \item[Psetup] - function pointer to preconditioner setup routine
  \item[Psolve] - function pointer to preconditioner solve routine
  \item[PData] - pointer to structure for \id{Psetup}, \id{Psolve}
  \item[s] - vector pointer for supplied scaling matrix
    (default is \id{NULL})
  \item[r] - a {\nvector} which holds the preconditioned linear system
    residual
  \item[p, z, Ap] - {\nvector}s used for workspace by the
    {\pcg} algorithm. 
\end{description}

This solver is constructed to perform the following operations:
\begin{itemize}
\item During construction all {\nvector} solver data is allocated,
  with vectors cloned from a template {\nvector} that is input, and
  default solver parameters are set.
\item User-facing ``set'' routines may be called to modify default
  solver parameters.
\item Additional ``set'' routines are called by the {\sundials} solver
  that interfaces with {\sunlinsolpcg} to supply the 
  \id{ATimes}, \id{PSetup} and \id{Psolve} function pointers and
  \id{s} scaling vector.
\item In the ``initialize'' call, the solver parameters are checked
  for validity.
\item In the ``setup'' call, any non-\id{NULL} \id{PSetup} function is
  called.  Typically, this is provided by the {\sundials} solver
  itself, that translates between the generic \id{PSetup} function and
  the solver-specific routine (solver-supplied or user-supplied).
\item In the ``solve'' call the {\pcg} iteration is performed.  This
  will include scaling and preconditioning if those options have been
  supplied.
\end{itemize}

\noindent The header file to be included when using this module 
is \id{sunlinsol/sunlinsol\_pcg.h}. \\
%%
%%----------------------------------------------
%%
The {\sunlinsolpcg} module defines implementations of all
``iterative'' linear solver operations listed in Table
\ref{t:sunlinsolops}:
\begin{itemize}
\item \id{SUNLinSolGetType\_PCG}
\item \id{SUNLinSolInitialize\_PCG}
\item \id{SUNLinSolSetATimes\_PCG}
\item \id{SUNLinSolSetPreconditioner\_PCG}
\item \id{SUNLinSolSetScalingVectors\_PCG} -- since {\pcg} only
  supports symmetric scaling, the second {\nvector} argument to this
  function is ignored
\item \id{SUNLinSolSetup\_PCG}
\item \id{SUNLinSolSolve\_PCG}
\item \id{SUNLinSolNumIters\_PCG}
\item \id{SUNLinSolResNorm\_PCG}
\item \id{SUNLinSolResid\_PCG}
\item \id{SUNLinSolLastFlag\_PCG}
\item \id{SUNLinSolSpace\_PCG}
\item \id{SUNLinSolFree\_PCG}
\end{itemize}
The module {\sunlinsolpcg} provides the following additional
user-callable routines: 
%%
\begin{itemize}

%%--------------------------------------

\item \ID{SUNPCG}

  This function creates and allocates memory for a {\pcg}
  \id{SUNLinearSolver}.  Its arguments are an {\nvector}, a flag
  indicating to use preconditioning, and the number of linear
  iterations to allow. 

  This routine will perform consistency checks to ensure that it is
  called with a consistent {\nvector} implementation (i.e.~that it
  supplies the requisite vector operations).  If \id{y} is
  incompatible then this routine will return \id{NULL}.

  A \id{maxl} argument that is $\le0$ will result in the default
  value (5).

  Since the {\pcg} algorithm is designed to only support symmetric
  preconditioning, then any of the \id{pretype} inputs \id{PREC\_LEFT}
  (1), \id{PREC\_RIGHT} (2), or \id{PREC\_BOTH} (3) will result in use
  of the symmetric preconditioner;  any other integer input will
  result in the default (no preconditioning).

  \verb|SUNLinearSolver SUNPCG(N_Vector y, int pretype, int maxl);|

%%--------------------------------------

\item \ID{SUNPCGSetPrecType}

  This function updates the flag indicating use of preconditioning.
  As above, any one of the input values, \id{PREC\_LEFT} (1),
  \id{PREC\_RIGHT} (2) and \id{PREC\_BOTH} (3) will enable
  preconditioning; \id{PREC\_NONE} (0) disables preconditioning.

  This routine will return with one of the error codes
  \id{SUNLS\_ILL\_INPUT} (illegal \id{pretype}), \id{SUNLS\_MEM\_NULL}
  (\id{S} is \id{NULL}) or \id{SUNLS\_SUCCESS}.
  
  \verb|int SUNPCGSetPrecType(SUNLinearSolver S, int pretype);|

%%--------------------------------------

\item \ID{SUNPCGSetMaxl}

  This function updates the number of linear solver iterations to
  allow. 

  A \id{maxl} argument that is $\le0$ will result in the default
  value (5).

  This routine will return with one of the error codes
  \id{SUNLS\_MEM\_NULL} (\id{S} is \id{NULL}) or \id{SUNLS\_SUCCESS}.
  
  \verb|int SUNPCGSetMaxl(SUNLinearSolver S, int maxl);|

\end{itemize}
%%
%%------------------------------------
%%
For solvers that include a Fortran interface module, the
{\sunlinsolpcg} module also includes the Fortran-callable
function \id{FSUNPCGInit(code, pretype, maxl, ier)} to initialize
this {\sunlinsolpcg} module for a given {\sundials} solver.
Here \id{code} is an input solver id (1 for {\cvode}, 2 for {\ida}, 3
for {\kinsol}, 4 for {\arkode}); \id{pretype} and \id{maxl} are the
same as for the C function \ID{SUNPCG}; \id{ier} is an error return
flag equal 0 for success and -1 for failure.  All of these input
arguments should be declared so as to match C type \id{int}).  This
routine must be called \emph{after} the {\nvector} object has been
initialized.  Additionally, when using {\arkode} with non-identity
mass matrix, the Fortran-callable function 
\id{FSUNMassPCGInit(pretype, maxl, ier)} initializes this
{\sunlinsolpcg} module for solving mass matrix linear systems.

The \id{SUNPCGSetPrecType} and \id{SUNPCGSetMaxl} routines also
support Fortran interfaces for the system and mass matrix solvers:
\begin{itemize}
\item \id{FSUNPCGSetPrecType(code, pretype, ier)} -- all arguments
  should be commensurate with a C \id{int}
\item \id{FSUNMassPCGSetPrecType(pretype, ier)}
\item \id{FSUNPCGSetMaxl(code, maxl, ier)} -- all arguments
  should be commensurate with a C \id{int}
\item \id{FSUNMassPCGSetMaxl(maxl, ier)}
\end{itemize}

\section{SUNLinearSolver Examples}\label{ss:sunlinsol_examples}

There are \id{SUNLinearSolver} examples that may be installed for each
implementation; these make use of the functions in \id{test\_sunlinsol.c}.
These example functions show simple usage of the \id{SUNLinearSolver} family
of functions.  The inputs to the examples depend on the linear solver type,
and are output to \texttt{stdout} if the example is run without the
appropriate number of command-line arguments.

\noindent The following is a list of the example functions in \id{test\_sunlinsol.c}:
\begin{itemize}
\item \id{Test\_SUNLinSolGetType}: Verifies the returned solver type against
  the value that should be returned.
\item \id{Test\_SUNLinSolInitialize}: Verifies that \id{SUNLinSolInitialize}
  can be called and returns successfully.
\item \id{Test\_SUNLinSolSetup}: Verifies that \id{SUNLinSolSetup} can
  be called and returns successfully.
\item \id{Test\_SUNLinSolSolve}: Given a {\sunmatrix} object $A$,
  {\nvector} objects $x$ and $b$ (where $Ax=b$) and a desired solution
  tolerance \texttt{tol}, this routine clones $x$ into a new vector $y$,
  calls \\ \noindent
  \id{SUNLinSolSolve} to fill $y$ as the solution to $Ay=b$ (to
  the input tolerance), verifies that each entry in $x$ and $y$
  match to within \texttt{10*tol}, and overwrites $x$ with $y$ prior
  to returning (in case the calling routine would like to investigate
  further).
\item \id{Test\_SUNLinSolSetATimes} (iterative solvers only): Verifies that
  \id{SUNLinSolSetATimes} can be called and returns successfully.
\item \id{Test\_SUNLinSolSetPreconditioner} (iterative solvers only):
  Verifies that \\ \noindent
  \id{SUNLinSolSetPreconditioner} can be called and
  returns successfully.
\item \id{Test\_SUNLinSolSetScalingVectors} (iterative solvers only):
  Verifies that \\ \noindent
  \id{SUNLinSolSetScalingVectors} can be called and
  returns successfully.
\item \id{Test\_SUNLinSolLastFlag}: Verifies that \id{SUNLinSolLastFlag} can
  be called, and outputs the result to \texttt{stdout}.
\item \id{Test\_SUNLinSolNumIters} (iterative solvers only): Verifies that
  \id{SUNLinSolNumIters} can be called, and outputs the result to
  \texttt{stdout}.
\item \id{Test\_SUNLinSolResNorm} (iterative solvers only): Verifies that
  \id{SUNLinSolResNorm} can be called, and that the result is
  non-negative.
\item \id{Test\_SUNLinSolResid} (iterative solvers only): Verifies that
  \id{SUNLinSolResid} can be called.
\item \id{Test\_SUNLinSolSpace} verifies that \id{SUNLinSolSpace} can be
  called, and outputs the results to \texttt{stdout}.
\end{itemize}
We'll note that these tests should be performed in a particular
order.  For either direct or iterative linear
solvers, \id{Test\_SUNLinSolInitialize} must be called
before \id{Test\_SUNLinSolSetup}, which must be called
before \id{Test\_SUNLinSolSolve}.  Additionally, for iterative linear
solvers \\ \noindent
\id{Test\_SUNLinSolSetATimes}, \id{Test\_SUNLinSolSetPreconditioner}
and \\ \noindent
\id{Test\_SUNLinSolSetScalingVectors} should be called
before \id{Test\_SUNLinSolInitialize};
similarly \id{Test\_SUNLinSolNumIters}, \id{Test\_SUNLinSolResNorm}
and \id{Test\_SUNLinSolResid} should be called
after \id{Test\_SUNLinSolSolve}.  These are called in the appropriate
order in all of the example problems.

