%===================================================================================
\section{Introduction}\label{s:ex_intro}
%===================================================================================

This report is intended to serve as a companion document to the User
Documentation of {\ida} \cite{ida2.4.0_ug}.  It provides details, with
listings, on the example programs supplied with the {\ida} distribution
package.

The {\ida} distribution contains examples of four types: serial
{\C} examples, parallel {\C} examples, and serial and parallel {\F}
examples.  The following lists summarize all of these examples.

The {\ida} distribution contains, in the \id{sundials/examples/ida/serial}
directory, the following four serial examples (using the {\nvecs} module):
%%
\begin{itemize}

\item \id{idadenx}
  solves the Robertson chemical kinetics problem~\cite{Rob:66}, which consists
  of two differential equations and one algebraic constraint.  It also uses
  the rootfinding feature of {\ida}.

  The problem is solved with the {\idadense} linear solver using
  a user-supplied Jacobian.

\item \id{idabanx1}
  solves a 2-D heat equation, semidiscretized to a DAE on the unit square.

  This program solves the problem with the {\idaband} linear solver and
  the default difference-quotient Jacobian approximation. For purposes of
  illustration, \id{IDACalcIC} is called to compute correct values at the
  boundary, given incorrect values as input initial guesses. The constraint
  $u > 0.0$ is imposed for all components.

\item \id{idakryx}
  solves the same 2-D heat equation problem as \id{idabanx1}, with the Krylov
  linear solver {\idaspgmr}. The preconditioner uses only the diagonal elements
  of the Jacobian.

\item \id{idabanx2}
  solves a system of PDEs modelling a food web problem, with predator-prey
  interaction and diffusion, on the unit square in 2-D.

  The PDEs are discretized in space to a system of DAEs which are solved
  using the {\idaband} linear solver with the default difference-quotient 
  Jacobian approximation.

\item \id{idakrydem\_lin}
  solves the same problem as \id{idakryx}, with three Krylov linear solvers
  {\idaspgmr}, {\idaspbcg}, and {\idasptfqmr}.  The preconditioner uses only
  the diagonal elements of the Jacobian.

\end{itemize}

In the \id{sundials/examples/ida/parallel} directory, the {\ida} 
distribution contains the following four parallel examples 
(using the {\nvecp} module):
%%
\begin{itemize}

\item \id{idakryx1\_p}
  solves the same 2-D heat equation problem as \id{idakryx}, with {\idaspgmr}
  in parallel, and with a user-supplied diagonal preconditioner,
  
\item \id{idakryx1\_bbd\_p}
  solves the same problem as \id{idakryx1\_p}.

  This program solves the problem in parallel, using the Krylov linear solver
  {\idaspgmr} and the band-block diagonal preconditioner {\idabbdpre} with
  half-bandwidths equal to $1$.

\item \id{idakryx2\_p}
  solves the same food web problem as \id{idabanx2}, but with {\idaspgmr}
  and a user-supplied preconditioner.
  
  The preconditioner supplied to {\idaspgmr} is the block-diagonal part of 
  the Jacobian with $n_s \times n_s$ blocks arising from the reaction terms only
  ($n_s$ is the number of species in the model).

\item \id{idakryx2\_bbd\_p}
  solves the same food web problem as \id{idakryx2\_p}.

  This program solves the problem using {\idaspgmr} in parallel and the
  {\idabbdpre} preconditioner.

\end{itemize}

With the {\fida} module, in the two directories
\id{sundials/examples/ida/fcmix\_serial} and
\id{sundials/examples/ida/fcmix\_parallel}, are the following
examples for the {\F}-{\C} interface:
%
\begin{itemize}
\item \id{fidadenx} is a serial chemical kinetics example (\id{BDF}/{\dense})
       with rootfinding, equivalent to \id{idadenx}.

\item \id{fidakryx\_bbd\_p} is a parallel example (\id{BDF}/{\spgmr}/{\idabbdpre})
       equivalent to the example \id{idakryx1\_bbd\_p}.

\end{itemize}

\vspace{0.2in}\noindent 
In the following sections, we give detailed descriptions of some (but
not all) of these examples.  The Appendices contain complete listings
of those examples described below.  We also give our output files for
each of these examples, but users should be cautioned that their
results may differ slightly from these.  Solution
values may differ within tolerances, and differences in cumulative
counters, such as numbers of steps or Newton iterations, may differ
from one machine environment to another by as much as 10\% to 20\%.

In the descriptions below, we make frequent references to the {\ida}
User Document \cite{ida2.4.0_ug}.  All citations to specific sections
(e.g. \S\ref{s:types}) are references to parts of that User Document, unless
explicitly stated otherwise.

\vspace{0.2in}\noindent {\bf Note}. 
The examples in the {\ida} distribution are written in such a way as
to compile and run for any combination of configuration options during
the installation of {\sundials} (see \ugref{s:install}). As a consequence,
they contain portions of code that will not be typically present in a
user program. For example, all example programs make use of the
variable \id{SUNDIALS\_EXTENDED\_PRECISION} to test if the solver libraries
were built in extended precision and use the appropriate conversion 
specifiers in \id{printf} functions.  Similarly, the {\F} examples in
{\fida} are automatically pre-processed to generate source code that
corresponds to the manner in which the {\ida} libraries were built
(see \ugref{s:ex_fortran} in this document for more details).
