%% This is a shared SUNDIALS TEX file with a description of the
%% OpenMP nvector implementation
%%

In situations where a user has a multi-core processing unit capable of
running multiple parallel threads with shared memory, {\sundials} provides
an implementation of {\nvector} using OpenMP, called {\nvecopenmp}, and
an implementation using Pthreads, called {\nvecpthreads}.  
Testing has shown that vectors should be of length at least $100,000$ 
before the overhead associated with creating and using the threads is
made up by the parallelism in the vector calculations. 

The OpenMP {\nvector} implementation provided with {\sundials},
{\nvecopenmp}, defines the {\em content} field of \id{N\_Vector} to be a structure 
containing the length of the vector, a pointer to the beginning of a contiguous 
data array, a boolean flag {\em own\_data} which specifies the ownership 
of {\em data}, and the number of threads.  
Operations on the vector are threaded using OpenMP.
%%
\begin{verbatim} 
struct _N_VectorContent_OpenMP {
  long int length;
  booleantype own_data;
  realtype *data;
  int num_threads;
};
\end{verbatim}
%%
%%--------------------------------------------
%%
The following six macros are provided to access the content of an {\nvecopenmp}
vector. The suffix \id{\_OMP} in the names denotes the OpenMP version.
%%
\begin{itemize}

\item \ID{NV\_CONTENT\_OMP}                             
    
  This routine gives access to the contents of the OpenMP
  vector \id{N\_Vector}.
  
  The assignment \id{v\_cont} $=$ \id{NV\_CONTENT\_OMP(v)} sets           
  \id{v\_cont} to be a pointer to the OpenMP \id{N\_Vector} content  
  structure.                                             
                                                            
  Implementation: 
  
  \verb|#define NV_CONTENT_OMP(v) ( (N_VectorContent_OpenMP)(v->content) )|
  
\item \ID{NV\_OWN\_DATA\_OMP}, \ID{NV\_DATA\_OMP}, \ID{NV\_LENGTH\_OMP}, \ID{NV\_NUM\_THREADS\_OMP}


  These macros give individual access to the parts of    
  the content of a OpenMP \id{N\_Vector}.                        
                                                               
  The assignment \id{v\_data = NV\_DATA\_OMP(v)} sets \id{v\_data} to be     
  a pointer to the first component of the data for the \id{N\_Vector} \id{v}. 
  The assignment \id{NV\_DATA\_OMP(v) = v\_data} sets the component array of \id{v} to     
  be \id{v\_data} by storing the pointer \id{v\_data}.                   
  
  The assignment \id{v\_len = NV\_LENGTH\_OMP(v)} sets \id{v\_len} to be     
  the length of \id{v}. On the other hand, the call \id{NV\_LENGTH\_OMP(v) = len\_v} 
  sets the length of \id{v} to be \id{len\_v}.
                                                            
  The assignment \id{v\_num\_threads = NV\_NUM\_THREADS\_OMP(v)} sets \id{v\_num\_threads} to be     
  the number of threads from \id{v}. On the other hand, the call \id{NV\_NUM\_THREADS\_OMP(v) = num\_threads\_v} 
  sets the number of threads for \id{v} to be \id{num\_threads\_v}.
                                                            
  Implementation: 
  
  \verb|#define NV_OWN_DATA_OMP(v) ( NV_CONTENT_OMP(v)->own_data )|

  \verb|#define NV_DATA_OMP(v) ( NV_CONTENT_OMP(v)->data )|
  
  \verb|#define NV_LENGTH_OMP(v) ( NV_CONTENT_OMP(v)->length )|

  \verb|#define NV_NUM_THREADS_OMP(v) ( NV_CONTENT_OMP(v)->num_threads )|

\item \ID{NV\_Ith\_OMP}                                               
                                                            
  This macro gives access to the individual components of the data
  array of an \id{N\_Vector}.

  The assignment \id{r = NV\_Ith\_OMP(v,i)} sets \id{r} to be the value of 
  the \id{i}-th component of \id{v}. The assignment \id{NV\_Ith\_OMP(v,i) = r}   
  sets the value of the \id{i}-th component of \id{v} to be \id{r}.        
  
  Here $i$ ranges from $0$ to $n-1$ for a vector of length $n$.

  Implementation:

  \verb|#define NV_Ith_OMP(v,i) ( NV_DATA_OMP(v)[i] )|

\end{itemize}
%%
%%----------------------------------------------
%%
The {\nvecopenmp} module defines OpenMP implementations of all vector operations listed
in Table \ref{t:nvecops}. Their names are obtained from those in Table \ref{t:nvecops}
by appending the suffix \id{\_OpenMP} (e.g.\id{N\_VDestroy\_OpenMP}).
The module {\nvecopenmp} provides the following additional user-callable routines:
%%
\begin{itemize}

%%--------------------------------------

\item \ID{N\_VNew\_OpenMP}

  This function creates and allocates memory for a OpenMP \id{N\_Vector}.
  Arguments are the vector length and number of threads.

  \verb|N_Vector N_VNew_OpenMP(long int vec_length, int num_threads);|

%%--------------------------------------

\item \ID{N\_VNewEmpty\_OpenMP}

  This function creates a new OpenMP \id{N\_Vector} with an empty (\id{NULL}) data array.

  

  \verb|N_Vector N_VNewEmpty_OpenMP(long int vec_length, int num_threads);|

%%--------------------------------------

\item \ID{N\_VMake\_OpenMP}

 This function creates and allocates memory for a OpenMP vector
 with user-provided data array.

 

 \verb|N_Vector N_VMake_OpenMP(long int vec_length, realtype *v_data, int num_threads);|

%%--------------------------------------

\item \ID{N\_VCloneVectorArray\_OpenMP}

 This function creates (by cloning) an array of \id{count} OpenMP vectors.

 

 \verb|N_Vector *N_VCloneVectorArray_OpenMP(int count, N_Vector w);|

%%--------------------------------------

\item \ID{N\_VCloneVectorArrayEmpty\_OpenMP}

 This function creates (by cloning) an array of \id{count} OpenMP vectors, each with an
 empty (\id{NULL}) data array.

 

 \verb|N_Vector *N_VCloneVectorArrayEmpty_OpenMP(int count, N_Vector w);|

%%--------------------------------------

\item \ID{N\_VDestroyVectorArray\_OpenMP}

 This function frees memory allocated for the array of \id{count} variables of type
 \id{N\_Vector} created with \id{N\_VCloneVectorArray\_OpenMP} or with
 \id{N\_VCloneVectorArrayEmpty\_OpenMP}.

 

 \verb|void N_VDestroyVectorArray_OpenMP(N_Vector *vs, int count);|

%%--------------------------------------

\item \ID{N\_VPrint\_OpenMP}

 This function prints the content of a OpenMP vector to \id{stdout}.

 
 
 \verb|void N_VPrint_OpenMP(N_Vector v);|

\end{itemize}
%%
%%------------------------------------
%%
\paragraph{\bf Notes}                                                      
           
\begin{itemize}
                                        
\item
  When looping over the components of an \id{N\_Vector} \id{v}, it is     
  more efficient to first obtain the component array via       
  \id{v\_data = NV\_DATA\_OMP(v)} and then access \id{v\_data[i]} within the     
  loop than it is to use \id{NV\_Ith\_OMP(v,i)} within the loop.        

\item
  {\warn}\id{N\_VNewEmpty\_OpenMP}, \id{N\_VMake\_OpenMP}, 
  and \id{N\_VCloneVectorArrayEmpty\_OpenMP} set the field 
  {\em own\_data} $=$ \id{FALSE}. 
  \id{N\_VDestroy\_OpenMP} and \id{N\_VDestroyVectorArray\_OpenMP}
  will not attempt to free the pointer {\em data} for any \id{N\_Vector} with
  {\em own\_data} set to \id{FALSE}. In such a case, it is the user's responsibility to
  deallocate the {\em data} pointer.
                                     
\item
  {\warn}To maximize efficiency, vector operations in the {\nvecopenmp} implementation
  that have more than one \id{N\_Vector} argument do not check for
  consistent internal representation of these vectors. It is the user's 
  responsibility to ensure that such routines are called with \id{N\_Vector}
  arguments that were all created with the same internal representations.

\end{itemize}

For solvers that include a Fortran interface module, the {\nvecopenmp}
module slso includes a Fortran-callable function
\id{FNVINITOMP(code, NEQ, NUMTHREADS, IER)}, to initialize this
{\nvecopenmp} module.  Here \id{code} is an input solver id
(1 for {\cvode}, 2 for {\ida}, 3 for {\kinsol}, 4 for {\arkode}); NEQ is
the problem size (declared so as to match C type \id{long int});
NUMTHREADS is the number of threads; and IER is an error return flag
equal 0 for success and -1 for failure.
