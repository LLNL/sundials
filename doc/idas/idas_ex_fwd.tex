%===================================================================================
\section{Forward sensitivity analysis example problems}\label{s:fwd_ex}
%===================================================================================

For all the {\idas} examples, either of the two sensitivity method options,
\id{IDA\_SIMULTANEOUS} or \id{IDA\_STAGGERED}, can be used, 
and sensitivities may be included in the error test or not 
(\id{errconS} set on \id{TRUE} or \id{FALSE}, respectively, in the
call to \id{IDASetSensErrCon}).

Abbreviated descriptions of one serial example (\id{idasSlCrank\_FSA\_dns})
and a parallel one (\id{idasBruss\_FSA\_kry\_bbd\_p}) are provided in the following
two subsections.  For details on the other examples, the reader is
directed to the comments in their source files.

%--------------------------------------------------------------------
\subsection{A serial dense example: idasSlCrank\_FSA\_dns}
\label{ss:idasSlCrank_FSA_dns}

The \id{idasSlCrank\_FSA\_dns} program solves a system of index-2 DAEs, modeling a
planar slider-crank mechanism.  The problem is obtained through a stabilized index
reduction (Gear-Gupta-Leimkuhler) starting from the index-3 DAE equations of motion
derived using three generalized coordinates and two algebraic position constraints.

The following output is generated by \id{idasSlCrank\_FSA\_dns} when computing
sensitivities with the \id{IDA\_SIMULTANEOUS} method and full error
control (\id{idasSlCrank\_FSA\_dns -sensi sim t}):

\includeOutput{idasSlCrank\_FSA\_dns}{../../examples/idas/serial/idasSlCrank_FSA_dns.out}


%----------------------------------------------------------------------------------

\subsection{A parallel example using IDABBDPRE: idasBruss\_FSA\_kry\_bbd\_p}
\label{ss:idasBruss_FSA_kry_bbd_p}

The \id{idasBruss\_FSA\_kry\_bbd\_p} program solves the two-species time-dependent
PDE known as the Brusselator problem, using the {\idaspgmr} linear solver and the
{\idabbdpre} preconditioner.

The PDEs are discretized by central differencing on a 2D spatial mesh.
Dirichlet boundary conditions result in  algebraic equations on the boundary 
and differential equations in the interior of the domain mesh.
The system is actually implemented on submeshes, processor by processor.

Here the forward sensitivity capability in {\idas} is used to compute solution
sensitivities with respect to two of the problem parameters, and then the gradient
of a model output functional, written as the final time value of the spatial integral
of the first PDE component.

The following output is generated by \id{idasBruss\_FSA\_kry\_bbd\_p} when computing
sensitivities with the \id{IDA\_SIMULTANEOUS} method and full error control:

\id{mpirun -np 4 idasBruss\_FSA\_kry\_bbd\_p -sensi sim t}

\includeOutput{idasBruss\_FSA\_kry\_bbd\_p}{../../examples/idas/parallel/idasBruss_FSA_kry_bbd_p.out}
