%===============================================================================
\section{Introduction}\label{s:ex_intro}
%===============================================================================

This report is intended to serve as a companion document to the User
Documentation of {\kinsol} \cite{kinsol2.4.0_ug}.  It provides details, with
listings, on the example programs supplied with the {\kinsol} distribution
package.

The {\kinsol} distribution contains examples of four types: serial
{\C} examples, parallel {\C} examples, and serial and parallel {\F}
examples.  The following lists summarize all of these examples.

Supplied in the \id{sundials/kinsol/examples\_ser} directory are the
following serial examples (using the {\nvecs} module):

\begin{itemize}
\item \id{kinbanx}
  solves a simple 2-D elliptic PDE on a unit square.
 \newline
  This program solves the problem with the {\kinband} linear solver.
\item \id{kindenx1}
  solves the Ferraris-Tronconi problem.
  \newline
  This program solves the problem with the {\kindense} linear solver
  and uses different combinations of globalization and Jacobian
  update strategies with different initial guesses.
\item \id{kindenx2}
  solves a nonlinear system from robot kinematics.
  \newline
  This program solves the problem with the {\kindense} linear solver
  and a user-supplied Jacobian routine.
\item \id{kinkryx}
  solves a nonlinear system that arises from a system of partial
  differential equations describing a six-species food web population
  model, with predator-prey interation and diffusion on the unit
  square in two dimensions.
  \newline
  This program solves the problem with the {\kinspgmr} linear solver
  and a user-supplied preconditioner. The preconditioner is a
  block-diagonal matrix based on the partial derivatives of the
  interaction terms only.
\item \id{kinkrydem\_lin}
  solves the same problem as \id{kinkryx}, but with three Krylov
  linear solvers: \id{kinspgmr}, \id{kinspbcg}, and \id{kinsptfqmr}.
\end{itemize}
%%
Supplied in the \id{sundials/kinsol/examples\_par} directory are
the following parallel examples (using the {\nvecp} module):
\begin{itemize}
\item \id{kinkryx\_p}
  is a parallel implementation of \id{kinkryx}.
\item \id{kinkryx\_bbd\_p}
  solves the same problem as \id{kinkryx\_p}, with a block-diagonal matrix
  with banded blocks as a preconditioner, generated by difference quotients,
  using the {\kinbbdpre} module.
\end{itemize}
%%
With the {\fkinsol} module, in the directories 
\id{sundials/kinsol/fcmix/examples\_ser} and
\id{sundials/kinsol/fcmix/examples\_par}, are the following examples for
the {\F}-{\C} interface:
\begin{itemize}
\item \id{fkinkryx}
  is a serial example, which solves a nonlinear system of the form
  $u_i^2 = i^2$ using an approximate diagonal preconditioner.
\item \id{fkinkryx\_p}
  is a parallel implementation of \id{fkinkryx}.
\end{itemize}

\vspace{0.2in}\noindent 
In the following sections, we give detailed descriptions of some (but
not all) of these examples.  The Appendices contain complete listings
of those examples described below.  We also give our output files for
each of these examples, but users should be cautioned that their
results may differ slightly from these.  Differences in solution
values may differ within the tolerances, and differences in cumulative
counters, such as numbers of Newton iterations, may differ
from one machine environment to another by as much as 10\% to 20\%.

In the descriptions below, we make frequent references to the {\kinsol}
User Document \cite{kinsol2.4.0_ug}.  All citations to specific sections
(e.g. \S\ref{s:types}) are references to parts of that User Document, unless
explicitly stated otherwise.

\vspace{0.2in}\noindent {\bf Note}. 
The examples in the {\kinsol} distribution are written in such a way as
to compile and run for any combination of configuration options used during
the installation of {\sundials} (see \S\ref{s:install}). As a consequence,
they contain portions of code that will not be typically present in a
user program. For example, all {\C} example programs make use of the
variable \id{SUNDIALS\_EXTENDED\_PRECISION} to test if the solver libraries
were built in extended precision and use the appropriate conversion 
specifiers in \id{printf} functions. Similarly, the {\F} examples in
{\fkinsol} are automatically pre-processed to generate source code that
corresponds to the precision in which the {\kinsol} libraries were built
(see \S\ref{s:ex_fortran} in this document for more details).

