%===================================================================================
\chapter{Introduction}\label{s:intro}
%===================================================================================

{\ida} is part of a software family called {\sundials}: 
SUite of Nonlinear and DIfferential/ALgebraic equation Solvers.  
This suite consists of {\cvode}, {\kinsol}, and {\ida}, and variants of these
with sensitivity analysis capabilities.


IDA is a general purpose solver for the initial value problem for
systems of differential-algebraic equations (DAEs).  The name IDA
stands for Implicit Differential-Algebraic solver.  IDA is based on
DASPK \cite{BHP:94,BHP:98}, but is written in ANSI-standard C
rather than Fortran 77.  Its most notable feature is that, in the
solution of the underlying nonlinear system at each time step, it
offers a choice of Newton/direct methods or an Inexact Newton/Krylov
(iterative) method.  Thus {\ida} shares significant modules previously
written within CASC at LLNL to support the ordinary differential
equation (ODE) solvers {\cvode} \cite{CoHi:94,CoHi:96} and {\pvode}
\cite{ByHi:98,ByHi:99}, and also the nonlinear system solver {\kinsol}
\cite{TaHi:98}.

The Newton/Krylov method uses the GMRES (Generalized Minimal RESidual)
linear iterative method \cite{SaSc:86}, and requires almost no matrix
storage for solving the Newton equations as compared to direct
methods.  However, the GMRES algorithm allows for a user-supplied
preconditioner matrix, and for most problems preconditioning is
essential for an efficient solution.

\index{IDA@{\ida}!motivation for writing in C|(}
There are several motivations for choosing the {\C} language for {\ida}.
First, a general movement away from {\F} and toward {\C} in scientific
computing is apparent.  Second, the pointer, structure, and dynamic
memory allocation features in C are extremely useful in software of
this complexity, with the great variety of method options offered.
Finally, we prefer {\C} over {\CPP} for {\ida} because of the wider
availability of {\C} compilers, the potentially greater efficiency of {\C},
and the greater ease of interfacing the solver to applications written
in extended {\F}.
\index{IDA@{\ida}!motivation for writing in C|)}

\section{Changes in version {\idarelease}}
The major changes from the previous version involve a redesign of the
user interface across the entire {\sundials} suite. We have eliminated the
mechanism of providing optional inputs and extracting optional statistics 
from the solver through the \id{iopt} and \id{ropt} arrays. Instead,
{\ida} now provides a set of routines (with prefix \id{IDASet})
to change the default values for various quantities controlling the
solver and a set of extraction routines (with prefix \id{IDAGet})
to extract statistics after return from the main solver routine.
Similarly, each linear solver module provides its own set of {\id{Set}-}
and {\id{Get}-type} routines. For more details see \S\ref{ss:optional_input}
and \S\ref{ss:optional_output}.

Additionally, the interfaces to several user-supplied routines
(such as those providing Jacobians and preconditioner information) 
were simplified by reducing the number
of arguments. The same information that was previously accessible
through such arguments can now be obtained through {\id{Get}-type}
functions.

Installation of {\ida} (and all of {\sundials}) has been completely 
redesigned and is now based on configure scripts.

\section{Reading this User Guide}\label{ss:reading}

This user guide is a combination of general usage instructions and
specific example programs.  We expect that some readers will want to
concentrate on the general instructions, while others will refer
mostly to the examples, and the organization is intended to
accommodate both styles.

The structure of this document is as follows:
\begin{itemize}
\item
  In \S\ref{s:install} we begin with instructions for the installation of 
  {\ida}, within the structure of {\sundials}.
\item
  In \S\ref{s:math}, we give short descriptions of the numerical 
  methods implemented by {\ida} for the solution of initial value problems
  for systems of DAEs.
\item
  The following section describes the structure of the {\sundials} suite
  of solvers (\S\ref{ss:sun_org}) and the software organization of the {\ida}
  solver (\S\ref{ss:ida_org}). 
\item
  In \S\ref{s:simulation}, we give an overview of the usage of {\ida},
  as well as a complete description of the user interface and of the 
  user-defined routines for integration of IVP DAEs.
\item
  Section \ref{s:nvector} gives a brief overview of the generic {\nvector} module 
  shared among the various components of {\sundials}, as well as details on the two {\nvector}
  implementations provided with {\sundials}: a serial implementation
  (\S\ref{ss:nvec_ser}) and a parallel implementation, based on MPI
  (\S\ref{ss:nvec_par}).
\item
  Section \ref{s:gen_linsolv} describes in detail the generic linear solvers shared 
  by all {\sundials} solvers.
\end{itemize}

Finally, the reader should be aware of the following notational conventions
in this user guide:  program listings and identifiers (such as \id{IDAMalloc}) 
within textual explanations appear in typewriter type style; 
fields in {\C} structures (such as {\em content}) appear in italics;
and packages or modules, such as {\cvdense}, are written in all capitals. 
In the Index, page numbers that appear in bold indicate the main reference
for that entry.

\paragraph{Acknowledgments.}
We wish to acknowledge the contributions to previous versions of the
{\ida} code and user guide of Allan G. Taylor.

