% LaTeX source for joint paper on SUNDIALS
% Version of 18 August 2003

\documentclass[acmtoms]{acmtrans2m}
\usepackage{amsmath, amssymb}
\usepackage{epsfig}
\usepackage{tabularx}
\usepackage{showlabels}

\newcommand{\id}[1]{{\tt #1}}

\newcommand{\dfdy}{\frac{\partial f}{\partial y}}
\newcommand{\dfdyI}{\partial f / \partial y}
\newcommand{\dfdpi}{\frac{\partial f}{\partial p_i}}
\newcommand{\dfdpiI}{\partial f / \partial p_i}

\acmVolume{V}
\acmNumber{N}
\acmYear{Y}
\acmMonth{Month}

\markboth{R. Serban and A.C. Hindmarsh}
{CVODES: An ODE solver with sensitivity analysis capabilities}

\title{CVODES: An ODE solver with sensitivity analysis capabilities}

\author{RADU SERBAN and ALAN C. HINDMARSH \\
  Center for Applied Scientific Computing \\
  Lawrence Livermore National Laboratory}

\begin{abstract}
CVODES, which is part of the SUNDIALS software suite,
is a stiff and nonstiff ordinary differential equation initial 
value problem solver with sensitivity analysis capabilities. 
CVODES is written in a data-independent manner, with a highly 
modular structure to allow incorporation of different 
preconditioning and/or linear solver methods. It shares with the 
other SUNDIALS solvers several common modules, most notably the 
generic kernel of vector operations and a set of generic linear 
solvers and preconditioners.

CVODES solves the IVP by one of two methods -- backward differentiation 
formula or Adams-Moulton -- both implemented in a variable-step, 
variable-order form. The forward sensitivity module in CVODES implements 
the simultaneous corrector method, as well as two flavors of 
staggered corrector methods. Its adjoint sensitivity module 
provides a combination of checkpointing and cubic Hermite interpolation 
for the efficient generation of the forward solution during the adjoint 
system integration.

We describe the current capabilities of CVODES, its design principles 
and connection to the SUNDIALS suite, and the user interface. 
Finally, we mention current and future development efforts for CVODES, 
particularly in the direction of automatic generation of the sensitivity 
right hand sides using automatic differentiation and/or complex-step 
techniques.
\end{abstract}

\category{G.4}{Mathematical Software}{Algorithm design and analysis}

\category{G.1.7}{Numerical Analysis}{Ordinary Differential Equations}
[Initial value problems \and Multistep methods \and Stiff equations]

%\category{G.1.m}{Miscellaneous}{Sensitivity Analysis}
%[Forward methods \and Adjoint methods]

\terms{Algorithms, Design}

\keywords{ODEs, Forward Sensitivity Analysis, Adjoint Sensitivity Analysis}

%=======================================================================

\begin{document}

\setcounter{page}{1}

\begin{bottomstuff}
Authors' address: 
Radu Serban, 
Lawrence Livermore National Laboratory, P.O. Box 808, L-560,
Livermore, CA 94551; email: {\tt radu@llnl.gov};
Alan C. Hindmarsh,
Lawrence Livermore National Laboratory, P.O. Box 808, L-560,
Livermore, CA 94551; email: {\tt alanh@llnl.gov};
\newline
This work was performed under the auspices of the
U.S. Department of Energy by the University of California,
Lawrence Livermore National Laboratory, under contract No.
W-7405-Eng-48.
\end{bottomstuff}

\maketitle

%=======================================================================
% Include sections

\section{Introduction}

{\sf need for time integrators and solvers, need for robust
software that can be added to existing codes, introduce SUNDIALS
and its basic design principles (restrict to minimal info. from
user, let user supply data structures underneath, allow swap of
linear solvers), introduce need for sensitivity and ease of adding
it to existing codes for these problem classes. }

LLNL has a long history of research and development in ordinary
differential equation (ODE) methods and software, and closely related
areas, with emphasis on applications to partial differential equations
(PDEs).  Among the popular Fortran solvers written at LLNL are the
following:
\vspace*{-.19in}
\begin{itemize}

\item VODE: a solver for ODE initial value problems for stiff/nonstiff
systems, with direct solution of linear systems, by Brown, Byrne, and
Hindmarsh \cite{BBH:89}.

\item VODPK: a variant of VODE with preconditioned Krylov (GMRES
iteration) solution of the linear systems in place of direct methods,
by Brown, Byrne, and Hindmarsh \cite{Byr:92}.

\item NKSOL: a Newton-Krylov (GMRES) solver for nonlinear algebraic
systems, by Brown and Saad \cite{BrSa:90}.

\item DASPK: a solver for differential-algebraic equation (DAE)
systems (a variant of DASSL) with both direct and preconditioned
Krylov solution methods for the linear systems, by Brown, Hindmarsh,
and Petzold \cite{BHP:94}.

\end{itemize}

In recent years, there has been special interest in two kinds of
extensions of this software.  One is the extension to parallel
solution of large problems, especially on massively parallel machines.
The other is extensions to do sensitivity analysis, which centers
on the calculation of sensitivity of solution with respect to model
parameters.

Starting in 1993, the push to solve large systems in parallel
motivated work to write or rewrite solvers in C.  The first result of
that effort was CVODE.  This was a rewrite in ANSI standard C of the
VODE and VODPK solvers combined, for serial machines
\cite{CoHi:94,CoHi:96}.  The next result of this effort was PVODE, a
parallel extension of CVODE \cite{ByHi:98,ByHi:99}.

Similar rewrites of NKSOL and DASPK were then done, using the same
general design as CVODE and PVODE.  The resulting solvers are called
KINSOL and IDA, respectively.

More recently, we have changed the naming of these codes, in order
to be consistent throughout the family.  Specifically, there is
one solver, CVODE, in two versions -- serial and parallel.  Thus
we refer to the parallel version of CVODE, rather than PVODE.



\section{Algorithms}\label{s:algorithms}

CVODES solves initial value problems (IVP) for systems of ODEs. 
Such problems can be stated as
\begin{equation}\label{e:ivp}
\dot{y} = f(t,\,y) \, , \quad y(t_0) = y_0 \, ,
\end{equation}
where $y \in {\bf R}^N$ and $\dot{y}\,=dy/dt$.
That is, (\ref{e:ivp}) represents a system of $N$ ordinary
differential equations and their initial conditions at some $t_0$. The
dependent variable is $y$ and the independent variable is $t$. The
independent variable need not appear explicitly in the vector valued
function $f$.

Additionally, if (\ref{e:ivp}) depends (through its right-hand side and/or its initial
conditions) on some parameters $p \in {\bf R}^{N_p}$, i.e.
\begin{equation}\label{e:ivp_p}
\dot{y}  = f(t,\,y,\,p) \, , \quad y(t_0)  = y_0(p) \, ,
\end{equation}
CVODES can also compute first order derivative information, performing either
{\em forward sensitivity analysis} or {\em adjoint sensitivity analysis}.
In the first case, CVODES computes the sensitivities of the solution with respect to the 
parameters $p$, while in the second case, CVODES computes the gradient of a 
{\em derived function} with respect to the parameters $p$.

In the remaining of this section we describe the algorithms implemented in CVODES,
with emphasis on sensitivity analysis. In Section~\ref{ss:integration} we give only a 
brief overview of the ODE integration algorithm to introduce some of the
quantities needed in the sequel. 
Since CVODES shares the main integration engine with CVODE, the interested reader is
directed to~\cite{HBGLSSW:03}.

%-----------------------------------------------------------------------------------

\subsection{ODE Integration}\label{ss:integration}

The IVP is solved by one of two numerical methods. These are the
backward differentiation formula (BDF) and the Adams-Moulton formula. 
Both are implemented in a variable-stepsize, variable-order form. The BDF
uses a fixed-leading-coefficient form. These formulas can both be
represented by a linear multistep formula 
\begin{equation}\label{e:lmm}
\sum_{i=0}^{K_1}\alpha_{n,i}y_{n-i} + h_n\sum_{i=0}^{K_2}\beta_{n,i} 
\dot{y}_{n-i}=0
\end{equation}
where the $N$-vector $y_n$ is the computed approximation to $y(t_n)$,
the exact solution of (\ref{e:ivp}) at $t_n$. The stepsize is
$h_n=t_n-t_{n-1}$.  The coefficients $\alpha_{n,i}$ and $\beta_{n,i}$
are uniquely determined by the particular integration formula, the
history of the stepsize, and the normalization $\alpha_{n,0}=-1$. The
Adams-Moulton formula is recommended for nonstiff ODEs and is
represented by (\ref{e:lmm}) with $K_1=1$ and $K_2=q-1$. The order
of this formula is $q$ and its values range from 1 through 12. For
stiff ODEs, BDF should be selected and is represented by 
(\ref{e:lmm}) with $K_1=q$ and $K_2=0$. For BDF, the order $q$ may
take on values from 1 through 5. In the case of either formula, the
integration begins with $q=1$, and after that $q$ varies automatically
and dynamically.

For either BDF or the Adams formula, $\dot{y}_n$ denotes
$f(t_n,\,y_n)$. That is, (\ref{e:lmm}) is an implicit formula, and 
the nonlinear equation 
\begin{equation}\label{e:nonlinear}
  \begin{split}
    G(y_n) &\equiv  y_n-h_n\beta_{n,0}f(t_n,\,y_n) - a_n=0   \\
    a_n &= \sum_{i>0}(\alpha_{n,i}y_{n-i}+h_n\beta_{n,i}\dot{y}_{n-i}) 
  \end{split}
\end{equation}
must be solved for $y_{n}$ at each time step. For nonstiff problems,
a functional (or fixpoint) iteration is normally used which does not
require the solution of a linear system of equations. For stiff
problems, a Newton iteration is used and for each iteration an
underlying linear system must be solved. This linear system of
equations has the form
\begin{equation}\label{e:Newton}
M[y_{n(m+1)}-y_{n(m)}]=-G(y_{n(m)}) \, ,
\end{equation}
where $y_{n(m)}$ is the $m$th approximation to $y_n$, and $M$
approximates $\partial G/ \partial y$:
\begin{equation} \label{e:N_Matrix}
M \approx I-\gamma J, ~~~~ J = \frac{\partial f}{\partial y}, ~~~~
    \gamma = h_n\beta_{n,0} ~.
\end{equation}
At present, aside from a diagonal Jacobian approximation, the other
options implemented in CVODES for solving the linear systems
(\ref{e:Newton}) are:
(a) a direct method with dense treatment of the Jacobian,
(b) a direct method with band treatment of the Jacobian, and
(c) an iterative method SPGMR (scaled, preconditioned
GMRES) \cite{BrHi:89}, which is a Krylov subspace method. In most
cases, performance of SPGMR is improved by user-supplied
preconditioners. The user may precondition the system on the left, on
the right, on both the left and right, or use no preconditioner.
In most cases of interest to the CVODES user, the technique of
integration will involve BDF and the Newton method coupled with one of the 
linear solver modules.

The integrator computes an estimate $E_{n}$ of the local error at each time
step, and strives to satisfy the following inequality
\begin{equation*}
\left\| E_n\right\|_{WRMS} < 1 ~,
\end{equation*}
where $\|\cdot\|_{WRMS}$ is the weighted root-mean-square norm~\cite{BCP:96}
defined in terms of the user-defined relative and absolute tolerances. 
Since these tolerances define the allowed error per step, they should be 
chosen conservatively. Experience indicates that a conservative choice yields 
a more economical solution than error tolerances that are too large.
The error control mechanism in CVODES varies the stepsize and order
in an attempt to take minimum number of steps while satisfying the local
error test. 

CVODES also incorporates an algorithm for special treatment of
quadratures depending on the solution $y$ of the (\ref{e:ivp}) or
(\ref{e:ivp_p}). Evaluation of integrals of the form
$G = \int_{t_0}^{t_f} g(t,y,p) dt$ can be done efficiently using the
underlying linear multistep method interpolant polynomials by
appending to (\ref{e:ivp_p}) an additional ODE
\begin{equation}\label{e:quad_eqns}
\dot\phi = g(t,y,p) \, , \quad \phi(t_0) = 0 \, ,
\end{equation}
in which case $G = \phi(t_f)$. In the context of an implicit ODE
integrator, since the right-hand side of (\ref{e:quad_eqns}) does not
depend on $\phi$, such equations need not participate in the solution of
the nonlinear system~(\ref{e:nonlinear}). CVODES allows the user to
identify these equations separately from those in~(\ref{e:ivp_p}) and
provides the option of including or excluding $\phi$ from the error
control algorithm.
%
The main reason for including this option in CVODES was the need for
efficient quadrature computation in the context of adjoint sensitivity 
analysis (see Section~\ref{ss:adj_sensitivity}).

A complete description of the CVODES integration algortihm, including 
the nonlinear solver convergence, error control mechanism, and heuristics 
related to stopping criteria and finite-difference parameter selection, is
given in~\cite{HBGLSSW:03}.

%-----------------------------------------------------------------------------------

\subsection{Forward Sensitivity Analysis}\label{ss:fwd_sensitivity}

Typically, the governing equations of complex, large-scale models
depend on various parameters,  through the right-hand side vector 
and/or through the vector of initial conditions, as in (\ref{e:ivp_p}).
In addition to numerically solving the ODEs, it may be desirable to
determine the sensitivity of the results with respect to the model
parameters. 
Such sensitivity information can be used to estimate which
parameters are most influential in affecting the behavior of the
simulation or to evaluate optimization gradients (in the setting of dynamic
optimization, parameter estimation, optimal control, etc.).

The {\em solution sensitivity} with respect to the model parameter
$p_i$ is defined as the vector 
$s_i (t) = {\partial y(t)}/{\partial p_i}$
and satisfies the following {\em forward sensitivity equations}
(or in short {\em sensitivity equations}):
\begin{equation}\label{e:sens_eqns}
\dot{s_i}  = \frac{\partial f}{\partial y} s_i + \frac{\partial f}{\partial p_i} \, ,
\quad s_i(t_0)  = \frac{\partial y_{0}(p)}{\partial p_i} \, ,
\end{equation}
which are obtained by applying the chain rule of differentiation to the original
ODEs (\ref{e:ivp_p}). 

When performing forward sensitivity analysis, CVODES carries out the time integration 
of the combined system, (\ref{e:ivp_p}) and (\ref{e:sens_eqns}), by viewing it as an ODE
system of size $N(N_s+1)$, where $N_s$ represents a subset of model parameters $p_i$, 
with respect to which sensitivities are desired ($N_s \le N_p$). 
However, major efficiency improvements can be obtained by taking advantage of the special 
form of the sensitivity equations as linearizations of the original ODEs. 
In particular, for stiff systems, in which case CVODES employs a Newton iteration, 
the original ODE system and all sensitivity systems share the same Jacobian matrix, 
and therefore the same iteration matrix $M$ in (\ref{e:N_Matrix}).

The sensitivity equations are solved with the same linear multistep formula that
was selected for the original ODEs and, if Newton iteration was selected, the
same linear solver is used in the correction phase for both state and sensitivity 
variables. In addition, CVODES offers the option of including
({\em full error control}) or excluding
({\em partial error control}) the sensitivity variables from the local 
error test.

\subsubsection{Forward sensitivity methods}
In what follows we briefly describe three methods that have been proposed for the 
solution of the combined ODE and sensitivity system for the vector
${\hat y} = [y, s_1, \ldots , s_{N_s}]$.
Due to its inefficiency, especially for large-scale problems, the first approach 
is not implemented in CVODES.

\begin{itemize}

\item[{\em Staggered Direct.}]
  In this approach \cite{CaSt:85}, the nonlinear system (\ref{e:nonlinear}) is first 
  solved and, once an acceptable numerical solution is obtained, the sensitivity 
  variables at the new step are found by directly solving (\ref{e:sens_eqns}) 
  after the BDF discretization is used to eliminate ${\dot s}_i$. 
  Although the system matrix of the above linear system is based on the exact same 
  information as the matrix $M$ in (\ref{e:N_Matrix}), it must be updated and factored 
  at every step of the integration as $M$ is updated only ocasionally. 
  The computational cost associated with these matrix updates and factrorizations 
  makes this method unattractive when compared with the methods described below and 
  is therefore not implemented in CVODES.
  
\item[{\em Simultaneous Corrector.}] 
  In this method \cite{MaPe:97}, the BDF discretization is applied simultaneously
  to both the original equations (\ref{e:ivp_p}) and the sensitivity systems
  (\ref{e:sens_eqns}) resulting in the following nonlinear system 
  \begin{equation*}
    {\hat G}({\hat y}_n) \equiv  
    {\hat y}_n - h_n\beta_{n,0} {\hat f}(t_n,\,{\hat y}_n) - {\hat a}_n = 0 \, ,
  \end{equation*}
  where
  ${\hat f} = [ f(t,y,p), \ldots , (\dfdyI)(t,y,p) s_i + (\dfdpiI)(t,y,p) , \ldots ]$
  and ${\hat a}_n$ are the terms in the BDF discretization that depend on the
  solution at previous integration steps.
  This combined nonlinear system can be solved as in (\ref{e:Newton}) using
  a modified Newton method by solving the corrector equation
  \begin{equation}\label{e:Newton_sim}
    {\hat M}[{\hat y}_{n(m+1)}-{\hat y}_{n(m)}]=-{\hat G}({\hat y}_{n(m)})
  \end{equation}
  at each iteration, where 
  \begin{equation*}
    {\hat M} = 
    \begin{bmatrix}
      M              &        &        &        &   \\
      \gamma J_1     & M      &        &        &   \\
      \gamma J_2     & 0      & M      &        &   \\
      \vdots         & \vdots & \ddots & \ddots &   \\
      \gamma J_{N_s} & 0      & \ldots & 0      & M 
    \end{bmatrix} \, ,
  \end{equation*}
  $M$ is defined as in (\ref{e:N_Matrix}), and 
  $J_i = ({\partial}/{\partial y})\left[ (\dfdyI) s_i + (\dfdpiI) \right]$.
  It can be shown that a 2-step quadratic convergence can be attained by only
  using the block-diagonal portion of ${\hat M}$ in the corrector equation
  (\ref{e:Newton_sim}). This results in a decoupling that allows the reuse of 
  $M$ without additional matrix factorizations. However, the products
  $(\dfdyI)s_i$ as well as the vectors $\dfdpiI$ must still be reevaluated at 
  each step of the iterative process (\ref{e:Newton_sim}) to update the 
  sensitivity portions of the residual ${\hat G}$.
  
\item[{\em Staggered corrector.}] In this approach \cite{FTB:97}, as in the staggered direct method,
  the nonlinear system (\ref{e:nonlinear}) is solved first using the Newton iteration
  (\ref{e:Newton}). Then, a separate Newton iteration is used to solve the
  sensitivity system (\ref{e:sens_eqns}):
  \begin{multline}\label{e:stgr_iterations}
    M [s_{i , n(m+1)} - s_{i , n(m)}]= \\
    s_{i, n(m)} - 
    \gamma \left( \dfdy (t_n , y_n, p) s_{i , n(m)} + \dfdpi (t_n , y_n , p) \right)
    -a_{i,n} \, ,
  \end{multline}
  where $a_{i,n} = \sum_{j>0}(\alpha_{n,j}s_{i , n-j}+h_n\beta_{n,j}\dot{s}_{i , n-j})$.
  In other words, a modified-Newton iteration is used to solve a linear system.
  In this approach, the vectors $\dfdpiI$ need be updated only once per integration step, 
  after the state correction phase (\ref{e:Newton}) has converged. Note also that 
  Jacobian-related data can be reused at all iterations (\ref{e:stgr_iterations})
  to evaluate the products $(\dfdyI) s_i$.
\end{itemize}  

CVODES implements the simultaneous corrector method and two flavors of the 
staggered corrector method which differ only if the sensitivity variables are
included in the error control test.
In the {\em full error control} case, 
the first variant of the staggered corrector method requires the convergence of 
the iterations (\ref{e:stgr_iterations}) for all $N_s$ sensitivity sytems and then 
performs the error test on the sensitivity variables. The second variant of the method
will perform the error test for each sensitivity vector $s_i,\,i=1,2,\ldots,N_s$
individually, as they pass the convergence test. Differences in performance
between the two variants may therefore be noticed whenever one of the sensitivity 
vectors $s_i$ fails a convergence or error test. 

An important observation is that the staggered corrector method, combined with 
the SPGMR linear solver effectively results in a staggered direct method. 
Indeed, SPGMR requires only the action of the matrix $M$ on a vector and
this can be provided with the current Jacobian information. Therefore, the
modified Newton procedure (\ref{e:stgr_iterations}) will theoretically converge 
after one iteration.

\subsubsection{Selection of the absolute tolerances for sensitivity variables}
If the sensitivities are considered in the error test, CVODES provides an 
automated estimation of absolute tolerances for the sensitivity variables 
based on the absolute tolerance for the corresponding state variable.
The relative tolerance for sensitivity variables is set to be the same as for 
the state variables. The selection of absolute tolerances for the sensitivity 
variables is based on the observation that the sensitivity vector $s_i$ will have 
units of $[y]/[p_i]$.
With this, the absolute tolerance for the $j$-th component of the sensitivity
vector $s_i$ is set to ${atol_j}/{|{\bar p}_i|}$,
where $atol$ are the absolute tolerances for the state variables and $\bar p$
is a vector of scaling factors that are dimensionally consistent with
the model parameters $p$ and give indication of their order of magnitude.
This choice of relative and absolute tolerances is equivalent 
to requiring that the weighted root-mean-square norm of the sensitivity 
vector $s_i$ with weights based on $s_i$ is the same as the
weighted root-mean-square norm of the vector of scaled sensitivities 
${\bar s}_i = |{\bar p}_i| s_i$ with weights based on the state variables
(the scaled sensitivities ${\bar s}_i$ being dimensionally consistent with the
state variables).

\subsubsection{Evaluation of the sensitivity right-hand side}
There are several methods for evaluating the right-hand side of the sensitivity 
systems (\ref{e:sens_eqns}): analytic evaluation, automatic differentiation, 
complex-step approximation, finite differences (or directional derivatives).
CVODES provides all the software hooks for implementing interfaces to
automatic differentiation or complex-step approximation and future versions
will provide these capabilities.
At the present time, besides the option for analytical sensitivity right hand 
sides (user-provided), CVODES can evaluate these quantities using various
finite difference-based approximations to evaluate the terms $(\dfdyI) s_i$ 
and $(\dfdpiI)$, or using directional derivatives to evaluate
$\left[ (\dfdyI) s_i + (\dfdpiI) \right]$.
As is typical for finite differences, the proper choice of perturbations is a 
delicate matter. CVODES takes into account several problem-related features;
the relative ODE error tolerance $rtol$, the machine unit roundoff $U$,
the scale factor ${\bar p}_i$, and the weighted root-mean-square norm of the 
sensitivity vector $s_i$.

Using central finite differences asd an example, the two terms 
$({\partial f}/{\partial y}) s_i$ 
and ${\partial f}/{\partial p_i}$ in the right-hand side of (\ref{e:sens_eqns}) 
can be evaluated separately:
\begin{gather}
  \frac{\partial f}{\partial y} s_i \approx \frac{f(t, y+\sigma_y s_i, p)-
    f(t, y-\sigma_y s_i, p)}{2\,\sigma_y} \, , \label{e:fd2} \\
  \frac{\partial f}{\partial p_i} \approx \frac{f(t,y,p + \sigma_i e_i)-
    f(t,y,p - \sigma_i e_i)}{2\,\sigma_i} \, , \tag{\ref{e:fd2}'} \\
  \sigma_i = |{\bar p}_i| \sqrt{\max(rtol, U)} \, , \quad
  \sigma_y = \frac{1}{\max(1/\sigma_i, \|s_i\|_{WRMS}/|{\bar p}_i|)} \, , \nonumber
\end{gather}
simultaneously:
\begin{gather}
  \frac{\partial f}{\partial y} s_i + \frac{\partial f}{\partial p_i} \approx
  \frac{f(t, y+\sigma s_i, p + \sigma e_i) -
    f(t, y-\sigma s_i, p - \sigma e_i)}{2\,\sigma} \, , \label{e:dd2} \\
  \sigma = \min(\sigma_i, \sigma_y) \, , \nonumber
\end{gather}
or adaptively switching between (\ref{e:fd2})+(\ref{e:fd2}') and (\ref{e:dd2}), 
depending on the relative size of the estimated finite difference 
increments $\sigma_i$ and $\sigma_y$.

%-----------------------------------------------------------------------------------

\subsection{Adjoint Sensitivity Analysis}\label{ss:adj_sensitivity}

In the {\em forward sensitivity approach} described in the previous
section, obtaining sensitivities with respect to $N_s$ parameters is roughly
equivalent to solving an ODE system of size $(1+N_s) N$. This can become 
prohibitively expensive, especially for large-scale problems, if sensitivities
with respect to many parameters are desired.
In this situation, the {\em adjoint sensitivity method} is a very
attractive alternative, provided that we do not need the solution sensitivities
$s_i$, but rather the gradients with respect to model parameters of a relatively 
few derived functionals of the solution. In other words, if $y(t)$ is the solution
of (\ref{e:ivp_p}), we wish to evaluate the gradient ${dG}/{dp}$ of
\begin{equation}\label{e:G}
G(p) = \int_{t_0}^{t_f} g(t, y, p) dt \, ,
\end{equation}
or, alternatively, the gradient ${dg}/{dp}$ of the function $g(t, x, p)$ 
at time $t_f$. The function $g$ must be smooth enough that $\partial g / \partial y$ 
and $\partial g / partial p$ exist and are bounded. 
In what follows, we only provide the final results for the gradients of both $G$ and 
$g(t_f)$. For details on the derivation see~\cite{CLPS:03}.
%
The gradient of $G$ with respect to $p$ is nothing but
\begin{equation}\label{e:dGdp}
  \frac{dG}{dp} = \lambda^T(t_0) s(t_0) + 
  \int_{t_0}^{t_f} \left( \frac{\partial g}{\partial p} + 
    \lambda^T \frac{\partial f}{\partial p} \right) dt,
\end{equation}
where $\lambda$ is solution of
\begin{equation}\label{e:adj_eqns}
{\dot \lambda} = -\left( \dfdy \right)^T \lambda - 
\left( \frac{\partial g}{\partial y} \right)^T \, ,
\quad \lambda(t_f) = 0
\end{equation}
and $s(t_0) = dy_0/dp$.
%
The gradient of $g(t_f,y,p)$ with respect to $p$ can be then obtained
by using the Leibnitz differentiation rule. Indeed, from (\ref{e:G}),
$({dg}/{dp})(t_f) = {d}/{dt_f}({dG}/{dp})$
and therefore, taking into account that $dG/dp$ in (\ref{e:dGdp}) depends on $t_f$
both through the upper integration limit and through $\lambda$ and that $\lambda(t_f) = 0$, 
\begin{equation}\label{e:dgdp}
  \frac{dg}{dp}(t_f) = 
  \frac{\partial g}{\partial p}(t_f) +
  \mu^T(t_0) s(t_0) + 
  \int_{t_0}^{t_f} \mu^T \frac{\partial f}{\partial p} dt \, ,
\end{equation}
where $\mu$ is the sensitivity of $\lambda$ with respect to the final integration 
limit and thus satisfies the following equation, obtained by taking the total derivative
with respect to $t_f$ of (\ref{e:adj_eqns}):
\begin{equation}\label{e:adj1_eqns}
{\dot \mu} = -\left( \dfdy \right)^T \mu \, ,
\quad \mu(t_f) = \left( \frac{\partial g}{\partial y}(t_f) \right)^T \, .
\end{equation}
The final condition on $\mu(t_f)$ follows from 
$(\partial\lambda/\partial t) + (\partial\lambda/\partial t_f) = 0$ at $t_f$, and
therefore, $\mu(t_f) = -{\dot\lambda}(t_f)$. 

The first thing to notice about the adjoint system (\ref{e:adj_eqns}) is that there is 
no explicit specification of the parameters $p$; this implies that, once the solution
$\lambda$ is found, the formula (\ref{e:dGdp}) can then be used to find the gradient
of $G$ with respect to any of the parameters $p$. The same holds true for the system
(\ref{e:adj1_eqns}) and the formula (\ref{e:dgdp}) for gradients of $g(t_f,y,p)$. 
The second important remark is that the adjoint systems are terminal value problems 
which depend on the solution $y(t)$ of the original IVP (\ref{e:ivp_p}). 
Therefore, a procedure is needed for providing the states $y$ obtained during a forward 
integration phase of (\ref{e:ivp_p}) to CVODES during the backward integration phase 
of (\ref{e:adj_eqns}) or (\ref{e:adj1_eqns}). 
The approach adopted in CVODES, based on {\em check-pointing} is described next.

During the backward integration, the evaluation of the right hand side 
of the adjoint system requires, at the current time, the states $y$ which
were computed in the forward integration phase.
Since CVODES implements variable-stepsize integration formulas,
it is unlikely that the states will be available at the desired time and
therefore some form of interpolation is needed. The CVODES implementation
being also variable-order, it is possible that during the forward
integration phase the order may be reduced as low as 1st order,
which means that there may be points in time where only $y$ and ${\dot y}$
are available. Therefore, CVODES employs a cubic Hermite interpolation
algorithm. However, especially for large-scale problems and long integration
intervals, the number and size of the vectors $y$ and ${\dot y}$ that would 
need to be stored make this approach computationally intractable. 

CVODES settles for a compromise between storage space and execution time by
implementing a so-called {\em check-pointing scheme}. At the cost of
at most one additional forward integration, this approach offers the best possible 
estimate of memory requirements for adjoint sensitivity analysis. To begin with,
based on the problem size $N$ and the available memory, the user decides on 
the number $N_d$ of data pairs $y$-${\dot y}$ that can be kept in memory for 
the purpose of interpolation. Then, during the first forward integration stage, 
every $N_d$ integration steps a check point is formed by saving enough information
(either in memory or on disk if needed) to allow for a hot restart, that is a restart
which will exactly reproduce the forward integration. In order to avoid storing
Jacobian-related data at each check point, a reevaluation of the iteration matrix
is forced before each check point. At the end of this stage, we are left with $N_c$ 
check points, including one at $t_0$.
During the backward integration stage, the adjoint variables are integrated
from $t_f$ to $t_0$ going from one check point to the previous one.
The backward integration from check point $i+1$ to check point $i$ is preceeded
by a forward integration from $i$ to $i+1$ during which $N_d$ data pairs 
$y$-${\dot y}$ are generated and stored in memory for interpolation.
%
This procedure is illustrated in Fig.~\ref{f:ckpnt}.
%
\begin{figure}
\centerline{\psfig{figure=ckpnt.eps,width=4in}}
\caption {Illustration of the check-pointing algorithm for generation of 
  the forward solution during the integration of the adjoint system.}
\label{f:ckpnt}
\end{figure}

This approach transfers the uncertainty in the number of integration
steps in the forward integration phase to uncertainty in the final number of check 
points. However, $N_c$ is much smaller than the number of steps taken during
the forward integration and there is no major penalty for writting and then reading
check point data to/from a temporary file.
%
Note that, at the end of the first forward integration stage, data pairs 
$y$-${\dot y}$ are available from the last check point to the end of the integration 
interval. If no check points are necessary, i.e $N_d$ is larger than the 
number of integration steps taken in the solution of (\ref{e:ivp_p}),
the total cost of an adjoint sensitivity computation can be as low as one forward
plus one backward integration.
%
In addition, CVODES provides the capability of reusing a set of check points
for multiple backward integrations, thus allowing for efficient computation of
gradients of several functionals (\ref{e:G}).

Finally, we note that the adjoint sensitivity module in CVODES provides the 
infrastructure to integrate backwards in time any ODE terminal value problem
dependent on the solution of the IVP (\ref{e:ivp_p}), including
adjoint systems (\ref{e:adj_eqns}) or (\ref{e:adj1_eqns}), as well as any other
quadrature ODEs that may be needed in evaluating the integrals in (\ref{e:dGdp}) 
or (\ref{e:dgdp}). In particular, for ODE systems arising from semi-discretization
of time-dependent PDEs, this feature allows for integration of either the 
discretized adjoint PDE system or the adjoint of the discretized PDE.

\section{Code Organization}\label{s:organization}

As mentioned before, the SUNDIALS family of solvers consists of 
CVODE (for ODE systems), KINSOL (for nonlinear algebraic
systems), and IDA (for DAE systems).  
In addition, variants of these which also do sensitivity analysis calculations are
available (CVODES) or in development (IDAS and KINSOLS).
%
The overall organization of the CVODES package,a s well as its relationship
to SUNDIALS, is shown in Fig.~\ref{f:cvsorg}.  
The basic elements of the structure are a module for
the basic integration algorithm (including forward sensitivity analysis),
a module for adjoint sensitivity analysis, and a set of modules for the solution
of linear systems that arise in the case of a stiff system.  
\begin{figure}
\centerline{\psfig{figure=cvsorg.eps,width=\textwidth}}
\caption {Overall structure diagram of the CVODES package.
  Modules specific to CVODES are distinguished by rounded boxes, while 
  generic solver and auxiliary modules are in square boxes.}
\label{f:cvsorg}
\end{figure}

The central integration module deals with the evaluation of integration coefficients,
the functional or Newton iteration process, estimation of local error,
selection of stepsize and order, and interpolation to user output
points, among other issues.  Although this module contains logic for
the basic Newton iteration algorithm, it has no knowledge of the
method being used to solve the linear systems that arise.  For any
given user problem, one of the linear system modules is specified, and
is then invoked as needed during the integration. 

In addition, if forward sensitivity analysis is turned on, the main module 
will integrate the forward sensitivity equations, simultaneously with the original IVP.
The sensitivities variables may or may not be included in the local error control
mechanism of the main integrator.
CVODES provides three different strategies of dealing with the correction
stage for the sensitivity variables, simultaneous corrector and
two variants of staggered corrector (see Section~\ref{ss:fwd_sensitivity}).
The CVODES package includes an algorithm for the approximation of the sensitivity 
equations right hand sides by difference quotients, but the user has the option of 
supplying these right hand sides directly.

The adjoint sensitivity module provides the infrastructure needed for the 
integration backwards in time of any system of ODEs which depends on the solution 
of the original IVP, in particular the adjoint system and any quadratures required
in evaluating the gradient of the objective functional.
This module deals with the set-up of the check points, interpolation of the forward 
solution during the backward integration, and backward integration of the adjoint
equations. 

At present, the package includes the following four CVODES linear system
modules:
(a) CVSDENSE (LU factorization and backsolving with dense matrices),
(b) CVSBAND (LU factorization and backsolving with banded matrices),
(c) CVSDIAG (an internally generated diagonal approximation to the 
Jacobian), and
(d) CVSSPGMR (scaled preconditioned GMRES method).
This set of linear solver modules is intended to be expanded in the
future as new algorithms are developed.

In the case of the direct CVSDENSE and CVSBAND methods, the package includes
an algorithm for the approximation of the Jacobian by difference
quotients, but the user also has the option of supplying the Jacobian
(or an approximation to it) directly. In the case of the iterative
CVSSPGMR method, the package includes and algorithm for the approximation
by difference quotients of the product between the Jacobian matrix and
a vector of appropriate length. Again, the user has the option of providing
a routine for this operation.
In the case of CVSPGMR, the preconditioning must be supplied by the user, 
in two phases: setup (preprocessing of Jacobian data) and solve.
While there is no default choice of preconditioner analogous to the 
difference quotient approximation in the direct case, the references
\cite{BrHi:89,Byr:92}, together with
the example and demonstration programs included with CVODES, offer
considerable assistance in building preconditioners.

Each CVODES linear solver module consists of five routines, devoted to (1)
memory allocation and initialization, (2) setup of the matrix data
involved, (3) solution of the system, (4) solution of the system in the
context of forward sensitivity analysis, and (5) freeing of memory.  The
setup and solution phases are separate because the evaluation of
Jacobians and preconditioners is done only periodically during the
integration, as required to achieve convergence. The call list within
the central CVODES module to each of the five associated functions is
fixed, thus allowing the central module to be completely independent
of the linear system method.

These modules are also decomposed in another way.
Each of the modules CVSDENSE, CVSBAND, and CVSSPGMR is a set of 
interface routines built on top of a generic solver module, 
named DENSE, BAND, and SPGMR, respectively.  
The interfaces deal with the use of these methods in the CVODES context, 
whereas the generic solver is independent of the context.
While the generic solvers here were generated with SUNDIALS in mind, our
intention is that they be usable in other applications as
general-purpose solvers.  This separation also allows for any generic
solver to be replaced by an improved version, with no necessity to
revise the CVODES package elsewhere.

CVODES also provides two preconditioner modules. The first one, 
CVSBANDPRE, is intended to be used on serial computers and provides
a banded difference quotient Jacobian based preconditioner and solver
routines for use with CVSPGMR. The second preconditioner module, 
CVBBDPRE, developed for parallel computers, generates a 
preconditioner that is a block-diagonal matrix with each block being 
a band matrix. A detailed description of these two modules, including
usage guidlines, is given in ~\cite{HBGLSSW:03}.

All state information used by CVODES to solve a given problem is saved
in a structure, and a pointer to that structure is returned to the
user.  There is no global data in the CVODES package, and so in this
respect it is reentrant. State information specific to the linear
solver is saved in a separate structure, a pointer to which resides in
the CVODES memory structure. The reentrancy of CVODES was motivated
by the anticipated multicomputer extension, but is also essential
during adjoint sensitivity analysis where the check-pointing algorithm
leads to interleaved forward and backward integration passes. 

Figure~\ref{f:cvsorg} does not show any of the user-supplied routines 
for CVODES. At a minimum, the user must provide a routine for the evaluation 
of the ODE right-hand side and, if performing adjoint sensitivity analysis,
a routine for the evaluation of the right-hand side of the adjoint system. 
Optional user-provided routines include, depending on the options chosen, 
functions for Jacobian evaluation (direct cases) or Jacobian-vector products 
(Krylov case), setup and solution of Krylov preconditioners, a function providing 
the integrand of any additional quadrature equations, and a routine for
providing the right-hand side of the sensitivity equations (for forward sensitivity
analysis). Depending on the options selected for the solution of the adjoint
system, the user may have to provide corresponding Jacobian and/or preconditioner
routines.

One of the most important characteristics of the design of CVODES 
(shared by all solvers across SUNDIALS) is the fact that it is implemented 
in a data-independent manner, in that the solver does not need any information
regarding the underlying structure of the data on which it operates.

The CVODES solver acts on vectors through a generic NVECTOR module,
which defines an NVECTOR structure specification, a data-independent NVECTOR type, 
a set of abstract vector operations, and a set of wrappers for accessing the actual vector
operations of the implementation under which an NVECTOR was created. Because
details of vector operations are thus encapsulated within each specific
NVECTOR implementation, CVODES is thus independent of a specific
implementation. This allows the solver to be precompiled as a binary
library and allows more than one NVECTOR implementation to be used within
a single program. This feature is essential for the efficient integration of
quadrature variables (see Section~\ref{ss:integration} as well as for
adjoint sensitivity analysis when, for some problems, the adjoint variables
are more conveniently organized in a structure different from that of
the variables in the forward problem.

A particular NVECTOR implementation, such as the serial and parallel 
implementations included with SUNDIALS or a user-provided implementation,
must provide the following:
(1) actual implementation of the routines for operations on N-vectors, 
such as creation, destruction, summation, and dot product;
(2) a routine to construct an NVECTOR specification structure
for this particular implementation which defines the data necessary
for constructing a new N-vector and attaches the vector operations
to the new structure; and
(3) a destructor for the NVECTOR specification structure.


\section{Usage} 
\label{s:usage}

A general approach for using SUNDIALS is given below. 
{\sf The guiding philosophy of using SUNDIALS is ...}
The outline conveys the basic elements of what is needed to properly
specify and solve a problem, the order in which certain tasks must be
done, the options available for modifying certain solver parameters,
heuristics and algorithms, and the options for extracting the solution
and/or solver statistics.
Complete details and additional examples are in the documentation that
accompanies each solver in SUNDIALS.

\begin{enumerate}

\item \label{sun_headers}
SUNDIALS contains header files that define various constants,
enumerations, macros, data types and function prototypes.  At a
minimum, the user must include header files that declare: the SUNDIALS
data types for real, integer, and boolean variables; the content and
operations that can be performed on vectors in either a serial or
parallel environment; and, the functions needed by the solver to
setup, compute, and extract the solution. Typically, additional header
files will be specified to declare the preconditioning and/or linear
solver methods to be used.

\item \label{sun_problem}
The user must provide a function for evaluating the equations to be
solved. Optionally, a user-defined data structure can be created and
passed to this function. The user must also initialize the vector
specification data structure, give the problem size and error
tolerances, provide a suitable initial guess or initial values, along
with other details for the problem.

\item \label{sun_create}
The next step is to call a routine for initializing a block of memory
that will be used in solving the problem. The memory block is created
with certain default values for the solver, such as the use of
standard output for writing warning and error messages, or {\tt NULL}
as a default value for the pointer to the user-specified data
structure to be passed in evaluating the user's function.

\item \label{sun_set}
At this stage, the default values in the solver memory block can be
changed if so desired. Choices and default values are given in the
following subsections for each of the basic solvers.

\item \label{sun_malloc}
The user now calls the appropriate routine to perform any required
memory allocation, after checking the initialized memory block for
errors in the default or optional inputs.

\item \label{sun_linear}
Typically, preconditioning and/or linear solver methods are needed for
solving the linear systems that may arise. These methods can now be
attached to the block of memory allocated for the solver.

\item \label{sun_solve}
Solve the problem. Information and solver statistics are available
through extraction functions.

\item \label{sun_reinit}
Re-initialization (SUNDIALS solver, linear solver)?

\item \label{sun_finalize}
To complete the process, the user must make the appropriate calls to
free memory that was allocated in the previous steps.

\end{enumerate}

For a parallel machine environment, the appropriate header file for
MPI must be specified in Step~\ref{sun_headers}; and, the MPI communicator,
active set of processors, local and global vectors lengths must be
initialized or given in Step~\ref{sun_problem}. Finally, in
Step~\ref{sun_finalize}, memory allocated for MPI must be freed.

For forward sensitivity analysis, the appropriate forward sensitivity
header file must be specified in Step~\ref{sun_headers} and the
definition of the sensitivity problem must be given before
Step~\ref{sun_solve}. To do so, the user must create an array of
real parameters upon which the problem solution depends; attach a pointer to
this array to the user-defined data structure f\_data; specify the
number of sensitivities to be computed; and, provide an array to
specify the parameters for which solution sensitivities are to be
computed. Memory allocated for sensitivity analysis should be freed in
Step~\ref{sun_finalize}.

{\sf For adjoint sensitivity analysis, we refer the interested reader to
the user's guide.}

\subsection{CVODE and CVODES} 
\label{ss:CVODE_usage}

The following options can be set by the user; otherwise, the option
retains the default setting that appears in the brackets {\tt []}:
a pointer to the user-defined data structure that is passed to the
user's $f$ routine [{\tt NULL}]; a pointer to an error file where all
warning and error messages will be written [{\tt NULL} means standard
output]; the maximum order used by the Adams method [12] or BDF method
[5]; the maximum number of internal steps to be taken in its attempt
to reach the next output time [500]; the maximum number of warning
messages issued when the internal step size is below machine epsilon
[10]; a flag to activate stability limit detection [{\tt FALSE}]; the
initial step size [estimated by the solver]; the minimum absolute step
size value allowed [0.0]; the maximum absolute step size value allowed
[infinity]; and, the independent variable value past which the
solution is not to proceed [infinity].
Additional optional inputs include: the maximum number of Newton
iterations [3]; the maximum number of convergence failures [10]; the
maximum number of error test failures [7]; and, the proportionality
coefficient in the nonlinear convergence test [0.1].

If desired, the user can request the following information from the
solver: the amount of integer workspace allocated by CVODE; the amount
of real workspace allocated by CVODE; the cumulative number of
internal steps taken; the number of calls to the user's
f function; the number of calls made to the linear solver's setup
routine; the number of local error test failures that have occurred;
the order used during the last internal step; the number of order
reductions due to stability limit detection; the actual initial step
size used; the step size used for the last internal step; the step
size to be attempted on the next internal step; the current internal
time reached by the solver; a suggested factor by which the user's
tolerances should be scaled when too much accuracy has been requested
for some internal time steps; the vector containing the error weights
for the state variables; and, the vector containing the estimated
local errors at the current internal time step.

\subsection{KINSOL} 
\label{ss:KINSOL_usage}

The following options can be set by the user; otherwise, the option
retains the default setting that appears in the brackets {\tt []}:
a pointer to the user data that will be passed to the user's $F$
routine [{\tt NULL}]; a pointer to an error file where all
warning and error messages will be written [{\tt NULL} means standard
output]; an integer value between 0--3 to indicated the desired amount
of convergence-related output [{\tt 0} means no statistics printed];
the maximum allowable number of nonlinear iterations [{\tt
MXITER\_DEFAULT}]; a flag to control the initial call to the
preconditioner setup routine [{\tt FALSE} forces the initial call];
the maximum number of steps calling the preconditioner solve without
calling the preconditioner setup [10]; a flag indicating which of
three methods to use for computing the coefficient ($\eta$) in the linear
solver convergence tolerance epsilon~(?) [{\tt ETACHOICE1}]; the
constant value of $\eta$ for {\tt ETACONSTANT} [0.1]; the parameter
values for $\eta$ in the case {\tt ETACHOICE2} [egamma=0.9,
ealpha=2.0]; the flag to control the lower bound on the linear solver
convergence tolerance ($\epsilon$) [{\tt FALSE}]; the maximum
allowable length of a Newton step [{\tt default}]; the relative error
in computing {\tt func(uu)} [{\tt unit roundoff}]; a scalar constant
which restricts the update of uu [{\tt infinity}]; a real scalar value
containing the stopping tolerance; a real scalar value containing the
stopping tolerance on the maximum scaled step [{\tt default}]; a
pointer to an array of constraints on uu [{\tt NULL}].

If desired, the user can request the following information from the
solver: the amount of integer workspace allocated by KINSOL; the
amount of real workspace allocated by KINSOL; the number of calls to
the user's {\tt func} function; the number of nonlinear iterations
performed; the total number of times the $\beta$ condition could not
be met in the line search algorithm; the number of backtrack
operations done in the linesearch algorithm; the scaled norm at a
given iteration; and, the last step length in the global strategy
routine.

\subsection{IDA} 
\label{ss:IDA_usage}

The following options can be set by the user; otherwise, the option
retains the default setting that appears in the brackets {\tt []}:
a pointer to the user data structure that will be passed to the user's
$F$ routine [{\tt NULL}]; a file pointer to an error file where
all warning and error messages will be written [{\tt NULL}]; the
maximum order used by the BDF method [5]; the maximum number of
internal steps to be taken in an attempt to reach the next output time
[500]; the initial step size [estimated by the solver]; the maximum
absolute step size allowed [{\tt infinity}]; the independent variable
value past which the solution is not to proceed [{\tt infinity}]; the
factor in nonlinear convergence test for use during integration [1.0];
a flag to indicate whether or not to suppress algebraic variables in
the local error test [{\tt FALSE}]; a vector which indicates whether a
given component is either an algebraic or differential variable [0.0
and 1.0 indicate algebraic and differential variables, respectively];
a vector defining inequality constraints for each component of the
solution vector [0.0 means the corresponding component has no
constraint; etc.].
Additional optional inputs include: the maximum number of Newton
iterations [3]; the maximum number of convergence failures [10]; the
maximum number of error test failures [10]; and, the constant in the
nonlinear solver convergence test [0.33].

If desired, the user can request the following information from the
solver: the amount of integer workspace allocated by IDA; the amount
of real workspace allocated by IDA; the cumulative number of internal
steps taken; the number of calls to the user's {\tt res} function; the
number of calls made to the linear solver's setup routine; the number
of local error test failures that have occurred; the number of
backtrack operations done in the linesearch algorithm of the
consistent initialization routine; the order used during the last
internal step; the order to be used on the next internal step; the
actual initial step size used; the step size for the last internal
step, or the last value of the artificial step size in calculating
consistent initial conditions; the step size to be attempted on the
next internal step; the current internal time reached; a suggested
factor by which the user's tolerances should be scaled when too much
accuracy has been requested for some internal step; the vector
containing the error weights for the state variables; and, the vector
containing the estimated local error at the current internal time steps.

\subsection{Preconditioners and Linear Solvers}
\label{ss:linear_usage}

For the scaled preconditioned GMRES solver, the following options can
be set by the user; otherwise, the option retains the default setting
that appears in the brackets {\tt []}:
a classical Gram-Schmidt orthogonalization can be used [modified Gram-Schmidt];
the factor by which the tolerance on the
nonlinear iteration is multiplied to get a tolerance on the linear
iteration [0.05]; the preconditioner setup routine [NULL]; the
preconditioner solver routine [NULL]; a pointer to the user
preconditioner data [NULL]; the Jacobian-vector product routine
[internal finite difference approximation]; and, a pointer to user
Jacobian data. The optional outputs are: the amount of integer
workspace; the amount of real workspace; the number of preconditioner
evaluations; the number of calls made to the preconditioner solve
routine; the number of linear iterations; the number of linear
convergence failures; the number of calls to the Jacobian-vector
product routines; and, the number of calls to the user's function due
to finite difference Jacobian-vector products.

For the band block diagonal preconditioner, the optional outputs are:
the amount of integer workspace used, the amount of real workspace
used; and, the number of calls to the local function that approximates
the user's function.

For the dense linear solver, the user can set optional inputs so that a
user-supplied dense Jacobian approximation is used instead of the
default difference quotient routine. Also, a pointer can be set so
that user data is passed each time the user-supplied dense Jacobian
routine is called. The optional outputs are: the amount of integer
workspace used; the amount of real workspace used; the number of calls
made to the user-supplied Jacobian evaluation routine; and, the number
of calls to the user's function due to the default difference quotient
routine.

For the band linear solver, the user can set optional inputs so that a
user-supplied band Jacobian approximation routine is used instead of
the default difference quotient routine. Furthermore, a pointer can be
set so that user data is passed each time the user-supplied banded
Jacobian routine is called. The optional outputs are: the amount
of integer workspace used; the amount of real workspace used; the
number of calls made to the Jacobian evaluation routine; and, the
number of calls made to the user's function due to the default
difference quotient routine.

For the diagonal linear solver, the optional outputs are: the amount
of integer workspace used; the amount of real workspace used; and, the
number of calls due to computing the diagonal Jacobian via finite differences.

\subsection{Sensitivity Analysis}
\label{ss:sensitivity_analysis}

{\sf Is this needed?}

\subsection{Fortran Usage} 
\label{ss:Fortran_usage}

For two of the SUNDIALS solvers -- CVODE and KINSOL, users with
Fortran applications are accommodated.  This is done with a set of
interface routines that connect the C solver with the user's Fortran
routines.

In order to achieve portability for these interfaces, all of the
Fortran user-supplied routines have fixed names.

These interfaces are provided as separate modules, called FCVODE and
FKINSOL, for CVODE and KINSOL, respectively.  In each case, small
examples programs are provided.

For complete details,
see the user documentation for each of the solvers in SUNDIALS.





\section{Conclusions}
\label{s:conclusions}

{\sf Make some concluding remarks and comments on future development 
directions.}





%=======================================================================
% Bibliography
\bibliographystyle{acmtrans}
\bibliography{../biblio}

%=======================================================================
% The article should end by the following lines. 
% The actual dates will be supplied by the Editor-in-Chief.

\begin{received}
Received Month Year;
revised Month Year; accepted Month Year
\end{received}

\end{document}
