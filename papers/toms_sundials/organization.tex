\section{Code Organization}
\label{s:organization}

The writing of CVODE from the Fortran solvers VODE and VODPK initiated a
complete redesign and reorganization of LLNL nonlinear solver coding.  This
was done partly to exploit features of C not present in Fortran, partly to
achieve a more object-oriented design, partly to maximize the reuse of code
modules, and partly to facilitate the extension from a serial to a parallel
implementation.

The process of modularization has continued with the collection of CVODE,
KINSOL, and IDA into SUNDIALS. The SUNDIALS distribution now contains a
number of common modules in a shared directory. Additionally, compilation of
SUNDIALS is now independent of any prior specification of a particular
NVECTOR implementation, facilitating the use of binary libraries. The
current NVECTOR structure also allows use of multiple implementations within
the same code, as may be required to meet user needs.

The features of the design of CVODE include the following:
\begin{itemize}
\item Memory allocation is heavily used.
\item The linear solver modules are separate from the core integrator,
so that the latter is independent of the method for solving linear
systems.
\item Each linear solver has generic solver, which is independent of
the ODE contest, together with an interface to the CVODE core
integrator module.
\item The vector kernels (linear sums, dot products, norms, etc.) on
$N-$vectors are isolated in a separate NVECTOR module.
\end{itemize}

Figure \ref{fig-sunorg} shows the overall structure of SUNDIALS solvers,
with the various separate modules. The evolution of SUNDIALS has been
directed keeping the entire set of solvers in mind. Thus, CVODE, KINSOL and
IDA share much in their organization and have a number of common modules.
The separation of the linear solvers from the core integrators allows for
easy addition of linear solvers not currently included in SUNDIALS. At the
bottom level is the NVECTOR module, providing key vector operations such as
creation, destruction, summation, and dot-products on potentially
distributed data vectors. Example serial and parallel NVECTOR
implementations are included with SUNDIALS. Two small modules defining
several data types and elementary mathematical operations are also included.

\begin{figure}[tp]
\centerline{\psfig{figure=sunorg.eps,width=\textwidth}}
\caption{Overall structure of the SUNDIALS  package.}
\label{fig-sunorg}
\end{figure}

A number of necessary and optional user-supplied routines for the SUNDIALS
solvers are not shown in \mbox{Figure \ref{fig-sunorg}}. The user must
provide a routine for the evaluation of $f$ (CVODE) or $F$ (KINSOL and
IDA). The user provided routines may include, depending on the options
chosen, routines for Jacobian evaluation (direct cases) or Jacobian-vector
products (Krylov case), and routines for the setup and solution of Krylov
preconditioners.

\subsection{Shared Modules - NVECTOR}

A generic NVECTOR implementation is used within the SUNDIALS solvers to
operate on NVECTORS. This generic implementation defines an NVECTOR
structure specification, a data-independent NVECTOR type, a set of
operations, and a set of wrappers for access to the actual operation set of
the implementation under which an NVECTOR was created. Because details of
vector operations are thus encapsulated within each specific implementation,
compilation of SUNDIAL solvers is now independent of implementation,
allowing the solvers to reside in precompiled libraries.

A particular NVECTOR implementation, such as the serial and parallel example
implementations included with SUNDIALS of a user provided implementation,
must provide certain functionalities. Each implementation must provide a
specification of the data needed to generate an new NVECTOR and a menu to
routines for operations on NVECTORS, including, for example, creation,
destruction, summation, element-by-element inversion, and dot-product. Each
NVECTOR includes data and descriptive fields as well as a pointer to the
specification for the implementation to which it belongs.

If neither the serial nor parallel package NVECTOR implementation is
suitable, the user can provide one or more NVECTOR implementations.  For
example, it might (and has been) more practical to substitute a more complex
data structure.


\subsection{Shared Modules - Linear Solvers}

As can be seen in \mbox{Figure \ref{fig-sunorg}}, three linear solver
packages are currently included with SUNDIALS: a direct dense matrix solver
(DENSE), a direct band solver (BAND), and an iterative Krylov solver
(SPGMR). These are stand-along packages in their own right.

The shared linear solvers are accessed from SUNDIALS solvers via solver
specific wrappers. Thus, SPGMR is accessed via CVSPGMR, IDASPGMR, and
KINSPGMR for CVODE (and CVODES), IDA, and KINSOL, respectively. For the
DENSE solver, the wrappers are CVDENSE and IDADENSE for CVODE/CVODES and
IDA, respectively. Similar wrappers for BAND are CVBAND and IDABAND. KINSOL,
because of its design specific to large problems, does not interface with
the direct solvers.
