\section{Sensitivity Analysis}

When solving an ODE or DAE system or a nonlinear algebraic system, in
addition to the solution $y$ or $u$, we often want its sensitivity
(first-order) with respect to parameters in the problem (or with
respect to initial conditions).

Consider the different problem types in turn.

\subsection{ODEs}

Here the system is assumed to involve a vector of parameters,
$p = (p_1,\cdots,p_m)$, so that we can write
\[ \dot{y} = f(t,y,p),~~ y(t_0) = y_0(p) \]
(including the case where components of the initial value vector $y_0$
may be parameters).  In addition to $y$ as a function of $t$, we want
the array $s = \partial y / \partial p ~~ (N \times m)$, whose columns
are the sensitivities of $y$ with respect to the $p_i$.

Each column $s_i = \partial y / \partial p_i$ satisfies another ODE
\[ \dot{s}_i = J s_i + \partial f / \partial p_i ~~~~
               (J = \partial f / \partial y) \]
with initial values $s_i(t_0) = \partial y_0 / \partial p_i$.

A code called {\bf SensPVODE}, by Lee, Hindmarsh, and Brown, was
written as an extension of the parallel version of CVODE for this
situation \cite{LHB:00}.  It integrates the extended ODE system for
$Y = (y,w_1,\cdots,w_m)$, where $w_i = \bar{p}_i s_i$ and $\bar{p}_i$
is a scale factor for $p_i$ (it may equal $p_i$ if this is nonzero).

In SensPVODE, the evaluation of $\dot{w}_i = \bar{p}_i \dot{s}_i$ is
done by difference quotients, for which there is a range of choices
in differencing strategy, or by Automatic Differentiation.

It is important to note that the Jacobian of the extended system, of
size $N(m+1)$, is approximated by $diag[J,\cdots,J]$.  So when the
linear systems are solved with the SPGMR algorithm, an appropriate
preconditioner is the block-diagonal matrix $diag[P,\cdots,P]$, where
$P$ is the preconditioner used for the original problem $\dot{y} = f$.
Thus the linear systems for the extended problem involve additional
solve operations compared to the original problem, but no additional
matrix setup operations.

{\sf Think through this carefully and point to CVODES paper as
appropriate - do not overlap with that paper here except for
general introductory stuff.}

In a second approach to the ODE sensitivity analysis, a code called
{\bf CVODES} was written by Serban and Hindmarsh.  It has two
different modes of operation.  In the first mode, it can be used to
integrate an extended system $Y = (y,s_1,\cdots,s_m)$ forward.  (Here
the scale factors $\bar{p}_i$ are not included in $Y$, but are used to
set tolerances imposed on $Y$.)  This integration also includes two
new options for the corrector iteration that is done on every step.
In the second mode, CVODES can be used to carry out adjoint
sensitivity analysis, in which the original system for $y$ is
integrated forward, an adjoint system is then integrated backward, and
finally the desired sensitivities are obtained from the backward
solution.  This backward (adjoint) approach is more practical than
the forward approach when the number of parameters $p_i$ is large.

To be specific about how the adjoint approach works, we consider
the following situation.  Assume as before that $f$ and/or $y_0$
involves the parameter vector $p$, and also that there is a
function $g(t,y,p)$ such that what we desire in the end is the
sensitivity array $(dg/dp)|_{t=t_f}$, where $g$ is a vector of
size $N_g$ and $t_f$ is the final time.  We first integrate the
original problem $\dot{y} = f$ forward from $t_0$ to $t_f$. The
next step in the procedure is to integrate from $t_f$ to $t_0$ the
adjoint system whose size is $N \times N_g$,
\[ \dot{\mu} = -J^* \mu ~,~~~
   \mu(t_f) = \left. \left( \frac{\partial g}{\partial y}
                     \right)^* \right| _{t=t_f} ~. \]
When this backward integration is complete, then the desired
sensitivity array is given by
\[ (dg/dp)|_{t=t_f} = \mu^*(t_0)s(t_0)
   ~+~ \int_{t_0}^{t_f} \mu^* f_p dt ~+~ g_p|_{t=t_f} ~. \]
In the backward integration, we regenerate $y(t)$ values, as needed
in right-hand side of the adjoint system, using a check-point scheme
combined with cubic Hermite interpolation.

Other situations, with different forms for the desired sensitvity
information, are covered by different adjoint systems.


\subsection{Nonlinear Systems}

In the case of a nonlinear system, the sensitivity equations are
considerably simpler.  If the system is written $F(u,p) = 0$ and we
define $s = \partial u / \partial p$, then for the individual
sensitivity vectors $s_i$,

\[ J s_i = -\partial F / \partial p_i ~~\mbox{where} ~~
          J = \partial F / \partial u ~. \]

A code called {\bf SensKINSOL}, by Grant, Hindmarsh, and Taylor
\cite{GHT:03} solves for $u$ (if not done already by KINSOL), then
solves linear systems for the scaled sensitivities $w_i = \bar{p}_i s_i$.


\subsection{DAEs}

The case of a DAE system is similar to that for ODEs.  Writing the
system as $F(t,y,y',p) = 0$ and defining $s = \partial y / \partial p$
as before, we obtain DAEs for the individual sensitivity vectors,
\[ \frac{\partial F}{\partial y} s_i + \frac{\partial F}{\partial y'} s'_i
                                 + \frac{\partial F}{\partial p_i}  = 0 \]

By analogy with SensPVODE, a code called {\bf SensIDA}, by Lee and
Hindmarsh was written to integrate the extended DAE system for
$Y = (y,w_1,\cdots,w_m)$, where $w_i = \bar{p}_i s_i$.

Also by analogy with the ODE case, the Newton matrix for this extended
system is approximated by $diag[J,\cdots,J]$, where $J$ is the Newton
matrix for original system.


