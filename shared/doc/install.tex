Generally speaking, the installation procedure outlined below will work on commodity 
{\linux}/{\unix} systems without modification, and so 
the typical user may certainly opt to just skip to the step-by-step installation 
instructions given below.  Users are still encouraged, however, to carefully read this
entire section before attempting to install the {\sundials} suite.  
In lieu of reading the installation guide, the user may invoke the 
configuration utility script with the help flag to view a complete listing of 
available optional parameters which may be done by issuing 
\begin{verbatim}
   % ./configure --help 
\end{verbatim}
from a shell command prompt.

Regarding terminology, \textit{build\_tree} refers to the directory under which the 
user wants to build and/or install the {\sundials} package.  
Also, \textit{source\_tree} refers to the directory where the {\sundials} 
source code is located.  The chosen \textit{build\_tree} may be different from the \textit{source\_tree}, 
thus allowing for multiple installations of the {\sundials} suite with different 
configuration options within a common \textit{source\_tree}.

Concerning the installation procedure outlined below, after invoking the \textit{tar} 
command with the appropriate options the contents of the {\sundials} archive 
(or the \textit{source\_tree}) will be extracted to a directory named \id{sundials}.  
Since the name of the extracted directory is not version-specific it is recommended that the 
user refrain from extracting the archive to a directory containing a previous version/release 
of the {\sundials} suite.  If the user is only upgrading and the previous 
installation of {\sundials} is not needed, then the user may remove the 
previous installation by issuing 
\begin{verbatim}
   % rm -Rf sundials
\end{verbatim}
from a shell command prompt.

With reference only to the parallel releases of the {\sundials} suite, the 
upshot of following the installation/build procedure given below will be the automatic 
compilation of support for a parallel execution environment if the subdirectory named 
\id{sundials/nvec\_par} exists (which is \textit{not} included 
with the serial {\sundials} packages), and a functional {\mpi} (Message-Passing Interface) 
implementation is detected by the configuration script during the preliminary system analysis stage.  
Passing the \id{--without-mpi} (or equivalently \id{--with-mpi=no}) command-line 
option to the configuration script will cause support for {\mpi} to be completely omitted.

Even though the installation procedure given below presupposes that the user will use the default 
vector kernels supplied with the distribution, using the {\sundials} suite with a 
user-supplied vector kernel normally will not require any changes to the build procedure.  

%%==========================================================================================

\section{Installation steps}

To install the {\sundials} suite issue the following commands from a shell command prompt:
\begin{enumerate}
\item \id{gunzip} \textit{sundials\_file}\id{.tar.gz}
\item \id{tar -xf} \textit{sundials\_file}\id{.tar}
\item \id{cd} {\em build\_tree}
\item {\em path\_to\_source\_tree}\id{/configure} [\textit{options}]

{\em Note}: If \id{./configure} does not work then try \id{sh ./configure} instead.
\item \id{make}
\item \id{make install}
\item \id{make examples}
\end{enumerate}

By default, the {\sundials} libraries and header files are installed under subdirectories 
(named \texttt{\textbf{lib}} and \texttt{\textbf{include}}, respectively) of the directory 
from within which the configuration script was initially invoked.

If system storage space conservation is a priority, then the user may, upon completion of the software 
installation, issue 
\begin{verbatim}
   % make clean
\end{verbatim}
and/or 
\begin{verbatim}
   % make examples_clean
\end{verbatim}
(will delete all example executables) from a shell command prompt to remove unneeded object files and 
thus help to minimize the total required storage space.

%%==========================================================================================

\section{Configuration options}

The installation procedure given above will generally work without modification; however, if the system 
includes multiple {\mpi} implementations, then certain configure script-related flags may 
be used to indicate which {\mpi} implementation should be used.  Also, if the user wants to 
use non-default language compilers, then, again, the necessary 
shell environment variables must be appropriately redefined.

The remainder of this section provides explanations of configure script-related optional parameters.  
An unabridged, release-specific listing of the available flags (with brief descriptions) may be viewed by 
issuing \id{./configure --help} from a shell command prompt.


%%
%% General options
%%

\begin{config}
  
\item \id{--prefix=PREFIX}
  
  Location for architecture-independent files.
  
  Default: \id{PREFIX =} {\em build\_tree}.
  
\item \id{--includedir=DIR}
  
  Alternate location for header files.
  
  Default: \id{DIR = PREFIX/include}.
  
\item \id{--libdir=DIR}
  
  Alternate location for libraries.
  
  Default: \id{DIR = PREFIX/lib}.
  
\item \id{--disable-examples}
  
  All example programs are built unless this option is used (or, equilavently to \id{--enable-examples=no}).  
  The example executables are created under the following subdirectories of the associated solver: 
  
  \begin{config}
  \item {\em build\_tree}/{\em solver}/\id{examples\_ser/}: serial examples.
  \item {\em build\_tree}/{\em solver}/\id{examples\_par/}: parralel {\mpi} examples.
  \item {\em build\_tree}/{\em solver}/\id{fcmix}/\id{examples\_ser/}: serial {\F} examples.
  \item {\em build\_tree}/{\em solver}/\id{fcmix}/\id{examples\_par/}: parallel {\F} examples.
  \end{config}
  
  Note that some of these subdirectories may not exist, depending on the solver and/or the configuration
  options used.
  
\item \id{--disable-}{\em solver}
  
  Each existing solver module will be built by default, unless explicitely disabled with this option.
  {\em solver} can be one of: \id{cvode}, \id{cvodes}, \id{ida}, or \id{kinsol}.
  
\item \id{--without-f77}
  
  A {\F} language compiler is only used if compilation of the included example subroutines has been 
  enabled (default) and the {\cvode} and/or {\kinsol} module(s) have/has been enabled (both enabled by default).
  This behaviour can be overwritten by using the option \id{--without-f77}.
  The only consequence of disabling {\F} support is that the {\F} example programs will not be built.
  
\item \id{--with-cppflags=ARG}
  
  Specify additional {\C} preprocessor flags 
  (e.g., \id{ARG = -I<include dir>} if there are header files in nonstandard locations).
  
\item \id{--with-cflags=ARG}
  
  Specify additional {\C} compilation flags.
  
\item \id{--with-fflags=ARG}
  
  Specify additional {\C} compilation flags.
  
\item \id{--with-ldflags=ARG}
  
  Specify additional linker flags
  (e.g., \id{ARG = -L<lib dir>} if there are libraries in nonstandard locations).
  
\item \id{--with-libs=ARG}
  
  Specify additional libraries to be used.
  (e.g., \id{ARG = -l<foo>} to link the library \id{libfoo.a}).
  
\item \id{--with-single-prec}
  
  By default, {\sundials} will define a real number (internally referred to as \id{realtype}) 
  to be a double-precision floating-point numeric via preprocessor directives in the header file
  \id{sundialstypes.h} (located under the \id{sundials/shared/include/} subdirectory); 
  however, the \id{--with-single-prec} configure option may be used to build {\sundials} with 
  \id{realtype} alternately defined as a single-precision floating-point numeric data type.
  
\item \id{--with-f77underscore=ARG}
  
  The option deals with the {\F}/{\C} interface and is used to specify the number of underscores to 
  append to function names so {\F} routines can link with the associated {\sundials} libraries.
  
  Default: \id{ARG = one}.
  
\end{config}

%%
%% Environment variables
%%

\vspace{0.25in}
\noindent The following environment variables can be locally defined for use in the configuration of {\sundials}:

\begin{config}
  
\item \id{CC}
  
\item \id{CFLAGS}
  
\item \id{F77}
  
\item \id{FFLAGS}
  
  Since the configuration script uses the first {\C} and {\F} language compilers found in the current executable 
  search path, then each relevant shell variable (\id{CC} and \id{F77}) must be locally (re)defined in order to 
  use a different compiler.  For example, to use \id{xcc} (executable name of chosen compiler) as the {\C} language 
  compiler use \id{CC=xcc}.
  
\end{config}

%%
%% Parallel options
%%

\vspace{0.25in}
\noindent The following configuration options are only applicable to the parallel {\sundials} packages.

\begin{config}
  
\item \id{--without-mpi}
  
  The parallelized vector kernel may be disabled by using the \id{--without-mpi} option
  (or equivalently \id{--with-mpi=no}).
  
\item \id{--with-mpicc=ARG}
\item \id{--with-mpif77=ARG}
  
  The configuration utility script will use by default the {\mpi} compiler scriptss named
  \id{mpicc} and \id{mpif77} to compile the parallelized {\sundials} subroutines; 
  however, for reasons of compatibility, different executable names may be specified via the above options.

\item \id{--with-mpi-root=MPIDIR}

  This option may be used to specify which {\mpi} implementation should be used.  
  The {\sundials} configuration script will automatically check under the subdirectories 
  \id{MPIDIR/include/} and \id{MPIDIR/lib/} for the necessary header files and libraries, and will also 
  search the subdirectory \id{MPIDIR/bin/} for the {\C} and {\F} {\mpi} compiler scripts.
  
\item \id{--with-mpi-incdir=INCDIR}
\item \id{--with-mpi-libdir=LIBDIR}
\item \id{--with-mpi-lib=ARG}

  These options may be used if the user would prefer not to use a preexisting {\mpi} compiler script, 
  but instead would rather use a serial complier and provide the flags necessary to compile the {\mpi}-aware 
  subroutines in {\sundials}.

  Often an {\mpi} implementation will have unique library names and so it may be necessary to specify the 
  appropriate libraries to use (e.g., \id{ARG=-lmpi})

  Default: \id{INCDIR=MPIDIR/include}, \id{LIBDIR=MPIDIR/lib}.

\end{config}


%%
%% Cross-compilation
%%

\vspace{0.25in}
\noindent If the {\sundials} suite will be cross-compiled (meaning the build procedure will not be completed on the actual 
destination system, but rather on an alternate system with a different architecture) then the next two options should be specified:

\begin{config}

\item \id{--build=BUILD}

  This particular command-line parameter is used to specify the canonical system/platform name for the alternate 
  system on which the software build procedure will actually be completed.  

\item \id{--host=HOST}

  If cross-compiling then the user must specify the canonical system/platform name for the destination system.


\end{config}


%%==========================================================================================

\section{Configuration examples}

The following examples are meant to help demonstrate proper usage of the optional flags.

\begin{verbatim}
% configure CC=gcc F77=g77 --with-cflags=-g --with-fflags=-g \
            --with-mpicc=/usr/apps/mpich/1.2.4/bin/mpicc \ 
            --with-mpif77=/usr/apps/mpich/1.2.4/bin/mpif77
\end{verbatim}


\begin{verbatim}
% configure CC=gcc F77=g77 --disable-examples \
            --with-mpicc=no --with-mpif77=no \
            --with-mpi-root=/usr/apps/mpich/1.2.4 \
            --with-mpi-libs=-lmpich
\end{verbatim}