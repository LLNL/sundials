Generally speaking, the installation procedure outlined in
\S\ref{ss:install_steps} below will work on commodity {\linux}/{\unix} systems
without modification. Users are still encouraged, however, to carefully read
the entire chapter before attempting to install the {\sundials} suite, in case
non-default choices are desired for compilers, compilation options, or the like.
In lieu of reading the option list below, the user may invoke the configuration
script with the help flag to view a complete listing of available options, which
may be done by issuing 
\begin{verbatim}
   % ./configure --help 
\end{verbatim}
from within the \id{sundials} directory.

In the descriptions below, {\em build\_tree} refers to the directory under
which the user wants to build and/or install the {\sundials} package. By
default, the {\sundials} libraries and header files are installed under the
subdirectories {\em build\_tree}\id{/lib} and {\em build\_tree}\id{/include},
respectively. Also, {\em source\_tree} refers to the directory where the
{\sundials} source code is located. The chosen {\em build\_tree} may be
different from the {\em source\_tree}, thus allowing for multiple installations
of the {\sundials} suite with different configuration options.

Concerning the installation procedure outlined below, after invoking the
\id{tar} command with the appropriate options, the contents of the
{\sundials} archive (or the {\em source\_tree}) will be extracted to a
directory named \id{sundials}. Since the name of the extracted directory
is not version-specific it is recommended that the user refrain from
extracting the archive to a directory containing a previous version/release
of the {\sundials} suite. If the user is only upgrading and the previous 
installation of {\sundials} is not needed, then the user may remove the
previous installation by issuing 
\begin{verbatim}
   % rm -rf sundials
\end{verbatim}
from a shell command prompt.

Even though the installation procedure given below presupposes that the user
will use the default vector modules supplied with the distribution, using the
{\sundials} suite with a user-supplied vector module normally will not require
any changes to the build procedure.  

%%===============================================================================

\section{Installation steps}\label{ss:install_steps}

To install the {\sundials} suite, given a downloaded file named
{\em sundials\_file}\id{.tar.gz}, issue the following commands from
a shell command prompt, while within the directory where {\em source\_tree}
is to be located.  


\begin{enumerate}
\item \id{gunzip} {\em sundials\_file}\id{.tar.gz}
\item \id{tar -xf} {\em sundials\_file}\id{.tar}\hspace{2em} [creates \id{sundials} directory]
\item \id{cd} {\em build\_tree}
\item {\em path\_to\_source\_tree}\id{/configure} {\em options}\hspace{2em} [options can be absent]
\item \id{make}
\item \id{make install}
\item \id{make examples}
\item If system storage space conservation is a priority, then issue \\
\verb+   make clean+ \\
and/or \\
\verb+   make examples_clean+ \\
from a shell command prompt to remove unneeded object files.
\end{enumerate}

%%===============================================================================

\section{Configuration options}\label{ss:configuration_options}

The installation procedure given above will generally work without modification;
however, if the system includes multiple {\mpi} implementations, then certain
configure script-related options may be used to indicate which {\mpi}
implementation should be used. Also, if the user wants to use non-default
language compilers, then, again, the necessary shell environment variables must
be appropriately redefined.
%%
The remainder of this section provides explanations of available configure script
options.


\subsection*{General options}

%%
%% General options
%%

\begin{config}
  
\item \id{--prefix=PREFIX}
  
  Location for architecture-independent files.
  
  Default: \id{PREFIX=}{\em build\_tree}
  
\item \id{--includedir=DIR}
  
  Alternate location for installation of header files.
  
  Default: \id{DIR=PREFIX/include}
  
\item \id{--libdir=DIR}
  
  Alternate location for installation of libraries.
  
  Default: \id{DIR=PREFIX/lib}

\item \id{--disable-examples}
  
  All available example programs are automatically built unless this option is
  given. The example executables are stored under the following subdirectories
  of the associated solver: 
  
  \begin{config}
  \item {\em build\_tree}/{\em solver}/\id{examples\_ser} : serial {\C} examples
  \item {\em build\_tree}/{\em solver}/\id{examples\_par} : parallel {\C} examples ({\mpi}-enabled)
  \item {\em build\_tree}/{\em solver}/\id{fcmix}/\id{examples\_ser} : serial {\F} examples
  \item {\em build\_tree}/{\em solver}/\id{fcmix}/\id{examples\_par} : parallel {\F} examples ({\mpi}-enabled)
  \end{config}
  
  {\em Note}: Some of these subdirectories may not exist depending upon the
  solver and/or the configuration options given.
  
\item \id{--disable-}{\em solver}

  Although each existing solver module is built by default, support for a
  given solver can be explicitly disabled using this option. 
  The valid values for {\em solver} are: \id{cvode}, \id{cvodes}, 
  \id{ida}, and \id{kinsol}.
  
\item \id{--with-cppflags=ARG}

  Specify additional {\C} preprocessor flags 
  (e.g., \id{ARG=-I<include\_dir>} if necessary header files are located in nonstandard locations).

\item \id{--with-cflags=ARG}

  Specify additional {\C} compilation flags.

\item \id{--with-ldflags=ARG}

  Specify additional linker flags 
  (e.g., \id{ARG=-L<lib\_dir>} if required libraries are located in nonstandard locations).

\item \id{--with-libs=ARG}

  Specify additional libraries to be used 
  (e.g., \id{ARG=-l<foo>} to link with the library named \id{libfoo.a} or \id{libfoo.so}).

\item \id{--with-precision=ARG}

  By default, {\sundials} will define a real number (internally referred to as
  \id{realtype}) to be a double-precision floating-point numeric data type
  (\id{double} {\C}-type); however, this option may be used to build {\sundials}
  with \id{realtype} alternatively defined as a single-precision floating-point
  numeric data type (\id{float} {\C}-type) if \id{ARG=single}, or as a
  \id{long double} {\C}-type if \\ \id{ARG=extended}.

  Default: \id{ARG=double}

  {\warn}Users should {\em not} build {\sundials} with support for
  single-precision floating-point arithmetic on 32- or 64-bit systems.
  This will almost certainly result in unreliable numerical solutions.
  The configuration option \id{--with-precision=single} is intended for
  systems on which single-precision arithmetic involves at least 14 decimal
  digits.

\end{config}

%%
%% Fortran support
%%

\subsection*{Options for Fortran support}

\begin{config}

\item \id{--disable-f77}

  Using this option will disable all {\F} support. The {\fcvode},
 {\fkinsol}, {\fida}, and {\fnvector} modules will not be built,
 regardless of availability.

\item \id{--with-fflags=ARG}

  Specify additional {\F} compilation flags.

\end{config}

\noindent The configuration script will attempt to automatically determine the
function name mangling scheme required by the specified {\F} compiler, but the
following two options may be used to override the default behavior.

\begin{config}

\item \id{--with-f77underscore=ARG}

  This option pertains to the {\fcvode}, {\fkinsol}, {\fida}, and {\fnvector}
  {\F}-{\C} interface modules and is used to specify the number of underscores to
  append to function names so {\F} routines can properly link with the associated
  {\sundials} libraries. Valid values for \id{ARG} are: \id{none}, \id{one}
  and \id{two}.

  Default: \id{ARG=one}

\item \id{--with-f77case=ARG}

  Use this option to specify whether the external names of the {\fcvode},
  {\fkinsol}, {\fida}, and {\fnvector} {\F}-{\C} interface functions should be
  lowercase or uppercase so {\F} routines can properly link with the associated
  {\sundials} libraries. Valid values for \id{ARG} are: \id{lower} and \id{upper}.

  Default: \id{ARG=lower}

\end{config}


%%
%% Parallel options
%%

\subsection*{Options for MPI support}

\noindent The following configuration options are only applicable to the parallel {\sundials} packages:

\begin{config}
  
\item \id{--disable-mpi}

  Using this option will completely disable {\mpi} support.

\item \id{--with-mpicc=ARG}
\item \id{--with-mpif77=ARG}

  By default, the configuration utility script will use the {\mpi} compiler
  scripts named \id{mpicc} and \id{mpif77} to compile the parallelized
  {\sundials} subroutines; however, for reasons of compatibility, different
  executable names may be specified via the above options. Also, \id{ARG=no}
  can be used to disable the use of {\mpi} compiler scripts, thus causing
  the serial {\C} and {\F} compilers to be used to compile the parallelized
  {\sundials} functions and examples.

\item \id{--with-mpi-root=MPIDIR}

  This option may be used to specify which {\mpi} implementation should be used.
  The {\sundials} configuration script will automatically check under the
  subdirectories \id{MPIDIR/include} and \id{MPIDIR/lib} for the necessary
  header files and libraries. The subdirectory \id{MPIDIR/bin} will also be
  searched for the {\C} and {\F} {\mpi} compiler scripts, unless the user uses
  \id{--with-mpicc=no} or \id{--with-mpif77=no}.

\item \id{--with-mpi-incdir=INCDIR}
\item \id{--with-mpi-libdir=LIBDIR}
\item \id{--with-mpi-libs=LIBS}

  These options may be used if the user would prefer not to use a preexisting
  {\mpi} compiler script, but instead would rather use a serial complier and
  provide the flags necessary to compile the {\mpi}-aware subroutines in
  {\sundials}.

  Often an {\mpi} implementation will have unique library names and so it may
  be necessary to specify the appropriate libraries to use (e.g.,
  \id{LIBS=-lmpich}).

  Default: \id{INCDIR=MPIDIR/include} and \id{LIBDIR=MPIDIR/lib}

\item \id{--with-mpi-flags=ARG}

  Specify additional {\mpi}-specific flags.

\end{config}


\subsection*{Options for library support}
%%
%% Shared or Static Libraries
%%

\noindent By default, only static libraries are built, but the following option
may be used to build shared libraries on supported platforms.

\begin{config}

\item \id{--enable-shared}

  Using this particular option will result in both static and shared versions
  of the available {\sundials} libraries being built if the system supports
  shared libraries. To build only shared libraries also specify \id{--disable-static}.

  {\em Note}: The {\fcvode}, {\fkinsol}, and {\fida} libraries can only be built as
  static libraries because they contain references to externally defined symbols,
  namely user-supplied {\F} subroutines.  Although the {\F} interfaces to the serial
  and parallel implementations of the supplied {\nvector} module do not contain any
  unresolvable external symbols, the libraries are still built as static libraries
  for the purpose of consistency.

\end{config}

\subsection*{Options for cross-compilation}

%%
%% Cross-compilation
%%

\noindent If the {\sundials} suite will be cross-compiled (meaning the build
procedure will not be completed on the actual destination system, but rather
on an alternate system with a different architecture) then the following two
options should be used:

\begin{config}

\item \id{--build=BUILD}

  This particular option is used to specify the canonical system/platform name
  for the build system.

\item \id{--host=HOST}

  If cross-compiling, then the user must use this option to specify the canonical
  system/platform name for the destination system.

\end{config}

\subsection*{Environment variables}

%%
%% Environment variables
%%

\noindent The following environment variables can be locally (re)defined for use 
during the configuration of {\sundials}. See the next section for illustrations of these.

\begin{config}

\item \id{CC}

\item \id{F77}

  Since the configuration script uses the first {\C} and {\F} compilers found in
  the current executable search path, then each relevant shell variable (\id{CC}
  and \id{F77}) must be locally (re)defined in order to use a different compiler. 
  For example, to use \id{xcc} (executable name of chosen compiler) as the {\C}
  language compiler, use \id{CC=xcc} in the configure step.

\item \id{CFLAGS}

\item \id{FFLAGS}

  Use these environment variables to override the default {\C} and {\F}
  compilation flags.

\end{config}


%%===============================================================================


\section{Configuration examples}

The following examples are meant to help demonstrate proper usage of the configure options:

\begin{verbatim}
   % configure CC=gcc F77=g77 --with-cflags=-g3 --with-fflags=-g3 \
               --with-mpicc=/usr/apps/mpich/1.2.4/bin/mpicc \ 
               --with-mpif77=/usr/apps/mpich/1.2.4/bin/mpif77
\end{verbatim}

\noindent The above example builds {\sundials} using \id{gcc} as the serial {\C}
compiler, \id{g77} as the serial {\F} compiler, \id{mpicc} as the parallel {\C}
compiler, \id{mpif77} as the parallel {\F} compiler, and appends the \id{-g3}
compilaton flag to the list of default flags.

\begin{verbatim}
   % configure CC=gcc --disable-examples --with-mpicc=no \
               --with-mpi-root=/usr/apps/mpich/1.2.4 \
               --with-mpi-libs=-lmpich
\end{verbatim}

\noindent This example again builds {\sundials} using \id{gcc} as the serial
{\C} compiler, but the \id{--with-mpicc=no} option explicitly disables the use
of the corresponding {\mpi} compiler script. In addition, since the 
\id{--with-mpi-root} option is given, the compilation flags
\id{-I/usr/apps/mpich/1.2.4/include} and \id{-L/usr/apps/mpich/1.2.4/lib} are passed
to \id{gcc} when compiling the {\mpi}-enabled functions. 
The \id{--disable-examples} option disables the examples (which means
a {\F} compiler is not required).
The \id{--with-mpi-libs} option is required so that the configure
script can check if \id{gcc} can link with the appropriate {\mpi}
library.


%%===============================================================================

\section{Installed libraries and exported header files}

Using the standard {\sundials} build system, the command
\begin{verbatim}
   % make install
\end{verbatim}
will install the libraries under {\em libdir} and the public
header files under {\em incdir}. The default values for these directories
are {\em build\_tree}\id{/lib} and {\em build\_tree}\id{/include}, respectively,
but can be changed using the configure script options \id{--prefix}, \id{--includedir}
and \id{--libdir} (see \S\ref{ss:configuration_options}). For example, a global installation
of {\sundials} on a {\sc *NIX} system could be accomplished using
\begin{verbatim}
   % configure --prefix=/usr/local
\end{verbatim}
Although all installed libraries reside under {\em libdir}, the public header files
are further organized into subdirectories under {\em incdir}.

The installed libraries and exported header files are listed for reference in 
Table~\ref{t:sundials_files}. 
The file extension .{\em lib} is typically \id{.so} for shared libraries and 
\id{.a} for static libraries (see {\em Options for library support} for additional details).

A typical user program need not explicitly include any of the shared {\sundials} header
files from under the {\em incdir}\id{/sundials} directory since they are explicitly included by the
appropriate solver header files ({\em e.g.}, \id{cvode\_dense.h} includes 
\id{sundials\_dense.h}). However, it is both legal and safe to do so
({\em e.g.}, the functions declared in \id{sundials\_smalldense.h} 
could be used in building a preconditioner).

\begin{table}
\centering
\caption{
  SUNDIALS libraries and header files (names are relative to {\em libdir}
  for libraries and to {\em incdir} for header files)
}\label{t:sundials_files}
\medskip
\begin{tabular}{|l|l|ll|} 
\hline %% --------------------------------------------------
{\shared} & Libraries    & n/a                               &                                 \\ 
\cline{2-4}
          & Header files & sundials/sundials\_types.h        & sundials/sundials\_math.h   \\
          &              & sundials/sundials\_config.h       & sundias/sundials\_nvector.h\\
          &              & sundials/sunials\_smalldense.h    & sundials/sundials\_dense.h   \\
          &              & sundials/sundials\_iterative.h    & sundials/sundials\_band.h\\
          &              & sundials/sundials\_spbcgs.h       & sundials/sundials\_sptfqmr.h\\
          &              & sundials/sundials\_spgmr.h        &                        \\ 
\hline %% --------------------------------------------------
{\nvecs}  & Libraries    & libsundials\_nvecserial.{\em lib} & libsundials\_fnvecserial.a  \\ 
\cline{2-4}
          & Header files & nvector\_serial.h                 &                       \\ 
\hline %% --------------------------------------------------
{\nvecp}  & Libraries    & libsundials\_nvecparallel.{\em lib} & libsundials\_fnvecparallel.a \\
\cline{2-4}
          & Header files & nvector\_parallel.h               &                    \\ 
\hline %% --------------------------------------------------
{\cvode}  & Libraries    & libsundials\_cvode.{\em lib}      & libsundials\_fcvode.a \\
\cline{2-4}
          & Header files & cvode.h                           &                       \\
          &              & cvode/cvode\_dense.h              & cvode/cvode\_band.h   \\
          &              & cvode/cvode\_diag.h               & cvode/cvode\_spils.h  \\
          &              & cvode/cvode\_bandpre.h            & cvode/cvode\_bbdpre.h \\
          &              & cvode/cvode\_spgmr.h              & cvode/cvode\_spbcgs.h \\
          &              & cvode/cvode\_sptfqmr.h            & cvode/cvode\_impl.h   \\
\hline %% --------------------------------------------------
{\cvodes} & Libraries    & libsundials\_cvodes.{\em lib}     &                        \\
\cline{2-4}
          & Header files & cvodes.h                          & cvodea.h              \\
          &              & cvodes/cvodes\_dense.h            & cvodes/cvodes\_band.h   \\
          &              & cvodes/cvodes\_diag.h             & cvodes/cvodes\_spils.h  \\
          &              & cvodes/cvodes\_bandpre.h          & cvodes/cvodes\_bbdpre.h \\
          &              & cvodes/cvodes\_spgmr.h            & cvodes/cvodes\_spbcgs.h \\
          &              & cvodes/cvodes\_sptfqmr.h          & cvodes/cvodes\_impl.h   \\
          &              & cvodes/cvodea\_impl.h             &                     \\
\hline %% --------------------------------------------------
{\ida}    & Libraries    & libsundials\_ida.{\em lib}        & libsundials\_fida.a \\
\cline{2-4}
          & Header files & ida.h                             &                     \\
          &              & ida/ida\_dense.h                  & ida/ida\_band.h     \\
          &              & ida/ida\_spils.h                  & ida/ida\_spgmr.h    \\
          &              & ida/ida\_spbcgs.h                 & ida/ida\_sptfqmr.h  \\
          &              & ida/ida\_bbdpre.h                 & ida/ida\_impl.h     \\
\hline %% --------------------------------------------------
{\kinsol} & Libraries    & libsundials\_kinsol.{\em lib}     & libsundials\_fkinsol.a \\
\cline{2-4}
          & Header files & kinsol.h                          &                     \\
          &              & kinsol/kinsol\_dense.h            & kinsol/kinsol\_band.h     \\
          &              & kinsol/kinsol\_spils.h            & kinsol/kinsol\_spgmr.h    \\
          &              & kinsol/kinsol\_spbcgs.h           & kinsol/kinsol\_sptfqmr.h  \\
          &              & kinsol/kinsol\_bbdpre.h           & kinsol/kinsol\_impl.h     \\
\hline %% --------------------------------------------------
\end{tabular}
\end{table}

%%===============================================================================

\section{Building SUNDIALS without the configure script}\label{ss:no_config}

If the \id{configure} script cannot be used ({\em e.g.}, when building {\sundials}
under Microsoft Windows without using Cygwin), or if the user prefers to own the build
process ({\em e.g.}, when {\sundials} is incorporated into a larger project with its
own build system), then the header and source files for a given module can be 
copied from the {\em source\_tree} to some other location and compiled separately.

The following files are required to compile a {\sundials} solver module:
\begin{itemize}
\item public header files located under 
{\em source\_tree}\id{/}{\em solver}\id{/include}
\item implementation header files and source files located under
{\em source\_tree}\id{/}{\em solver}\id{/source}
\item (optional) {\F}/{\C} interface files located under
{\em source\_tree}\id{/}{\em solver}\id{/fcmix}
\item shared public header files located under
{\em source\_tree}\id{/shared/include}
\item shared source files located under
{\em source\_tree}\id{/shared/source}
\item (optional) {\nvecs} header and source files located under
{\em source\_tree}\id{/nvec\_ser}
\item (optional) {\nvecp} header and source files located under
{\em source\_tree}\id{/nvec\_par}
\item configuration header file \id{sundials\_config.h} (see below)
\end{itemize}

A sample header file that, appropriately modified, can be used
as \id{sundials\_config.h} (otherwise created automatically by the \id{configure} script)
is provided below. The various preprocessor macros defined within \id{sundials\_config.h}
have the following uses:
\begin{itemize}

\item Precision of the {\sundials} \id{realtype} type\\ \\
  Only one of the macros \id{SUNDIALS\_SINGLE\_PRECISION}, \id{SUNDIALS\_DOUBLE\_PRECISION} and \\
  \id{SUNDIALS\_EXTENDED\_PRECISION} should be defined to indicate if the {\sundials}
  \id{realtype} type is an alias for \id{float}, \id{double}, or \id{long double},
  respectively.

\item Use of generic math functions\\ \\
  If \id{SUNDIALS\_USE\_GENERIC\_MATH} is defined, then the functions
  in \id{sundials\_math.(h,c)} will use the \id{pow}, \id{sqrt} and \id{fabs}
  functions from the standard math library (see \id{math.h}), irregardless of the definition of \id{realtype}.
  Otherwise, if \id{realtype} is defined to be an alias for the \id{float} {\C}-type, then
  {\sundials} will use \id{powf}, \id{sqrtf} and \id{fabsf}.
  If \id{realtype} is instead defined to be a synonym for the \id{long double} {\C}-type,
  then \id{powl}, \id{sqrtl} and \id{fabsl} will be used.

  {\em Note}: Although the \id{powf}/\id{powl}, \id{sqrtf}/\id{sqrtl} and \id{fabsf}/\id{fabsl}
  routines are not specified in the ANSI {\C} standard, they are ISO C99 requirements. Consequently,
  these routines will only be used if available.

\item {\F} name-mangling scheme\\ \\
  The macros given below are used to transform the {\C}-language function names defined in the
  {\F}-{\C} inteface modules in a manner consistent with the preferred {\F} compiler, thus
  allowing native {\C} functions to be called from within a {\F} subroutine. The name-mangling
  scheme can be specified either by appropriately defining the parameterized macros (using the
  stringization operator, \id{\#\#}, if necessary)
  \begin{itemize}
  \item \id{F77\_FUNC(name,NAME)}
  \item \id{F77\_FUNC\_(name,NAME)}
  \end{itemize}
  or by defining {\em one} macro from each of the following lists:
  \begin{itemize}
  \item \id{SUNDIALS\_CASE\_LOWER} or \id{SUNDIALS\_CASE\_UPPER}
  \item \id{SUNDIALS\_UNDERSCORE\_NONE}, \id{SUNDIALS\_UNDERSCORE\_ONE}, or \id{SUNDIALS\_UNDERSCORE\_TWO}
  \end{itemize}

  For example, to specify that mangled {\C}-language function names should be lowercase with one
  underscore appended include either
\begin{verbatim}
  #define F77_FUNC(name,NAME) name ## _
  #define F77_FUNC_(name,NAME) name ## _
\end{verbatim}
  or
\begin{verbatim}
  #define SUNDIALS_CASE_LOWER 1
  #define SUNDIALS_UNDERSCORE_ONE 1
\end{verbatim}
  in the \id{sundials\_config.h} header file.

\item Use of an {\mpi} communicator other than \id{MPI\_COMM\_WORLD} in {\F}\\ \\
  If the macro \id{SUNDIALS\_MPI\_COMM\_F2C} is defined, then the {\mpi} implementation
  used to build {\sundials} defines the type \id{MPI\_Fint} and the function \id{MPI\_Comm\_f2c},
  and it is possible to use {\mpi} communicators other than \id{MPI\_COMM\_WORLD} with the
  {\F}-{\C} interface modules.
\end{itemize}


\begin{verbatim}
/*
 * -----------------------------------------------------------------
 * Copyright (c) 2005, The Regents of the University of California.
 * Produced at the Lawrence Livermore National Laboratory.
 * All rights reserved.
 * For details, see sundials/shared/LICENSE.
 * -----------------------------------------------------------------
 * SUNDIALS configuration header file
 * -----------------------------------------------------------------
 */
 
/* ------------------------------  
 * Define SUNDIALS version number
 * ------------------------------ */

#define SUNDIALS_PACKAGE_VERSION "2.2.0"
 
/* ------------------------------------------------- 
 * Define precision of SUNDIALS data type 'realtype'
 * ------------------------------------------------- */

/* Define SUNDIALS data type 'realtype' as 'double' */
#define SUNDIALS_DOUBLE_PRECISION 1

/* Define SUNDIALS data type 'realtype' as 'float' */
/* #define SUNDIALS_SINGLE_PRECISION 1 */

/* Define SUNDIALS data type 'realtype' as 'long double' */
/* #define SUNDIALS_EXTENDED_PRECISION 1 */

/* --------------------------
 * Use generic math functions
 * -------------------------- */

#define SUNDIALS_USE_GENERIC_MATH 1
  
/* -----------------------------------------
 * FCMIX: Define Fortran name-mangling macro
 * ----------------------------------------- */

#define F77_FUNC(name,NAME) name ## _
#define F77_FUNC_(name,NAME) name ## _

/* ------------------------------------
 * FCMIX: Define case of function names
 * ------------------------------------ */
 
/* FCMIX: Make function names lowercase */
/* #define SUNDIALS_CASE_LOWER 1 */

/* FCMIX: Make function names uppercase */
/* #define SUNDIALS_CASE_UPPER 1 */

/* ---------------------------------------------------------------
 * FCMIX: Define number of underscores to append to function names
 * --------------------------------------------------------------- */

/* FCMIX: Do NOT append any underscores to functions names */
/* #define SUNDIALS_UNDERSCORE_NONE 1 */

/* FCMIX: Append ONE underscore to function names */
/* #define SUNDIALS_UNDERSCORE_ONE 1 */

/* FCMIX: Append TWO underscores to function names */
/* #define SUNDIALS_UNDERSCORE_TWO 1 */
 
/* ----------------------------------------------------------
 * FNVECTOR: Allow user to specify different MPI communicator
 * ---------------------------------------------------------- */

#define SUNDIALS_MPI_COMM_F2C 1

\end{verbatim}
