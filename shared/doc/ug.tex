\usepackage{epsfig, supertabular, makeidx}
\usepackage{amsmath, amssymb}

% Package to add Bibliography and Index to TOC
% but not the TOC itself :-)
\usepackage[nottoc]{tocbibind}

\usepackage{hangcaption}
\usepackage{subfigure}

\usepackage{color}

% Packages needed for the cover page
\usepackage{calc, epsfig, chngpage}
\input{pstricks}\input{pst-node}\input{llnlCoverPage}

% Package useful for debuging
\usepackage{showlabels}

%----- Define some colors
\definecolor{gray}{rgb}{0.5,0.5,0.5} 

%----- Page formatting
\setlength{\oddsidemargin}{0in}
\setlength{\evensidemargin}{0in}
\setlength{\textwidth}{6.5in}
\setlength{\textheight}{8.5in}

%----- SUNDIALS MODULES
\newcommand{\sundials}{{\normalfont\scshape sundials}}
\newcommand{\nvector}{{\normalfont\scshape nvector}}
\newcommand{\nvecp}{{\normalfont\scshape nvector\_parallel}}
\newcommand{\nvecs}{{\normalfont\scshape nvector\_serial}}
\newcommand{\cvode}{{\normalfont\scshape cvode}}
\newcommand{\pvode}{{\normalfont\scshape pvode}}
\newcommand{\cvodes}{{\normalfont\scshape cvodes}}
\newcommand{\ida}{{\normalfont\scshape ida}}
\newcommand{\idas}{{\normalfont\scshape idas}}
\newcommand{\kinsol}{{\normalfont\scshape kinsol}}
\newcommand{\kinsols}{{\normalfont\scshape kinsols}}

%----- OTHER PACKAGES
\newcommand{\vode}{{\normalfont\scshape vode}}
\newcommand{\vodpk}{{\normalfont\scshape vodpk}}
\newcommand{\lsode}{{\normalfont\scshape lsode}}

%----- CVODES COMPONENTS
\newcommand{\cvdense}{{\normalfont\scshape cvdense}}
\newcommand{\cvband}{{\normalfont\scshape cvband}}
\newcommand{\cvdiag}{{\normalfont\scshape cvdiag}}
\newcommand{\cvspgmr}{{\normalfont\scshape cvspgmr}}
\newcommand{\cvbandpre}{{\normalfont\scshape cvbandpre}}
\newcommand{\cvbbdpre}{{\normalfont\scshape cvbbdpre}}
\newcommand{\stald}{{\normalfont\scshape stald}}

%----- SHARED COMPONENTS
\newcommand{\sundialsmath}{{\normalfont\scshape sundialsmath}}
\newcommand{\dense}{{\normalfont\scshape dense}}
\newcommand{\band}{{\normalfont\scshape band}}
\newcommand{\spgmr}{{\normalfont\scshape spgmr}}

%----- C and Fortran languages
\newcommand{\C}{{\sc C}}
\newcommand{\CPP}{{\sc C++}}
\newcommand{\F}{{\sc Fortran}}

%------ Serial or Parallel
\newcommand{\p}{[{\bf P}]}
\newcommand{\s}{[{\bf S}]}

%------ Appendix in text
\newcommand{\A}{App. }

%------ Index entries
\newcommand{\ID}[1]{{\tt #1}{\index{#1@\texttt{#1}|textbf}}}
\newcommand{\Id}[1]{{\tt #1}{\index{#1@\texttt{#1}}}}
\newcommand{\id}[1]{{\tt #1}}

%------ Cross reference to the user guide
\newcommand{\ugref}[1]{\ref{#1} in the user guide}

%%----- Shortcuts for math formulas
\newcommand{\mb}[1]{{\mbox{\scriptsize #1}}}
\newcommand{\dfdy}{\frac{\partial f}{\partial y}}
\newcommand{\dfdyI}{\partial f / \partial y}
\newcommand{\dfdpi}{\frac{\partial f}{\partial p_i}}
\newcommand{\dfdpiI}{\partial f / \partial p_i}
\newcommand{\rhomax}{\rho_{\max}}

%%--------------------------------
\newcommand{\frontug}
{
  \maketitle

  %% Leave an empty page
  \newpage\thispagestyle{empty}
  \vspace*{1.0in}
  
  %% Start roman numbering
  \pagestyle{plain}\pagenumbering{roman}
  \tableofcontents
  \listoftables
  \listoffigures
  
  \newpage\thispagestyle{empty}
  \vspace*{1.0in}
  
  %% Move to new page and start arabic numbering
  \newpage\pagestyle{plain}\pagenumbering{arabic}
}

%%--------------------------------
\newcommand{\frontex}
{
  \maketitle

  %% Leave an empty page
  \newpage\thispagestyle{empty}
  \vspace*{1.0in}
  
  %% Start roman numbering
  \pagestyle{plain}\pagenumbering{roman}
  \tableofcontents
  
  %% Move to new page and start arabic numbering
  \newpage\pagestyle{plain}\pagenumbering{arabic}
}

%%---------------------------------
%% Steps used in scheleton programs
%%---------------------------------
\newcounter{Stepsctr}
\newenvironment{Steps}
{\stepcounter{Stepsctr}
  \begin{list}{\arabic{Stepsctr}. }{
      \usecounter{Stepsctr}
      \setlength{\parsep}{0.5em}
      \setlength{\labelsep}{0em}
      \settowidth{\labelwidth}{99. }
      \setlength{\leftmargin}{\labelwidth+\labelsep}}}
  {\end{list}}
%%-----------------------------------------------------
%% Underlying list environemnt for function definitions
%%-----------------------------------------------------
\newenvironment{Ventry}[1][\quad]
{\begin{list}{}{
      \setlength{\rightmargin}{0em}
      \setlength{\topsep}{0.15in}
      \setlength{\itemsep}{0em}
      \renewcommand{\makelabel}[1]{##1\hfill}
      \settowidth{\labelwidth}{#1}
      \setlength{\leftmargin}{\labelwidth+\labelsep}}}
  {\end{list}}
%%----------------------------------
%% List of function arguments
%%---------------------------------
\newenvironment{args}[1][\quad]
{\begin{list}{}{
      \setlength{\rightmargin}{0em}
      \setlength{\topsep}{0em}
      \setlength{\itemsep}{0em}
      \renewcommand{\makelabel}[1]{\id{##1}\hfill}
      \settowidth{\labelwidth}{\id{#1}}
      \setlength{\leftmargin}{\labelwidth+\labelsep}}}
  {\end{list}}
%%---------------------------------
%% User-callable function
%%---------------------------------
\newcommand{\ucfunction}[6]{
  \begin{Ventry}[Return value]
  \item[\fbox{\id{#1}}]{}
  \item[Call]{\id{#2}}
  \item[Description]{#3}
  \item[Arguments]{#4}
  \item[Return value]{#5}
  \addNotes{#6}
  \end{Ventry}
}
%%---------------------------------
%% User-supplied function
%%---------------------------------
\newcommand{\usfunction}[6]{
  \begin{Ventry}[Return value]
  \item[\fbox{\id{#1}}]{}
  \item[Definition]{\id{\begin{tabular}[t]{@{}r@{}l@{}}#2\end{tabular}}}
  \item[Purpose]{#3}
  \item[Arguments]{#4}
  \item[Return value]{#5}
  \addNotes{#6}
  \end{Ventry}
}
%%---------------------------------
\makeatletter
\long\def\addNotes#1{\def\@tempa{#1}\ifx\@tempa\empty\else\item[Notes]{#1}\fi}
\makeatother